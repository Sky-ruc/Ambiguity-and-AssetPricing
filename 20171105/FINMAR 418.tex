% !TEX TS-program = latex
\documentclass[10pt]{article}
\usepackage{amsmath, amssymb}
\usepackage{mathrsfs}
\usepackage{graphicx}
\usepackage{array,color}
\usepackage{booktabs}
\usepackage{multirow}
\usepackage{textcomp}
%\usepackage{geometry}
\usepackage{lscape}
\usepackage{tikz}
\usepackage{graphicx}
\usepackage{epsfig}
\usepackage{colortbl}
\usepackage{pstricks-add}
\setlength{\topmargin}{-0.40 in}
\setlength{\oddsidemargin}{0.1in}
\setlength{\textwidth}{6.25 in}
\setlength{\textheight}{9.00 in}
\renewcommand{\baselinestretch}{1.48}
\definecolor{Red}{rgb}{1,0,0}
\newcommand{\hhred}{\textcolor{red}}
\newtheorem{prop}{Proposition}
\newtheorem{cor}{Corollary}
\newtheorem{thm}{Theorem}
\newtheorem{lemma}{Lemma}
\newtheorem{example}{Example}
\newtheorem{assumption}{Assumption}
\newtheorem{definition}{Definition}
\newtheorem{claim}{Claim}
\newtheorem{remark}{Remark}
\newtheorem{result}{Result}
\newcommand{\HIDE}[1]{}
\usepackage{mathptmx}
\usepackage{xcolor}
\usepackage{longtable}
\usepackage{dcolumn}
%\usepackage[]{footmisc}
\usepackage{rotating,booktabs}
\usepackage{booktabs,threeparttable}
%\usepackage{tabularx}
\definecolor{Red}{rgb}{1,0,0}
\definecolor{Blue}{rgb}{0,0,1}
\definecolor{Green}{rgb}{0,1,0}
\definecolor{magenta}{rgb}{1,0,0.6}
\definecolor{lightblue}{rgb}{0,0.5,1}
\definecolor{lightpurple}{rgb}{0.6,0.4,1}
\definecolor{gold}{rgb}{0.6,0.5,0}
\definecolor{orange}{rgb}{1,0.4,0}
\definecolor{hotpink}{rgb}{1,0,0.5}
\definecolor{newcolor2}{rgb}{0.5,0.3,0.5}
\definecolor{newcolor}{rgb}{0,0.3,1}
\definecolor{newcolor3}{rgb}{1,0,0.35}
\definecolor{darkgreen1}{rgb}{0,0.35, 0}
\definecolor{darkgreen}{rgb}{0,0.6, 0}
\definecolor{darkred}{rgb}{0.75,0,0}
%\xdefinecolor{olive}{cmyk}{0.64,0,0.95,0.4}
%\xdefinecolor{purpleish}{cmyk}{0.75,0.75,0,0}

\begin{document}
\countdef\pageno=0\pageno=1
\overfullrule=0pt

\title{Limited participation under ambiguity of correlation
\footnote{\baselineskip1.3em This work is supported by the Social Sciences and Humanities Research Council of Canada (SSHRC) Insight Development Grant ($\#$ 430-2012-0698), the National Natural Science Foundation of China (NSFC Grant Numbers: 71273271 and 71573220), the Major Basic Research Plan of Renmin University of China (Grant Number: 14XNL001), and the KPMG research scholar award (Faculty of Business Administration, University of Regina). All errors are our own.}}
\author{{\small Helen Hui Huang} \\ {\small Faculty of Business Administration, University of Regina, Regina, SK, S4S 0A2, Canada} \\ {\small E-mail: helen.huang@uregina.ca} \\ {\small Shunming Zhang}\footnote{\baselineskip1.3em Corresponding author. School of Finance, Renmin University of China, Beijing 100872, P.R.China. E-mail: szhang@ruc.edu.cn. URL: http://sf.ruc.edu.cn/archives/5523} \\ {\small China Financial Policy Research Center, Renmin University of China, Beijing 100872, P.R.China} \\ {\small E-mail: szhang@ruc.edu.cn} \\ {\small Wei Zhu} \\ {\small Department of Statistics and Actuarial Science, University of Hong Kong, Pokfulam, Hong Kong} \\ {\small E-mail: michael\_wzhu@hku.hk}}
%\author{Helen H. Huang\footnote{\baselineskip1.3em Faculty of Business Administration, University of Regina, Regina, SK, S4S 0A2, Canada. E-mail: helen.huang@uregina.ca.} \qquad Shunming Zhang\footnote{\baselineskip1.3em Correspondence author.}\footnote{\baselineskip1.3em China Financial Policy Research Center, Renmin University of China, Beijing 100872, P.R.China. E-mail: szhang@ruc.edu.cn.} \qquad Wei Zhu\footnote{\baselineskip1.3em Department of Statistics and Actuarial Science, University of Hong Kong, Pokfulam, Hong Kong. E-mail: michael\_wzhu@hku.hk.}}
\date{\today}
\maketitle

\begin{abstract}
In this paper, we investigate the implications of correlation ambiguity for investor behaviors and asset prices. In our model, individuals' decision making incorporates both risk and ambiguity, and we demonstrate that limited participation arises from the rational decision by na\"ive investors to avoid correlation ambiguity. In equilibrium, the asset with lower quality generates positive excess returns. Comparative static analysis of the equilibrium result suggests that changes in the fraction of na\"ive investors and ambiguity level can alter equilibrium types and flight to quality phenomenon is observed. However, their impacts on asset prices are non-monotonic. 

{\it JEL classification}: G02, G11, D80, D81.

{\it Keywords}: Ambiguity aversion; Correlation ambiguity; General equilibrium; Limited participation; Flight to quality.
\end{abstract}

\newpage

\section{Introduction}

\quad \ 
Ambiguity aversion arises when individuals incorporate both risk and ambiguity in their decision making. Knight (1921) makes a distinction between known odds (risk) and unknown odds (ambiguity). Gilboa and Schmeidler (1989) and Schmeidler (1989) conduct axiomatic analyses for decision making with ambiguity aversion. So far, ambiguity has been widely applied to many kinds of asset pricing models (e.g., Chen and Epstein, 2002; Illeditsch, 2009; Epstein and Ji, 2012, 2013) and provides explanatory power ffor some behavioral anomalies in financial markets, such as limited participation (e.g., Cao, Wang, and Zhang, 2005; Easley and O'Hara, 2009), negative skewness in returns (Epstein and Schneider, 2008), and so on. However, researchers focus more on ambiguity of mean and variance, and ignore the ambiguity of correlation, an equally important parameter of the economy. Correlation is ubiquitous in financial markets and plays a central role in portfolio choices (Markowitz, 1952; Samuelson, 1967) and asset pricing models (e.g., Sharpe, 1964; Lintner, 1965; Duffie and Singleton, 2003). Thus, it is natural and of importance for us to study the ambiguity of correlation. 

In this article, we focus on the implications of correlation ambiguity for three reasons. First, we think that it is natural and intuitive for na\"ive investors to perceive ambiguity in the correlation between assets rather than in mean or variance of individual assets. Unlike the expected payoff and volatility, correlation measures between the prices of different assets are rarely disclosed. Hence, it is rather difficult for non-professional investors to form up the precise perception of connections among assets. That is, by observing the data of two assets, one can hardly tell whether they are more likely to be positively or negatively correlated, let alone estimating the true value of the parameter. Second, as with all other types of ambiguity, the introduction of correlation ambiguity appeals to the robustness of decision making. Although there are many statistical methods for professionals to estimate correlation coefficients, a single estimate is far away reliable for future investment decisions, especially when the economic environment shifts rapidly. Thus it is a wiser choice for professionals to make investment decisions with references to more than one prior, instead of acting on a single prior. Third, theoretically speaking, as it turns out, ambiguity of correlation can indeed generate interesting implications for the financial markets, which we elaborate later.

The economy we investigate in this paper is a natural extension of the models in Cao, Wang, and Zhang (2005) and Easley and O'Hara (2009). The economy has one risk-free asset and two risky assets. The risky assets have normally distributed payoffs. Disparate from Easley and O'Hara (2009) where ambiguity lies in the expected payoffs and variances of risky assets, heterogeneous beliefs of the correlation coefficient between two risky assets are assumed among investors, while they have common knowledge of the means and variances of those two assets.\footnote{\baselineskip1.3em Here we assume the existence of two different types of investors. However, for simplicity, heterogeneity among investors in the same category is ignored.} Sophisticated investors are standard expected utility maximizing agents and have rational expectations of the parameters of the economy. Na\"ive investors are ambiguity-averse agents who have rational expectations on the marginal distributions of assets but perceive the correlation coefficient of the assets as ambiguous. The decision-making of these na\"ive investors is described by the maxmin utility model proposed by Gilboa and Schmeidler (1989).

As compared to Easley and O'Hara (2009), the demand function of na\"ive investors shows three new and interesting features. First, nonparticipating decisions on risky assets arise from the rational decisions by na\"ive investors to avoid ambiguity. The decisions to hold incomplete portfolios are determined by the set of correlation coefficients that are considered by na\"ive investors. Second, the demand functions of na\"ive investors are continuous and have kinks at certain prices. Third, na\"ive investors trade in the same direction as sophisticated investors. That is, when na\"ive investors hold a non-zero position on a risky asset, the sophisticated investors' position is in the same direction. Under this setup, we show that na\"ive investors' demand function intersects with the demand of sophisticated investors, which implies that na\"ive investors might hold larger positions than sophisticated investors. This result differs from the result of Easley and O'Hara (2009).

In equilibrium, the product of the standard deviation and per capita endowment acts as a measure of quality. The unique equilibrium prevailing in the economy has three alternative types. When the quality of an asset is comparatively small (the ratio against the other asset's quality smaller than a threshold that is determined by the true value of the correlation coefficient and degree of ambiguity), nonparticipation on this asset will occur as an endogenous result. Note that the limited participation result in our paper is different from that in Easley and O'Hara (2009) in that in our result non-participation on both assets cannot happen. We also show that under equilibrium, na\"ive investors could trade more intensively than sophisticated investors in one of the risky assets.

Further analysis suggests another two key implications of correlation ambiguity for asset prices and limited market participation. First, the capital asset pricing model (CAPM) analysis reveals that the asset with lower quality generates positive excess returns, no matter whether it is held by na\"ive investors. Second, the flight to quality phenomenon will be observed when parameters in the economy change. Specifically, when the fraction of na\"ive investors drops or the level of ambiguity increases, the prevailing participating equilibrium might be altered to a non-participating one, and naive investors will only hold the assets with higher quality. However, the influence of parameter changes on asset prices are non-monotonic, which suggests that policies pertaining to ambiguity can have profound impacts on asset prices and social welfare. In addition, increasing the market participation of one risky asset will not cause the nonparticipation of the other asset. 

Our model is closely related to a large and growing literature on the behavioral phenomena and asset pricing anomalies in the financial markets, in the presence of ambiguity aversion. Cao, Wang, and Zhang (2005) and Easley and O'Hara (2009) use the maxmin expected utility model to demonstrate limited participation as a result of ambiguity about the asset's expected payoff and variance. Condie and Ganguli (2011) demonstrate the existence and robustness of partially revealing rational expectations equilibria in economies with ambiguity-averse preferences. Easley, O'Hara, and Yang (2014) investigate the effect of ambiguity in hedge fund strategies on asset prices and aggregate welfare, thus providing a profound interpretation of disclosure policies implemented upon hedge funds.

Our paper is different and thus complements the above papers in two ways. First, our source of ambiguity is different. In the literature on limited participation in the presence of ambiguity, investors are ambiguous about individual assets' expected payoffs and volatility. Easley, O'Hara, and Yang (2014) also investigate the ambiguity of correlation, however the ambiguity is assumed to be about the opaque traders' effective risk tolerance, which is a function of the correlation between the equity asset and the extra investment opportunities to which only opaque traders have access. Instead, we directly assume the ambiguity of the correlation of two risky assets, and our general equilibrium framework generates the result that na\"ive investors' trade-off between holding all risky assets and holding an under-diversified portfolio. Second, correlation ambiguity suggests some interactions between non-participations. For example, our results suggest that an increase in market participation of one risky asset as a result of a changing degree of ambiguity will not cause the non-participation of the other asset. 

This paper is structured as follows. In Section 2, we develop a multi-asset model that includes na\"ive and sophisticated investors. Individual demand functions are calculated. In Section 3, we present how limited participation decisions are made, solve for market equilibrium, and show that three possible types of equilibriums can prevail in the economy. We analyze the CAPM in Section 4 and investigate how parameter changes affect equilibrium types and prices. We conclude in Section 5. Proofs are provided in the Appendix.

%\newpage

\section{Basic model}
%\renewcommand{\theequation}{2.\arabic{equation}}
%\setcounter{equation}0

\quad \
Our model extends Easley and O'Hara (2009) by allowing risky assets to be dependent. We analyze an economy with three assets. One risk-free asset is money, which has a constant price of 1 and is in zero supply; two risky financial assets are normally distributed with payoffs, $ \tilde X_i $, $ i = 1, 2 $. These risky assets can be interpreted as individual stocks, bonds, mutual funds, exchange-traded funds (ETFs), or even deposits. The mean payoff and variance for asset $i$ are $ \mu_i $ and $ \sigma_i^2 $. The correlation coefficient between the payoffs of risky assets is measured by $ \rho $, differing from Easley and O'Hara (2009) where the two risky assets are independent. Therefore, the payoffs of risky assets follow a two-dimensional normal distribution $ \tilde X \!\!\!\!\!\! X \sim {\bf N} (\mu \!\!\!\!\!\! \mu, \Sigma \!\!\!\!\! \Sigma (\rho)) $, where:
\begin{eqnarray*}
\tilde X \!\!\!\!\!\! X = \left( \begin{matrix} \tilde X_1 \\ \tilde X_2 \end{matrix} \right), \qquad \mu \!\!\!\!\!\! \mu = \left( \begin{matrix} \mu_1 \\ \mu_2 \end{matrix} \right), \qquad \Sigma \!\!\!\!\! \Sigma (\rho) = \left( \begin{matrix} \sigma_1^2 & \rho \sigma_1 \sigma_2 \\ \rho \sigma_1 \sigma_2 & \sigma_2^2 \end{matrix} \right).
\end{eqnarray*}

All investors have constant absolute risk aversion (CARA) utility for wealth, with the risk aversion parameter set equal to $ \alpha $:
\begin{eqnarray}
u (w) = - e^{- \alpha w}.
\end{eqnarray}

Two types of investors with heterogeneous beliefs participate in the economy: sophisticated investors ($S$) and na\"ive investors ($N$). Na\"ive investors constitute a fraction $ \theta \in [0, 1] $ of all investors, while sophisticated investors constitute the remaining fraction $ 1 - \theta $. The sophisticated investors are standard expected utility (EU) maximizers with rational expectations about payoff parameters. Let $ \hat \rho $ denote the true correlation coefficient. Since our sophisticated investors have rational expectations, they know the true value $ \hat \rho $. From their viewpoint, the payoffs follow normally distributed $ \tilde X \!\!\!\!\!\! X \sim {\bf N} (\mu \!\!\!\!\!\! \mu, \Sigma \!\!\!\!\! \Sigma (\hat \rho)) $. The na\"ive investors know about the means and variances of the risky assets' payoffs; however, they differ from sophisticated investors in that they do not know the exact value of the correlation coefficient. They consider an interval $ [\underline{\rho}, \overline{\rho}] \subset [- 1, 1] $ of values possible with $ - 1 < \underline{\rho} < \overline{\rho} < 1 $ and they do not have a prior among them. For any $ \rho \in [\underline{\rho}, \overline{\rho}] $, na\"ive investors must face the possible economy $ \tilde X \!\!\!\!\!\! X \sim {\bf N} (\mu \!\!\!\!\!\! \mu, \Sigma \!\!\!\!\! \Sigma (\rho)) $, so they take them all into account when they make decisions. Following Gilboa and Schmeidler's (1989) axiomatic foundation for ambiguity aversion, we model these na\"ive investors as choosing a portfolio to maximize their minimum expected utility over the set of possible distributions. To make our analysis of the equilibrium interaction between sophisticated and na\"ive investors interesting, we assume that the true parameter value for sophisticated investors is a convex combination of the extreme values considered possible by the na\"ive investors, $ \hat \rho \in [\underline{\rho}, \overline{\rho}] $.

The per capita endowments of the risky assets are $ (Z^0_1, Z^0_2) $. The exact distribution of endowments over investors does not affect their demand for risky assets, so we do not specify it. We denote a typical investor's wealth by $w$. Where no confusion would occur, we drop the investor index. The investor's budget constraint is:
\begin{eqnarray}
w = m + p_1 z_1 + p_2 z_2,
\end{eqnarray}
where $m$ is the quantity of money, $ p_i $ is the price of asset $i$, and $ z_i $ is the quantity demanded of risky asset $i$. Investors are allowed to go long or short for each asset. If an investor chooses portfolio $ (m, z_1, z_2) $, her random next period wealth will be:
\begin{eqnarray}
\tilde w = m + \tilde X_1 z_1 + \tilde X_2 z_2.
\end{eqnarray}
Equivalently, we indicate the investor choice as $ (w - p_1 z_1 - p_2 z_2, z_1, z_2) $, then her random next period wealth will be written as:
\begin{eqnarray*}
\tilde{w}  = w + (\tilde{X}_1 - p_1) z_1 + (\tilde{X}_2 - p_2) z_2.
\end{eqnarray*}

For a sophisticated investor with CARA utility of wealth and correlation coefficient parameter $ \hat \rho $, the expected utility of this random wealth is a strictly increasing transformation of:
\begin{eqnarray}
f (z_1, z_2, \hat{\rho}) = (\mu_1 - p_1) z_1 + (\mu_2 - p_2) z_2 - \frac12 \alpha \left[ \sigma_1^2 z_1^2 + 2 \hat{\rho} \sigma_1 \sigma_2 z_1 z_2 + \sigma_2^2 z_2^2 \right] + w.
\end{eqnarray}

Calculation shows that the sophisticated investor's demand function for risky assets is given by:
\begin{eqnarray}
Z_S^* = \left( \begin{matrix} Z_{S 1}^* \\ Z_{S 2}^* \end{matrix} \right) 
= \dfrac1{\alpha \sigma_1^2 \sigma_2^2 (1 - {\hat \rho}^2)} \left( \begin{matrix} \sigma_2^2 (\mu_1 - p_1) - {\hat \rho} \sigma_1 \sigma_2 (\mu_2 - p_2) \\ \sigma_1^2 (\mu_2 - p_2) - {\hat \rho} \sigma_1 \sigma_2 (\mu_1 - p_1) \end{matrix} \right).
\end{eqnarray}
We define $ R_i = \dfrac{\mu_i - p_i}{\sigma_i} $ to be the Sharpe ratio (Sharpe, 1966), which measures an average of how much additional profit one can get by taking one more unit of risk for $ i = 1, 2 $. So equation (5) can be written as:
\begin{eqnarray}
Z_S^* = \left( \begin{matrix} Z_{S 1}^* \\ Z_{S 2}^* \end{matrix} \right) = \dfrac1{\alpha (1 - {\hat \rho}^2)} \left( \begin{matrix} \dfrac{R_1 - {\hat \rho} R_2}{\sigma_1} \\ \dfrac{R_2 - {\hat \rho} R_1}{\sigma_2} \end{matrix} \right).
\end{eqnarray}

A na\"ive investor evaluates the expected utility of wealth for each correlation coefficient parameter and chooses the portfolio that maximizes the minimum of these expected utilities. In effect, the na\"ive investor tries to avoid the worst case outcomes and so chooses a portfolio that explicitly limits exposure to such adverse outcomes. The expected utility of random wealth, given the correlation coefficient parameter $ \rho $, is a strictly increasing transformation of:
\begin{eqnarray}
f (z_1, z_2, \rho) & = & (\mu_1 - p_1) z_1 + (\mu_2 - p_2) z_2 - \frac12 \alpha \left[ \sigma_1^2 z_1^2 + 2 \rho \sigma_1 \sigma_2 z_1 z_2 + \sigma_2^2 z_2^2 \right] + w \nonumber \\
& = & \sigma_1 R_1 z_1 + \sigma_2 R_2 z_2 - \frac12 \alpha \left[ \sigma_1^2 z_1^2 + 2 \rho \sigma_1 \sigma_2 z_1 z_2 + \sigma_2^2 z_2^2 \right] + w.
\end{eqnarray}
Thus, the na\"ive investor's decision problem can be written as two-level mathematical programming:
\begin{eqnarray}
\max_{(z_1, z_2)} \min_{\rho \in [\underline{\rho}, \overline{\rho}]} f (z_1, z_2, \rho) = \sigma_1 R_1 z_1 + \sigma_2 R_2 z_2 - \frac12 \alpha \left[ \sigma_1^2 z_1^2 + 2 \rho \sigma_1 \sigma_2 z_1 z_2 + \sigma_2^2 z_2^2 \right] + w.
\end{eqnarray}

By examining the minimization problem, we reveal that for any portfolio the minimum occurs at the minimum possible correlation if trading strategies for the two assets have different signs and at the maximum possible correlation if trading strategies for the two assets have the same signs. Whether the minimum occurs at the maximum or minimum mean payoff depends on the sign of $ z_1 z_2 $. 
\begin{eqnarray}
\min_{\rho \in [\underline{\rho}, \overline{\rho}]} f (z_1, z_2, \rho) 
= \left\{ \begin{matrix} 
f (z_1, z_2, \underline{\rho}), & \text{if} \ z_1 z_2 < 0 \\ 
f (z_1, z_2, \overline{\rho}), & \text{if} \ z_1 z_2 > 0 \\
f (0, z_2, \rho), & \text{if} \ z_1 = 0 \\ 
f (z_1, 0, \rho), & \text{if} \ z_2 = 0.
\end{matrix} \right.
\end{eqnarray}

Equation (9) illustrates a segmented curved surface. It indicates that for any portfolio, the minimum occurs at the endpoints of the interval $ [\underline{\rho}, \overline{\rho}] $. Consequently, what matters to a na\"ive investor is not the correlation coefficient values inside the set, but rather the extreme values of the correlation coefficient. Whether the minimum occurs at $ \underline{\rho} $ or $ \overline{\rho} $ depends on the investor's positions on the assets. The minimum occurs at $ \underline{\rho} $ if the investor goes long on one risky asset and short on the other and at $ \overline{\rho} $ if the investor goes long (or short) on both risky assets.
%We can use Sion's Theorem to calculate the na\"ive investor's demand function.

In Appendix A1, we solve minimization problem (9) and obtain the four separate solutions for the corresponding optimal problems. The four solutions are merged into the global solution to two-level mathematical programming (8).\footnote{\baselineskip1.3em There is another approach to work out the mathematical programming (8) by using Sion's Theorem.} The results are listed in Theorem 1.

\vskip 8 pt

\begin{thm}
The na\"ive investor's demand function for risky assets is given by:
\begin{eqnarray}
Z_N^* = \left( \begin{matrix} Z_{N 1}^* \\ N_{N 2}^* \end{matrix} \right) = \left\{ \begin{matrix}
\dfrac1{\alpha (1 - \underline{\rho}^2)} \left( \begin{matrix} \dfrac{R_1 - \underline{\rho} R_2}{\sigma_1} \\ \dfrac{R_2 - \underline{\rho} R_1}{\sigma_2} \end{matrix} \right), \quad \text{if} \quad \left\{ \begin{matrix} R_1 < \underline{\rho} R_2 \\ R_2 > \underline{\rho} R_1 \end{matrix} \right. \quad \text{or} \quad \left\{ \begin{matrix} R_1 > \underline{\rho} R_2 \\ R_2 < \underline{\rho} R_1 \end{matrix} \right. \\
\dfrac1{\alpha} \left( \begin{matrix} 0 \\ \dfrac{R_2}{\sigma_2} \end{matrix} \right), \quad \text{if} \quad \left\{ \begin{matrix} \overline{\rho} R_2 \leqslant R_1 \leqslant \underline{\rho} R_2 \\ R_2 < 0 \end{matrix} \right. \quad \text{or} \quad \left\{ \begin{matrix} \underline{\rho} R_2 \leqslant R_1 \leqslant \overline{\rho} R_2 \\ R_2 > 0 \end{matrix} \right. \\
\dfrac1{\alpha} \left( \begin{matrix} \dfrac{R_1}{\sigma_1} \\ 0 \end{matrix} \right), \quad \text{if} \quad \left\{ \begin{matrix} R_1 < 0 \\ \overline{\rho} R_1 \leqslant R_2 \leqslant \underline{\rho} R_1 \end{matrix} \right. \quad \text{or} \quad \left\{ \begin{matrix} R_1 > 0 \\ \underline{\rho} R_1 \leqslant R_2 \leqslant \overline{\rho} R_1 \end{matrix} \right. \\
\dfrac1{\alpha (1 - \overline{\rho}^2)} \left( \begin{matrix} \dfrac{R_1 - \overline{\rho} R_2}{\sigma_1} \\ \dfrac{R_2 - \overline{\rho} R_1}{\sigma_2} \end{matrix} \right), \quad \text{if} \quad \left\{ \begin{matrix} R_1 < \overline{\rho} R_2 \\ R_2 < \overline{\rho} R_1 \end{matrix} \right. \quad \text{or} \quad \left\{ \begin{matrix} R_1 > \overline{\rho} R_2 \\ R_2 > \overline{\rho} R_1. \end{matrix} \right.
\end{matrix} \right.
\end{eqnarray}
\end{thm}

The discussion of the unique properties of na\"ive investor's demand function for risky assets is postponed to next section.
We also consider the equilibrium condition: the per capita demand for assets equals per capita supply. Equating the demands from equations (6) and (10) to this supply then results in:
\begin{eqnarray}
(1 - \theta) Z_S^* + \theta Z_N^* = Z^0
\end{eqnarray}
or $ (1 - \theta) Z_{S i}^* + \theta Z_{N i}^* = Z_i^0 $ for $ i = 1, 2 $.
Depending on the parameters of the economy, there are four possible types of solutions to this equation.

%\newpage

\section{Characterization of equilibrium}
%\renewcommand{\theequation}{3.\arabic{equation}}
%\setcounter{equation}0

\quad \ 
In this section, we check the existence of equilibrium. We firstly present various interesting propositions of na\"ive investors' demand functions. We then calculate the equilibrium price according to the four cases in Section 2. We obtain the general equilibrium under three different conditions and hence three types of equilibria. These different types of equilibria can explain the limited participation phenomenon.

\subsection{Properties of na\"ive investors' demand functions}

\quad \ 
Now we examine the properties of the na\"ive investor's demand function. Figures A2.1 - A2.10 in Appendix A2 depict the demand function for asset 1 against its price $ p_1 $, fixing $ p_2 $, the price of asset 2. Meanwhile, Figures A2.11 - A2.20 in Appendix A2 depict the demand for asset 1 against $ p_2 $, fixing $ p_1 $. Figure A2 shows some interesting properties of the behaviors of na\"ive investors. First, the na\"ive investor's demand function is continuous in price but it has kinks at several prices: $ \mu_1 - \underline{\rho} \dfrac{\sigma_1}{\sigma_2} (\mu_2 - p_2) $, $ \mu_1 - \overline{\rho} \dfrac{\sigma_1}{\sigma_2} (\mu_2 - p_2) $, $ \mu_1 - \dfrac1{\underline{\rho}} \dfrac{\sigma_1}{\sigma_2} (\mu_2 - p_2) $, and $ \mu_1 - \dfrac1{\overline{\rho}} \dfrac{\sigma_1}{\sigma_2} (\mu_2 - p_2) $ for $ p_1 $ and $ \mu_2 - \underline{\rho} \dfrac{\sigma_2}{\sigma_1} (\mu_1 - p_1) $, $ \mu_2 - \overline{\rho} \dfrac{\sigma_2}{\sigma_1} (\mu_1 - p_1) $, $ \mu_2 - \dfrac1{\underline{\rho}} \dfrac{\sigma_2}{\sigma_1} (\mu_1 - p_1) $, and $ \mu_2 - \dfrac1{\overline{\rho}} \dfrac{\sigma_2}{\sigma_1} (\mu_1 - p_1) $ for $ p_2 $. This feature is very similar to the demand function in Easley and O'Hara (2009).

Second, limited participation phenomenon can be observed. Na\"ive investors will not participate in the transaction of asset 1 when $ \mu_1 - \overline{\rho} \dfrac{\sigma_1}{\sigma_2} (\mu_2 - p_2) \leqslant p_1 \leqslant \mu_1 - \underline{\rho} \dfrac{\sigma_1}{\sigma_2} (\mu_2 - p_2) $ if $ p_2 < \mu_2 $ and when $ \mu_1 - \underline{\rho} \dfrac{\sigma_1}{\sigma_2} (\mu_2 - p_2) \leqslant p_1 \leqslant \mu_1 - \overline{\rho} \dfrac{\sigma_1}{\sigma_2} (\mu_2 - p_2) $ if $ p_2 > \mu_2 $; while na\"ive investors will not participate in the transaction of asset 2 when $ \mu_2 - \overline{\rho} \dfrac{\sigma_2}{\sigma_1} (\mu_1 - p_1) \leqslant p_2 \leqslant \mu_2 - \underline{\rho} \dfrac{\sigma_2}{\sigma_1} (\mu_1 - p_1) $ if $ p_1 < \mu_1 $ and when $ \mu_2 - \underline{\rho} \dfrac{\sigma_2}{\sigma_1} (\mu_1 - p_1) \leqslant p_2 \leqslant \mu_2 - \overline{\rho} \dfrac{\sigma_2}{\sigma_1} (\mu_1 - p_1) $ if $ p_1 > \mu_1 $. This non-participation phenomenon occurs because a na\"ive investor faces the ambiguity that lies in the correlation coefficient when the investor trades both assets. Under the maxmin utility framework by Gilboa and Schmeidler (1989), the na\"ive investor is extremely pessimistic and heavily influenced by the worst possible state (i.e., the maximum and minimum possible correlation coefficients). Therefore, if the Sharpe ratio of the risky asset is too small to be held, but not so negative enough for na\"ive investors to short, na\"ive investors are better holding just one risky asset, which keeps them away from the ambiguity of correlation coefficient. \footnote{\baselineskip1.3em Note that when a na\"ive investor holds just one asset, the investor has a rational expectation of this asset and acts as if this asset is the only risky asset in the economy.}

Third, the na\"ive investor's decision about whether to hold assets is independent of the set of correlation coefficients the investor believes possible as discrete or continuous. All that matters for the participation decision is the two extreme values, $ \underline{\rho} $ and $ \overline{\rho} $. As for the choice of which asset is traded, it depends on the relation between $ R_i = \dfrac{\mu_i - p_i}{\sigma_i} $ for $ i = 1, 2 $, $ \underline{\rho} $ and $ \overline{\rho} $. When non-participation occurs, neither $ \underline{\rho} $ nor $ \overline{\rho} $ affects the quantity of the position a na\"ive investor holds. However, one of these two extreme values will affect how much is held if the investor trades both risky assets. The other values of the correlation coefficient the investor considers possible are irrelevant to whether the investor participate or how much is held when participating.

Fourth, Figures A2 shows that $ Z_{S i}^* Z_{N i}^* \geqslant 0 $ for $ i = 1, 2 $. Na\"ive and sophisticated investors trade in the same direction, except for the scenarios when non-participation occurs, i.e., $ Z_{N i}^* = 0 $. When na\"ive investors go long (or short) on some asset, sophisticated investors will also go long (or short). 

\vskip 8 pt

\begin{prop}
$ Z_{S i}^* Z_{N i}^* \geqslant 0 $ for $ i = 1, 2 $.
\end{prop}

\vskip 8 pt

First of all, Proposition 1 helps us to eliminate impossible cases of equilibrium. Under the assumption that both assets have positive supply, this proposition immediately rules out the equilibriums in which sophisticated investors short either or both of the risky assets. Thus, we only have three types of equilibriums possible in our economy. Second, this result tells us that these two types of investors are simultaneously on the demand side or the supply side of the risky assets. This prevents sophisticated investors from exploiting the na\"ive investor's lack of information or confidence when the latter wrongly shorts or longs the assets. We can also consider this as a result of the na\"ive investor's prudence in trading, since the investor will not long or short an asset when sophisticated investors do otherwise.

\vskip 8 pt

When modeling ambiguity averse decision makers, they are often depicted the literature as either lacking experience and knowledge (or confidence), or seeking a decision that is robust to a set of different values for a parameter. However, decision maker uncertainty about parameters (or robust about parameters) is not equal to holding a more conservative trading strategy. Easley and O'Hara (2009) suggest that the sophisticated investor always holds a larger amount (in absolute value) of the risky asset than does the na\"ive investor. They argue that when compared to sophisticated investors who avoid risk and require compensation, a na\"ive investor also avoids ambiguity in the distribution of payoffs, thus reducing the size of the position in the risky asset. However, we posit in Proposition 2 that even when ambiguity aversion distorts the na\"ive investors' behaviors, they might still select a more aggressive position to hold.\footnote{\baselineskip1.3em This is an interesting and counter intuitive feature that draws our attention when we observe the demand curves of the two types of investors. We can easily observe that in Figure A2 the sophisticated investors' demand function intersects the na\"ive investors' demand function on sereral intervals.} The specific conditions under which the na\"ive investors hold larger positions than sophisticated investors are given in Proposition 2.

\vskip 8 pt

\begin{prop}
Na\"ive investors might hold larger positions (long or short) than sophisticated investors. Specifically, for asset 1, we have $ |Z_{N1}^*| > |Z_{S1}^*|$ if and only if one of the following four scenarios occurs: 
%$ \left\{ \begin{matrix} \hat{\rho} < 0, \quad R_1 < 0 \\ \hat{\rho} R_1 \leqslant R_2 < \dfrac{\underline{\rho} + {\hat \rho}}{1 + \underline{\rho} {\hat \rho}} R_1 \end{matrix} \right. $, or $ \left\{ \begin{matrix} \hat{\rho} < 0, \quad R_1 > 0 \\ \dfrac{\underline{\rho} + {\hat \rho}}{1 + \underline{\rho} {\hat \rho}} R_1 < R_2 \leqslant \hat{\rho} R_1 \end{matrix} \right. $, or $ \left\{ \begin{matrix} \hat{\rho} > 0, \quad R_1 < 0 \\ \dfrac{\overline{\rho} + {\hat \rho}}{1 + \overline{\rho} {\hat \rho}} R_1 < R_2 \leqslant \hat{\rho} R_1 \end{matrix} \right. $ or $ \left\{ \begin{matrix} \hat{\rho} > 0, \quad R_1 > 0 \\ \hat{\rho} R_1 \leqslant R_2 < \dfrac{\overline{\rho} + {\hat \rho}}{1 + \overline{\rho} {\hat \rho}} R_1 \end{matrix} \right. $.
\begin{enumerate}
\item [(1).] $ \hat{\rho} < 0 $, $ R_1 < 0 $ and $ \hat{\rho} R_1 \leqslant R_2 < \dfrac{\underline{\rho} + {\hat \rho}}{1 + \underline{\rho} {\hat \rho}} R_1 $,
\item [(2).] $ \hat{\rho} < 0 $, $ R_1 > 0 $ and $ \dfrac{\underline{\rho} + {\hat \rho}}{1 + \underline{\rho} {\hat \rho}} R_1 < R_2 \leqslant \hat{\rho} R_1 $, 
\item [(3).] $ \hat{\rho} > 0 $, $ R_1 < 0 $ and $ \dfrac{\overline{\rho} + {\hat \rho}}{1 + \overline{\rho} {\hat \rho}} R_1 < R_2 \leqslant \hat{\rho} R_1 $,
\item [(4).] $ \hat{\rho} > 0 $, $ R_1 > 0 $ and $ \hat{\rho} R_1 \leqslant R_2 < \dfrac{\overline{\rho} + {\hat \rho}}{1 + \overline{\rho} {\hat \rho}} R_1 $.
\end{enumerate}
For asset 2, the result is symmetric, that is $|Z_{N2}^{*}| > |Z_{S2}^{*}|$ if and only if one of the following four scenarios occurs: 
%$ \left\{ \begin{matrix} \hat{\rho} < 0, \quad R_2 < 0 \\ \hat{\rho} R_2 \leqslant R_1 < \dfrac{\underline{\rho} + {\hat \rho}}{1 + \underline{\rho} {\hat \rho}} R_2 \end{matrix} \right. $, or $ \left\{ \begin{matrix} \hat{\rho} < 0, \quad R_2 > 0 \\ \dfrac{\underline{\rho} + {\hat \rho}}{1 + \underline{\rho} {\hat \rho}} R_2 < R_1 \leqslant \hat{\rho} R_2 \end{matrix} \right. $, or $ \left\{ \begin{matrix} \hat{\rho} > 0, \quad R_2 < 0 \\ \dfrac{\overline{\rho} + {\hat \rho}}{1 + \overline{\rho} {\hat \rho}} R_2 < R_1 \leqslant \hat{\rho} R_2 \end{matrix} \right. $ or $ \left\{ \begin{matrix} \hat{\rho} > 0, \quad R_2 > 0 \\ \hat{\rho} R_2 \leqslant R_1 < \dfrac{\overline{\rho} + {\hat \rho}}{1 + \overline{\rho} {\hat \rho}} R_2 \end{matrix} \right. $.  
\begin{enumerate}
\item [(1).] $ \hat{\rho} < 0 $, $ \hat{\rho} R_2 \leqslant R_1 < \dfrac{\underline{\rho} + {\hat \rho}}{1 + \underline{\rho} {\hat \rho}} R_2 $ and $ R_2 < 0 $,
\item [(2).] $ \hat{\rho} < 0 $, $ \dfrac{\underline{\rho} + {\hat \rho}}{1 + \underline{\rho} {\hat \rho}} R_2 < R_1 \leqslant \hat{\rho} R_2 $ and $ R_2 > 0 $, 
\item [(3).] $ \hat{\rho} > 0 $, $ \dfrac{\overline{\rho} + {\hat \rho}}{1 + \overline{\rho} {\hat \rho}} R_2 < R_1 \leqslant \hat{\rho} R_2 $ and $ R_2 < 0 $,
\item [(4).] $ \hat{\rho} > 0 $, $ \hat{\rho} R_2 \leqslant R_1 < \dfrac{\overline{\rho} + {\hat \rho}}{1 + \overline{\rho} {\hat \rho}} R_2 $ and $ R_2 > 0 $.
\end{enumerate}
\end{prop}
	
\subsection{General equilibrium}

\quad \ 
According to Proposition 1, if the na\"ive investors short one of the risky assets, the sophisticated investors also short the same asset, which makes it impossible for the economy to reach an equilibrium since the supply of this asset is strictly positive. So the scenario in which $ Z_{N 1}^* Z_{N 2}^* < 0 $ will not happen in equilibrium. We now explore the other three scenarios: $ Z_{N 1}^* = 0 $, $ Z_{N 1}^* Z_{N 2}^* > 0 $, and $ Z_{N 2}^* = 0 $.

\vskip 8 pt

\underline{Scenario 1}: For $ Z_{N 1}^* = 0 $, the na\"ive investors do not participate in the trading of asset 1. General equilibrium exists only when the na\"ive investors hold a long position of asset 2, $ \left\{ \begin{matrix} \underline{\rho} R_2 \leqslant R_1 \leqslant \overline{\rho} R_2 \\ R_2 > 0 \end{matrix} \right. $. The equilibrium prices for the two risky assets are given by:
\begin{eqnarray}
& p_1 = \mu_1 - \alpha \sigma_1 \dfrac{(1 - \theta {\hat \rho}^2) \sigma_1 Z_1^0 + (1 - \theta) {\hat \rho} \sigma_2 Z_2^0}{1 - \theta} & \\
& p_2 = \mu_2 - \alpha \sigma_2 ({\hat \rho} \sigma_1 Z_1^0 + \sigma_2 Z_2^0). &
\end{eqnarray}
The equilibrium condition can be equivalently written as $ \dfrac{\sigma_1 Z_1^0}{\sigma_2 Z_2^0} \leqslant \dfrac{(1 - \theta) (\overline{\rho} - {\hat \rho})}{(1 - \theta {\hat \rho}^2) - (1 - \theta) {\hat \rho} \overline{\rho}} $.

\vskip 8 pt

\underline{Scenario 2}: For $ Z_{N 1}^* Z_{N 2}^* > 0 $, the na\"ive investors trade both risky assets. General equilibrium exists only when na\"ive investors hold long positions on both risky assets, $ \left\{ \begin{matrix} R_1 > \overline{\rho} R_2 \\ R_2 > \overline{\rho} R_1 \end{matrix} \right. $. The equilibrium prices for the two risky assets are given by:
\begin{eqnarray}
& p_1 = \mu_1 - \alpha \sigma_1 \dfrac{\left[ \dfrac{1 - \theta}{1 - {\hat \rho}^2} + \dfrac{\theta}{1 - \overline{\rho}^2} \right] \sigma_1 Z_1^0 + \left[ \dfrac{1 - \theta}{1 - {\hat \rho}^2} {\hat \rho} + \dfrac{\theta}{1 - \overline{\rho}^2} \overline{\rho} \right] \sigma_2 Z_2^0}{\left[ \dfrac{1 - \theta}{1 - {\hat \rho}^2} + \dfrac{\theta}{1 - \overline{\rho}^2} \right]^2 - \left[ \dfrac{1 - \theta}{1 - {\hat \rho}^2} {\hat \rho} + \dfrac{\theta}{1 - \overline{\rho}^2} \overline{\rho} \right]^2} & \\
& p_2 = \mu_2 - \alpha \sigma_2 \dfrac{\left[ \dfrac{1 - \theta}{1 - {\hat \rho}^2} {\hat \rho} + \dfrac{\theta}{1 - \overline{\rho}^2} \overline{\rho} \right] \sigma_1 Z_1^0 + \left[ \dfrac{1 - \theta}{1 - {\hat \rho}^2} + \dfrac{\theta}{1 - \overline{\rho}^2} \right] \sigma_2 Z_2^0}{\left[ \dfrac{1 - \theta}{1 - {\hat \rho}^2} + \dfrac{\theta}{1 - \overline{\rho}^2} \right]^2 - \left[ \dfrac{1 - \theta}{1 - {\hat \rho}^2} {\hat \rho} + \dfrac{\theta}{1 - \overline{\rho}^2} \overline{\rho} \right]^2}. &
\end{eqnarray}
The equilibrium condition can be equivalently written as:
\begin{eqnarray*}
\dfrac{(1 - \theta) (\overline{\rho} - {\hat \rho})}{(1 - \theta {\hat \rho}^2) - (1 - \theta) {\hat \rho} \overline{\rho}} < \dfrac{\sigma_1 Z_1^0}{\sigma_2 Z_2^0} < \dfrac{(1 - \theta {\hat \rho}^2) - (1 - \theta) {\hat \rho} \overline{\rho}}{(1 - \theta) (\overline{\rho} - {\hat \rho})}.
\end{eqnarray*}

\vskip 8 pt

\underline{Scenario 3}: For $ Z_{N 2}^* = 0 $, the na\"ive investors do not participate in the trading of asset 2. General equilibrium exists only when the na\"ive investors hold a long position of asset 1, $ \left\{ \begin{matrix} R_1 > 0 \\ \underline{\rho} R_1 \leqslant R_2 \leqslant \overline{\rho} R_1 \end{matrix} \right. $. The equilibrium prices for the two risky assets are given by:
\begin{eqnarray}
& p_1 = \mu_1 - \alpha \sigma_1 (\sigma_1 Z_1^0 + {\hat \rho} \sigma_2 Z_2^0) & \\
& p_2 = \mu_2 - \alpha \sigma_2 \dfrac{(1 - \theta) {\hat \rho} \sigma_1 Z_1^0 + (1 - \theta {\hat \rho}^2) \sigma_2 Z_2^0}{1 - \theta}. &
\end{eqnarray}
The equilibrium condition can be equivalently written as $ \dfrac{(1 - \theta {\hat \rho}^2) - (1 - \theta) {\hat \rho} \overline{\rho}}{(1 - \theta) (\overline{\rho} - {\hat \rho})} \leqslant \dfrac{\sigma_1 Z_1^0}{\sigma_2 Z_2^0} $.

\vskip 8 pt

We define a measure of quality of risky asset $i$ as the product of standard deviation and per capita endowment, $ \sigma_i Z_i^0 $. We denote the ratio of quality of the two risky assets as:
\begin{eqnarray*}
E_{1 2} = \dfrac{\sigma_1 Z_1^0}{\sigma_2 Z_2^0} \qquad \text{and} \qquad E_{2 1} = \dfrac{\sigma_2 Z_2^0}{\sigma_1 Z_1^0}.
\end{eqnarray*}
We define, for $ \overline{\rho} \in ({\hat \rho}, 1) $,
\begin{eqnarray*}
h (\theta, \overline{\rho}, {\hat \rho}) = \dfrac{(1 - \theta) (\overline{\rho} - {\hat \rho})}{(1 - \theta {\hat \rho}^2) - (1 - \theta) {\hat \rho} \overline{\rho}} \quad \text{and} \quad H (\theta, \overline{\rho}, {\hat \rho}) = \dfrac{(1 - \theta {\hat \rho}^2) - (1 - \theta) {\hat \rho} \overline{\rho}}{(1 - \theta) (\overline{\rho} - {\hat \rho})},
\end{eqnarray*}
\begin{eqnarray*}
& q (\theta, \overline{\rho}, {\hat \rho}) = \dfrac{\dfrac{1 - \theta}{1 - {\hat \rho}^2} {\hat \rho} + \dfrac{\theta}{1 - \overline{\rho}^2} \overline{\rho}}{\left[ \dfrac{1 - \theta}{1 - {\hat \rho}^2} + \dfrac{\theta}{1 - \overline{\rho}^2} \right]^2 - \left[ \dfrac{1 - \theta}{1 - {\hat \rho}^2} {\hat \rho} + \dfrac{\theta}{1 - \overline{\rho}^2} \overline{\rho} \right]^2}, & \\
& Q (\theta, \overline{\rho}, {\hat \rho}) = \dfrac{\dfrac{1 - \theta}{1 - {\hat \rho}^2} + \dfrac{\theta}{1 - \overline{\rho}^2}}{\left[ \dfrac{1 - \theta}{1 - {\hat \rho}^2} + \dfrac{\theta}{1 - \overline{\rho}^2} \right]^2 - \left[ \dfrac{1 - \theta}{1 - {\hat \rho}^2} {\hat \rho} + \dfrac{\theta}{1 - \overline{\rho}^2} \overline{\rho} \right]^2} &
\end{eqnarray*}
and
\begin{eqnarray*}
D (\theta, \overline{\rho}, {\hat \rho}) = \dfrac{\dfrac{1}{1 - \overline{\rho}^2} \dfrac{\overline{\rho} - {\hat \rho}}{1 - {\hat \rho}^2}}{\left[ \dfrac{1 - \theta}{1 - {\hat \rho}^2} + \dfrac{\theta}{1 - \overline{\rho}^2} \right]^2 - \left[ \dfrac{1 - \theta}{1 - {\hat \rho}^2} {\hat \rho} + \dfrac{\theta}{1 - \overline{\rho}^2} \overline{\rho} \right]^2},
\end{eqnarray*}
then $ 0 < h (\theta, \overline{\rho}, {\hat \rho}) < 1 < H (\theta, \overline{\rho}, {\hat \rho}) = \dfrac{1}{h (\theta, \overline{\rho}, {\hat \rho})} $ and $ q (\theta, \overline{\rho}, {\hat \rho}) < Q (\theta, \overline{\rho}, {\hat \rho}) $.
We summarize the above analysis and present the existence of general equilibrium in Theorem 2. 

\vskip 16 pt

\begin{thm}
There exists a unique equilibrium in the markets. It is one of three types:

[1] {\bf Non-participating in Asset 1}. If the quality ratio is small, $ E_{1 2} \leqslant h (\theta, \overline{\rho}, {\hat \rho}) $, the equilibrium prices of the risky assets are:
\begin{eqnarray*}
p_1 = \mu_1 - \alpha \sigma_1 \left[ \dfrac{1 - \theta {\hat \rho}^2}{1 - \theta} \sigma_1 Z_1^0 + {\hat \rho} \sigma_2 Z_2^0 \right] \quad \text{and} \quad p_2 = \mu_2 - \alpha \sigma_2 \left[ {\hat \rho} \sigma_1 Z_1^0 + \sigma_2 Z_2^0 \right].
\end{eqnarray*}
In equilibrium, a sophisticated investor holds positions of risky assets:
\begin{eqnarray*}
Z_{S 1}^* = \dfrac{1}{1 - \theta} Z_1^0 > 0 \qquad \text{and} \qquad Z_{S 2}^* = \left[ - \dfrac{\theta}{1 - \theta} {\hat \rho} E_{1 2} + 1 \right] Z_2^0 > 0
\end{eqnarray*}
and a na\"ive investor holds positions of risky assets:
\begin{eqnarray*}
Z_{N 1}^* = 0 \qquad \text{and} \qquad Z_{N 2}^* = \left[ {\hat \rho} E_{1 2} + 1 \right] Z_2^0 > 0.
\end{eqnarray*}

[2] {\bf Participating in Both Assets}. If the quality ratio is moderate, $ h (\theta, \overline{\rho}, {\hat \rho}) < E_{1 2} < H (\theta, \overline{\rho}, {\hat \rho}) $, the equilibrium prices of the risky assets are:
\begin{eqnarray*}
& p_1 = \mu_1 - \alpha \sigma_1 \left[ Q (\theta, \overline{\rho}, {\hat \rho}) \sigma_1 Z_1^0 + q (\theta, \overline{\rho}, {\hat \rho}) \sigma_2 Z_2^0 \right] & \\
& p_2 = \mu_2 - \alpha \sigma_2 \left[ q (\theta, \overline{\rho}, {\hat \rho}) \sigma_1 Z_1^0 + Q (\theta, \overline{\rho}, {\hat \rho}) \sigma_2 Z_2^0 \right]. &
\end{eqnarray*}
In equilibrium, a sophisticated investor holds positions of risky assets:
\begin{eqnarray*}
& & Z_{S 1}^* = \left[ Q (\theta, \overline{\rho}, {\hat \rho}) - \theta {\hat \rho} D (\theta, \overline{\rho}, {\hat \rho}) + \theta D (\theta, \overline{\rho}, {\hat \rho}) E_{2 1} \right] Z_1^0 > 0 \\
& & Z_{S 2}^* = \left[ \theta D (\theta, \overline{\rho}, {\hat \rho}) E_{1 2} + Q (\theta, \overline{\rho}, {\hat \rho}) - \theta {\hat \rho} D (\theta, \overline{\rho}, {\hat \rho}) \right] Z_2^0 > 0
\end{eqnarray*}
and a na\"ive investor holds positive positions of risky assets:
\begin{eqnarray*}
& & Z_{N 1}^* = \left[ Q (\theta, \overline{\rho}, {\hat \rho}) + (1 - \theta) \overline{\rho} D (\theta, \overline{\rho}, {\hat \rho}) + (\theta - 1) D (\theta, \overline{\rho}, {\hat \rho}) E_{2 1} \right] Z_1^0 > 0 \\
& & Z_{N 2}^* = \left[ (\theta - 1) D (\theta, \overline{\rho}, {\hat \rho}) E_{1 2} + Q (\theta, \overline{\rho}, {\hat \rho}) + (1 - \theta) \overline{\rho} D (\theta, \overline{\rho}, {\hat \rho}) \right] Z_2^0 > 0.
\end{eqnarray*}

[3] {\bf Non-participating in Asset 2}. If the quality ratio is big, $ H (\theta, \overline{\rho}, {\hat \rho}) \leqslant E_{1 2} $, the equilibrium prices of the risky assets are:
\begin{eqnarray*}
p_1 = \mu_1 - \alpha \sigma_1 \left[ \sigma_1 Z_1^0 + {\hat \rho} \sigma_2 Z_2^0 \right] \quad \text{and} \quad p_2 = \mu_2 - \alpha \sigma_2 \left[ {\hat \rho} \sigma_1 Z_1^0 + \dfrac{1 - \theta {\hat \rho}^2}{1 - \theta} \sigma_2 Z_2^0 \right].
\end{eqnarray*}
In equilibrium, a sophisticated investor holds positions of risky assets:
\begin{eqnarray*}
Z_{S 1}^* = \left[ 1 - \dfrac{\theta}{1 - \theta} {\hat \rho} E_{2 1} \right] Z_1^0 > 0 \qquad \text{and} \qquad Z_{S 2}^* = \dfrac{1}{1 - \theta} Z_2^0 > 0
\end{eqnarray*}
and a na\"ive investor holds positions of risky assets:
\begin{eqnarray*}
Z_{N 1}^* = \left[ 1 + {\hat \rho} E_{2 1} \right] Z_1^0 > 0 \qquad \text{and} \qquad Z_{N 2}^* = 0.
\end{eqnarray*}
\end{thm}

%\vskip 8 pt

{\bf Remark 1}. According to symmetry, the conditions in Theorem 2 can be written as
[1] General equilibrium is {\bf Non-participating in Asset 1} if the quality ratio is big, $ H (\theta, \overline{\rho}, {\hat \rho}) \leqslant E_{2 1} $;
[2] General equilibrium is {\bf Participating in Both Assets} if the quality ratio is moderate, $ h (\theta, \overline{\rho}, {\hat \rho}) < E_{2 1} < H (\theta, \overline{\rho}, {\hat \rho}) $; and 
[3] General equilibrium is {\bf Non-participating in Asset 2} if the quality ratio is small, $ E_{2 1} \leqslant h (\theta, \overline{\rho}, {\hat \rho}) $.

\vskip 8 pt

{\bf Remark 2}. From Theorem 2 we know that the minimum correlation coefficient $ \underline{\rho} $ does not matter with respect to deciding which equilibrium type occurs in the economy or affecting equilibrium prices. Its elimination directly results from our model settings. First note that we assume that the per capita endowment for each risky asset is positive. $ \underline{\rho} $ enters the expressions of the na\"ive investors' demand function only given the specific scenario that they sell one risky asset and buy  another. However, from Theorem 1, we know that this scenario indicates that they hold a negative position in one of the risky assets. According to Proposition 1, sophisticated investors also hold a negative position for this asset, making it impossible to meet the equilibrium condition pertaining to this asset.

\vskip 8 pt

{\bf Remark 3}. Both $ h (\theta, \overline{\rho}, {\hat \rho}) $ and $ H (\theta, \overline{\rho}, {\hat \rho}) $ form the thresholds that determine whether the equilibrium is a participating one or not. These two critical values are determined by the fraction of na\"ive investors and the maximum correlation coefficient. Furthermore, $ h (\theta, \overline{\rho}, {\hat \rho}) $ is increasing in the maximum correlation coefficient with $ h (\theta, {\hat \rho}, {\hat \rho}) = 0 $ and $ h (\theta, 1, {\hat \rho}) = \dfrac{1 - \theta}{1 + \theta {\hat \rho}} < 1 $, and $ H (\theta, \overline{\rho}, {\hat \rho}) $ is decreasing in the maximum correlation coefficient with $ H (\theta, {\hat \rho}, {\hat \rho}) = \infty $ and $ H (\theta, 1, {\hat \rho}) = \dfrac{1 + \theta {\hat \rho}}{1 - \theta} > 1 $.
%\hhred{\bf Remark 3}. Both $ h (\theta, \overline{\rho}, {\hat \rho}) $ and $ H (\theta, \overline{\rho}, {\hat \rho}) $ form the thresholds which determine whether the equlibrium is a participating one or not. Note that $ \Delta \rho = \overline{\rho} -\hat{\rho} $ measures the degree of ambiguity that na\"ive investors face, thus these two critical values are determined by fraction of na\"ive investors and degree of ambiguity. Furthermore, $ h (\theta, \overline{\rho}, {\hat \rho}) $ is increasing in degree of ambiguity and $ H (\theta, \overline{\rho}, {\hat \rho}) $ is decreasing in degree of ambiguity.

\vskip 8 pt

{\bf Remark 4}. As shown in the previous subsection, the decision on whether to hold an asset is made by comparing the Sharpe ratios of the two assets. In equilibrium, after the Sharpe ratios are solved endogenously, comparing the Sharpe ratio is equivalent to comparing the exogenous quality of risky asset $ \sigma_i Z_i^0 $ for $ i = 1, 2 $. If the quality of asset 1 is small enough relative to asset 2 (the quality ratio $ E_{12} $ smaller than the threshold $ h (\theta, \overline{\rho}, {\hat \rho}) $), na\"ive investors make non-participation decisions on asset 1. If the quality of asset 2 is small enough relative to asset 1 (the quality ratio $ E_{21} $ smaller than the threshold $ h (\theta, \overline{\rho}, {\hat \rho}) $), na\"ive investors make non-participation decisions on asset 2. Thus the higher the quality, the more favorable the asset is. footnote{\baselineskip1.3em In the financial markets, for example in the stock market, the larger companies supply more stocks; however, generally they tend to have lower volatility on payoffs. Thus the quality is determined by the two opposing effects and it remains an empirical question as to which asset is of higher quality.} In the next section, we show more implications of this quality measure. 

\vskip 8 pt

{\bf Remark 5}. Another interesting feature of the limited participation in our model is that non-participation on both risky assets can not happen under equilibrium. This can be observed directly from Theorem 2. This phenomenon is unique since it cannot be found in the models of ambiguity for expected payoffs or volatilities. Intuitively, if a na\"ive investor decides not to trade an asset with low quality, the individual avoids the ambiguity of correlation and invests rationally in the other risky asset.

\vskip 8 pt

We can also compare the positions that both sophisticated investors and na\"ive investors hold. We list the results as follows:

\vskip 8 pt

\begin{prop}
In equilibrium, na\"ive investors might hold larger positions (long or short) than sophisticated investors. If the true correlation coefficient is positive, $ {\hat \rho} > 0 $, then we have the following results:
\begin{enumerate}
\item [(1).] On non-participating equilibrium in asset 1, $ E_{1 2} \leqslant h (\theta, \overline{\rho}, {\hat \rho}) $, and hence $ Z_{N 2}^* > Z_{S 2}^* > 0 $.
\item [(2).] On participating equilibrium in both assets, $ h (\theta, \overline{\rho}, {\hat \rho}) < E_{1 2} < H (\theta, \overline{\rho}, {\hat \rho}) $, and hence $ Z_{N 1}^* > Z_{S 1}^* > 0 $ if and only if $ E_{2 1} < (1 - \theta) \overline{\rho} + \theta {\hat \rho} $ and $ Z_{N 2}^* > Z_{S 2}^* > 0 $ if and only if $ E_{1 2} < (1 - \theta) \overline{\rho} + \theta {\hat \rho} $.
\item [(3).] On non-participating equilibrium in asset 2, $ E_{2 1} \leqslant h (\theta, \overline{\rho}, {\hat \rho}) $, and hence $ Z_{N 1}^* > Z_{S 1}^* > 0 $.
\end{enumerate}
\end{prop}

\vskip 8 pt

From the proposition 3, we can see that na\"ive investors trade more intensively on asset 1 only when its quality is large enough, or in other words, the quality ratio is less than the threshold $ (1 - \theta) \overline{\rho} + \theta {\hat \rho} $. 

%\underline{Scenario 1}: On non-participating equilibrium in asset 1, $ E_{1 2} \leqslant h (\theta, \overline{\rho}, {\hat \rho}) $, and hence $ Z_{N 1}^* = 0 < \dfrac{Z_1^0}{1 - \theta} = Z_{S 1}^* $ and $ 0 < Z_{S 2}^* = Z_2^0 - \dfrac{\theta}{1 - \theta} {\hat \rho} \dfrac{\sigma_1}{\sigma_2} Z_1^0 < {\hat \rho} \dfrac{\sigma_1}{\sigma_2} Z_1^0 + Z_2^0 = Z_{N 2}^* $ if $\hat{\rho}>0$.

%\underline{Scenario 2}: On participating equilibrium in both assets, $ h (\theta, \overline{\rho}, {\hat \rho}) < E_{1 2} < H (\theta, \overline{\rho}, {\hat \rho}) $, 
%\begin{eqnarray*}
%& Z_{N 1}^* - Z_{S 1}^* = \dfrac{1}{\sigma_1} D (\theta, \overline{\rho}, {\hat \rho}) \left\{ \left[ (1 - \theta) \overline{\rho} + \theta {\hat \rho} \right] \sigma_1 Z_1^0 - \sigma_2 Z_2^0 \right\} & \\
%& Z_{N 2}^* - Z_{S 2}^* = \dfrac{1}{\sigma_2} D (\theta, \overline{\rho}, {\hat \rho}) \left\{ \left[ (1 - \theta) \overline{\rho} + \theta {\hat \rho} \right] \sigma_2 Z_2^0 - \sigma_1 Z_1^0 \right\}. &
%\end{eqnarray*}
%If $ {\hat \rho} < 0 $, then $ (1 - \theta) \overline{\rho} + \theta {\hat \rho} < h (\theta, \overline{\rho}, {\hat \rho}) < E_{2 1} < H (\theta, \overline{\rho}, {\hat \rho}) $, and hence $ Z_{N 1}^* < Z_{S 1}^* $ and $ Z_{N 2}^* < Z_{S 2}^* $.
%If $ 0 < {\hat \rho} $, then $ h (\theta, \overline{\rho}, {\hat \rho}) < (1 - \theta) \overline{\rho} + \theta {\hat \rho} < H (\theta, \overline{\rho}, {\hat \rho}) $. Therefore we have
%\begin{itemize}
%\item If $ h (\theta, \overline{\rho}, {\hat \rho}) < E_{2 1} < (1 - \theta) \overline{\rho} + \theta {\hat \rho} $, then $ Z_{N 1}^* > Z_{S 1}^* $;
%\item If $ (1 - \theta) \overline{\rho} + \theta {\hat \rho} < E_{2 1} < H (\theta, \overline{\rho}, {\hat \rho}) $, then $ Z_{N 1}^* < Z_{S 1}^* $;
%\item If $ h (\theta, \overline{\rho}, {\hat \rho}) < E_{1 2} < (1 - \theta) \overline{\rho} + \theta {\hat \rho} $, then $ Z_{N 2}^* > Z_{S 2}^* $;
%\item If $ (1 - \theta) \overline{\rho} + \theta {\hat \rho} < E_{1 2} < H (\theta, \overline{\rho}, {\hat \rho}) $, then $ Z_{N 2}^* < Z_{S 2}^* $.
%\end{itemize}

%\underline{Scenario 3}: On non-participating equilibrium in asset 2, $ H (\theta, \overline{\rho}, {\hat \rho}) \leqslant E_{1 2} $, and hence $ Z_{N 2}^* = 0 < \dfrac{Z_2^0}{1 - \theta} = Z_{S 2}^* $ and $ 0 < Z_{S 1}^* = Z_1^0 - \dfrac{\theta}{1 - \theta} {\hat \rho} \dfrac{\sigma_2}{\sigma_1} Z_2^0 < Z_1^0 + {\hat \rho} \dfrac{\sigma_2}{\sigma_1} Z_2^0 = Z_{N 1}^* $ if $\hat{\rho}>0$.

\subsection{Equilibrium regions}

\quad \ 
As stated in Theorem 2, if the quality ratio is small enough, $ E_{1 2} \leqslant h (\theta, \overline{\rho}, {\hat \rho}) $, or large enough, $ E_{1 2} \geqslant H (\theta, \overline{\rho}, {\hat \rho}) $, then there exists a unique non-participating equilibrium in the markets, and na\"ive investors will not trade either asset. Otherwise, if the ratio is between the two thresholds, $ h (\theta, \overline{\rho}, {\hat \rho}) < E_{1 2} < H (\theta, \overline{\rho}, {\hat \rho}) $, then there exists a unique participating equilibrium in the markets, and na\"ive investors will hold long positions in both assets. Figure 1 reports the three equilibrium regions in plane $ \overline{\rho} - O - E_{1 2} $ for the three cases in Theorem 2. For any given maximum correlation coefficient $ \overline{\rho} $ and the quality ratio $ E_{1 2} $, the point $ \left( \overline{\rho}, E_{1 2} \right) $ lies in one of three regions shown in the figure, thus implying three different types of equilibria. If the point $ \left( \overline{\rho}, E_{1 2} \right) $ is in the curved triangle $ (\hat{\rho}, 1) \times \left( 0, h (\theta, \overline{\rho}, {\hat \rho}) \right] $, then there exists a non-participating equilibrium in which na\"ive traders do not trade asset 1. If the point $ \left( \overline{\rho}, E_{1 2} \right) $ is in the curved triangle $ (\hat{\rho}, 1) \times \left[ H (\theta, \overline{\rho}, {\hat \rho}), \infty \right) $, then there exists a non-participating equilibrium in which na\"ive traders do not trade asset 2. If the point $ \left( \overline{\rho}, E_{1 2} \right) $ is in the curved trapezoid $ (\hat{\rho}, 1) \times \left( h (\theta, \overline{\rho}, {\hat \rho}), H (\theta, \overline{\rho}, {\hat \rho}) \right) $, then there exists a participating equilibrium in which na\"ive traders trade both assets.

As we know, $ H (\theta, \overline{\rho}, {\hat \rho}) = \dfrac{1 - {\hat \rho}^2}{(1 - \theta) (\overline{\rho} - {\hat \rho})} - {\hat \rho} $ is increasing in $ \theta $ and decreasing in the level of ambiguity $ \Delta \rho = \overline{\rho} - \hat{\rho} $ that na\"ive investors face. Thus, as the fraction of na\"ive traders increases or the level of ambiguity decreases, the area of curved trapezoid $ (\hat{\rho}, 1) \times \left( h (\theta, \overline{\rho}, {\hat \rho}), \right. $ $ \left. H (\theta, \overline{\rho}, {\hat \rho}) \right) $ is increasing and the areas of two curved triangles $ (\hat{\rho}, 1) \times \left( 0, h (\theta, \overline{\rho}, {\hat \rho}) \right] $ and $ (\hat{\rho}, 1) \times \left[ H (\theta, \overline{\rho}, {\hat \rho}), \infty \right) $ are decreasing, and then the possibility of participating equilibrium increases and the possibility of non-participating equilibrium decreases. 

%\vskip 16 pt
\newpage

\centerline{\bf Figure 1: Equilibrium regions}

\begin{center}
\psset{xunit=10,yunit=0.25}
\begin{pspicture}(-0.55,0)(1.05,20)
\psline(-0.55,0)(1.05,0)
\psline(0,0)(0,20)
\psline(1,0)(1,20)
%\rput[l](-0.025,-1){$O$}
\rput[l](1,-1){$1$}
\rput[l](1.06,-1){$ \overline{\rho} $}
\rput[l](-0.17,19){$ E_{1 2} $}
\psline[linewidth=1.6pt,linecolor=green](-0.1,0)(-0.1,20)
\rput[l](-0.015,-1){$ \hat{\rho} $}
\psline[linewidth=1.6pt,linecolor=red]
%(0.01,200.0000) 
%(0.02,100.0000) 
%(0.03,66.6667) 
%(0.04,50.0000) 
%(0.05,40.0000) 
%(0.06,33.3333) 
%(0.07,28.5714) 
%(0.08,25.0000) 
%(0.09,22.2222) 
(0.10,20.0000) 
(0.11,18.1818) 
(0.12,16.6667) 
(0.13,15.3846) 
(0.14,14.2857) 
(0.15,13.3333) 
(0.16,12.5000) 
(0.17,11.7647) 
(0.18,11.1111) 
(0.19,10.5263) 
(0.20,10.0000) 
(0.21,9.5238) 
(0.22,9.0909) 
(0.23,8.6957) 
(0.24,8.3333) 
(0.25,8.0000) 
(0.26,7.6923) 
(0.27,7.4074) 
(0.28,7.1429) 
(0.29,6.8966) 
(0.30,6.6667) 
(0.31,6.4516) 
(0.32,6.2500) 
(0.33,6.0606) 
(0.34,5.8824) 
(0.35,5.7143) 
(0.36,5.5556) 
(0.37,5.4054) 
(0.38,5.2632) 
(0.39,5.1282) 
(0.40,5.0000) 
(0.41,4.8780) 
(0.42,4.7619) 
(0.43,4.6512) 
(0.44,4.5455) 
(0.45,4.4444) 
(0.46,4.3478) 
(0.47,4.2553) 
(0.48,4.1667) 
(0.49,4.0816) 
(0.50,4.0000) 
(0.51,3.9216) 
(0.52,3.8462) 
(0.53,3.7736) 
(0.54,3.7037) 
(0.55,3.6364) 
(0.56,3.5714) 
(0.57,3.5088) 
(0.58,3.4483) 
(0.59,3.3898) 
(0.60,3.3333) 
(0.61,3.2787) 
(0.62,3.2258) 
(0.63,3.1746) 
(0.64,3.1250) 
(0.65,3.0769) 
(0.66,3.0303) 
(0.67,2.9851) 
(0.68,2.9412) 
(0.69,2.8986) 
(0.70,2.8571) 
(0.71,2.8169) 
(0.72,2.7778) 
(0.73,2.7397) 
(0.74,2.7027) 
(0.75,2.6667) 
(0.76,2.6316) 
(0.77,2.5974) 
(0.78,2.5641) 
(0.79,2.5316) 
(0.80,2.5000) 
(0.81,2.4691) 
(0.82,2.4390) 
(0.83,2.4096) 
(0.84,2.3810) 
(0.85,2.3529) 
(0.86,2.3256) 
(0.87,2.2989) 
(0.88,2.2727) 
(0.89,2.2472) 
(0.90,2.2222) 
(0.91,2.1978) 
(0.92,2.1739) 
(0.93,2.1505) 
(0.94,2.1277) 
(0.95,2.1053) 
(0.96,2.0833) 
(0.97,2.0619) 
(0.98,2.0408) 
(0.99,2.0202) 
(1.00,2.0000)
\psline[linewidth=1.6pt,linecolor=blue]
(0.01,0.0050) 
(0.02,0.0100) 
(0.03,0.0150) 
(0.04,0.0200) 
(0.05,0.0250) 
(0.06,0.0300) 
(0.07,0.0350) 
(0.08,0.0400) 
(0.09,0.0450) 
(0.10,0.0500) 
(0.11,0.0550) 
(0.12,0.0600) 
(0.13,0.0650) 
(0.14,0.0700) 
(0.15,0.0750) 
(0.16,0.0800) 
(0.17,0.0850) 
(0.18,0.0900) 
(0.19,0.0950) 
(0.20,0.1000) 
(0.21,0.1050) 
(0.22,0.1100) 
(0.23,0.1150) 
(0.24,0.1200) 
(0.25,0.1250) 
(0.26,0.1300) 
(0.27,0.1350) 
(0.28,0.1400) 
(0.29,0.1450) 
(0.30,0.1500) 
(0.31,0.1550) 
(0.32,0.1600) 
(0.33,0.1650) 
(0.34,0.1700) 
(0.35,0.1750) 
(0.36,0.1800) 
(0.37,0.1850) 
(0.38,0.1900) 
(0.39,0.1950) 
(0.40,0.2000) 
(0.41,0.2050) 
(0.42,0.2100) 
(0.43,0.2150) 
(0.44,0.2200) 
(0.45,0.2250) 
(0.46,0.2300) 
(0.47,0.2350) 
(0.48,0.2400) 
(0.49,0.2450) 
(0.50,0.2500) 
(0.51,0.2550) 
(0.52,0.2600) 
(0.53,0.2650) 
(0.54,0.2700) 
(0.55,0.2750) 
(0.56,0.2800) 
(0.57,0.2850) 
(0.58,0.2900) 
(0.59,0.2950) 
(0.60,0.3000) 
(0.61,0.3050) 
(0.62,0.3100) 
(0.63,0.3150) 
(0.64,0.3200) 
(0.65,0.3250) 
(0.66,0.3300) 
(0.67,0.3350) 
(0.68,0.3400) 
(0.69,0.3450) 
(0.70,0.3500) 
(0.71,0.3550) 
(0.72,0.3600) 
(0.73,0.3650) 
(0.74,0.3700) 
(0.75,0.3750) 
(0.76,0.3800) 
(0.77,0.3850) 
(0.78,0.3900) 
(0.79,0.3950) 
(0.80,0.4000) 
(0.81,0.4050) 
(0.82,0.4100) 
(0.83,0.4150) 
(0.84,0.4200) 
(0.85,0.4250) 
(0.86,0.4300) 
(0.87,0.4350) 
(0.88,0.4400) 
(0.89,0.4450) 
(0.90,0.4500) 
(0.91,0.4550) 
(0.92,0.4600) 
(0.93,0.4650) 
(0.94,0.4700) 
(0.95,0.4750) 
(0.96,0.4800) 
(0.97,0.4850) 
(0.98,0.4900) 
(0.99,0.4950) 
(1.00,0.5000)
\psline[linewidth=1.6pt,linecolor=green](1,2.0000)(1,0.5000)
\psline[linewidth=1.6pt,linecolor=red](-0.50,10)(-0.45,10)
\rput[l](-0.44,10){$ H (\theta, \overline{\rho}, {\hat \rho}) $}
\psline[linewidth=1.6pt,linecolor=blue](-0.50,5)(-0.45,5)
\rput[l](-0.44,5){$ h (\theta, \overline{\rho}, {\hat \rho}) $}
\rput[l](0.30,10){Non-participating in Asset 2}
\rput[l](0.05,2){Participating in Both Assets}
\rput[l](0.35,-2){Non-participating in Asset 1}
\put(7.75,-0.4){\vector(0,1){0.45}}
\end{pspicture}
\end{center}

%\newpage

\section{Implications for asset prices and market non-participation}
%\renewcommand{\theequation}{4.\arabic{equation}}
%\setcounter{equation}0

\quad \ 
In this section, we take a deeper look into the equilibrium and its implications, and analyze how changes in the economy affect equilibrium types and equilibrium prices. We are particularly interested in the CAPM analysis, which presents implications for asset prices, and in changes in the fraction of na\"ive investors and in the maximum correlation coefficient, which demonstrate the interactions of non-participations and flight to quality phenomenon. 

\subsection{CAPM analysis}

\quad \ 
To shed light on the pricing effects of correlation ambiguity and non-participation, we now turn to the return of risky assets and see whether they have excess excess return (alpha) that deviates from the CAPM. For asset $i$, the return is defined as:
\begin{equation*}
\tilde{Y}_i = \dfrac{\tilde{X}_i}{p_i} - 1, \qquad i = 1, 2,
\end{equation*}
and the return on the market (holding the entire supply of both risky assets) is:
\begin{equation*}
\tilde{Y}_M = \dfrac{\tilde{X}_1 Z_1^0 + \tilde{X}_2 Z_2^0}{p_1 Z_1^0 + p_2 Z_2^0} - 1. 
\end{equation*}

Suppose that now the equilibrium prevailing in the economy is type 1, that is, na\"ive investors do not participate in the trading of asset 1. Then from the Theorem 2, we see that equilibrium prices are:
\begin{eqnarray*}
p_1 = \mu_1 - \alpha \sigma_1 \left[ \dfrac{1 - \theta {\hat \rho}^2}{1 - \theta} \sigma_1 Z_1^0 + {\hat \rho} \sigma_2 Z_2^0 \right] \quad \text{and} \quad p_2 = \mu_2 - \alpha \sigma_2 \left[ {\hat \rho} \sigma_1 Z_1^0 + \sigma_2 Z_2^0 \right].
\end{eqnarray*}

Now imagine that there is a representative agent (A) in this economy with the common CARA utility function. In order for the equilibrium prices to be the same, the representative agent must hold the following beliefs: 
\begin{equation*}
\mu_i^A = \mu_i, \quad \sigma_1^A = \sigma_1 \sqrt{\dfrac{1 - \theta \hat{\rho}^2}{1 - \theta}}, \quad \sigma_2^A = \sigma_2 \quad \text{and} \quad \rho^A = \hat{\rho} \sqrt{\dfrac{1 - \theta}{1 - \theta \hat{\rho}^2}}. 
\end{equation*}

The representative agent has correct beliefs about mean payoffs, so she will also have correct beliefs mean returns: $ \tilde{Y}_i $ for asset $i$ and $ \tilde{Y}_M $ for the market portfolio. Further, as assets are priced correctly from the point of view of the representative agent, the CAPM must hold from her perspective. That is,\footnote{\baselineskip1.3em Since we normalize the payoff of the risk-free asset to 0, the excess return of $ \tilde{Y}_i $ and $ \tilde{Y}_M $ is $ \overline{Y}_i $ and $ \overline{Y}_M $, respectively.}
\begin{equation}
\overline{Y}_i = \beta_i^A \overline{Y}_M, \qquad i = 1, 2,
\end{equation}
in which $ \beta_i^A $ is the beta for asset $i$ according to the representative agent and is given by:
\begin{eqnarray*}
& \beta_1^A = \dfrac{Cov^A (\tilde{Y}_M, \tilde{Y}_1)}{Var^A (\tilde{Y}_M)} = \dfrac{p_1 Z_1^0 + p_2 Z_2^0}{p_1} \dfrac{\left[ \sigma_1^A \right]^2 Z_1^0 + \rho^A \sigma_1^A \sigma_2^A Z_2^0}{\left[ \sigma_1^A \right]^2 \left[ Z_1^0 \right]^2 + 2 \rho^A \sigma_1^A \sigma_2^A Z_1^0 Z_2^0 + \left[ \sigma_2^A \right]^2 \left[ Z_2^0 \right]^2}, & \\
& \beta_2^A = \dfrac{Cov^A (\tilde{Y}_M, \tilde{Y}_2)}{Var^A (\tilde{Y}_M)} = \dfrac{p_1 Z_1^0 + p_2 Z_2^0}{p_2} \dfrac{\rho^A \sigma_1^A \sigma_2^A Z_1^0 + \left[ \sigma_2^A \right]^2 Z_2^0}{\left[ \sigma_1^A \right]^2 \left[ Z_1^0 \right]^2 + 2 \rho^A \sigma_1^A \sigma_2^A Z_1^0 Z_2^0 + \left[ \sigma_2^A \right]^2 \left[ Z_2^0 \right]^2}. &
\end{eqnarray*}

Note that $ \rho^A \sigma_1^A = \hat{\rho} \sigma_1 $, so the above expressions can be written in terms of the actual parameters as:
\begin{eqnarray*}
& \beta_1^A = \dfrac{Cov^A (\tilde{Y}_M, \tilde{Y}_1)}{Var^A (\tilde{Y}_M)} = \dfrac{p_1 Z_1^0 + p_2 Z_2^0}{p_1} \dfrac{\dfrac{1 - \theta \hat{\rho}^2}{1 - \theta} \sigma_1^2 Z_1^0 + \hat{\rho} \sigma_1 \sigma_2 Z_2^0}{\dfrac{1 - \theta \hat{\rho}^2}{1 - \theta} \sigma_1^2 \left[ Z_1^0 \right]^2 + 2 \hat{\rho} \sigma_1 \sigma_2 Z_1^0 Z_2^0 + \sigma_2^2 \left[ Z_2^0 \right]^2}, & \\
& \beta_2^A = \dfrac{Cov^A (\tilde{Y}_M, \tilde{Y}_2)}{Var^A (\tilde{Y}_M)} = \dfrac{p_1 Z_1^0 + p_2 Z_2^0}{p_2}
\dfrac{\hat{\rho} \sigma_1 \sigma_2 Z_1^0 + \sigma_2^2 Z_2^0}{\dfrac{1 - \theta \hat{\rho}^2}{1 - \theta} \sigma_1^2 \left[ Z_1^0 \right]^2 + 2 \hat{\rho} \sigma_1 \sigma_2 Z_1^0 Z_2^0 + \sigma_2^2 \left[ Z_2^0 \right]^2}. &
\end{eqnarray*}
In this expression, the covariance of the return on the market and the return on asset $i$, and the variance of the return on the market are calculated using the artificial beliefs of the representative agent, rather than correct beliefs.

The representative agent's beliefs about the variance of returns to asset 1 and the correlation coefficient are incorrect as $\sigma_1^A = \sigma_1 \sqrt{\dfrac{1 - \theta \hat{\rho}^2}{1 - \theta}} $ and $ \rho^A = \hat{\rho} \sqrt{\dfrac{1 - \theta}{1 - \theta \hat{\rho}^2}} $, and so her beliefs about the variance of the market and the covariance of the market and returns to each asset are incorrect. Thus, the betas calculated from the representative agent's point of view are not the betas that would be computed from actual payoff data. Now consider an outside econometrician who has the rational belief on the whole economy, that is, she knows the true value of the correlation coefficient $ \hat{\rho} $. Thus from her point of view, 
\begin{eqnarray*}
& \beta_1 = \dfrac{Cov (\tilde{Y}_M, \tilde{Y}_1)}{Var (\tilde{Y}_M)} = \dfrac{p_1 Z_1^0 + p_2 Z_2^0}{p_1} \dfrac{\sigma_1^2 Z_1^0 + \hat{\rho} \sigma_1 \sigma_2 Z_2^0}{\sigma_1^2 \left[ Z_1^0 \right]^2 + 2 \hat{\rho} \sigma_1 \sigma_2 Z_1^0 Z_2^0 + \sigma_2^2 \left[ Z_2^0 \right]^2}, & \\
& \beta_2 = \dfrac{Cov (\tilde{Y}_M, \tilde{Y}_2)}{Var (\tilde{Y}_M)} = \dfrac{p_1 Z_1^0 + p_2 Z_2^0}{p_2} \dfrac{\hat{\rho} \sigma_1 \sigma_2 Z_1^0 + \sigma_2^2 Z_2^0}{\sigma_1^2 \left[ Z_1^0 \right]^2 + 2 \hat{\rho} \sigma_1 \sigma_2 Z_1^0 Z_2^0 + \sigma_2^2 \left[ Z_2^0 \right]^2}, &
\end{eqnarray*}
where the variances and covariances are computed using the true distribution of equilibrium returns.

Both of the actual betas differ from the representative agent's Betas and thus both assets are mispriced if we consider the CAPM model in the actual economy. This mispricing can be captured in $ \alpha_i $, the market-adjusted returns:
\begin{equation*}
\alpha_i = \overline{Y}_i - \beta_i \overline{Y}_M = (\beta_i^A - \beta_i) \overline{Y}_M, \qquad i = 1, 2, 
\end{equation*}
in which the second equation is due to (18).

To determine the sign of $ \alpha_i $, we have to compare $ \beta_i^A $ and $ \beta_i $. Observing the expressions for the betas, they are of the same form as the function $ f (x) = \dfrac{a x + b}{c x + d} $, whose derivative is:
\begin{equation*}
f^{\prime} (x) = \dfrac{a d - b c}{(c x + d)^2}.
\end{equation*}
$ f^{\prime} (x) > 0 $ if and only if $ a d > b c $. Now if we let $ a = \sigma_1^2 Z_1^0 $, $ b = \hat{\rho} \sigma_1 \sigma_2 Z_2^0 $, $ c = \sigma_1^2 \left[ Z_1^0 \right]^2 $ and $ d = 2 \hat{\rho} \sigma_1 \sigma_2 Z_1^0 Z_2^0 + \sigma_2^2 \left[ Z_2^0 \right]^2 $, then $ a d - b c = \hat{\rho} \sigma_1^3 \sigma_2 \left[ Z_1^0 \right]^2 Z_2^0 + \sigma_1^2 \sigma_2^2 Z_1^0 \left[ Z_2^0 \right]^2 = (\hat{\rho} \sigma_1 Z_1^0 + \sigma_2 Z_2^0) \sigma_1^2 \sigma_2 Z_1^0 Z_2^0 > 0 $. Note that $ \dfrac{1 - \theta\hat{\rho}^2}{1 - \theta} > 1 $, so 
\begin{equation*}
\alpha_1 = (\beta_1^A - \beta_1) \overline{Y}_M = \dfrac{p_1 Z_1^0 + p_2 Z_2^0}{p_1} \left[ f \left( \dfrac{1 - \theta\hat{\rho}^2}{1 - \theta} \right) - f (1) \right] \overline{Y}_M > 0,
\end{equation*} 
since according to the first scenario of Theorem 2,
\begin{equation*}
\overline{Y}_M = \dfrac{(\mu_1 - p_1) Z_1^0 + (\mu_2 - p_2) Z_2^0}{p_1 Z_1^0 + p_2 Z_2^0} = \alpha\dfrac{\dfrac{1 - \theta\hat{\rho}^2}{1 - \theta} \sigma_1^2 \left[ Z_1^0 \right]^2 + \hat{\rho} \sigma_1 \sigma_2 Z_1^0 Z_2^0 + \sigma_2^2 \left[ Z_2^0 \right]^2}{p_1 Z_1^0 + p_2 Z_2^0} > 0.
\end{equation*}

Using the same approach, now let $ a = 0 $, $ b = \hat{\rho} \sigma_1 \sigma_2 Z_1^0 + \sigma_2^2 Z_2^0 $, $ c = \sigma_1^2 \left[ Z_1^0 \right]^2 $, and $ d = 2 \hat{\rho} \sigma_1 \sigma_2 Z_1^0 Z_2^0 + \sigma_2^2 \left[ Z_2^0 \right]^2 $, then $ a d - b c = - (\hat{\rho} \sigma_1 Z_1^0 + \sigma_2 Z_2^0) \sigma_1^2 \sigma_2 Z_1^0 Z_2^0 < 0 $. Then
\begin{equation*}
\alpha_2 = (\beta_2^A - \beta_2) \overline{Y}_M = \dfrac{p_1 Z_1^0 + p_2 Z_2^0}{p_1} \left[ f \left( \dfrac{1 - \theta \hat{\rho}^2}{1 - \theta} \right) - f (1) \right] \overline{Y}_M < 0. 
\end{equation*}

By the symmetry of equilibrium prices, we conclude that when na\"ive investors do not participate in asset 2, $\alpha_2 > 0 $ and $ \alpha_1 < 0 $. In general, there is a positive excess return to hold the asset that is not held by na\"ive investors and a negative excess expected return to holding the other risky asset. This occurs because in order to attract the sophisticated investors to hold the supply of the asset with ambiguous returns (according to the na\"ive investors), its price must be low and, thus, its returns must be high. Conversely, the ambiguity-averse investors overweight their portfolios in the non-ambiguous asset, thus increasing its price and lowering its returns.

Next, we adopt the same approach to show the mispricing effect of correlation ambiguity when the equilibrium is a participating one. When na\"ive investors trade both risky assets, the equilibrium prices are:
\begin{eqnarray*}
& p_1 = \mu_1 - \alpha \sigma_1 \left[ Q (\theta, \overline{\rho}, {\hat \rho}) \sigma_1 Z_1^0 + q (\theta, \overline{\rho}, {\hat \rho}) \sigma_2 Z_2^0 \right] & \\
& p_2 = \mu_2 - \alpha \sigma_2 \left[ q (\theta, \overline{\rho}, {\hat \rho}) \sigma_1 Z_1^0 + Q (\theta, \overline{\rho}, {\hat \rho}) \sigma_2 Z_2^0 \right]. &
\end{eqnarray*}

For the economy of representative agent (A) to have the same equilibrium prices, the agent must hold the belief that $ \mu_i^A = \mu_i $, $ \left[ \sigma_i^A \right]^2 = \sigma_i^2 Q (\theta, \overline{\rho}, {\hat \rho}) $ and $ \rho^A = \dfrac{q (\theta, \overline{\rho}, {\hat \rho})}{Q (\theta, \overline{\rho}, {\hat \rho})} $.
%\begin{eqnarray*}
%& \left[ \sigma_i^A \right]^2 = \sigma_i^2 Q (\theta, \overline{\rho}, {\hat \rho}) & \\
%& \rho^A = \dfrac{q (\theta, \overline{\rho}, {\hat \rho})}{Q (\theta, \overline{\rho}, {\hat \rho})}. &
%\end{eqnarray*}

According to the representative agent, the beta for the two risky assets is written in terms of the actual parameters as:
\begin{equation*}
\beta_1^A = \dfrac{p_1 Z_1^0 + p_2 Z_2^0}{p_1} \dfrac{\sigma_1^2 Z_1^0 + \rho^A \sigma_1 \sigma_2 Z_2^0}{\sigma_1^2 \left[ Z_1^0 \right]^2 + 2 \rho^A \sigma_1 \sigma_2 Z_1^0 Z_2^0 + \sigma_2^2 \left[ Z_2^0 \right]^2}, 
\end{equation*}
\begin{equation*}
\beta_2^A = \dfrac{p_1 Z_1^0 + p_2 Z_2^0}{p_2} \dfrac{\rho^A \sigma_1 \sigma_2 Z_1^0 + \sigma_2^2 Z_2^0}{\sigma_1^2 \left[ Z_1^0 \right]^2 + 2 \rho^A \sigma_1 \sigma_2 Z_1^0 Z_2^0 + \sigma_2^2 \left[ Z_2^0 \right]^2}. 
\end{equation*}

Thus, $ \alpha_i $ can be calculated as:
\begin{equation*}
\alpha_i = \overline{Y}_i - \beta_i \overline{Y}_{M} = (\beta_i^A - \beta_i) \overline{Y}_M, \qquad i = 1, 2.
\end{equation*}

Now let the function $ f (x) = \dfrac{a x + b}{c x + d} $ have the parameters given by $ a = \sigma_1 \sigma_2 Z_2^0 $, $ b = \sigma_1^2 Z_1^0 $, $ c = 2 \sigma_1 \sigma_2 Z_1^0 Z_2^0 $, and $ d = \sigma_1^2 \left[ Z_1^0 \right]^2 + \sigma_2^2 \left[ Z_2^0 \right]^2 $. Since $ a d - b c = \left( \sigma_2^2 \left[ Z_2^0 \right]^2 - \sigma_1^2 \left[ Z_1^0 \right]^2 \right) \sigma_1 \sigma_2 Z_2^0 $, $ f (\cdot) $ is an increasing function if and only if $ \sigma_1 Z_1^0 < \sigma_2 Z_2^0 $. Since 
\begin{equation*}
\overline{Y}_M = Q (\theta, \overline{\rho}, {\hat \rho}) \dfrac{\alpha \left[ \left[ Z_1^0 \right]^2 + \rho^A \sigma_1 \sigma_2 Z_1^0 Z_2^0 + \sigma_2^2 \left[ Z_2^0 \right]^2 \right]}{p_1 Z_1^0 + p_2 Z_2^0} > 0, 
\end{equation*}
and $ \rho^A > \hat{\rho} $, we have that 
\begin{equation*}
\alpha_1 = (\beta_1^A - \beta_1) \overline{Y}_M = \dfrac{p_1 Z_1^0 + p_2 Z_2^0}{p_1} \left[ f (\rho^A) - f (\hat{\rho}) \right]\overline{Y}_M > 0,
\end{equation*}
if and only if $ \sigma_1 Z_1^0 < \sigma_2 Z_2^0 $; by symmetry we know that at the same time $ \alpha_2 < 0 $. Similarly, if $ \sigma_1 Z_1^0 > \sigma_2 Z_2^0 $ then $ \alpha_1 < 0 $ and $ \alpha_2 > 0 $. 

Note that under a non-participating equilibrium, na\"ive investors will decide to not hold the lower quality asset. Thus to summarize, under three types of equilibria, the asset with lower quality will generate positive excess returns, while the asset with higher quality will receive negative excess returns. That is, no matter whether the economy is under a participating equilibrium or a non-participating one, na\"ive investors will favor the asset with higher quality even to an irrational degree, making its price increase and the return lower than what standard models forecast. From this we can see that correlation ambiguity can be considered as a novel approach to complement the study of cross-sectional performance of individual stocks. 

The above analysis is summarized in Proposition 4. 

\begin{prop}
No matter whether or not the equilibrium is a participating one, the risky asset with lower quality will generate positive excess returns, while the asset with higher quality will generate negative excess returns.
\end{prop}

\subsection{Fraction of na\"ive investors}

\quad \ 
Now we show how the fraction of na\"ive investors influences the equilibrium outcome by investigating whether equilibrium types alter and how equilibrium prices change as $ \theta $ varies. By symmetry, we just show the analysis for the type [1] equilibrium, and a similar result for the type [3] equilibrium will be achieved immediately.

First, we assume that the economy is at type [1] equilibrium when $ \theta $ is small enough. Then the equilibrium prices of the risky assets are:
\begin{eqnarray*}
& p_1 = \mu_1 - \alpha \sigma_1 \left\{ \dfrac{1 - {\hat \rho}^2}{1 - \theta} \sigma_1 Z_1^0 + {\hat \rho} \left[ {\hat \rho} \sigma_1 Z_1^0 + \sigma_2 Z_2^0 \right] \right\} & \\
& p_2 = \mu_2 - \alpha \sigma_2 \left[ {\hat \rho} \sigma_1 Z_1^0 + \sigma_2 Z_2^0 \right] &
\end{eqnarray*}
and it is required that the quality of asset 1 is small enough compared to asset 2, $ E_{1 2} \leqslant h (\theta, \overline{\rho}, {\hat \rho}) $. $ p_2 $ is not affected by $ \theta $ while $ p_1 $ is a decreasing function of $ \theta $, and declines to negative infinite if $ \theta $ moves toward 1. Thus, if $ \theta $ increases to a certain value, the equilibrium type will shift to a participating one. This calculation shows that the critical value of the shift is $ \theta_1 = 1 - \dfrac{1 - {\hat \rho}^2}{\overline{\rho} - {\hat \rho}} \dfrac{1}{{\hat \rho} + E_{2 1}} < 1 $. $ \theta_1 > 0 $ is equivalent to $ E_{1 2} < \dfrac{\overline{\rho} - {\hat \rho}}{1 - \overline{\rho} {\hat \rho}} $, which is obvious because $ h (\theta, \overline{\rho}, {\hat \rho}) < \dfrac{\overline{\rho} - {\hat \rho}}{1 - {\hat \rho} \overline{\rho}} $. Hence $ 0 < \theta_1 < 1 $. Thus, if type [1] equilibrium can occur, $ \theta_1 \in (0, 1) $, which is consistent with its economic meaning. When $ \theta $ shifts to $ \theta > \theta_1 $, we have a participating equilibrium, which requires $ h (\theta, \overline{\rho}, {\hat \rho}) < E_{1 2} < H (\theta, \overline{\rho}, {\hat \rho}) $. Since $ h (\theta, \overline{\rho}, {\hat \rho}) $ is decreasing in $ \theta $ and $ H (\theta, \overline{\rho}, {\hat \rho}) = \dfrac{1}{h (\theta, \overline{\rho}, {\hat \rho})} $ is decreasing, the equilibrium type remains unchanged when $ \theta $ continues to increase.

Second, we assume that the economy is at type [3] equilibrium when $ \theta $ is small enough. Then the equilibrium prices of the risky assets are:
\begin{eqnarray*}
& p_1 = \mu_1 - \alpha \sigma_1 \left[ \sigma_1 Z_1^0 + {\hat \rho} \sigma_2 Z_2^0 \right] & \\
& p_2 = \mu_2 - \alpha \sigma_2 \left\{ \dfrac{1 - {\hat \rho}^2}{1 - \theta} \sigma_2 Z_2^0 + {\hat \rho} \left[ \sigma_1 Z_1^0 + {\hat \rho} \sigma_2 Z_2^0 \right] \right\} &
\end{eqnarray*}
and it is required that the quality of asset 1 is big enough compared to asset 2, $ H (\theta, \overline{\rho}, {\hat \rho}) \leqslant E_{1 2} $. $ p_1 $ is not affected by $ \theta $. $ p_2 $ is a decreasing function of $ \theta $, and declines to negative infinite if $ \theta $ moves toward 1. Thus, if $ \theta $ increases to a certain value, the equilibrium type will shift to a participating one. We can calculate that the critical value of the shift is $ \theta_2 = 1 - \dfrac{1 - {\hat \rho}^2}{\overline{\rho} - {\hat \rho}} \dfrac{1}{E_{1 2} + {\hat \rho}} < 1 $. $ \theta_2 > 0 $ is equivalent to $ E_{1 2} > \dfrac{1 - \overline{\rho} {\hat \rho}}{\overline{\rho} - {\hat \rho}} $, which is obvious because $ \dfrac{1 - \overline{\rho} {\hat \rho}}{\overline{\rho} - {\hat \rho}} < H (\theta, \overline{\rho}, {\hat \rho}) $. Hence $ 0 < \theta_2 < 1 $. Thus, if type [3] equilibrium can occur, $ \theta_2 \in (0, 1) $, which is consistent with its economic meaning. When $ \theta $ shifts to $ \theta > \theta_2 $, we have a participating equilibrium, which requires $ h (\theta, \overline{\rho}, {\hat \rho}) < E_{1 2} < H (\theta, \overline{\rho}, {\hat \rho}) $. Since $ H (\theta, \overline{\rho}, {\hat \rho}) $ is increasing in $ \theta $, the equilibrium keeps the same type when $ \theta $ continues to decrease.

In the above analysis, we assume that at first the economy is in a nonparticipating equilibrium when $ \theta $ is small enough, which is impossible if $ \dfrac{\overline{\rho} - {\hat \rho}}{1 - {\hat \rho} \overline{\rho}} < E_{1 2} \leqslant 1 $, since under such a scenario $\theta_{1}$ would be negative. Thus if $ \dfrac{\overline{\rho} - {\hat \rho}}{1 - {\hat \rho} \overline{\rho}} < E_{1 2} \leqslant 1 $, the equilibrium that occurs in the economy can be only a participating one, and as a result, changes in $ \theta $ cannot alter the type of the equilibrium. Similarly, when $ 1 \leqslant E_{1 2} < \dfrac{1 - {\hat \rho} \overline{\rho}}{\overline{\rho} - {\hat \rho}} $, no matter how small $ \theta $ is, the equilibrium cannot be type [3]. We summarize our analysis in Proposition 5. 

\vskip 8 pt

\begin{prop}
When $ \dfrac{\overline{\rho} - {\hat \rho}}{1 - {\hat \rho} \overline{\rho}} < E_{1 2} < \dfrac{1 - {\hat \rho} \overline{\rho}}{\overline{\rho} - {\hat \rho}} $, the equilibrium can be only a participating one, and changes in $ \theta $ do not alter the equilibrium type. But if $ E_{1 2} $ is not in the interval, increasing $ \theta $ may shift the equilibrium type from a non-participating one to a participating one.
\end{prop}

\vskip 8 pt

Intuitively, one might think that with the fraction of na\"ive investors increasing, the equilibrium prices will increase monotonically, since na\"ive investors are usually thought to be more conservative then sophisticated investors. However, as mentioned in Section 2, we find that na\"ive investors might hold larger positions than sophisticated investors, thus increasing $ \theta $ might increase the demand for the asset and decrease the equilibrium price. As a result, the equilibrium prices might be non-monotonic when $ \theta $ varies from 0 to 1. We show the result in Proposition 6:

\vskip 8 pt

\begin{prop}
If $ E_{1 2} < 1 $, then the equilibrium price of asset 1 is decreasing in $ \theta $, and if $ E_{1 2} > \dfrac{1 - {\hat \rho} \overline{\rho}}{\overline{\rho} - {\hat \rho}} $, the equilibrium price of asset 1 is increasing then decreasing in $ \theta $. 
If $ E_{2 1} < 1 $, then the equilibrium price of asset 2 is decreasing in $ \theta $, and if $ E_{2 1} > \dfrac{1 - {\hat \rho} \overline{\rho}}{\overline{\rho} - {\hat \rho}} $, the equilibrium price of asset 2 is increasing then decreasing in $ \theta $. 
\end{prop}

\subsection{Maximum correlation coefficient}

\quad \ 
As we point out in Section 3, under equilibrium only the maximum correlation coefficient $ \overline{\rho} $ of na\"ive investors' prior set $ [\underline{\rho}, \overline{\rho}] $ matters to their decision of limited participation and asset prices, thus $ \Delta \rho = \overline{\rho} - \hat{\rho} $ can be considered as a measure for the level of ambiguity they are facing. $\Delta \rho $ is exogenously determined together by agents' experiences, knowledge or confidence, as well as the market environment. We next investigate how changes in the maximum correlation coefficient (or equivalently, level of ambiguity) affect the equilibrium type and asset prices. 

Fixing other parameters constant, increasing $ \overline{\rho} $ can alter the equilibrium types and flight to quality phenomenon can be observed. To see this mathematically, note that the participating equilibrium occurs if and only if $ h (\theta, \overline{\rho}, {\hat \rho}) < E_{12} < H (\theta, \overline{\rho}, {\hat \rho})$. When $ \overline{\rho} $ varies from $ \hat{\rho} $ to $1$, $ h (\theta, \overline{\rho}, {\hat \rho}) $ increases from $0$ to $ \dfrac{1 - \theta}{1 + \theta \hat{\rho}} < 1 $, and $ H (\theta, \overline{\rho}, {\hat \rho}) $ decreases from infinity to $ \dfrac{1 + \theta \hat{\rho}}{1 - \theta} > 1 $. Thus, if the quality ratio $ E_{12} $ satisfies $ \dfrac{1 - \theta}{1 + \theta\hat{\rho}} \leqslant E_{12} \leqslant \dfrac{1 + \theta\hat{\rho}}{1 - \theta} $, the equilibrium must be a participating one. And if $ E_{12} < \dfrac{1 - \theta}{1 + \theta \hat{\rho}} $ (or $ E_{12} > \dfrac{1 + \theta \hat{\rho}}{1 - \theta} $), increasing the maximum correlation coefficient can alter the relation between $ E_{12} $ and $ h (\theta, \overline{\rho}, {\hat \rho}) $ (or $ H (\theta, \overline{\rho}, {\hat \rho}) $), thus switching the equilibrium from a participating one to a non-participating one, and na\"ive investors will just trade the higher quality asset. 

We can also understand this intuitively. Note that to hold both assets, na\"ive investors must tolerate the correlation ambiguity. When the quality ratio is small or large enough, the two assets have a significant distinction. As the market becomes more uncertain to them and $ \overline{\rho} $ increases, to avoid ambiguity they will choose to hold only the asset with higher quality, resulting in a non-participating equilibrium. The result is summarized in Proposition 7. 

\vskip 8 pt

\begin{prop}
When $ \dfrac{1 - \theta}{1 + \theta \hat{\rho}} \leqslant E_{12} \leqslant \dfrac{1 + \theta \hat{\rho}}{1 - \theta} $, changes in $ \overline{\rho} $ do not alter the equilibrium type and the equilibrium can be only a participating one. However, if $ E_{1 2} $ is not in the interval, increasing $ \overline{\rho} $ may shift the equilibrium type from a participating one to a non-participating one, and na\"ive investors will hold the higher quality asset. 
\end{prop}

Propositions 5, 6, and 7 show that if the economy is in a non-participating equilibrium and we want to increase the market participation by increasing the fraction of na\"ive investors or decreasing the maximum correlation coefficient, the non-participation on the other asset will not occur. This is true since the higher quality asset will always be held by na\"ive investors under equilibrium. 

\vskip 8 pt

Based on our analysis, reducing the maximum correlation coefficient will increase the market participation. As shown in Caballero and Krishnamurthy (2007) and Easley and O'Hara (2009), a few mechanisms (such as lender of the last resort facility, deposit insurance) and regulations (such as unlisted securities disclosure, suitability rules) can be socially desirable, since in an economy with some na\"ive investors, these policies would reduce their maximum correlation coefficient and likely help them to make objectively better decisions. Our analysis also supports these observations. Take regulations regarding related transactions for example.\footnote{\baselineskip1.3em Both FASB and SEC set rules to regulate related party transaction disclosure. For example, FASB describes disclosure requirements for related party transaction in FAS 57.} Complex related party (RP) transactions influence the financial statements of the related parties in a subtle way (Kohlbeck and Mayhew, 2004), but disclosure requirements and other related rules will help to reduce the ambiguity of the correlation among them, thus making the correlation less ambiguous for na\"ive investors. 

As for equilibrium prices, it is obvious according to Theorem 2 that if the economy is at a non-participating equilibrium, then changes in $ \overline{\rho} $ have no effect on asset prices since na\"ive investors trade only one risky asset. As for the participating equilibrium, how changes in $ \overline{\rho} $ affect the asset prices are shown in Proposition 8.

\begin{prop}
Under the participating equilibrium, when $ E_{1 2} \geqslant \dfrac{2 (1 - \theta) (1 + \theta \hat{\rho})}{(1 - \theta)^2 + (1 + \theta\hat{\rho})^2} $, $ p_1 $ is a decreasing function of $ \overline{\rho} $. And when $ E_{1 2} < \dfrac{2 (1 - \theta) (1 + \theta \hat{\rho})}{(1 - \theta)^2 + (1 + \theta\hat{\rho})^2} $, $ p_1 $ first decreases then increases when $ \overline{\rho} $ changes from $ \hat{\rho} $ to $1$. The conclusion for asset 2's price is symmetric. 
\end{prop}

Thus although policies on reducing $ \overline{\rho} $ can effectively encourage market participation for one asset (with lower quality), it might have some negative impact on the other asset's price. Thus in this scenario, how to measure the total social effect of the related policies becomes essential.\footnote{\baselineskip1.3em Easley, O'Hara, and Yang (2014) discuss the social welfare effect of the degree of ambiguity. In their model, the welfare function is the certainty equivalent of transparent traders' ex ante equilibrium utility. They find that decreasing the degree of ambiguity will not always result in an increase in the welfare function.} Our result suggests that the evaluation of the policies should be include consideration of ambiguity, and careful calibration is needed to evaluate the impact of those policies.  

\section{Conclusions}
\renewcommand{\theequation}{5.\arabic{equation}}
\setcounter{equation}0

\quad \ 
In this paper, we examine the role of correlation ambiguity in financial markets. In particular, we extend the multi-asset model in Easley and O'Hara (2009) by assuming that a portion of the agents in the economy perceive the correlation of the risky assets as ambiguous. Tactfully defining a Sharpe ratio [analogous to Sharpe (1966)] results in technical convenience. Ambiguity of correlation generates four scenarios in the demand function, thus making the demand curves continuous but kinked. Under certain conditions, ambiguity-averse na\"ive investors rationally choose to limit participation so that they can avoid ambiguity of correlation. Na\"ive investors trade both risky assets in the same direction as the sophisticated investors, but it is not necessarily true that they will hold more conservative positions. 

The properties of demand functions lead to three different types of equilibria. Equilibrium regions are demonstrated in a $ \overline{\rho} - O - E_{12} $ plane. The product of standard deviation and per capita endowment, $ \sigma_i Z_i^0 $ acts as a measure of quality for asset $i$. Limited participation with the lower quality asset occurs endogenously, when the ratio of quality is small or large enough. CAPM analysis suggests that the lower quality asset always generates a positive excess return. Comparative static analysis of the equilibria shows that changes in the fraction of na\"ive investors or the maximum correlation coefficient can affect market participation and flight to quality can be observed if the equilibrium shifts from a participating one to a non-participating one. However, the pricing effects of changes in the two parameters are non-monotonic, suggesting profound social effects from relevant policies.

There are two directions that our model can be extended towards. First, we can further consider the implications of heterogeneity among ambiguity-averse agents in a manner similar to Cao, Wang, and Zhang (2005), so that further insight on how different levels of ambiguity will affect investment decisions can be achieved. Second, another natural extension is to consider learning of ambiguity correlation under a dynamic setting. There is a large literature on the time-varying features of correlation structure, such as the GARCH process for the covariance matrix by Bollerslev, Engle, and Wooldridge (1988), and counter-cyclical patterns documented by Aydemir (2008). Thus if we consider a model in which agents consider the correlation as a stochastic process and are ambiguous about the specification of the process, then we can study the dynamics of financial markets under correlation ambiguity. 

%\newpage

\section*{Appendix}
\renewcommand{\theequation}{A.\arabic{equation}}
\setcounter{equation}0

\subsection*{A.1 \quad Proof of Theorem 1}

\quad \
To solve for the optimal solution to the original programming (9), we need to consider the following four problems:
\begin{enumerate}
\item $ \max\limits_{z_1 z_2 < 0} \min\limits_{\rho \in [\underline{\rho}, \overline{\rho}]} f (z_1, z_2, \rho) = \max\limits_{z_1 z_2 < 0} f (z_1, z_2, \underline{\rho}) $,
\item $ \max\limits_{z_1 z_2 > 0} \min\limits_{\rho \in [\underline{\rho}, \overline{\rho}]} f (z_1, z_2, \rho) = \max\limits_{z_1 z_2 > 0} f (z_1, z_2, \overline{\rho}) $,
\item $ \max\limits_{z_1 = 0} \min\limits_{\rho \in [\underline{\rho}, \overline{\rho}]} f (z_1, z_2, \rho) = \max\limits_{z_2} f (0, z_2, \rho) $,
\item $ \max\limits_{z_2 = 0} \min\limits_{\rho \in [\underline{\rho}, \overline{\rho}]} f (z_1, z_2, \rho) = \max\limits_{z_1} f (z_1, 0, \rho) $,
\end{enumerate}
and then try to find the maximum value that corresponds to the optimal solution for the original problem (9).

We first consider problem 1: $ \max\limits_{z_1 z_2 < 0} \min\limits_{\rho \in [\underline{\rho}, \overline{\rho}]} f (z_1, z_2, \rho) = \max\limits_{z_1 z_2 < 0} f (z_1, z_2, \underline{\rho}) $, where
\begin{eqnarray*}
f (z_1, z_2, \underline{\rho}) = \sigma_1 R_1 z_1 + \sigma_2 R_2 z_2 - \frac12 \alpha [\sigma_1^2 z_1^2 + 2 \underline{\rho} \sigma_1 \sigma_2 z_1 z_2 + \sigma_2^2 z_2^2] + w.
\end{eqnarray*}
Calculation shows that the na\"ive investor's demand function for risky assets is given by:
\begin{eqnarray}
Z_N^* = \left( \begin{matrix} Z_{N 1}^* \\ Z_{N 2}^* \end{matrix} \right) = \dfrac1{\alpha (1 - \underline{\rho}^2)} \left( \begin{matrix} \dfrac{R_1 - \underline{\rho} R_2}{\sigma_1} \\ \dfrac{R_2 - \underline{\rho} R_1}{\sigma_2} \end{matrix} \right)
\end{eqnarray}
if $ Z_{N 1}^* Z_{N 2}^* < 0 $, that is, $ (R_1 - \underline{\rho} R_2) (R_2 - \underline{\rho} R_1) < 0 $, or equivalently:
\begin{eqnarray}
(N1.1) \quad \left\{ \begin{matrix} R_1 < \underline{\rho} R_2 \\ R_2 > \underline{\rho} R_1 \end{matrix} \right. \qquad \text{or} \qquad (N1.2) \quad \left\{ \begin{matrix} R_1 > \underline{\rho} R_2 \\ R_2 < \underline{\rho} R_1 \end{matrix} \right..
\end{eqnarray}
Thus
\begin{eqnarray*}
\max\limits_{z_1 z_2 < 0} \min\limits_{\rho \in [\underline{\rho}, \overline{\rho}]} f (z_1, z_2, \rho) = f (Z_{N 1}^*, Z_{N 2}^*, \underline{\rho}) = \dfrac{R_1^2 - 2 \underline{\rho} R_1 R_2 + R_2^2}{2 \alpha (1 - \underline{\rho}^2)} + w.
\end{eqnarray*}

We then consider problem 2: $ \max\limits_{z_1 z_2 > 0} \min\limits_{\rho \in [\underline{\rho}, \overline{\rho}]} f (z_1, z_2, \rho) = \max\limits_{z_1 z_2 > 0} f (z_1, z_2, \overline{\rho}) $, where
\begin{eqnarray*}
f (z_1, z_2, \overline{\rho}) = \sigma_1 R_1 z_1 + \sigma_2 R_2 z_2 - \frac12 \alpha [\sigma_1^2 z_1^2 + 2 \overline{\rho} \sigma_1 \sigma_2 z_1 z_2 + \sigma_2^2 z_2^2] + w.
\end{eqnarray*}
Calculation shows that the na\"ive investor's demand function for risky assets is given by:
\begin{eqnarray}
Z_N^* = \left( \begin{matrix} Z_{N 1}^* \\ Z_{N 2}^* \end{matrix} \right) = \dfrac1{\alpha (1 - \overline{\rho}^2)} \left( \begin{matrix} \dfrac{R_1 - \overline{\rho} R_2}{\sigma_1} \\ \dfrac{R_2 - \overline{\rho} R_1}{\sigma_2} \end{matrix} \right)
\end{eqnarray}
if $ Z_{N 1}^* Z_{N 2}^* > 0 $, that is, $ (R_1 - \overline{\rho} R_2) (R_2 - \overline{\rho} R_1) > 0 $, or equivalently:
\begin{eqnarray}
(N2.1) \quad \left\{ \begin{matrix} R_1 < \overline{\rho} R_2 \\ R_2 < \overline{\rho} R_1 \end{matrix} \right. \qquad \text{or} \qquad (N2.2) \quad \left\{ \begin{matrix} R_1 > \overline{\rho} R_2 \\ R_2 > \overline{\rho} R_1 \end{matrix} \right..
\end{eqnarray}
Thus
\begin{eqnarray*}
\max\limits_{z_1 z_2 > 0} \min\limits_{\rho \in [\underline{\rho}, \overline{\rho}]} f (z_1, z_2, \rho) = f (Z_{N 1}^*, Z_{N 2}^*, \overline{\rho}) = \dfrac{R_1^2 - 2 \overline{\rho} R_1 R_2 + R_2^2}{2 \alpha (1 - \overline{\rho}^2)} + w.
\end{eqnarray*}

We next consider problem 3: $ \max\limits_{z_1 = 0} \min\limits_{\rho \in [\underline{\rho}, \overline{\rho}]} f (z_1, z_2, \rho) = \max\limits_{z_2} f (0, z_2, \rho) $.
Quadratic surface $ f (z_1, z_2, \rho) $ intercepts on Plane $ z_1 = 0 $ at Parabola 1:
\begin{eqnarray*}
f (0, z_2, \rho) = \sigma_2 R_2 z_2 - \frac12 \alpha \sigma_2^2 z_2^2 + w \qquad \text{and} \qquad z_1 = 0.
\end{eqnarray*}
The vertex of Parabola 1 is $ \left( 0, \dfrac{R_2}{\alpha \sigma_2} \right) $, 
\begin{eqnarray}
Z_N^* = \left( \begin{matrix} Z_{N 1}^* \\ Z_{N 2}^* \end{matrix} \right) = \dfrac1{\alpha} \left( \begin{matrix} 0 \\ \dfrac{R_2}{\sigma_2} \end{matrix} \right)
\end{eqnarray}
and $ \max\limits_{z_2} f (0, z_2, \rho) = f \left( 0, \dfrac{R_2}{\alpha \sigma_2}, \rho \right) = \dfrac{R_2^2}{2 \alpha} + w $.

Finally, we consider problem 4: $ \max\limits_{z_2 = 0} \min\limits_{\rho \in [\underline{\rho}, \overline{\rho}]} f (z_1, z_2, \rho) = \max\limits_{z_1} f (z_1, 0, \rho) $.
The quadratic surface $ f (z_1, z_2, \rho) $ intercepts on Plane $ z_2 = 0 $ at Parabola 2:
\begin{eqnarray*}
f (z_1, 0, \rho) = \sigma_1 R_1 z_1 - \frac12 \alpha \sigma_1^2 z_1^2 + w \qquad \text{and} \qquad z_2 = 0.
\end{eqnarray*}
The vertex of Parabola 2 is $ \left( \dfrac{R_1}{\alpha \sigma_1}, 0 \right) $,
\begin{eqnarray}
Z_N^* = \left( \begin{matrix} Z_{N 1}^* \\ Z_{N 2}^* \end{matrix} \right) = \dfrac1{\alpha} \left( \begin{matrix} \dfrac{R_1}{\sigma_1} \\ 0 \end{matrix} \right)
\end{eqnarray}
and $ \max\limits_{z_1} f (z_1, 0, \rho) = f \left( \dfrac{R_1}{\alpha \sigma_1}, 0, \rho \right) =\dfrac{R_1^2}{2 \alpha} + w $.

It is impossible that condition (A.2) $ (R_1 - \underline{\rho} R_2) (R_2 - \underline{\rho} R_1) < 0 $ and condition (A.4) $ (R_1 - \overline{\rho} R_2) (R_2 - \overline{\rho} R_1) > 0 $ hold simultaneously. That is,
\begin{eqnarray*}
\{ (R_1, R_2) | (R_1 - \underline{\rho} R_2) (R_2 - \underline{\rho} R_1) < 0 \} \cap \{ (R_1, R_2) | (R_1 - \overline{\rho} R_2) (R_2 - \overline{\rho} R_1) > 0 \} = \emptyset
\end{eqnarray*}
in Plane $ R_1 - O - R_2 $. So we consider the domain of programmings 3 and 4 as the intersection of two complements of conditions (A.2) and (A.4):
\begin{eqnarray*}
D & = & \{ (R_1, R_2) | (R_1 - \underline{\rho} R_2) (R_2 - \underline{\rho} R_1) < 0 \}^c \cap \{ (R_1, R_2) | (R_1 - \overline{\rho} R_2) (R_2 - \overline{\rho} R_1) > 0 \}^c \\
& = & \left[ \{ (R_1, R_2) | (R_1 - \underline{\rho} R_2) (R_2 - \underline{\rho} R_1) < 0 \} \cup \{ (R_1, R_2) | (R_1 - \overline{\rho} R_2) (R_2 - \overline{\rho} R_1) > 0 \} \right]^c \\
& = & \{ (R_1, R_2) | (R_1 - \underline{\rho} R_2) (R_2 - \underline{\rho} R_1) \geqslant 0 \} \cap \{ (R_1, R_2) | (R_1 - \overline{\rho} R_2) (R_2 - \overline{\rho} R_1) \leqslant 0 \}.
\end{eqnarray*}

We analyze the intersection by two approaches. We first consider the sign of $ R_1 $ for the following three cases.

Case 1.1: $ R_1 < 0 $, then $ R_2 - \underline{\rho} R_1 < R_2 - \overline{\rho} R_1 $.

(1.1.1) If $ R_2 - \underline{\rho} R_1 < R_2 - \overline{\rho} R_1 < 0 $, then $ R_1 - \underline{\rho} R_2 \leqslant 0 \leqslant R_1 - \overline{\rho} R_2 $, thus $ R_2 < 0 $ and $ \overline{\rho} R_2 \leqslant R_1 \leqslant \underline{\rho} R_2 $. 

(1.1.2) If $ R_2 - \underline{\rho} R_1 < R_2 - \overline{\rho} R_1 = 0 $, then $ R_1 - \underline{\rho} R_2 \leqslant 0 $, thus $ R_2 = \overline{\rho} R_1 $ and $ R_1 \leqslant \underline{\rho} R_2 $.

(1.1.3) If $ R_2 - \underline{\rho} R_1 < 0 < R_2 - \overline{\rho} R_1 $, then $ R_1 - \underline{\rho} R_2 \leqslant 0 $ and $ R_1 - \overline{\rho} R_2 \leqslant 0 $, thus $ \overline{\rho} R_1 < R_2 < \underline{\rho} R_1 $.

(1.1.4) If $ 0 = R_2 - \underline{\rho} R_1 < R_2 - \overline{\rho} R_1 $, then $ R_1 - \overline{\rho} R_2 \leqslant 0 $, thus $ R_2 = \underline{\rho} R_1 $ and $ R_1 \leqslant \overline{\rho} R_2 $.

(1.1.5) If $ 0 < R_2 - \underline{\rho} R_1 < R_2 - \overline{\rho} R_1 $, then $ R_1 - \overline{\rho} R_2 \leqslant 0 \leqslant R_1 - \underline{\rho} R_2 $, thus $ R_2 > 0 $ and $ \underline{\rho} R_2 \leqslant R_1 \leqslant \overline{\rho} R_2 $. 

Case 1.2: $ R_1 = 0 $, then $ - \underline{\rho} R_2^2 \geqslant 0 $ and $ - \overline{\rho} R_2^2 \leqslant 0 $, thus $ R_2 = 0 $ or $ \underline{\rho} \leqslant 0 \leqslant \overline{\rho} $.

Case 1.3: $ R_1 > 0 $, then $ R_2 - \overline{\rho} R_1 < R_2 - \underline{\rho} R_1 $.

(1.3.1) If $ R_2 - \overline{\rho} R_1 < R_2 - \underline{\rho} R_1 < 0 $, then $ R_1 - \underline{\rho} R_2 \leqslant 0 \leqslant R_1 - \overline{\rho} R_2 $, thus $ R_2 < 0 $ and $ \overline{\rho} R_2 \leqslant R_1 \leqslant \underline{\rho} R_2 $. 
(1.3.2) If $ R_2 - \overline{\rho} R_1 < R_2 - \underline{\rho} R_1 = 0 $, then $ 0 \leqslant R_1 - \overline{\rho} R_2 $, thus $ R_2 = \underline{\rho} R_1 $ and $ \underline{\rho} R_2 \leqslant R_1 $.

(1.3.3) If $ R_2 - \overline{\rho} R_1 < 0 < R_2 - \underline{\rho} R_1 $, then $ 0 \leqslant R_1 - \underline{\rho} R_2 $ and $ 0 \leqslant R_1 - \overline{\rho} R_2 $, thus $ \underline{\rho} R_1 < R_2 < \overline{\rho} R_1 $.

(1.3.4) If $ 0 = R_2 - \overline{\rho} R_1 < R_2 - \underline{\rho} R_1 $, then $ 0 \leqslant R_1 - \underline{\rho} R_2 $, thus $ R_2 = \overline{\rho} R_1 $ and $ \underline{\rho} R_2 \leqslant R_1 $.

(1.3.5) If $ 0 < R_2 - \overline{\rho} R_1 < R_2 - \underline{\rho} R_1 $, then $ R_1 - \overline{\rho} R_2 \leqslant 0 \leqslant R_1 - \underline{\rho} R_2 $, thus $ R_2 > 0 $ and $ \underline{\rho} R_2 \leqslant R_1 \leqslant \overline{\rho} R_2 $. 

Figure A1 reports the situation for three settings. There exist two viewpoints to watch for region $D$. The first one is from the sign of $ R_1 $. The part $ D \cap \{ R_1 < 0 \} $ is split into five pieces as (1.1.1), (1.1.2), (1.1.3), (1.1.4), and (1.1.5); and the part $ D \cap \{ R_1 > 0 \} $ is split into five pieces as (1.3.1), (1.3.2), (1.3.3), (1.3.4), and (1.3.5). The ten pieces are divided into four zones:
\begin{eqnarray*}
& (1.1.2) + (1.1.3) + (1.1.4) = \left\{ \begin{matrix} R_1 < 0 \\ \overline{\rho} R_1 \leqslant R_2 \leqslant \underline{\rho} R_1 \end{matrix} \right. \equiv (N-.1), & \\
& (1.3.2) + (1.3.3) + (1.3.4) = \left\{ \begin{matrix} R_1 > 0 \\ \underline{\rho} R_1 \leqslant R_2 \leqslant \overline{\rho} R_1 \end{matrix} \right. \equiv (N+.1), & \\
& (1.1.1) + (1.3.1) = \left\{ \begin{matrix} \overline{\rho} R_2 \leqslant R_1 \leqslant \underline{\rho} R_2 \\ R_2 < 0 \end{matrix} \right. \equiv (N-.2), & \\
& (1.1.5) + (1.3.5) = \left\{ \begin{matrix} \underline{\rho} R_2 \leqslant R_1 \leqslant \overline{\rho} R_2 \\ R_2 > 0 \end{matrix} \right. \equiv (N+.2). &
\end{eqnarray*}

We then consider the sign of $ R_2 $ for the following three cases.

Case 2.1: $ R_2 < 0 $, then $ R_1 - \underline{\rho} R_2 < R_1 - \overline{\rho} R_2 $.

(2.1.1) If $ R_1 - \underline{\rho} R_2 < R_1 - \overline{\rho} R_2 < 0 $, then $ R_2 - \underline{\rho} R_1 \leqslant 0 \leqslant R_2 - \overline{\rho} R_1 $, thus $ R_1 < 0 $ and $ \overline{\rho} R_1 \leqslant R_2 \leqslant \underline{\rho} R_1 $. 

(2.1.2) If $ R_1 - \underline{\rho} R_2 < R_1 - \overline{\rho} R_2 = 0 $, then $ R_2 - \underline{\rho} R_1 \leqslant 0 $, thus $ R_1 = \overline{\rho} R_2 $ and $ R_2 \leqslant \underline{\rho} R_1 $.

(2.1.3) If $ R_1 - \underline{\rho} R_2 < 0 < R_1 - \overline{\rho} R_2 $, then $ R_2 - \underline{\rho} R_1 \leqslant 0 $ and $ R_2 - \overline{\rho} R_1 \leqslant 0 $, thus $ \overline{\rho} R_2 < R_1 < \underline{\rho} R_2 $.

(2.1.4) If $ 0 = R_1 - \underline{\rho} R_2 < R_1 - \overline{\rho} R_2 $, then $ R_2 - \overline{\rho} R_1 \leqslant 0 $, thus $ R_1 = \underline{\rho} R_2 $ and $ R_2 \leqslant \overline{\rho} R_1 $.

(2.1.5) If $ 0 < R_1 - \underline{\rho} R_2 < R_1 - \overline{\rho} R_2 $, then $ R_2 - \overline{\rho} R_1 \leqslant 0 \leqslant R_2 - \underline{\rho} R_1 $, thus $ R_1 > 0 $ and $ \underline{\rho} R_1 \leqslant R_2 \leqslant \overline{\rho} R_1 $. 

Case 2.2: $ R_2 = 0 $, then $ - \underline{\rho} R_1^2 \geqslant 0 $ and $ - \overline{\rho} R_1^2 \leqslant 0 $, thus $ R_1 = 0 $ or $ \underline{\rho} \leqslant 0 \leqslant \overline{\rho} $.

Case 2.3: $ R_2 > 0 $, then $ R_1 - \overline{\rho} R_2 < R_1 - \underline{\rho} R_2 $.

(2.3.1) If $ R_1 - \overline{\rho} R_2 < R_1 - \underline{\rho} R_2 < 0 $, then $ R_2 - \underline{\rho} R_1 \leqslant 0 \leqslant R_2 - \overline{\rho} R_1 $, thus $ R_1 < 0 $ and $ \overline{\rho} R_1 \leqslant R_2 \leqslant \underline{\rho} R_1 $. 

(2.3.2) If $ R_1 - \overline{\rho} R_2 < R_1 - \underline{\rho} R_2 = 0 $, then $ 0 \leqslant R_2 - \overline{\rho} R_1 $, thus $ R_1 = \underline{\rho} R_2 $ and $ \underline{\rho} R_1 \leqslant R_2 $.

(2.3.3) If $ R_1 - \overline{\rho} R_2 < 0 < R_1 - \underline{\rho} R_2 $, then $ 0 \leqslant R_2 - \underline{\rho} R_1 $ and $ 0 \leqslant R_2 - \overline{\rho} R_1 $, thus $ \underline{\rho} R_2 < R_1 < \overline{\rho} R_2 $.

(2.3.4) If $ 0 = R_1 - \overline{\rho} R_2 < R_1 - \underline{\rho} R_2 $, then $ 0 \leqslant R_2 - \underline{\rho} R_1 $, thus $ R_1 = \overline{\rho} R_2 $ and $ \underline{\rho} R_1 \leqslant R_2 $.

(2.3.5) If $ 0 < R_1 - \overline{\rho} R_2 < R_1 - \underline{\rho} R_2 $, then $ R_2 - \overline{\rho} R_1 \leqslant 0 \leqslant R_2 - \underline{\rho} R_1 $, thus $ R_1 > 0 $ and $ \underline{\rho} R_1 \leqslant R_2 \leqslant \overline{\rho} R_1 $. 

Figure A1 also reports the situation for three settings from another viewpoint to watch for region $D$, which is from the sign of $ R_2 $. The part $ D \cap \{ R_2 < 0 \} $ is split into five pieces as (2.1.1), (2.1.2), (2.1.3), (2.1.4), and (2.1.5); and the part $ D \cap \{ R_2 > 0 \} $ is split into five pieces as (2.3.1), (2.3.2), (2.3.3), (2.3.4), and (2.3.5). The ten pieces are divided into four zones:
\begin{eqnarray*}
& (2.1.1) + (2.3.1) = (N-.1), & \\
& (2.1.5) + (2.3.5) = (N+.1), & \\
& (2.1.2) + (2.1.3) + (2.1.4) = (N-.2), & \\
& (2.3.2) + (2.3.3) + (2.3.4) = (N+.2). &
\end{eqnarray*}

The solution to programing 3 is of the form of expression (A.5), there is not a solution to programing 3 in $ (N-.1) $ and $ (N+.1) $, and expression (A.5) is the solution to programing 3 in $ (N-.2) $ and $ (N+.2) $. Therefore,
\begin{eqnarray*}
& Z_{N1}^* = 0 \quad \text{and} \quad Z_{N2}^* = \dfrac{R_2}{\alpha \sigma_2} < 0 \quad \text{in} \quad (N-.2) & \\
& Z_{N1}^* = 0 \quad \text{and} \quad Z_{N2}^* = \dfrac{R_2}{\alpha \sigma_2} > 0 \quad \text{in} \quad (N+.2). &
\end{eqnarray*}
Similarly, the solution to programing 4 is of the form of expression (A.6), there is not a solution to Programing 3 in $ (N-.2) $, and $ (N+.2) $, and expression (A.6) is the solution to programing 4 in $ (N-.1) $ and $ (N+.1) $. Therefore,
\begin{eqnarray*}
& Z_{N1}^* = \dfrac{R_1}{\alpha \sigma_1} < 0 \quad \text{and} \quad Z_{N2}^* = 0 \quad \text{in} \quad (N-.1) & \\
& Z_{N1}^* = \dfrac{R_1}{\alpha \sigma_1} > 0 \quad \text{and} \quad Z_{N2}^* = 0 \quad \text{in} \quad (N+.1). &
\end{eqnarray*}
The solutions to programmings 3 and 4 are consistent to the solution forms of programmings 1 and 2. From (A.1) - (A.2) - (A.3) - (A.4), 
\begin{eqnarray*}
& Z_{N1}^* < 0 \quad \text{and} \quad Z_{N2}^* > 0 \quad \text{in} \quad (N1.1), & \\
& Z_{N1}^* > 0 \quad \text{and} \quad Z_{N2}^* < 0 \quad \text{in} \quad (N1.2), & \\
& Z_{N1}^* < 0 \quad \text{and} \quad Z_{N2}^* < 0 \quad \text{in} \quad (N2.1), & \\
& Z_{N1}^* > 0 \quad \text{and} \quad Z_{N2}^* > 0 \quad \text{in} \quad (N2.2) & 
\end{eqnarray*}
and then we obtain Equation (10):
{\small \begin{eqnarray}
Z_N^* = \left( \begin{matrix} Z_{N 1}^* \\ N_{N 2}^* \end{matrix} \right) = \left\{ \begin{matrix}
\dfrac1{\alpha} \left( \begin{matrix} \dfrac{R_1 - \underline{\rho} R_2}{\sigma_1 (1 - \underline{\rho}^2)} \\ \dfrac{R_2 - \underline{\rho} R_1}{\sigma_2 (1 - \underline{\rho}^2)} \end{matrix} \right), \qquad \text{if} \ (N1.1) \left\{ \begin{matrix} R_1 < \underline{\rho} R_2 \\ R_2 > \underline{\rho} R_1 \end{matrix} \right. \ \text{or} \ (N1.2) \left\{ \begin{matrix} R_1 > \underline{\rho} R_2 \\ R_2 < \underline{\rho} R_1 \end{matrix} \right. \\
\dfrac1{\alpha} \left( \begin{matrix} 0 \\ \dfrac{R_2}{\sigma_2} \end{matrix} \right), \ \text{if} \ (N-.2) \left\{ \begin{matrix} \overline{\rho} R_2 \leqslant R_1 \leqslant \underline{\rho} R_2 \\ R_2 < 0 \end{matrix} \right. \ \text{or} \ (N+.2) \left\{ \begin{matrix} \underline{\rho} R_2 \leqslant R_1 \leqslant \overline{\rho} R_2 \\ R_2 > 0 \end{matrix} \right. \\
\dfrac1{\alpha} \left( \begin{matrix} \dfrac{R_1}{\sigma_1} \\ 0 \end{matrix} \right), \ \text{if} \ (N-.1) \left\{ \begin{matrix} R_1 < 0 \\ \overline{\rho} R_1 \leqslant R_2 \leqslant \underline{\rho} R_1 \end{matrix} \right. \ \text{or} \ (N+.1) \left\{ \begin{matrix} R_1 > 0 \\ \underline{\rho} R_1 \leqslant R_2 \leqslant \overline{\rho} R_1 \end{matrix} \right. \\
\dfrac1{\alpha} \left( \begin{matrix} \dfrac{R_1 - \overline{\rho} R_2}{\sigma_1 (1 - \overline{\rho}^2)} \\ \dfrac{R_2 - \overline{\rho} R_1}{\sigma_2 (1 - \overline{\rho}^2)} \end{matrix} \right), \qquad \text{if} \ (N2.1) \left\{ \begin{matrix} R_1 < \overline{\rho} R_2 \\ R_2 < \overline{\rho} R_1 \end{matrix} \right. \ \text{or} \ (N2.2) \left\{ \begin{matrix} R_1 > \overline{\rho} R_2 \\ R_2 > \overline{\rho} R_1 \end{matrix} \right.
\end{matrix} \right.
\end{eqnarray}}

\newpage

\centerline{\bf Figure A1.1 \quad Solution zones for $ 0 < \underline{\rho} < \overline{\rho} $}

\vskip 64 pt

\begin{center}
\psset{xunit=0.5,yunit=0.5}
\begin{pspicture}(-10,-10)(10,10)
\put(-5.5,0){\vector(1,0){11}}
\put(0,-5.5){\vector(0,1){11}}
\psline(-10,-2.5)(10,2.5)
\psline(-10,-5.0)(10,5.0)
\psline(-2.5,-10)(2.5,10)
\psline(-5.0,-10)(5.0,10)
\rput(0.3,-0.4)0
\rput(10.8,-0.5){$ R_1 $}
\rput(-0.5,10.6){$ R_2 $}
\rput(-11.5,-2.0){$ R_2 = \underline{\rho} R_1 $}
\rput(-11.5,-5.5){$ R_2 = \overline{\rho} R_1 $}
\rput(11.5,2.0){$ R_2 = \underline{\rho} R_1 $}
\rput(11.5,5.5){$ R_2 = \overline{\rho} R_1 $}
\rput(-1.5,-10.5){$ R_1 = \underline{\rho} R_2 $}
\rput(-6.0,-10.5){$ R_1 = \overline{\rho} R_2 $}
\rput(1.5,10.5){$ R_1 = \underline{\rho} R_2 $}
\rput(6.0,10.5){$ R_1 = \overline{\rho} R_2 $}
\rput[l](-9.0,5.0){\footnotesize $ (N1.1) \left\{ \begin{matrix} R_1 < \underline{\rho} R_2 \\ R_2 > \underline{\rho} R_1 \end{matrix} \right. $}
\rput[l](3.0,-5.0){\footnotesize $ (N1.2) \left\{ \begin{matrix} R_1 > \underline{\rho} R_2 \\ R_2 < \underline{\rho} R_1 \end{matrix} \right. $}
\psline[linewidth=1.6pt,linecolor=red](-10.0,10)(-10.0,-2.500)
\psline[linewidth=1.6pt,linecolor=red](-9.5,10)(-9.5,-2.375)
\psline[linewidth=1.6pt,linecolor=red](-9.0,10)(-9.0,6.6)
\psline[linewidth=1.6pt,linecolor=red](-9.0,3.4)(-9.0,-2.250)
\psline[linewidth=1.6pt,linecolor=red](-8.5,10)(-8.5,6.6)
\psline[linewidth=1.6pt,linecolor=red](-8.5,3.4)(-8.5,-2.125)
\psline[linewidth=1.6pt,linecolor=red](-8.0,10)(-8.0,6.6)
\psline[linewidth=1.6pt,linecolor=red](-8.0,3.4)(-8.0,-2.000)
\psline[linewidth=1.6pt,linecolor=red](-7.5,10)(-7.5,6.6)
\psline[linewidth=1.6pt,linecolor=red](-7.5,3.4)(-7.5,-1.875)
\psline[linewidth=1.6pt,linecolor=red](-7.0,10)(-7.0,6.6)
\psline[linewidth=1.6pt,linecolor=red](-7.0,3.4)(-7.0,-1.750)
\psline[linewidth=1.6pt,linecolor=red](-6.5,10)(-6.5,6.6)
\psline[linewidth=1.6pt,linecolor=red](-6.5,3.4)(-6.5,-1.625)
\psline[linewidth=1.6pt,linecolor=red](-6.0,10)(-6.0,6.6)
\psline[linewidth=1.6pt,linecolor=red](-6.0,3.4)(-6.0,-1.500)
\psline[linewidth=1.6pt,linecolor=red](-5.5,10)(-5.5,6.6)
\psline[linewidth=1.6pt,linecolor=red](-5.5,3.4)(-5.5,-1.375)
\psline[linewidth=1.6pt,linecolor=red](-5.0,10)(-5.0,6.6)
\psline[linewidth=1.6pt,linecolor=red](-5.0,3.4)(-5.0,-1.250)
\psline[linewidth=1.6pt,linecolor=red](-4.5,10)(-4.5,6.6)
\psline[linewidth=1.6pt,linecolor=red](-4.5,3.4)(-4.5,-1.125)
\psline[linewidth=1.6pt,linecolor=red](-4.0,10)(-4.0,6.6)
\psline[linewidth=1.6pt,linecolor=red](-4.0,3.4)(-4.0,-1.000)
\psline[linewidth=1.6pt,linecolor=red](-3.5,10)(-3.5,6.6)
\psline[linewidth=1.6pt,linecolor=red](-3.5,3.4)(-3.5,-0.875)
\psline[linewidth=1.6pt,linecolor=red](-3.0,10)(-3.0,-0.750)
\psline[linewidth=1.6pt,linecolor=red](-2.5,10)(-2.5,-0.625)
\psline[linewidth=1.6pt,linecolor=red](-2.0,10)(-2.0,-0.500)
\psline[linewidth=1.6pt,linecolor=red](-1.5,10)(-1.5,-0.375)
\psline[linewidth=1.6pt,linecolor=red](-1.0,10)(-1.0,-0.250)
\psline[linewidth=1.6pt,linecolor=red](-0.5,10)(-0.5,-0.125)
\psline[linewidth=1.6pt,linecolor=red](0.0,10)(0.0,0.000)
\psline[linewidth=1.6pt,linecolor=red](0.5,10)(0.5,2.00)
\psline[linewidth=1.6pt,linecolor=red](1.0,10)(1.0,4.00)
\psline[linewidth=1.6pt,linecolor=red](1.5,10)(1.5,6.00)
\psline[linewidth=1.6pt,linecolor=red](2.0,10)(2.0,8.00)
\psline[linewidth=1.6pt,linecolor=red](10.0,-10)(10.0,2.500)
\psline[linewidth=1.6pt,linecolor=red](9.5,-10)(9.5,2.375)
\psline[linewidth=1.6pt,linecolor=red](9.0,-10)(9.0,2.250)
\psline[linewidth=1.6pt,linecolor=red](8.5,-10)(8.5,2.125)
\psline[linewidth=1.6pt,linecolor=red](8.0,-10)(8.0,-6.6)
\psline[linewidth=1.6pt,linecolor=red](8.0,-3.4)(8.0,2.000)
\psline[linewidth=1.6pt,linecolor=red](7.5,-10)(7.5,-6.6)
\psline[linewidth=1.6pt,linecolor=red](7.5,-3.4)(7.5,1.875)
\psline[linewidth=1.6pt,linecolor=red](7.0,-10)(7.0,-6.6)
\psline[linewidth=1.6pt,linecolor=red](7.0,-3.4)(7.0,1.750)
\psline[linewidth=1.6pt,linecolor=red](6.5,-10)(6.5,-6.6)
\psline[linewidth=1.6pt,linecolor=red](6.5,-3.4)(6.5,1.625)
\psline[linewidth=1.6pt,linecolor=red](6.0,-10)(6.0,-6.6)
\psline[linewidth=1.6pt,linecolor=red](6.0,-3.4)(6.0,1.500)
\psline[linewidth=1.6pt,linecolor=red](5.5,-10)(5.5,-6.6)
\psline[linewidth=1.6pt,linecolor=red](5.5,-3.4)(5.5,1.375)
\psline[linewidth=1.6pt,linecolor=red](5.0,-10)(5.0,-6.6)
\psline[linewidth=1.6pt,linecolor=red](5.0,-3.4)(5.0,1.250)
\psline[linewidth=1.6pt,linecolor=red](4.5,-10)(4.5,-6.6)
\psline[linewidth=1.6pt,linecolor=red](4.5,-3.4)(4.5,1.125)
\psline[linewidth=1.6pt,linecolor=red](4.0,-10)(4.0,-6.6)
\psline[linewidth=1.6pt,linecolor=red](4.0,-3.4)(4.0,1.000)
\psline[linewidth=1.6pt,linecolor=red](3.5,-10)(3.5,-6.6)
\psline[linewidth=1.6pt,linecolor=red](3.5,-3.4)(3.5,0.875)
\psline[linewidth=1.6pt,linecolor=red](3.0,-10)(3.0,-6.6)
\psline[linewidth=1.6pt,linecolor=red](3.0,-3.4)(3.0,0.750)
\psline[linewidth=1.6pt,linecolor=red](2.5,-10)(2.5,0.625)
\psline[linewidth=1.6pt,linecolor=red](2.0,-10)(2.0,0.500)
\psline[linewidth=1.6pt,linecolor=red](1.5,-10)(1.5,0.375)
\psline[linewidth=1.6pt,linecolor=red](1.0,-10)(1.0,0.250)
\psline[linewidth=1.6pt,linecolor=red](0.5,-10)(0.5,0.125)
\psline[linewidth=1.6pt,linecolor=red](0.0,-10)(0.0,0.00)
\psline[linewidth=1.6pt,linecolor=red](-0.5,-10)(-0.5,-2.00)
\psline[linewidth=1.6pt,linecolor=red](-1.0,-10)(-1.0,-4.00)
\psline[linewidth=1.6pt,linecolor=red](-1.5,-10)(-1.5,-6.00)
\psline[linewidth=1.6pt,linecolor=red](-2.0,-10)(-2.0,-8.00)
\rput[l](-9.0,-5.5){\footnotesize $ (N2.1) \left\{ \begin{matrix} R_1 < \overline{\rho} R_2 \\ R_2 < \overline{\rho} R_1 \end{matrix} \right. $}
\rput[l](3.0,5.5){\footnotesize $ (N2.2) \left\{ \begin{matrix} R_1 > \overline{\rho} R_2 \\ R_2 > \overline{\rho} R_1 \end{matrix} \right. $}
\psline[linewidth=1.6pt,linecolor=blue](-10,-10)(-5.0,-10)
\psline[linewidth=1.6pt,linecolor=blue](-10,-9.5)(-4.75,-9.5)
\psline[linewidth=1.6pt,linecolor=blue](-10,-9.0)(-4.50,-9.0)
\psline[linewidth=1.6pt,linecolor=blue](-10,-8.5)(-4.25,-8.5)
\psline[linewidth=1.6pt,linecolor=blue](-10,-8.0)(-4.00,-8.0)
\psline[linewidth=1.6pt,linecolor=blue](-10,-7.5)(-3.75,-7.5)
\psline[linewidth=1.6pt,linecolor=blue](-10,-7.0)(-3.50,-7.0)
\psline[linewidth=1.6pt,linecolor=blue](-10,-6.5)(-9.2,-6.5)
\psline[linewidth=1.6pt,linecolor=blue](-3.5,-6.5)(-3.25,-6.5)
\psline[linewidth=1.6pt,linecolor=blue](-10,-6.0)(-9.2,-6.0)
\psline[linewidth=1.6pt,linecolor=blue](-3.5,-6.0)(-3.00,-6.0)
\psline[linewidth=1.6pt,linecolor=blue](-10,-5.5)(-9.2,-5.5)
\psline[linewidth=1.6pt,linecolor=blue](-3.5,-5.5)(-2.75,-5.5)
\psline[linewidth=1.6pt,linecolor=blue](-10,-5.0)(-9.2,-5.0)
\psline[linewidth=1.6pt,linecolor=blue](-3.5,-5.0)(-2.50,-5.0)
\psline[linewidth=1.6pt,linecolor=blue](-3.5,-4.5)(-2.25,-4.5)
\psline[linewidth=1.6pt,linecolor=blue](-8.0,-4.0)(-2.00,-4.0)
\psline[linewidth=1.6pt,linecolor=blue](-7.0,-3.5)(-1.75,-3.5)
\psline[linewidth=1.6pt,linecolor=blue](-6.0,-3.0)(-1.50,-3.0)
\psline[linewidth=1.6pt,linecolor=blue](-5.0,-2.5)(-1.25,-2.5)
\psline[linewidth=1.6pt,linecolor=blue](-4.0,-2.0)(-1.00,-2.0)
\psline[linewidth=1.6pt,linecolor=blue](-3.0,-1.5)(-0.75,-1.5)
\psline[linewidth=1.6pt,linecolor=blue](-2.0,-1.0)(-0.50,-1.0)
\psline[linewidth=1.6pt,linecolor=blue](-1.0,-0.5)(-0.25,-0.5)
\psline[linewidth=1.6pt,linecolor=blue](10,10)(5.0,10)
\psline[linewidth=1.6pt,linecolor=blue](10,9.5)(4.75,9.5)
\psline[linewidth=1.6pt,linecolor=blue](10,9.0)(4.50,9.0)
\psline[linewidth=1.6pt,linecolor=blue](10,8.5)(4.25,8.5)
\psline[linewidth=1.6pt,linecolor=blue](10,8.0)(4.00,8.0)
\psline[linewidth=1.6pt,linecolor=blue](10,7.5)(3.75,7.5)
\psline[linewidth=1.6pt,linecolor=blue](10,7.0)(3.50,7.0)
\psline[linewidth=1.6pt,linecolor=blue](10,6.5)(8.7,6.5)
\psline[linewidth=1.6pt,linecolor=blue](10,6.0)(8.7,6.0)
\psline[linewidth=1.6pt,linecolor=blue](10,5.5)(8.7,5.5)
\psline[linewidth=1.6pt,linecolor=blue](10,5.0)(8.7,5.0)
\psline[linewidth=1.6pt,linecolor=blue](2.75,4.5)(2.25,4.5)
\psline[linewidth=1.6pt,linecolor=blue](8.0,4.0)(2.00,4.0)
\psline[linewidth=1.6pt,linecolor=blue](7.0,3.5)(1.75,3.5)
\psline[linewidth=1.6pt,linecolor=blue](6.0,3.0)(1.50,3.0)
\psline[linewidth=1.6pt,linecolor=blue](5.0,2.5)(1.25,2.5)
\psline[linewidth=1.6pt,linecolor=blue](4.0,2.0)(1.00,2.0)
\psline[linewidth=1.6pt,linecolor=blue](3.0,1.5)(0.75,1.5)
\psline[linewidth=1.6pt,linecolor=blue](2.0,1.0)(0.50,1.0)
\psline[linewidth=1.6pt,linecolor=blue](1.0,0.5)(0.25,0.5)
\rput[l](-18.0,-3.75){\footnotesize $ (N-.1) \left\{ \begin{matrix} R_1 < 0 \\ \overline{\rho} R_1 \leqslant R_2 \leqslant \underline{\rho} R_1 \end{matrix} \right. $}
\rput[l](11.0,3.75){\footnotesize $ (N+.1) \left\{ \begin{matrix} R_1 > 0 \\ \underline{\rho} R_1 \leqslant R_2 \leqslant \overline{\rho} R_1 \end{matrix} \right. $}
\rput[l](-7.5,-12.0){\footnotesize $ (N-.2) \left\{ \begin{matrix} \overline{\rho} R_2 \leqslant R_1 \leqslant \underline{\rho} R_2 \\ R_2 < 0 \end{matrix} \right. $}
\rput[l](0.5,12.5){\footnotesize $ (N+.2) \left\{ \begin{matrix} \underline{\rho} R_2 \leqslant R_1 \leqslant \overline{\rho} R_2 \\ R_2 > 0 \end{matrix} \right. $}
\rput[l](-4.3,-9.0){\magenta (1.1.1)}
\rput[l](-9.0,-4.0){\magenta (1.1.2)}
\rput[l](-9.0,-3.0){\magenta (1.1.3)}
\rput[l](-9.0,-2.0){\magenta (1.1.4)}
%\rput[l](-2.0,9.0){\magenta (1.1.5)}
%\rput[l](0.1,-9.0){\magenta (1.3.1)}
\rput[l](7.0,2.0){\magenta (1.3.2)}
\rput[l](7.0,3.0){\magenta (1.3.3)}
\rput[l](7.0,4.0){\magenta (1.3.4)}
\rput[l](2.4,9.0){\magenta (1.3.5)}
\rput[l](-7.0,-2.25){\green (2.1.1)}
\rput[l](-5.0,-8.0){\green (2.1.2)}
\rput[l](-3.5,-7.0){\green (2.1.3)}
\rput[l](-3.0,-8.0){\green (2.1.4)}
%\rput[l](5.0,-0.7){\green (2.1.5)}
%\rput[l](-7.0,0.5){\green (2.3.1)}
\rput[l](1.0,8.0){\green (2.3.2)}
\rput[l](1.75,7.0){\green (2.3.3)}
\rput[l](3.0,8.0){\green (2.3.4)}
\rput[l](5.0,2.25){\green (2.3.5)}
\rput[l](-5.0,-15.0){$ (1.1.2) + (1.1.3) + (1.1.4) = (N-.1) $}
\rput[l](-5.0,-16.0){$ (1.3.2) + (1.3.3) + (1.3.4) = (N+.1) $}
\rput[l](-5.0,-17.0){$ (1.1.1) = (N-.2) \qquad (1.3.1) = \emptyset $}
\rput[l](-5.0,-18.0){$ (1.3.5) = (N+.2) \qquad (1.1.5) = \emptyset $}
\rput[l](-5.0,-20.0){$ (2.1.1) = (N-.1) \qquad (2.3.1) = \emptyset $}
\rput[l](-5.0,-21.0){$ (2.3.5) = (N+.1) \qquad (2.1.5) = \emptyset $}
\rput[l](-5.0,-22.0){$ (2.1.2) + (2.1.3) + (2.1.4) = (N-.2) $}
\rput[l](-5.0,-23.0){$ (2.3.2) + (2.3.3) + (2.3.4) = (N+.2) $}
\end{pspicture}
\end{center}

%\vskip 16 pt

\newpage

\centerline{\bf Figure A1.2 \quad Solution zones for $ 0 = \underline{\rho} < \overline{\rho} $}

\vskip 64 pt

\begin{center}
\psset{xunit=0.5,yunit=0.5}
\begin{pspicture}(-10,-10)(10,10)
\put(-5.5,0){\vector(1,0){11}}
\put(0,-5.5){\vector(0,1){11}}
%\psline(-10,0.0)(10,0.0)
\psline(-10,-5.0)(10,5.0)
%\psline(0.0,-10)(0.0,10)
\psline(-5.0,-10)(5.0,10)
\rput(0.3,-0.4)0
\rput(10.8,-0.5){$ R_1 $}
\rput(0.5,10.6){$ R_2 $}
\rput(-12.5,0.0){$ R_2 = \underline{\rho} R_1 $}
\rput(-11.5,-5.0){$ R_2 = \overline{\rho} R_1 $}
\rput(12.5,0.0){$ R_2 = \underline{\rho} R_1 $}
\rput(11.5,5.0){$ R_2 = \overline{\rho} R_1 $}
\rput(1.5,-10.5){$ R_1 = \underline{\rho} R_2 $}
\rput(-5.0,-10.5){$ R_1 = \overline{\rho} R_2 $}
\rput(-1.5,10.5){$ R_1 = \underline{\rho} R_2 $}
\rput(5.0,10.5){$ R_1 = \overline{\rho} R_2 $}
\rput[l](-9.0,5.0){\footnotesize $ (N1.1) \left\{ \begin{matrix} R_1 < \underline{\rho} R_2 \\ R_2 > \underline{\rho} R_1 \end{matrix} \right. $}
\rput[l](3.0,-5.0){\footnotesize $ (N1.2) \left\{ \begin{matrix} R_1 > \underline{\rho} R_2 \\ R_2 < \underline{\rho} R_1 \end{matrix} \right. $}
\psline[linewidth=1.6pt,linecolor=red](-10.0,10)(-10.0,0.000)
\psline[linewidth=1.6pt,linecolor=red](-9.5,10)(-9.5,0.000)
\psline[linewidth=1.6pt,linecolor=red](-9.0,10)(-9.0,6.6)
\psline[linewidth=1.6pt,linecolor=red](-9.0,3.4)(-9.0,0.000)
\psline[linewidth=1.6pt,linecolor=red](-8.5,10)(-8.5,6.6)
\psline[linewidth=1.6pt,linecolor=red](-8.5,3.4)(-8.5,0.000)
\psline[linewidth=1.6pt,linecolor=red](-8.0,10)(-8.0,6.6)
\psline[linewidth=1.6pt,linecolor=red](-8.0,3.4)(-8.0,0.000)
\psline[linewidth=1.6pt,linecolor=red](-7.5,10)(-7.5,6.6)
\psline[linewidth=1.6pt,linecolor=red](-7.5,3.4)(-7.5,0.000)
\psline[linewidth=1.6pt,linecolor=red](-7.0,10)(-7.0,6.6)
\psline[linewidth=1.6pt,linecolor=red](-7.0,3.4)(-7.0,0.000)
\psline[linewidth=1.6pt,linecolor=red](-6.5,10)(-6.5,6.6)
\psline[linewidth=1.6pt,linecolor=red](-6.5,3.4)(-6.5,0.000)
\psline[linewidth=1.6pt,linecolor=red](-6.0,10)(-6.0,6.6)
\psline[linewidth=1.6pt,linecolor=red](-6.0,3.4)(-6.0,0.000)
\psline[linewidth=1.6pt,linecolor=red](-5.5,10)(-5.5,6.6)
\psline[linewidth=1.6pt,linecolor=red](-5.5,3.4)(-5.5,0.000)
\psline[linewidth=1.6pt,linecolor=red](-5.0,10)(-5.0,6.6)
\psline[linewidth=1.6pt,linecolor=red](-5.0,3.4)(-5.0,0.000)
\psline[linewidth=1.6pt,linecolor=red](-4.5,10)(-4.5,6.6)
\psline[linewidth=1.6pt,linecolor=red](-4.5,3.4)(-4.5,0.000)
\psline[linewidth=1.6pt,linecolor=red](-4.0,10)(-4.0,6.6)
\psline[linewidth=1.6pt,linecolor=red](-4.0,3.4)(-4.0,0.000)
\psline[linewidth=1.6pt,linecolor=red](-3.5,10)(-3.5,6.6)
\psline[linewidth=1.6pt,linecolor=red](-3.5,3.4)(-3.5,0.000)
\psline[linewidth=1.6pt,linecolor=red](-3.0,10)(-3.0,0.000)
\psline[linewidth=1.6pt,linecolor=red](-2.5,10)(-2.5,0.000)
\psline[linewidth=1.6pt,linecolor=red](-2.0,10)(-2.0,0.000)
\psline[linewidth=1.6pt,linecolor=red](-1.5,10)(-1.5,0.000)
\psline[linewidth=1.6pt,linecolor=red](-1.0,10)(-1.0,0.000)
\psline[linewidth=1.6pt,linecolor=red](-0.5,10)(-0.5,0.000)
%\psline[linewidth=1.6pt,linecolor=red](0.0,10)(0.0,0.000)
\psline[linewidth=1.6pt,linecolor=red](10.0,-10)(10.0,0.000)
\psline[linewidth=1.6pt,linecolor=red](9.5,-10)(9.5,0.000)
\psline[linewidth=1.6pt,linecolor=red](9.0,-10)(9.0,0.000)
\psline[linewidth=1.6pt,linecolor=red](8.5,-10)(8.5,0.000)
\psline[linewidth=1.6pt,linecolor=red](8.0,-10)(8.0,-6.6)
\psline[linewidth=1.6pt,linecolor=red](8.0,-3.4)(8.0,0.000)
\psline[linewidth=1.6pt,linecolor=red](7.5,-10)(7.5,-6.6)
\psline[linewidth=1.6pt,linecolor=red](7.5,-3.4)(7.5,0.000)
\psline[linewidth=1.6pt,linecolor=red](7.0,-10)(7.0,-6.6)
\psline[linewidth=1.6pt,linecolor=red](7.0,-3.4)(7.0,0.000)
\psline[linewidth=1.6pt,linecolor=red](6.5,-10)(6.5,-6.6)
\psline[linewidth=1.6pt,linecolor=red](6.5,-3.4)(6.5,0.000)
\psline[linewidth=1.6pt,linecolor=red](6.0,-10)(6.0,-6.6)
\psline[linewidth=1.6pt,linecolor=red](6.0,-3.4)(6.0,0.000)
\psline[linewidth=1.6pt,linecolor=red](5.5,-10)(5.5,-6.6)
\psline[linewidth=1.6pt,linecolor=red](5.5,-3.4)(5.5,0.000)
\psline[linewidth=1.6pt,linecolor=red](5.0,-10)(5.0,-6.6)
\psline[linewidth=1.6pt,linecolor=red](5.0,-3.4)(5.0,0.000)
\psline[linewidth=1.6pt,linecolor=red](4.5,-10)(4.5,-6.6)
\psline[linewidth=1.6pt,linecolor=red](4.5,-3.4)(4.5,0.000)
\psline[linewidth=1.6pt,linecolor=red](4.0,-10)(4.0,-6.6)
\psline[linewidth=1.6pt,linecolor=red](4.0,-3.4)(4.0,0.000)
\psline[linewidth=1.6pt,linecolor=red](3.5,-10)(3.5,-6.6)
\psline[linewidth=1.6pt,linecolor=red](3.5,-3.4)(3.5,0.000)
\psline[linewidth=1.6pt,linecolor=red](3.0,-10)(3.0,-6.6)
\psline[linewidth=1.6pt,linecolor=red](3.0,-3.4)(3.0,0.000)
\psline[linewidth=1.6pt,linecolor=red](2.5,-10)(2.5,0.000)
\psline[linewidth=1.6pt,linecolor=red](2.0,-10)(2.0,0.000)
\psline[linewidth=1.6pt,linecolor=red](1.5,-10)(1.5,0.000)
\psline[linewidth=1.6pt,linecolor=red](1.0,-10)(1.0,0.000)
\psline[linewidth=1.6pt,linecolor=red](0.5,-10)(0.5,0.000)
%\psline[linewidth=1.6pt,linecolor=red](0.0,-10)(0.0,0.000)
\rput[l](-9.0,-5.5){\footnotesize $ (N2.1) \left\{ \begin{matrix} R_1 < \overline{\rho} R_2 \\ R_2 < \overline{\rho} R_1 \end{matrix} \right. $}
\rput[l](3.0,5.5){\footnotesize $ (N2.2) \left\{ \begin{matrix} R_1 > \overline{\rho} R_2 \\ R_2 > \overline{\rho} R_1 \end{matrix} \right. $}
\psline[linewidth=1.6pt,linecolor=blue](-10,-10)(-5.0,-10)
\psline[linewidth=1.6pt,linecolor=blue](-10,-9.5)(-4.75,-9.5)
\psline[linewidth=1.6pt,linecolor=blue](-10,-9.0)(-4.50,-9.0)
\psline[linewidth=1.6pt,linecolor=blue](-10,-8.5)(-4.25,-8.5)
\psline[linewidth=1.6pt,linecolor=blue](-10,-8.0)(-4.00,-8.0)
\psline[linewidth=1.6pt,linecolor=blue](-10,-7.5)(-3.75,-7.5)
\psline[linewidth=1.6pt,linecolor=blue](-10,-7.0)(-3.50,-7.0)
\psline[linewidth=1.6pt,linecolor=blue](-10,-6.5)(-9.2,-6.5)
\psline[linewidth=1.6pt,linecolor=blue](-3.5,-6.5)(-3.25,-6.5)
\psline[linewidth=1.6pt,linecolor=blue](-10,-6.0)(-9.2,-6.0)
\psline[linewidth=1.6pt,linecolor=blue](-3.5,-6.0)(-3.00,-6.0)
\psline[linewidth=1.6pt,linecolor=blue](-10,-5.5)(-9.2,-5.5)
\psline[linewidth=1.6pt,linecolor=blue](-3.5,-5.5)(-2.75,-5.5)
\psline[linewidth=1.6pt,linecolor=blue](-10,-5.0)(-9.2,-5.0)
\psline[linewidth=1.6pt,linecolor=blue](-3.5,-5.0)(-2.50,-5.0)
\psline[linewidth=1.6pt,linecolor=blue](-3.5,-4.5)(-2.25,-4.5)
\psline[linewidth=1.6pt,linecolor=blue](-8.0,-4.0)(-2.00,-4.0)
\psline[linewidth=1.6pt,linecolor=blue](-7.0,-3.5)(-1.75,-3.5)
\psline[linewidth=1.6pt,linecolor=blue](-6.0,-3.0)(-1.50,-3.0)
\psline[linewidth=1.6pt,linecolor=blue](-5.0,-2.5)(-1.25,-2.5)
\psline[linewidth=1.6pt,linecolor=blue](-4.0,-2.0)(-1.00,-2.0)
\psline[linewidth=1.6pt,linecolor=blue](-3.0,-1.5)(-0.75,-1.5)
\psline[linewidth=1.6pt,linecolor=blue](-2.0,-1.0)(-0.50,-1.0)
\psline[linewidth=1.6pt,linecolor=blue](-1.0,-0.5)(-0.25,-0.5)
%\psline[linewidth=1.6pt,linecolor=red](0.0,0.0)(0.0,0.0)
\psline[linewidth=1.6pt,linecolor=blue](10,10)(5.0,10)
\psline[linewidth=1.6pt,linecolor=blue](10,9.5)(4.75,9.5)
\psline[linewidth=1.6pt,linecolor=blue](10,9.0)(4.50,9.0)
\psline[linewidth=1.6pt,linecolor=blue](10,8.5)(4.25,8.5)
\psline[linewidth=1.6pt,linecolor=blue](10,8.0)(4.00,8.0)
\psline[linewidth=1.6pt,linecolor=blue](10,7.5)(3.75,7.5)
\psline[linewidth=1.6pt,linecolor=blue](10,7.0)(3.50,7.0)
\psline[linewidth=1.6pt,linecolor=blue](10,6.5)(8.7,6.5)
\psline[linewidth=1.6pt,linecolor=blue](10,6.0)(8.7,6.0)
\psline[linewidth=1.6pt,linecolor=blue](10,5.5)(8.7,5.5)
\psline[linewidth=1.6pt,linecolor=blue](10,5.0)(8.7,5.0)
\psline[linewidth=1.6pt,linecolor=blue](9.0,4.5)(8.7,4.5)
\psline[linewidth=1.6pt,linecolor=blue](2.75,4.5)(2.25,4.5)
\psline[linewidth=1.6pt,linecolor=blue](8.0,4.0)(2.00,4.0)
\psline[linewidth=1.6pt,linecolor=blue](7.0,3.5)(1.75,3.5)
\psline[linewidth=1.6pt,linecolor=blue](6.0,3.0)(1.50,3.0)
\psline[linewidth=1.6pt,linecolor=blue](5.0,2.5)(1.25,2.5)
\psline[linewidth=1.6pt,linecolor=blue](4.0,2.0)(1.00,2.0)
\psline[linewidth=1.6pt,linecolor=blue](3.0,1.5)(0.75,1.5)
\psline[linewidth=1.6pt,linecolor=blue](2.0,1.0)(0.50,1.0)
\psline[linewidth=1.6pt,linecolor=blue](1.0,0.5)(0.25,0.5)
%\psline[linewidth=1.6pt,linecolor=red](0.0,0.0)(0.0,0.0)
\rput[l](-18,-2.5){\footnotesize $ (N-.1) \left\{ \begin{matrix} R_1 < 0 \\ \overline{\rho} R_1 \leqslant R_2 \leqslant \underline{\rho} R_1 \end{matrix} \right. $}
\rput[l](11.0,2.5){\footnotesize $ (N+.1) \left\{ \begin{matrix} R_1 > 0 \\ \underline{\rho} R_1 \leqslant R_2 \leqslant \overline{\rho} R_1 \end{matrix} \right. $}
\rput[l](-6.25,-12.0){\footnotesize $ (N-.2) \left\{ \begin{matrix} \overline{\rho} R_2 \leqslant R_1 \leqslant \underline{\rho} R_2 \\ R_2 < 0 \end{matrix} \right. $}
\rput[l](-1.25,12.5){\footnotesize $ (N+.2) \left\{ \begin{matrix} \overline{\rho} R_2 \leqslant R_1 \leqslant \underline{\rho} R_2 \\ R_2 > 0 \end{matrix} \right. $}
\rput[l](-3.0,-9.0){\magenta (1.1.1)}
\rput[l](-9.0,-4.0){\magenta (1.1.2)}
\rput[l](-9.0,-2.0){\magenta (1.1.3)}
\rput[l](-9.0,0.0){\magenta (1.1.4)}
%\rput[l](-2.0,9.0){\magenta (1.1.5)}
%\rput[l](0.1,-9.0){\magenta (1.3.1)}
\rput[l](7.0,0.0){\magenta (1.3.2)}
\rput[l](7.0,2.0){\magenta (1.3.3)}
\rput[l](7.0,4.0){\magenta (1.3.4)}
\rput[l](1.0,9.0){\magenta (1.3.5)}
\rput[l](-7.0,-1.0){\green (2.1.1)}
\rput[l](-5.0,-8.0){\green (2.1.2)}
\rput[l](-3.0,-8.0){\green (2.1.3)}
\rput[l](-1.0,-8.0){\green (2.1.4)}
%\rput[l](5.0,-0.7){\green (2.1.5)}
%\rput[l](-7.0,0.5){\green (2.3.1)}
\rput[l](-1.0,8.0){\green (2.3.2)}
\rput[l](1.0,8.0){\green (2.3.3)}
\rput[l](3.0,8.0){\green (2.3.4)}
\rput[l](5.0,1.0){\green (2.3.5)}
\rput[l](-5.0,-15.0){$ (1.1.2) + (1.1.3) + (1.1.4) = (N-.1) $}
\rput[l](-5.0,-16.0){$ (1.3.2) + (1.3.3) + (1.3.4) = (N+.1) $}
\rput[l](-5.0,-17.0){$ (1.1.1) = (N-.2) \qquad (1.3.1) = \emptyset $}
\rput[l](-5.0,-18.0){$ (1.3.5) = (N+.2) \qquad (1.1.5) = \emptyset $}
\rput[l](-5.0,-20.0){$ (2.1.1) = (N-.1) \qquad (2.3.1) = \emptyset $}
\rput[l](-5.0,-21.0){$ (2.3.5) = (N+.1) \qquad (2.1.5) = \emptyset $}
\rput[l](-5.0,-22.0){$ (2.1.2) + (2.1.3) + (2.1.4) = (N-.2) $}
\rput[l](-5.0,-23.0){$ (2.3.2) + (2.3.3) + (2.3.4) = (N+.2) $}
\end{pspicture}
\end{center}

%\vskip 16 pt

\newpage

\centerline{\bf Figure A1.3 \quad Solution zones for $ \underline{\rho} < 0 < \overline{\rho} $}

\vskip 64 pt

\begin{center}
\psset{xunit=0.5,yunit=0.5}
\begin{pspicture}(-10,-10)(10,10)
\put(-5.5,0){\vector(1,0){11}}
\put(0,-5.5){\vector(0,1){11}}
\psline(-10,2.5)(10,-2.5)
\psline(-10,-5.0)(10,5.0)
\psline(2.5,-10)(-2.5,10)
\psline(-5.0,-10)(5.0,10)
\rput(0.3,-0.4)0
\rput(10.8,-0.5){$ R_1 $}
\rput(0.5,10.6){$ R_2 $}
\rput(-11.5,2.5){$ R_2 = \underline{\rho} R_1 $}
\rput(-11.5,-5.0){$ R_2 = \overline{\rho} R_1 $}
\rput(11.5,-2.5){$ R_2 = \underline{\rho} R_1 $}
\rput(11.5,5.0){$ R_2 = \overline{\rho} R_1 $}
\rput(2.5,-10.5){$ R_1 = \underline{\rho} R_2 $}
\rput(-5.0,-10.5){$ R_1 = \overline{\rho} R_2 $}
\rput(-2.5,10.5){$ R_1 = \underline{\rho} R_2 $}
\rput(5.0,10.5){$ R_1 = \overline{\rho} R_2 $}
\rput[l](-9.0,5.0){\footnotesize $ (N1.1) \left\{ \begin{matrix} R_1 < \underline{\rho} R_2 \\ R_2 > \underline{\rho} R_1 \end{matrix} \right. $}
\rput[l](3.0,-5.0){\footnotesize $ (N1.2) \left\{ \begin{matrix} R_1 > \underline{\rho} R_2 \\ R_2 < \underline{\rho} R_1 \end{matrix} \right. $}
\psline[linewidth=1.6pt,linecolor=red](-10.0,10)(-10.0,2.500)
\psline[linewidth=1.6pt,linecolor=red](-9.5,10)(-9.5,2.375)
\psline[linewidth=1.6pt,linecolor=red](-9.0,10)(-9.0,6.6)
\psline[linewidth=1.6pt,linecolor=red](-9.0,3.4)(-9.0,2.250)
\psline[linewidth=1.6pt,linecolor=red](-8.5,10)(-8.5,6.6)
\psline[linewidth=1.6pt,linecolor=red](-8.5,3.4)(-8.5,2.125)
\psline[linewidth=1.6pt,linecolor=red](-8.0,10)(-8.0,6.6)
\psline[linewidth=1.6pt,linecolor=red](-8.0,3.4)(-8.0,2.000)
\psline[linewidth=1.6pt,linecolor=red](-7.5,10)(-7.5,6.6)
\psline[linewidth=1.6pt,linecolor=red](-7.5,3.4)(-7.5,1.875)
\psline[linewidth=1.6pt,linecolor=red](-7.0,10)(-7.0,6.6)
\psline[linewidth=1.6pt,linecolor=red](-7.0,3.4)(-7.0,1.750)
\psline[linewidth=1.6pt,linecolor=red](-6.5,10)(-6.5,6.6)
\psline[linewidth=1.6pt,linecolor=red](-6.5,3.4)(-6.5,1.625)
\psline[linewidth=1.6pt,linecolor=red](-6.0,10)(-6.0,6.6)
\psline[linewidth=1.6pt,linecolor=red](-6.0,3.4)(-6.0,1.500)
\psline[linewidth=1.6pt,linecolor=red](-5.5,10)(-5.5,6.6)
\psline[linewidth=1.6pt,linecolor=red](-5.5,3.4)(-5.5,1.375)
\psline[linewidth=1.6pt,linecolor=red](-5.0,10)(-5.0,6.6)
\psline[linewidth=1.6pt,linecolor=red](-5.0,3.4)(-5.0,1.250)
\psline[linewidth=1.6pt,linecolor=red](-4.5,10)(-4.5,6.6)
\psline[linewidth=1.6pt,linecolor=red](-4.5,3.4)(-4.5,1.125)
\psline[linewidth=1.6pt,linecolor=red](-4.0,10)(-4.0,6.6)
\psline[linewidth=1.6pt,linecolor=red](-4.0,3.4)(-4.0,1.000)
\psline[linewidth=1.6pt,linecolor=red](-3.5,10)(-3.5,6.6)
\psline[linewidth=1.6pt,linecolor=red](-3.5,3.4)(-3.5,0.875)
\psline[linewidth=1.6pt,linecolor=red](-3.0,10)(-3.0,0.750)
\psline[linewidth=1.6pt,linecolor=red](-2.5,10)(-2.5,0.625)
\psline[linewidth=1.6pt,linecolor=red](-2.0,8)(-2.0,0.500)
\psline[linewidth=1.6pt,linecolor=red](-1.5,6)(-1.5,0.375)
\psline[linewidth=1.6pt,linecolor=red](-1.0,4)(-1.0,0.250)
\psline[linewidth=1.6pt,linecolor=red](-0.5,2)(-0.5,0.125)
%\psline[linewidth=1.6pt,linecolor=red](0.0,0)(0.0,0.000)
\psline[linewidth=1.6pt,linecolor=red](10.0,-10)(10.0,-2.500)
\psline[linewidth=1.6pt,linecolor=red](9.5,-10)(9.5,-2.325)
\psline[linewidth=1.6pt,linecolor=red](9.0,-10)(9.0,-2.250)
\psline[linewidth=1.6pt,linecolor=red](8.5,-10)(8.5,-2.125)
\psline[linewidth=1.6pt,linecolor=red](8.0,-10)(8.0,-6.6)
\psline[linewidth=1.6pt,linecolor=red](8.0,-3.4)(8.0,-2.000)
\psline[linewidth=1.6pt,linecolor=red](7.5,-10)(7.5,-6.6)
\psline[linewidth=1.6pt,linecolor=red](7.5,-3.4)(7.5,-1.875)
\psline[linewidth=1.6pt,linecolor=red](7.0,-10)(7.0,-6.6)
\psline[linewidth=1.6pt,linecolor=red](7.0,-3.4)(7.0,-1.750)
\psline[linewidth=1.6pt,linecolor=red](6.5,-10)(6.5,-6.6)
\psline[linewidth=1.6pt,linecolor=red](6.5,-3.4)(6.5,-1.625)
\psline[linewidth=1.6pt,linecolor=red](6.0,-10)(6.0,-6.6)
\psline[linewidth=1.6pt,linecolor=red](6.0,-3.4)(6.0,-1.500)
\psline[linewidth=1.6pt,linecolor=red](5.5,-10)(5.5,-6.6)
\psline[linewidth=1.6pt,linecolor=red](5.5,-3.4)(5.5,-1.375)
\psline[linewidth=1.6pt,linecolor=red](5.0,-10)(5.0,-6.6)
\psline[linewidth=1.6pt,linecolor=red](5.0,-3.4)(5.0,-1.250)
\psline[linewidth=1.6pt,linecolor=red](4.5,-10)(4.5,-6.6)
\psline[linewidth=1.6pt,linecolor=red](4.5,-3.4)(4.5,-1.125)
\psline[linewidth=1.6pt,linecolor=red](4.0,-10)(4.0,-6.6)
\psline[linewidth=1.6pt,linecolor=red](4.0,-3.4)(4.0,-1.000)
\psline[linewidth=1.6pt,linecolor=red](3.5,-10)(3.5,-6.6)
\psline[linewidth=1.6pt,linecolor=red](3.5,-3.4)(3.5,-0.875)
\psline[linewidth=1.6pt,linecolor=red](3.0,-10)(3.0,-6.6)
\psline[linewidth=1.6pt,linecolor=red](3.0,-3.4)(3.0,-0.750)
\psline[linewidth=1.6pt,linecolor=red](2.5,-10)(2.5,-0.625)
\psline[linewidth=1.6pt,linecolor=red](2.0,-8)(2.0,-0.500)
\psline[linewidth=1.6pt,linecolor=red](1.5,-6)(1.5,-0.375)
\psline[linewidth=1.6pt,linecolor=red](1.0,-4)(1.0,-0.250)
\psline[linewidth=1.6pt,linecolor=red](0.5,-2)(0.5,-0.125)
%\psline[linewidth=1.6pt,linecolor=red](0.0,0)(0.0,0.000)
\rput[l](-9.0,-5.5){\footnotesize $ (N2.1) \left\{ \begin{matrix} R_1 < \overline{\rho} R_2 \\ R_2 < \overline{\rho} R_1 \end{matrix} \right. $}
\rput[l](3.0,5.5){\footnotesize $ (N2.2) \left\{ \begin{matrix} R_1 > \overline{\rho} R_2 \\ R_2 > \overline{\rho} R_1 \end{matrix} \right. $}
\psline[linewidth=1.6pt,linecolor=blue](-10,-10)(-5.0,-10)
\psline[linewidth=1.6pt,linecolor=blue](-10,-9.5)(-4.75,-9.5)
\psline[linewidth=1.6pt,linecolor=blue](-10,-9.0)(-4.50,-9.0)
\psline[linewidth=1.6pt,linecolor=blue](-10,-8.5)(-4.25,-8.5)
\psline[linewidth=1.6pt,linecolor=blue](-10,-8.0)(-4.00,-8.0)
\psline[linewidth=1.6pt,linecolor=blue](-10,-7.5)(-3.75,-7.5)
\psline[linewidth=1.6pt,linecolor=blue](-10,-7.0)(-3.50,-7.0)
\psline[linewidth=1.6pt,linecolor=blue](-10,-6.5)(-9.2,-6.5)
\psline[linewidth=1.6pt,linecolor=blue](-3.5,-6.5)(-3.25,-6.5)
\psline[linewidth=1.6pt,linecolor=blue](-10,-6.0)(-9.2,-6.0)
\psline[linewidth=1.6pt,linecolor=blue](-3.5,-6.0)(-3.00,-6.0)
\psline[linewidth=1.6pt,linecolor=blue](-10,-5.5)(-9.2,-5.5)
\psline[linewidth=1.6pt,linecolor=blue](-3.5,-5.5)(-2.75,-5.5)
\psline[linewidth=1.6pt,linecolor=blue](-10,-5.0)(-9.2,-5.0)
\psline[linewidth=1.6pt,linecolor=blue](-3.5,-5.0)(-2.50,-5.0)
\psline[linewidth=1.6pt,linecolor=blue](-3.5,-4.5)(-2.25,-4.5)
\psline[linewidth=1.6pt,linecolor=blue](-8.0,-4.0)(-2.00,-4.0)
\psline[linewidth=1.6pt,linecolor=blue](-7.0,-3.5)(-1.75,-3.5)
\psline[linewidth=1.6pt,linecolor=blue](-6.0,-3.0)(-1.50,-3.0)
\psline[linewidth=1.6pt,linecolor=blue](-5.0,-2.5)(-1.25,-2.5)
\psline[linewidth=1.6pt,linecolor=blue](-4.0,-2.0)(-1.00,-2.0)
\psline[linewidth=1.6pt,linecolor=blue](-3.0,-1.5)(-0.75,-1.5)
\psline[linewidth=1.6pt,linecolor=blue](-2.0,-1.0)(-0.50,-1.0)
\psline[linewidth=1.6pt,linecolor=blue](-1.0,-0.5)(-0.25,-0.5)
%\psline[linewidth=1.6pt,linecolor=red](0.0,0.0)(0.0,0.0)
\psline[linewidth=1.6pt,linecolor=blue](10,10)(5.0,10)
\psline[linewidth=1.6pt,linecolor=blue](10,9.5)(4.75,9.5)
\psline[linewidth=1.6pt,linecolor=blue](10,9.0)(4.50,9.0)
\psline[linewidth=1.6pt,linecolor=blue](10,8.5)(4.25,8.5)
\psline[linewidth=1.6pt,linecolor=blue](10,8.0)(4.00,8.0)
\psline[linewidth=1.6pt,linecolor=blue](10,7.5)(3.75,7.5)
\psline[linewidth=1.6pt,linecolor=blue](10,7.0)(3.50,7.0)
\psline[linewidth=1.6pt,linecolor=blue](10,6.5)(8.7,6.5)
%\psline[linewidth=1.6pt,linecolor=blue](3.5,6.5)(3.25,6.5)
\psline[linewidth=1.6pt,linecolor=blue](10,6.0)(8.7,6.0)
%\psline[linewidth=1.6pt,linecolor=blue](3.5,6.0)(3.00,6.0)
\psline[linewidth=1.6pt,linecolor=blue](10,5.5)(8.7,5.5)
%\psline[linewidth=1.6pt,linecolor=blue](3.5,5.5)(2.75,5.5)
\psline[linewidth=1.6pt,linecolor=blue](10,5.0)(8.7,5.0)
%\psline[linewidth=1.6pt,linecolor=blue](3.5,5.0)(2.50,5.0)
\psline[linewidth=1.6pt,linecolor=blue](9.0,4.5)(8.7,4.5)
\psline[linewidth=1.6pt,linecolor=blue](2.75,4.5)(2.25,4.5)
\psline[linewidth=1.6pt,linecolor=blue](8.0,4.0)(2.00,4.0)
\psline[linewidth=1.6pt,linecolor=blue](7.0,3.5)(1.75,3.5)
\psline[linewidth=1.6pt,linecolor=blue](6.0,3.0)(1.50,3.0)
\psline[linewidth=1.6pt,linecolor=blue](5.0,2.5)(1.25,2.5)
\psline[linewidth=1.6pt,linecolor=blue](4.0,2.0)(1.00,2.0)
\psline[linewidth=1.6pt,linecolor=blue](3.0,1.5)(0.75,1.5)
\psline[linewidth=1.6pt,linecolor=blue](2.0,1.0)(0.50,1.0)
\psline[linewidth=1.6pt,linecolor=blue](1.0,0.5)(0.25,0.5)
%\psline[linewidth=1.6pt,linecolor=red](0.0,0.0)(0.0,0.0)
\rput[l](-18,-1.5){\footnotesize $ (N-.1) \left\{ \begin{matrix} R_1 < 0 \\ \overline{\rho} R_1 \leqslant R_2 \leqslant \underline{\rho} R_1 \end{matrix} \right. $}
\rput[l](11.0,1.5){\footnotesize $ (N+.1) \left\{ \begin{matrix} R_1 > 0 \\ \underline{\rho} R_1 \leqslant R_2 \leqslant \overline{\rho} R_1 \end{matrix} \right. $}
\rput[l](-5.0,-12.0){\footnotesize $ (N-.2) \left\{ \begin{matrix} \overline{\rho} R_2 \leqslant R_1 \leqslant \underline{\rho} R_2 \\ R_2 < 0 \end{matrix} \right. $}
\rput[l](-2.5,12.5){\footnotesize $ (N+.2) \left\{ \begin{matrix} \overline{\rho} R_2 \leqslant R_1 \leqslant \underline{\rho} R_2 \\ R_2 > 0 \end{matrix} \right. $}
\rput[l](-3.0,-9.0){\magenta (1.1.1)}
\rput[l](-9.0,-4.0){\magenta (1.1.2)}
\rput[l](-9.0,-1.0){\magenta (1.1.3)}
\rput[l](-9.0,2.0){\magenta (1.1.4)}
\rput[l](-2.0,9.0){\magenta (1.1.5)}
\rput[l](0.1,-9.0){\magenta (1.3.1)}
\rput[l](7.0,-2.0){\magenta (1.3.2)}
\rput[l](7.0,1.0){\magenta (1.3.3)}
\rput[l](7.0,4.0){\magenta (1.3.4)}
\rput[l](1.0,9.0){\magenta (1.3.5)}
\rput[l](-7.0,-1.0){\green (2.1.1)}
\rput[l](-5.0,-8.0){\green (2.1.2)}
\rput[l](-1.5,-8.0){\green (2.1.3)}
\rput[l](1.0,-8.0){\green (2.1.4)}
\rput[l](5.0,-0.7){\green (2.1.5)}
\rput[l](-7.0,0.5){\green (2.3.1)}
\rput[l](-3.0,8.0){\green (2.3.2)}
\rput[l](-0.5,8.0){\green (2.3.3)}
\rput[l](3.0,8.0){\green (2.3.4)}
\rput[l](5.0,1.0){\green (2.3.5)}
\rput[l](-5.0,-15.0){$ (1.1.2) + (1.1.3) + (1.1.4) = (N-.1) $}
\rput[l](-5.0,-16.0){$ (1.3.2) + (1.3.3) + (1.3.4) = (N+.1) $}
\rput[l](-5.0,-17.0){$ (1.1.1) + (1.3.1) = (N-.2) $}
\rput[l](-5.0,-18.0){$ (1.1.5) + (1.3.5) = (N+.2) $}
\rput[l](-5.0,-20.0){$ (2.1.1) + (2.3.1) = (N-.1) $}
\rput[l](-5.0,-21.0){$ (2.1.5) + (2.3.5) = (N+.1) $}
\rput[l](-5.0,-22.0){$ (2.1.2) + (2.1.3) + (2.1.4) = (N-.2) $}
\rput[l](-5.0,-23.0){$ (2.3.2) + (2.3.3) + (2.3.4) = (N+.2) $}
\end{pspicture}
\end{center}

%\vskip 16 pt

\newpage

\centerline{\bf Figure A1.4 \quad Solution zones for $ \underline{\rho} < \overline{\rho} = 0 $}

\vskip 64 pt

\begin{center}
\psset{xunit=0.5,yunit=0.5}
\begin{pspicture}(-10,-10)(10,10)
\put(-5.5,0){\vector(1,0){11}}
\put(0,-5.5){\vector(0,1){11}}
%\psline(-10,2.5)(10,-2.5)
\psline(-10,5.0)(10,-5.0)
%\psline(2.5,-10)(-2.5,10)
\psline(5.0,-10)(-5.0,10)
\rput(0.3,-0.4)0
\rput(10.8,-0.5){$ R_1 $}
\rput(-0.5,10.6){$ R_2 $}
\rput(-11.5,5.5){$ R_2 = \underline{\rho} R_1 $}
\rput(-12.5,-0.5){$ R_2 = \overline{\rho} R_1 $}
\rput(11.5,-5.5){$ R_2 = \underline{\rho} R_1 $}
\rput(11.5,0.5){$ R_2 = \overline{\rho} R_1 $}
\rput(6.0,-10.5){$ R_1 = \underline{\rho} R_2 $}
\rput(-1.5,-10.5){$ R_1 = \overline{\rho} R_2 $}
\rput(-6.0,10.5){$ R_1 = \underline{\rho} R_2 $}
\rput(1.5,10.5){$ R_1 = \overline{\rho} R_2 $}
\rput[l](-9.0,5.0){\footnotesize $ (N1.1) \left\{ \begin{matrix} R_1 < \underline{\rho} R_2 \\ R_2 > \underline{\rho} R_1 \end{matrix} \right. $}
\rput[l](3.0,-5.0){\footnotesize $ (N1.2) \left\{ \begin{matrix} R_1 > \underline{\rho} R_2 \\ R_2 < \underline{\rho} R_1 \end{matrix} \right. $}
\psline[linewidth=1.6pt,linecolor=red](-10.0,10)(-10.0,5.00)
\psline[linewidth=1.6pt,linecolor=red](-9.5,10)(-9.5,4.75)
\psline[linewidth=1.6pt,linecolor=red](-9.0,10)(-9.0,6.6)
\psline[linewidth=1.6pt,linecolor=red](-8.5,10)(-8.5,6.6)
\psline[linewidth=1.6pt,linecolor=red](-8.0,10)(-8.0,6.6)
\psline[linewidth=1.6pt,linecolor=red](-7.5,10)(-7.5,6.6)
\psline[linewidth=1.6pt,linecolor=red](-7.0,10)(-7.0,6.6)
\psline[linewidth=1.6pt,linecolor=red](-6.5,10)(-6.5,6.6)
\psline[linewidth=1.6pt,linecolor=red](-6.5,3.4)(-6.5,3.25)
\psline[linewidth=1.6pt,linecolor=red](-6.0,10)(-6.0,6.6)
\psline[linewidth=1.6pt,linecolor=red](-6.0,3.4)(-6.0,3.00)
\psline[linewidth=1.6pt,linecolor=red](-5.5,10)(-5.5,6.6)
\psline[linewidth=1.6pt,linecolor=red](-5.5,3.4)(-5.5,2.75)
\psline[linewidth=1.6pt,linecolor=red](-5.0,10)(-5.0,6.6)
\psline[linewidth=1.6pt,linecolor=red](-5.0,3.4)(-5.0,2.50)
\psline[linewidth=1.6pt,linecolor=red](-4.5,9)(-4.5,6.6)
\psline[linewidth=1.6pt,linecolor=red](-4.5,3.4)(-4.5,2.25)
\psline[linewidth=1.6pt,linecolor=red](-4.0,8)(-4.0,6.6)
\psline[linewidth=1.6pt,linecolor=red](-4.0,3.4)(-4.0,2.00)
\psline[linewidth=1.6pt,linecolor=red](-3.5,7)(-3.5,6.6)
\psline[linewidth=1.6pt,linecolor=red](-3.5,3.4)(-3.5,1.75)
\psline[linewidth=1.6pt,linecolor=red](-3.0,6)(-3.0,1.50)
\psline[linewidth=1.6pt,linecolor=red](-2.5,5)(-2.5,1.25)
\psline[linewidth=1.6pt,linecolor=red](-2.0,4)(-2.0,1.00)
\psline[linewidth=1.6pt,linecolor=red](-1.5,3)(-1.5,0.75)
\psline[linewidth=1.6pt,linecolor=red](-1.0,2)(-1.0,0.50)
\psline[linewidth=1.6pt,linecolor=red](-0.5,1)(-0.5,0.25)
%\psline[linewidth=1.6pt,linecolor=red](0.0,0)(0.0,0.00)
\psline[linewidth=1.6pt,linecolor=red](10.0,-10)(10.0,-5.00)
\psline[linewidth=1.6pt,linecolor=red](9.5,-10)(9.5,-4.75)
\psline[linewidth=1.6pt,linecolor=red](9.0,-10)(9.0,-4.50)
\psline[linewidth=1.6pt,linecolor=red](8.5,-10)(8.5,-4.25)
\psline[linewidth=1.6pt,linecolor=red](8.0,-10)(8.0,-6.6)
\psline[linewidth=1.6pt,linecolor=red](7.5,-10)(7.5,-6.6)
\psline[linewidth=1.6pt,linecolor=red](7.0,-10)(7.0,-6.6)
\psline[linewidth=1.6pt,linecolor=red](6.5,-10)(6.5,-6.6)
\psline[linewidth=1.6pt,linecolor=red](6.5,-3.4)(6.5,-3.25)
\psline[linewidth=1.6pt,linecolor=red](6.0,-10)(6.0,-6.6)
\psline[linewidth=1.6pt,linecolor=red](6.0,-3.4)(6.0,-3.00)
\psline[linewidth=1.6pt,linecolor=red](5.5,-10)(5.5,-6.6)
\psline[linewidth=1.6pt,linecolor=red](5.5,-3.4)(5.5,-2.75)
\psline[linewidth=1.6pt,linecolor=red](5.0,-10)(5.0,-6.6)
\psline[linewidth=1.6pt,linecolor=red](5.0,-3.4)(5.0,-2.50)
\psline[linewidth=1.6pt,linecolor=red](4.5,-9)(4.5,-6.6)
\psline[linewidth=1.6pt,linecolor=red](4.5,-3.4)(4.5,-2.25)
\psline[linewidth=1.6pt,linecolor=red](4.0,-8)(4.0,-6.6)
\psline[linewidth=1.6pt,linecolor=red](4.0,-3.4)(4.0,-2.00)
\psline[linewidth=1.6pt,linecolor=red](3.5,-7)(3.5,-6.6)
\psline[linewidth=1.6pt,linecolor=red](3.5,-3.4)(3.5,-1.75)
\psline[linewidth=1.6pt,linecolor=red](3.0,-3.4)(3.0,-1.50)
\psline[linewidth=1.6pt,linecolor=red](2.5,-5)(2.5,-1.25)
\psline[linewidth=1.6pt,linecolor=red](2.0,-4)(2.0,-1.00)
\psline[linewidth=1.6pt,linecolor=red](1.5,-3)(1.5,-0.75)
\psline[linewidth=1.6pt,linecolor=red](1.0,-2)(1.0,-0.50)
\psline[linewidth=1.6pt,linecolor=red](0.5,-1)(0.5,-0.25)
%\psline[linewidth=1.6pt,linecolor=red](0.0,0)(0.0,0.00)
\rput[l](-9.0,-5.5){\footnotesize $ (N2.1) \left\{ \begin{matrix} R_1 < \overline{\rho} R_2 \\ R_2 < \overline{\rho} R_1 \end{matrix} \right. $}
\rput[l](3.0,5.5){\footnotesize $ (N2.2) \left\{ \begin{matrix} R_1 > \overline{\rho} R_2 \\ R_2 > \overline{\rho} R_1 \end{matrix} \right. $}
\psline[linewidth=1.6pt,linecolor=blue](-10,-10)(0.00,-10)
\psline[linewidth=1.6pt,linecolor=blue](-10,-9.5)(0.00,-9.5)
\psline[linewidth=1.6pt,linecolor=blue](-10,-9.0)(0.00,-9.0)
\psline[linewidth=1.6pt,linecolor=blue](-10,-8.5)(0.00,-8.5)
\psline[linewidth=1.6pt,linecolor=blue](-10,-8.0)(0.00,-8.0)
\psline[linewidth=1.6pt,linecolor=blue](-10,-7.5)(0.00,-7.5)
\psline[linewidth=1.6pt,linecolor=blue](-10,-7.0)(0.00,-7.0)
\psline[linewidth=1.6pt,linecolor=blue](-10,-6.5)(-9.2,-6.5)
\psline[linewidth=1.6pt,linecolor=blue](-3.5,-6.5)(0.00,-6.5)
\psline[linewidth=1.6pt,linecolor=blue](-10,-6.0)(-9.2,-6.0)
\psline[linewidth=1.6pt,linecolor=blue](-3.5,-6.0)(0.00,-6.0)
\psline[linewidth=1.6pt,linecolor=blue](-10,-5.5)(-9.2,-5.5)
\psline[linewidth=1.6pt,linecolor=blue](-3.5,-5.5)(0.00,-5.5)
\psline[linewidth=1.6pt,linecolor=blue](-10,-5.0)(-9.2,-5.0)
\psline[linewidth=1.6pt,linecolor=blue](-3.5,-5.0)(0.00,-5.0)
\psline[linewidth=1.6pt,linecolor=blue](-10,-4.5)(-9.2,-4.5)
\psline[linewidth=1.6pt,linecolor=blue](-3.5,-4.5)(0.00,-4.5)
\psline[linewidth=1.6pt,linecolor=blue](-10,-4.0)(0.00,-4.0)
\psline[linewidth=1.6pt,linecolor=blue](-10,-3.5)(0.00,-3.5)
\psline[linewidth=1.6pt,linecolor=blue](-10,-3.0)(0.00,-3.0)
\psline[linewidth=1.6pt,linecolor=blue](-10,-2.5)(0.00,-2.5)
\psline[linewidth=1.6pt,linecolor=blue](-10,-2.0)(0.00,-2.0)
\psline[linewidth=1.6pt,linecolor=blue](-10,-1.5)(0.00,-1.5)
\psline[linewidth=1.6pt,linecolor=blue](-10,-1.0)(0.00,-1.0)
\psline[linewidth=1.6pt,linecolor=blue](-10,-0.5)(0.00,-0.5)
%\psline[linewidth=1.6pt,linecolor=red](-10,0.0)(0.00,0.0)
\psline[linewidth=1.6pt,linecolor=blue](10,10)(0.00,10)
\psline[linewidth=1.6pt,linecolor=blue](10,9.5)(0.00,9.5)
\psline[linewidth=1.6pt,linecolor=blue](10,9.0)(0.00,9.0)
\psline[linewidth=1.6pt,linecolor=blue](10,8.5)(0.00,8.5)
\psline[linewidth=1.6pt,linecolor=blue](10,8.0)(0.00,8.0)
\psline[linewidth=1.6pt,linecolor=blue](10,7.5)(0.00,7.5)
\psline[linewidth=1.6pt,linecolor=blue](10,7.0)(0.00,7.0)
\psline[linewidth=1.6pt,linecolor=blue](10,6.5)(8.7,6.5)
\psline[linewidth=1.6pt,linecolor=blue](2.75,6.5)(0.00,6.5)
\psline[linewidth=1.6pt,linecolor=blue](10,6.0)(8.7,6.0)
\psline[linewidth=1.6pt,linecolor=blue](2.75,6.0)(0.00,6.0)
\psline[linewidth=1.6pt,linecolor=blue](10,5.5)(8.7,5.5)
\psline[linewidth=1.6pt,linecolor=blue](2.75,5.5)(0.00,5.5)
\psline[linewidth=1.6pt,linecolor=blue](10,5.0)(8.7,5.0)
\psline[linewidth=1.6pt,linecolor=blue](2.75,5.0)(0.00,5.0)
\psline[linewidth=1.6pt,linecolor=blue](10,4.5)(8.7,4.5)
\psline[linewidth=1.6pt,linecolor=blue](2.75,4.5)(0.00,4.5)
\psline[linewidth=1.6pt,linecolor=blue](10,4.0)(0.00,4.0)
\psline[linewidth=1.6pt,linecolor=blue](10,3.5)(0.00,3.5)
\psline[linewidth=1.6pt,linecolor=blue](10,3.0)(0.00,3.0)
\psline[linewidth=1.6pt,linecolor=blue](10,2.5)(0.00,2.5)
\psline[linewidth=1.6pt,linecolor=blue](10,2.0)(0.00,2.0)
\psline[linewidth=1.6pt,linecolor=blue](10,1.5)(0.00,1.5)
\psline[linewidth=1.6pt,linecolor=blue](10,1.0)(0.00,1.0)
\psline[linewidth=1.6pt,linecolor=blue](10,0.5)(0.00,0.5)
%\psline[linewidth=1.6pt,linecolor=red](10,0.0)(0.00,0.0)
\rput[l](-18,2.5){\footnotesize $ (N-.1) \left\{ \begin{matrix} R_1 < 0 \\ \overline{\rho} R_1 \leqslant R_2 \leqslant \underline{\rho} R_1 \end{matrix} \right. $}
\rput[l](11.0,-2.5){\footnotesize $ (N+.1) \left\{ \begin{matrix} R_1 > 0 \\ \underline{\rho} R_1 \leqslant R_2 \leqslant \overline{\rho} R_1 \end{matrix} \right. $}
\rput[l](-1.0,-12.0){\footnotesize $ (N-.2) \left\{ \begin{matrix} \overline{\rho} R_2 \leqslant R_1 \leqslant \underline{\rho} R_2 \\ R_2 < 0 \end{matrix} \right. $}
\rput[l](-6.0,12.5){\footnotesize $ (N+.2) \left\{ \begin{matrix} \overline{\rho} R_2 \leqslant R_1 \leqslant \underline{\rho} R_2 \\ R_2 > 0 \end{matrix} \right. $}
%\rput[l](-3.0,-9.0){\magenta (1.1.1)}
\rput[l](-9.0,0.0){\magenta (1.1.2)}
\rput[l](-9.0,2.0){\magenta (1.1.3)}
\rput[l](-9.0,4.0){\magenta (1.1.4)}
\rput[l](-3.0,9.0){\magenta (1.1.5)}
\rput[l](1.0,-9.0){\magenta (1.3.1)}
\rput[l](7.0,-4.0){\magenta (1.3.2)}
\rput[l](7.0,-2.0){\magenta (1.3.3)}
\rput[l](7.0,0.0){\magenta (1.3.4)}
%\rput[l](1.0,9.0){\magenta (1.3.5)}
%\rput[l](-7.0,-1.0){\green (2.1.1)}
\rput[l](-1.0,-8.0){\green (2.1.2)}
\rput[l](0.75,-7.0){\green (2.1.3)}
\rput[l](3.0,-8.0){\green (2.1.4)}
\rput[l](5.0,-1.5){\green (2.1.5)}
\rput[l](-7.0,1.5){\green (2.3.1)}
\rput[l](-5.0,8.0){\green (2.3.2)}
\rput[l](-2.5,7.0){\green (2.3.3)}
\rput[l](-1.0,8.0){\green (2.3.4)}
%\rput[l](5.0,1.0){\green (2.3.5)}
\rput[l](-5.0,-15.0){$ (1.1.2) + (1.1.3) + (1.1.4) = (N-.1) $}
\rput[l](-5.0,-16.0){$ (1.3.2) + (1.3.3) + (1.3.4) = (N+.1) $}
\rput[l](-5.0,-17.0){$ (1.3.1) = (N-.2) \qquad (1.1.1) = \emptyset $}
\rput[l](-5.0,-18.0){$ (1.1.5) = (N+.2) \qquad (1.3.5) = \emptyset $}
\rput[l](-5.0,-20.0){$ (2.3.1) = (N-.1) \qquad (2.1.1) = \emptyset $}
\rput[l](-5.0,-21.0){$ (2.1.5) = (N+.1) \qquad (2.3.5) = \emptyset $}
\rput[l](-5.0,-22.0){$ (2.1.2) + (2.1.3) + (2.1.4) = (N-.2) $}
\rput[l](-5.0,-23.0){$ (2.3.2) + (2.3.3) + (2.3.4) = (N+.2) $}
\end{pspicture}
\end{center}

%\vskip 16 pt

\newpage

\centerline{\bf Figure A1.5 \quad Solution zones for $ \underline{\rho} < \overline{\rho} < 0 $}

\vskip 64 pt

\begin{center}
\psset{xunit=0.5,yunit=0.5}
\begin{pspicture}(-10,-10)(10,10)
\put(-5.5,0){\vector(1,0){11}}
\put(0,-5.5){\vector(0,1){11}}
\psline(-10,2.5)(10,-2.5)
\psline(-10,5.0)(10,-5.0)
\psline(2.5,-10)(-2.5,10)
\psline(5.0,-10)(-5.0,10)
\rput(-0.3,-0.4)0
\rput(10.8,0.5){$ R_1 $}
\rput(0.5,10.6){$ R_2 $}
\rput(-11.5,5.5){$ R_2 = \underline{\rho} R_1 $}
\rput(-11.5,2.0){$ R_2 = \overline{\rho} R_1 $}
\rput(11.5,-5.5){$ R_2 = \underline{\rho} R_1 $}
\rput(11.5,-2.0){$ R_2 = \overline{\rho} R_1 $}
\rput(6.0,-10.5){$ R_1 = \underline{\rho} R_2 $}
\rput(1.5,-10.5){$ R_1 = \overline{\rho} R_2 $}
\rput(-6.0,10.5){$ R_1 = \underline{\rho} R_2 $}
\rput(-1.5,10.5){$ R_1 = \overline{\rho} R_2 $}
\rput[l](-9.0,5.0){\footnotesize $ (N1.1) \left\{ \begin{matrix} R_1 < \underline{\rho} R_2 \\ R_2 > \underline{\rho} R_1 \end{matrix} \right. $}
\rput[l](3.0,-5.0){\footnotesize $ (N1.2) \left\{ \begin{matrix} R_1 > \underline{\rho} R_2 \\ R_2 < \underline{\rho} R_1 \end{matrix} \right. $}
\psline[linewidth=1.6pt,linecolor=red](-10.0,10)(-10.0,5.00)
\psline[linewidth=1.6pt,linecolor=red](-9.5,10)(-9.5,4.75)
\psline[linewidth=1.6pt,linecolor=red](-9.0,10)(-9.0,6.6)
\psline[linewidth=1.6pt,linecolor=red](-8.5,10)(-8.5,6.6)
\psline[linewidth=1.6pt,linecolor=red](-8.0,10)(-8.0,6.6)
\psline[linewidth=1.6pt,linecolor=red](-7.5,10)(-7.5,6.6)
\psline[linewidth=1.6pt,linecolor=red](-7.0,10)(-7.0,6.6)
\psline[linewidth=1.6pt,linecolor=red](-6.5,10)(-6.5,6.6)
\psline[linewidth=1.6pt,linecolor=red](-6.5,3.4)(-6.5,3.25)
\psline[linewidth=1.6pt,linecolor=red](-6.0,10)(-6.0,6.6)
\psline[linewidth=1.6pt,linecolor=red](-6.0,3.4)(-6.0,3.00)
\psline[linewidth=1.6pt,linecolor=red](-5.5,10)(-5.5,6.6)
\psline[linewidth=1.6pt,linecolor=red](-5.5,3.4)(-5.5,2.75)
\psline[linewidth=1.6pt,linecolor=red](-5.0,10)(-5.0,6.6)
\psline[linewidth=1.6pt,linecolor=red](-5.0,3.4)(-5.0,2.50)
\psline[linewidth=1.6pt,linecolor=red](-4.5,9)(-4.5,6.6)
\psline[linewidth=1.6pt,linecolor=red](-4.5,3.4)(-4.5,2.25)
\psline[linewidth=1.6pt,linecolor=red](-4.0,8)(-4.0,6.6)
\psline[linewidth=1.6pt,linecolor=red](-4.0,3.4)(-4.0,2.00)
\psline[linewidth=1.6pt,linecolor=red](-3.5,7)(-3.5,6.6)
\psline[linewidth=1.6pt,linecolor=red](-3.5,3.4)(-3.5,1.75)
\psline[linewidth=1.6pt,linecolor=red](-3.0,6)(-3.0,1.50)
\psline[linewidth=1.6pt,linecolor=red](-2.5,5)(-2.5,1.25)
\psline[linewidth=1.6pt,linecolor=red](-2.0,4)(-2.0,1.00)
\psline[linewidth=1.6pt,linecolor=red](-1.5,3)(-1.5,0.75)
\psline[linewidth=1.6pt,linecolor=red](-1.0,2)(-1.0,0.50)
\psline[linewidth=1.6pt,linecolor=red](-0.5,1)(-0.5,0.25)
%\psline[linewidth=1.6pt,linecolor=red](0.0,0)(0.0,0.00)
\psline[linewidth=1.6pt,linecolor=red](10.0,-10)(10.0,-5.00)
\psline[linewidth=1.6pt,linecolor=red](9.5,-10)(9.5,-4.75)
\psline[linewidth=1.6pt,linecolor=red](9.0,-10)(9.0,-4.50)
\psline[linewidth=1.6pt,linecolor=red](8.5,-10)(8.5,-4.25)
\psline[linewidth=1.6pt,linecolor=red](8.0,-10)(8.0,-6.6)
\psline[linewidth=1.6pt,linecolor=red](7.5,-10)(7.5,-6.6)
\psline[linewidth=1.6pt,linecolor=red](7.0,-10)(7.0,-6.6)
\psline[linewidth=1.6pt,linecolor=red](6.5,-10)(6.5,-6.6)
\psline[linewidth=1.6pt,linecolor=red](6.5,-3.4)(6.5,-3.25)
\psline[linewidth=1.6pt,linecolor=red](6.0,-10)(6.0,-6.6)
\psline[linewidth=1.6pt,linecolor=red](6.0,-3.4)(6.0,-3.00)
\psline[linewidth=1.6pt,linecolor=red](5.5,-10)(5.5,-6.6)
\psline[linewidth=1.6pt,linecolor=red](5.5,-3.4)(5.5,-2.75)
\psline[linewidth=1.6pt,linecolor=red](5.0,-10)(5.0,-6.6)
\psline[linewidth=1.6pt,linecolor=red](5.0,-3.4)(5.0,-2.50)
\psline[linewidth=1.6pt,linecolor=red](4.5,-9)(4.5,-6.6)
\psline[linewidth=1.6pt,linecolor=red](4.5,-3.4)(4.5,-2.25)
\psline[linewidth=1.6pt,linecolor=red](4.0,-8)(4.0,-6.6)
\psline[linewidth=1.6pt,linecolor=red](4.0,-3.4)(4.0,-2.00)
\psline[linewidth=1.6pt,linecolor=red](3.5,-7)(3.5,-6.6)
\psline[linewidth=1.6pt,linecolor=red](3.5,-3.4)(3.5,-1.75)
\psline[linewidth=1.6pt,linecolor=red](3.0,-3.4)(3.0,-1.50)
\psline[linewidth=1.6pt,linecolor=red](2.5,-5)(2.5,-1.25)
\psline[linewidth=1.6pt,linecolor=red](2.0,-4)(2.0,-1.00)
\psline[linewidth=1.6pt,linecolor=red](1.5,-3)(1.5,-0.75)
\psline[linewidth=1.6pt,linecolor=red](1.0,-2)(1.0,-0.50)
\psline[linewidth=1.6pt,linecolor=red](0.5,-1)(0.5,-0.25)
%\psline[linewidth=1.6pt,linecolor=red](0.0,0)(0.0,0.00)
\rput[l](-9.0,-5.5){\footnotesize $ (N2.1) \left\{ \begin{matrix} R_1 < \overline{\rho} R_2 \\ R_2 < \overline{\rho} R_1 \end{matrix} \right. $}
\rput[l](3.0,5.5){\footnotesize $ (N2.2) \left\{ \begin{matrix} R_1 > \overline{\rho} R_2 \\ R_2 > \overline{\rho} R_1 \end{matrix} \right. $}
\psline[linewidth=1.6pt,linecolor=blue](-10,-10)(2.5,-10)
\psline[linewidth=1.6pt,linecolor=blue](-10,-9.5)(2.375,-9.5)
\psline[linewidth=1.6pt,linecolor=blue](-10,-9.0)(2.250,-9.0)
\psline[linewidth=1.6pt,linecolor=blue](-10,-8.5)(2.125,-8.5)
\psline[linewidth=1.6pt,linecolor=blue](-10,-8.0)(2.000,-8.0)
\psline[linewidth=1.6pt,linecolor=blue](-10,-7.5)(1.875,-7.5)
\psline[linewidth=1.6pt,linecolor=blue](-10,-7.0)(1.750,-7.0)
\psline[linewidth=1.6pt,linecolor=blue](-10,-6.5)(-9.2,-6.5)
\psline[linewidth=1.6pt,linecolor=blue](-3.5,-6.5)(1.625,-6.5)
\psline[linewidth=1.6pt,linecolor=blue](-10,-6.0)(-9.2,-6.0)
\psline[linewidth=1.6pt,linecolor=blue](-3.5,-6.0)(1.500,-6.0)
\psline[linewidth=1.6pt,linecolor=blue](-10,-5.5)(-9.2,-5.5)
\psline[linewidth=1.6pt,linecolor=blue](-3.5,-5.5)(1.375,-5.5)
\psline[linewidth=1.6pt,linecolor=blue](-10,-5.0)(-9.2,-5.0)
\psline[linewidth=1.6pt,linecolor=blue](-3.5,-5.0)(1.250,-5.0)
\psline[linewidth=1.6pt,linecolor=blue](-10,-4.5)(-9.2,-4.5)
\psline[linewidth=1.6pt,linecolor=blue](-3.5,-4.5)(1.125,-4.5)
\psline[linewidth=1.6pt,linecolor=blue](-10,-4.0)(1.000,-4.0)
\psline[linewidth=1.6pt,linecolor=blue](-10,-3.5)(0.875,-3.5)
\psline[linewidth=1.6pt,linecolor=blue](-10,-3.0)(0.750,-3.0)
\psline[linewidth=1.6pt,linecolor=blue](-10,-2.5)(0.625,-2.5)
\psline[linewidth=1.6pt,linecolor=blue](-10,-2.0)(0.500,-2.0)
\psline[linewidth=1.6pt,linecolor=blue](-10,-1.5)(0.375,-1.5)
\psline[linewidth=1.6pt,linecolor=blue](-10,-1.0)(0.250,-1.0)
\psline[linewidth=1.6pt,linecolor=blue](-10,-0.5)(0.125,-0.5)
\psline[linewidth=1.6pt,linecolor=blue](-10,0.0)(0.0,0.0)
\psline[linewidth=1.6pt,linecolor=blue](-10,0.5)(-2.0,0.5)
\psline[linewidth=1.6pt,linecolor=blue](-10,1.0)(-4.0,1.0)
\psline[linewidth=1.6pt,linecolor=blue](-10,1.5)(-6.0,1.5)
\psline[linewidth=1.6pt,linecolor=blue](-10,2.0)(-8.0,2.0)
\psline[linewidth=1.6pt,linecolor=blue](10,10.0)(-2.500,10.0)
\psline[linewidth=1.6pt,linecolor=blue](10,9.5)(-2.375,9.5)
\psline[linewidth=1.6pt,linecolor=blue](10,9.0)(-2.250,9.0)
\psline[linewidth=1.6pt,linecolor=blue](10,8.5)(-2.125,8.5)
\psline[linewidth=1.6pt,linecolor=blue](10,8.0)(-2.000,8.0)
\psline[linewidth=1.6pt,linecolor=blue](10,7.5)(-1.875,7.5)
\psline[linewidth=1.6pt,linecolor=blue](10,7.0)(-1.750,7.0)
\psline[linewidth=1.6pt,linecolor=blue](10,6.5)(8.7,6.5)
\psline[linewidth=1.6pt,linecolor=blue](2.75,6.5)(-1.625,6.5)
\psline[linewidth=1.6pt,linecolor=blue](10,6.0)(8.7,6.0)
\psline[linewidth=1.6pt,linecolor=blue](2.75,6.0)(-1.500,6.0)
\psline[linewidth=1.6pt,linecolor=blue](10,5.5)(8.7,5.5)
\psline[linewidth=1.6pt,linecolor=blue](2.75,5.5)(-1.325,5.5)
\psline[linewidth=1.6pt,linecolor=blue](10,5.0)(8.7,5.0)
\psline[linewidth=1.6pt,linecolor=blue](2.75,5.0)(-1.250,5.0)
\psline[linewidth=1.6pt,linecolor=blue](10,4.5)(8.7,4.5)
\psline[linewidth=1.6pt,linecolor=blue](2.75,4.5)(-1.125,4.5)
\psline[linewidth=1.6pt,linecolor=blue](10,4.0)(-1.000,4.0)
\psline[linewidth=1.6pt,linecolor=blue](10,3.5)(-0.875,3.5)
\psline[linewidth=1.6pt,linecolor=blue](10,3.0)(-0.750,3.0)
\psline[linewidth=1.6pt,linecolor=blue](10,2.5)(-0.625,2.5)
\psline[linewidth=1.6pt,linecolor=blue](10,2.0)(-0.500,2.0)
\psline[linewidth=1.6pt,linecolor=blue](10,1.5)(-0.375,1.5)
\psline[linewidth=1.6pt,linecolor=blue](10,1.0)(-0.250,1.0)
\psline[linewidth=1.6pt,linecolor=blue](10,0.5)(-0.125,0.5)
\psline[linewidth=1.6pt,linecolor=blue](10,0.0)(0.000,0.0)
\psline[linewidth=1.6pt,linecolor=blue](10,-0.5)(2.0,-0.5)
\psline[linewidth=1.6pt,linecolor=blue](10,-1.0)(4.0,-1.0)
\psline[linewidth=1.6pt,linecolor=blue](10,-1.5)(6.0,-1.5)
\psline[linewidth=1.6pt,linecolor=blue](10,-2.0)(8.0,-2.0)
\rput[l](-18.0,4.0){\footnotesize $ (N-.1) \left\{ \begin{matrix} R_1 < 0 \\ \overline{\rho} R_1 < R_2 < \underline{\rho} R_1 \end{matrix} \right. $}
\rput[l](11.0,-4.0){\footnotesize $ (N+.1) \left\{ \begin{matrix} R_1 > 0 \\ \underline{\rho} R_1 < R_2 < \overline{\rho} R_1 \end{matrix} \right. $}
\rput[l](0.50,-12.0){\footnotesize $ (N-.2) \left\{ \begin{matrix} \overline{\rho} R_2 \leqslant R_1 \leqslant \underline{\rho} R_2 \\ R_2 < 0 \end{matrix} \right. $}
\rput[l](-7.50,12.5){\footnotesize $ (N+.2) \left\{ \begin{matrix} \overline{\rho} R_2 \leqslant R_1 \leqslant \underline{\rho} R_2 \\ R_2 > 0 \end{matrix} \right. $}
%\rput[l](-3.0,-9.0){\magenta (1.1.1)}
\rput[l](-9.0,2.0){\magenta (1.1.2)}
\rput[l](-9.0,3.0){\magenta (1.1.3)}
\rput[l](-9.0,4.0){\magenta (1.1.4)}
\rput[l](-4.3,9.0){\magenta (1.1.5)}
\rput[l](2.4,-9.0){\magenta (1.3.1)}
\rput[l](7.0,-4.0){\magenta (1.3.2)}
\rput[l](7.0,-3.0){\magenta (1.3.3)}
\rput[l](7.0,-2.0){\magenta (1.3.4)}
%\rput[l](1.0,9.0){\magenta (1.3.5)}
%\rput[l](-7.0,-1.0){\green (2.1.1)}
\rput[l](1.0,-8.0){\green (2.1.2)}
\rput[l](1.75,-7.0){\green (2.1.3)}
\rput[l](3.0,-8.0){\green (2.1.4)}
\rput[l](5.0,-2.25){\green (2.1.5)}
\rput[l](-7.0,2.25){\green (2.3.1)}
\rput[l](-5.0,8.0){\green (2.3.2)}
\rput[l](-3.5,7.0){\green (2.3.3)}
\rput[l](-3.0,8.0){\green (2.3.4)}
%\rput[l](5.0,1.0){\green (2.3.5)}
\rput[l](-5.0,-15.0){$ (1.1.2) + (1.1.3) + (1.1.4) = (N-.1) $}
\rput[l](-5.0,-16.0){$ (1.3.2) + (1.3.3) + (1.3.4) = (N+.1) $}
\rput[l](-5.0,-17.0){$ (1.3.1) = (N-.2) \qquad (1.1.1) = \emptyset $}
\rput[l](-5.0,-18.0){$ (1.1.5) = (N+.2) \qquad (1.3.5) = \emptyset $}
\rput[l](-5.0,-20.0){$ (2.3.1) = (N-.1) \qquad (2.1.1) = \emptyset $}
\rput[l](-5.0,-21.0){$ (2.1.5) = (N+.1) \qquad (2.3.5) = \emptyset $}
\rput[l](-5.0,-22.0){$ (2.1.2) + (2.1.3) + (2.1.4) = (N-.2) $}
\rput[l](-5.0,-23.0){$ (2.3.2) + (2.3.3) + (2.3.4) = (N+.2) $}
\end{pspicture}
\end{center}

%\vskip 16 pt

\newpage

\subsection*{A.2 \quad Figures of na\"ive investor's demand function}

\quad \
Note that the na\"ive investor's demand for asset $i$ depends upon prices of both risky assets and that the shape of the function graph varies with the sign of $ R_i $, so it is not easy to depict the graph of $ Z (p_1, p_2) $ in three-dimensional coordinate space. Instead, we use an approach of comparative static analysis. Since the formation of the demand function is symmetric for both assets, we present the demand function for only asset $1$ here. When studying $ Z_1 (p_1) $, how the demand for asset 1 changes with $ p_1 $, we fix the value of $ p_2 $ (and presume the sign of $ R_2 $).

Case 1: The price of asset 2 is given above the mean payoff, $ p_2 > \mu_2 $. According to different bounds of the correlation coefficient, we depict the demand function $ Z_1 (p_1) $.

For $ p_2 > \mu_2 $ and $ 0 < \underline{\rho} < \overline{\rho} $,
{\footnotesize \begin{eqnarray}
Z_{N 1}^* = \left\{ \begin{matrix}
\dfrac{\sigma_2^2 (\mu_1 - p_1) - \underline{\rho} \sigma_1 \sigma_2 (\mu_2 - p_2)}{\alpha \sigma_1^2 \sigma_2^2 (1 - \underline{\rho}^2)}, & \text{if} \qquad p_1 < \mu_1 - \underline{\rho} \dfrac{\sigma_1}{\sigma_2} (\mu_2 - p_2) \\
0, & \text{if} \qquad \mu_1 - \underline{\rho} \dfrac{\sigma_1}{\sigma_2} (\mu_2 - p_2) \leqslant p_1 \leqslant \mu_1 - \overline{\rho} \dfrac{\sigma_1}{\sigma_2} (\mu_2 - p_2) \\
\dfrac{\sigma_2^2 (\mu_1 - p_1) - \overline{\rho} \sigma_1 \sigma_2 (\mu_2 - p_2)}{\alpha \sigma_1^2 \sigma_2^2 (1 - \overline{\rho}^2)}, & \text{if} \qquad \mu_1 - \overline{\rho} \dfrac{\sigma_1}{\sigma_2} (\mu_2 - p_2) < p_1 < \mu_1 - \dfrac1{\overline{\rho}} \dfrac{\sigma_1}{\sigma_2} (\mu_2 - p_2) \\
\dfrac{\mu_1 - p_1}{\alpha \sigma_1^2}, & \text{if} \qquad \mu_1 - \dfrac1{\overline{\rho}} \dfrac{\sigma_1}{\sigma_2} (\mu_2 - p_2) \leqslant p_1 \leqslant \mu_1 - \dfrac1{\underline{\rho}} \dfrac{\sigma_1}{\sigma_2} (\mu_2 - p_2) \\
\dfrac{\sigma_2^2 (\mu_1 - p_1) - \underline{\rho} \sigma_1 \sigma_2 (\mu_2 - p_2)}{\alpha \sigma_1^2 \sigma_2^2 (1 - \underline{\rho}^2)}, & \text{if} \qquad \mu_1 - \dfrac1{\underline{\rho}} \dfrac{\sigma_1}{\sigma_2} (\mu_2 - p_2) < p_1.
\end{matrix} \right.
\end{eqnarray}}

For $ p_2 > \mu_2 $ and $ 0 = \underline{\rho} < \overline{\rho} $,
{\footnotesize \begin{eqnarray}
Z_{N 1}^* = \left\{ \begin{matrix}
\dfrac{\mu_1 - p_1}{\alpha \sigma_1^2}, & \text{if} \qquad p_1 < \mu_1 \\
0, & \text{if} \qquad \mu_1 \leqslant p_1 \leqslant \mu_1 - \overline{\rho} \dfrac{\sigma_1}{\sigma_2} (\mu_2 - p_2) \\
\dfrac{\sigma_2^2 (\mu_1 - p_1) - \overline{\rho} \sigma_1 \sigma_2 (\mu_2 - p_2)}{\alpha \sigma_1^2 \sigma_2^2 (1 - \overline{\rho}^2)}, & \text{if} \qquad \mu_1 - \overline{\rho} \dfrac{\sigma_1}{\sigma_2} (\mu_2 - p_2) < p_1 < \mu_1 - \dfrac1{\overline{\rho}} \dfrac{\sigma_1}{\sigma_2} (\mu_2 - p_2) \\
\dfrac{\mu_1 - p_1}{\alpha \sigma_1^2}, & \text{if} \qquad \mu_1 - \dfrac1{\overline{\rho}} \dfrac{\sigma_1}{\sigma_2} (\mu_2 - p_2) \leqslant p_1.
\end{matrix} \right.
\end{eqnarray}}

For $ p_2 > \mu_2 $ and $ \underline{\rho} < 0 < \overline{\rho} $,
{\footnotesize \begin{eqnarray}
Z_{N 1}^* = \left\{ \begin{matrix}
\dfrac{\mu_1 - p_1}{\alpha \sigma_1^2}, & \text{if} \qquad p_1 \leqslant \mu_1 - \dfrac1{\underline{\rho}} \dfrac{\sigma_1}{\sigma_2} (\mu_2 - p_2) \\
\dfrac{\sigma_2^2 (\mu_1 - p_1) - \underline{\rho} \sigma_1 \sigma_2 (\mu_2 - p_2)}{\alpha \sigma_1^2 \sigma_2^2 (1 - \underline{\rho}^2)}, & \text{if} \qquad \mu_1 - \dfrac1{\underline{\rho}} \dfrac{\sigma_1}{\sigma_2} (\mu_2 - p_2) < p_1 < \mu_1 - \underline{\rho} \dfrac{\sigma_1}{\sigma_2} (\mu_2 - p_2) \\
0, & \text{if} \qquad \mu_1 - \underline{\rho} \dfrac{\sigma_1}{\sigma_2} (\mu_2 - p_2) \leqslant p_1 \leqslant \mu_1 - \overline{\rho} \dfrac{\sigma_1}{\sigma_2} (\mu_2 - p_2) \\
\dfrac{\sigma_2^2 (\mu_1 - p_1) - \overline{\rho} \sigma_1 \sigma_2 (\mu_2 - p_2)}{\alpha \sigma_1^2 \sigma_2^2 (1 - \overline{\rho}^2)}, & \text{if} \qquad \mu_1 - \overline{\rho} \dfrac{\sigma_1}{\sigma_2} (\mu_2 - p_2) < p_1 < \mu_1 - \dfrac1{\overline{\rho}} \dfrac{\sigma_1}{\sigma_2} (\mu_2 - p_2) \\
\dfrac{\mu_1 - p_1}{\alpha \sigma_1^2}, & \text{if} \qquad \mu_1 - \dfrac1{\overline{\rho}} \dfrac{\sigma_1}{\sigma_2} (\mu_2 - p_2) \leqslant p_1.
\end{matrix} \right.
\end{eqnarray}}

For $ p_2 > \mu_2 $ and $ \underline{\rho} < \overline{\rho} = 0 $,
{\footnotesize \begin{eqnarray}
Z_{N 1}^* = \left\{ \begin{matrix}
\dfrac{\mu_1 - p_1}{\alpha \sigma_1^2}, & \text{if} \qquad p_1 \leqslant \mu_1 - \dfrac1{\underline{\rho}} \dfrac{\sigma_1}{\sigma_2} (\mu_2 - p_2) \\
\dfrac{\sigma_2^2 (\mu_1 - p_1) - \underline{\rho} \sigma_1 \sigma_2 (\mu_2 - p_2)}{\alpha \sigma_1^2 \sigma_2^2 (1 - \underline{\rho}^2)}, & \text{if} \qquad \mu_1 - \dfrac1{\underline{\rho}} \dfrac{\sigma_1}{\sigma_2} (\mu_2 - p_2) < p_1 < \mu_1 - \underline{\rho} \dfrac{\sigma_1}{\sigma_2} (\mu_2 - p_2) \\
0, & \text{if} \qquad \mu_1 - \underline{\rho} \dfrac{\sigma_1}{\sigma_2} (\mu_2 - p_2) \leqslant p_1 \leqslant \mu_1 \\
\dfrac{\mu_1 - p_1}{\alpha \sigma_1^2}, & \text{if} \qquad \qquad \mu_1 < p_1.
\end{matrix} \right.
\end{eqnarray}}

For $ p_2 > \mu_2 $ and $ \underline{\rho} < \overline{\rho} < 0 $,
{\footnotesize \begin{eqnarray}
Z_{N 1}^* = \left\{ \begin{matrix}
\dfrac{\sigma_2^2 (\mu_1 - p_1) - \overline{\rho} \sigma_1 \sigma_2 (\mu_2 - p_2)}{\alpha \sigma_1^2 \sigma_2^2 (1 - \overline{\rho}^2)}, & \text{if} \qquad p_1 < \mu_1 - \dfrac1{\overline{\rho}} \dfrac{\sigma_1}{\sigma_2} (\mu_2 - p_2) \\
\dfrac{\mu_1 - p_1}{\alpha \sigma_1^2}, & \text{if} \qquad \mu_1 - \dfrac1{\overline{\rho}} \dfrac{\sigma_1}{\sigma_2} (\mu_2 - p_2) \leqslant p_1 \leqslant \mu_1 - \dfrac1{\underline{\rho}} \dfrac{\sigma_1}{\sigma_2} (\mu_2 - p_2) \\
\dfrac{\sigma_2^2 (\mu_1 - p_1) - \underline{\rho} \sigma_1 \sigma_2 (\mu_2 - p_2)}{\alpha \sigma_1^2 \sigma_2^2 (1 - \underline{\rho}^2)}, & \text{if} \qquad \mu_1 - \dfrac1{\underline{\rho}} \dfrac{\sigma_1}{\sigma_2} (\mu_2 - p_2) < p_1 < \mu_1 - \underline{\rho} \dfrac{\sigma_1}{\sigma_2} (\mu_2 - p_2) \\
0, & \text{if} \qquad \mu_1 - \underline{\rho} \dfrac{\sigma_1}{\sigma_2} (\mu_2 - p_2) \leqslant p_1 \leqslant \mu_1 - \overline{\rho} \dfrac{\sigma_1}{\sigma_2} (\mu_2 - p_2) \\
\dfrac{\sigma_2^2 (\mu_1 - p_1) - \overline{\rho} \sigma_1 \sigma_2 (\mu_2 - p_2)}{\alpha \sigma_1^2 \sigma_2^2 (1 - \overline{\rho}^2)}, & \text{if} \qquad \qquad \mu_1 - \overline{\rho} \dfrac{\sigma_1}{\sigma_2} (\mu_2 - p_2) < p_1.
\end{matrix} \right.
\end{eqnarray}}

Expression (A.9) is a limit form of expressions (A.8) and (A.10) for $ \underline{\rho} = 0 $.
Expression (A.11) is a limit form of expressions (A.10) and (A.12) for $ \overline{\rho} = 0 $.
The the five settings: $ 0 < \underline{\rho} < \overline{\rho} $, $ 0 = \underline{\rho} < \overline{\rho} $, $ \underline{\rho} < 0 < \overline{\rho} $, $ \underline{\rho} < \overline{\rho} = 0 $, and $ \underline{\rho} < \overline{\rho} < 0 $ are shown in Figures A.2.1 - A.2.5, respectively. The panels show that the demand function $ Z_1 (p_1) $ is piecewise linear and monotonically decreasing in $ p_1 $.

Case 2: The price of asset 2 is given below the mean payoff, $ p_2 < \mu_2 $. According to different bounds of correlation coefficient, we depict the demand function $ Z_1 (p_1) $.

For $ p_2 < \mu_2 $ and $ 0 < \underline{\rho} < \overline{\rho} $,
{\footnotesize \begin{eqnarray}
Z_{N 1}^* = \left\{ \begin{matrix}
\dfrac{\sigma_2^2 (\mu_1 - p_1) - \underline{\rho} \sigma_1 \sigma_2 (\mu_2 - p_2)}{\alpha \sigma_1^2 \sigma_2^2 (1 - \underline{\rho}^2)}, & \text{if} \qquad p_1 < \mu_1 - \dfrac1{\underline{\rho}} \dfrac{\sigma_1}{\sigma_2} (\mu_2 - p_2) \\
\dfrac{\mu_1 - p_1}{\alpha \sigma_1^2}, & \text{if} \qquad \mu_1 - \dfrac1{\underline{\rho}} \dfrac{\sigma_1}{\sigma_2} (\mu_2 - p_2) \leqslant p_1 \leqslant \mu_1 - \dfrac1{\overline{\rho}} \dfrac{\sigma_1}{\sigma_2} (\mu_2 - p_2) \\
\dfrac{\sigma_2^2 (\mu_1 - p_1) - \overline{\rho} \sigma_1 \sigma_2 (\mu_2 - p_2)}{\alpha \sigma_1^2 \sigma_2^2 (1 - \overline{\rho}^2)}, & \text{if} \qquad \mu_1 - \dfrac1{\overline{\rho}} \dfrac{\sigma_1}{\sigma_2} (\mu_2 - p_2) < p_1 < \mu_1 - \overline{\rho} \dfrac{\sigma_1}{\sigma_2} (\mu_2 - p_2) \\
0, & \text{if} \qquad \mu_1 - \overline{\rho} \dfrac{\sigma_1}{\sigma_2} (\mu_2 - p_2) \leqslant p_1 \leqslant \mu_1 - \underline{\rho} \dfrac{\sigma_1}{\sigma_2} (\mu_2 - p_2) \\
\dfrac{\sigma_2^2 (\mu_1 - p_1) - \underline{\rho} \sigma_1 \sigma_2 (\mu_2 - p_2)}{\alpha \sigma_1^2 \sigma_2^2 (1 - \underline{\rho}^2)}, & \text{if} \qquad \mu_1 - \underline{\rho} \dfrac{\sigma_1}{\sigma_2} (\mu_2 - p_2) < p_1.
\end{matrix} \right.
\end{eqnarray}}

For $ p_2 < \mu_2 $ and $ 0 = \underline{\rho} < \overline{\rho} $,
{\footnotesize \begin{eqnarray}
Z_{N 1}^* = \left\{ \begin{matrix}
\dfrac{\mu_1 - p_1}{\alpha \sigma_1^2}, & \text{if} \qquad p_1 \leqslant \mu_1 - \dfrac1{\overline{\rho}} \dfrac{\sigma_1}{\sigma_2} (\mu_2 - p_2) \\
\dfrac{\sigma_2^2 (\mu_1 - p_1) - \overline{\rho} \sigma_1 \sigma_2 (\mu_2 - p_2)}{\alpha \sigma_1^2 \sigma_2^2 (1 - \overline{\rho}^2)}, & \text{if} \qquad \mu_1 - \dfrac1{\overline{\rho}} \dfrac{\sigma_1}{\sigma_2} (\mu_2 - p_2) < p_1 < \mu_1 - \overline{\rho} \dfrac{\sigma_1}{\sigma_2} (\mu_2 - p_2) \\
0, & \text{if} \qquad \mu_1 - \overline{\rho} \dfrac{\sigma_1}{\sigma_2} (\mu_2 - p_2) \leqslant p_1 \leqslant \mu_1 \\
\dfrac{\mu_1 - p_1}{\alpha \sigma_1^2}, & \text{if} \qquad \mu_1 < p_1.
\end{matrix} \right.
\end{eqnarray}}

For $ p_2 < \mu_2 $ and $ \underline{\rho} < 0 < \overline{\rho} $,
{\footnotesize \begin{eqnarray}
Z_{N 1}^* = \left\{ \begin{matrix}
\dfrac{\mu_1 - p_1}{\alpha \sigma_1^2}, & \text{if} \qquad p_1 \leqslant \mu_1 - \dfrac1{\overline{\rho}} \dfrac{\sigma_1}{\sigma_2} (\mu_2 - p_2) \\
\dfrac{\sigma_2^2 (\mu_1 - p_1) - \overline{\rho} \sigma_1 \sigma_2 (\mu_2 - p_2)}{\alpha \sigma_1^2 \sigma_2^2 (1 - \overline{\rho}^2)}, & \text{if} \qquad \mu_1 - \dfrac1{\overline{\rho}} \dfrac{\sigma_1}{\sigma_2} (\mu_2 - p_2) < p_1 < \mu_1 - \overline{\rho} \dfrac{\sigma_1}{\sigma_2} (\mu_2 - p_2) \\
0, & \text{if} \qquad \mu_1 - \overline{\rho} \dfrac{\sigma_1}{\sigma_2} (\mu_2 - p_2) \leqslant p_1 \leqslant \mu_1 - \underline{\rho} \dfrac{\sigma_1}{\sigma_2} (\mu_2 - p_2) \\
\dfrac{\sigma_2^2 (\mu_1 - p_1) - \underline{\rho} \sigma_1 \sigma_2 (\mu_2 - p_2)}{\alpha \sigma_1^2 \sigma_2^2 (1 - \underline{\rho}^2)}, & \text{if} \qquad \mu_1 - \underline{\rho} \dfrac{\sigma_1}{\sigma_2} (\mu_2 - p_2) < p_1 < \mu_1 - \dfrac1{\underline{\rho}} \dfrac{\sigma_1}{\sigma_2} (\mu_2 - p_2) \\
\dfrac{\mu_1 - p_1}{\alpha \sigma_1^2}, & \text{if} \qquad \mu_1 - \dfrac1{\underline{\rho}} \dfrac{\sigma_1}{\sigma_2} (\mu_2 - p_2) \leqslant p_1.
\end{matrix} \right.
\end{eqnarray}}

For $ p_2 < \mu_2 $ and $ \underline{\rho} < \overline{\rho} = 0 $,
{\footnotesize \begin{eqnarray}
Z_{N 1}^* = \left\{ \begin{matrix}
\dfrac{\mu_1 - p_1}{\alpha \sigma_1^2}, & \text{if} \qquad p_1 < \mu_1 \\
0, & \text{if} \qquad \mu_1 \leqslant p_1 \leqslant \mu_1 - \underline{\rho} \dfrac{\sigma_1}{\sigma_2} (\mu_2 - p_2) \\
\dfrac{\sigma_2^2 (\mu_1 - p_1) - \underline{\rho} \sigma_1 \sigma_2 (\mu_2 - p_2)}{\alpha \sigma_1^2 \sigma_2^2 (1 - \underline{\rho}^2)}, & \text{if} \qquad \mu_1 - \underline{\rho} \dfrac{\sigma_1}{\sigma_2} (\mu_2 - p_2) < p_1 < \mu_1 - \dfrac1{\underline{\rho}} \dfrac{\sigma_1}{\sigma_2} (\mu_2 - p_2) \\
\dfrac{\mu_1 - p_1}{\alpha \sigma_1^2}, & \text{if} \qquad \mu_1 - \dfrac1{\underline{\rho}} \dfrac{\sigma_1}{\sigma_2} (\mu_2 - p_2) \leqslant p_1.
\end{matrix} \right.
\end{eqnarray}}

For $ p_2 < \mu_2 $ and $ \underline{\rho} < \overline{\rho} < 0 $,
{\footnotesize \begin{eqnarray}
Z_{N 1}^* = \left\{ \begin{matrix}
\dfrac{\sigma_2^2 (\mu_1 - p_1) - \overline{\rho} \sigma_1 \sigma_2 (\mu_2 - p_2)}{\alpha \sigma_1^2 \sigma_2^2 (1 - \overline{\rho}^2)}, & \text{if} \qquad p_1 < \mu_1 - \overline{\rho} \dfrac{\sigma_1}{\sigma_2} (\mu_2 - p_2) \\
0, & \text{if} \qquad \mu_1 - \overline{\rho} \dfrac{\sigma_1}{\sigma_2} (\mu_2 - p_2) \leqslant p_1 \leqslant \mu_1 - \underline{\rho} \dfrac{\sigma_1}{\sigma_2} (\mu_2 - p_2) \\
\dfrac{\sigma_2^2 (\mu_1 - p_1) - \underline{\rho} \sigma_1 \sigma_2 (\mu_2 - p_2)}{\alpha \sigma_1^2 \sigma_2^2 (1 - \underline{\rho}^2)}, & \text{if} \qquad \mu_1 - \underline{\rho} \dfrac{\sigma_1}{\sigma_2} (\mu_2 - p_2) < p_1 < \mu_1 - \dfrac1{\underline{\rho}} \dfrac{\sigma_1}{\sigma_2} (\mu_2 - p_2) \\
\dfrac{\mu_1 - p_1}{\alpha \sigma_1^2}, & \text{if} \qquad \mu_1 - \dfrac1{\underline{\rho}} \dfrac{\sigma_1}{\sigma_2} (\mu_2 - p_2) \leqslant p_1 \leqslant \mu_1 - \dfrac1{\overline{\rho}} \dfrac{\sigma_1}{\sigma_2} (\mu_2 - p_2) \\
\dfrac{\sigma_2^2 (\mu_1 - p_1) - \overline{\rho} \sigma_1 \sigma_2 (\mu_2 - p_2)}{\alpha \sigma_1^2 \sigma_2^2 (1 - \overline{\rho}^2)}, & \text{if} \qquad \mu_1 - \dfrac1{\overline{\rho}} \dfrac{\sigma_1}{\sigma_2} (\mu_2 - p_2) < p_1.
\end{matrix} \right.
\end{eqnarray}}

Expression (A.14) is a limit form of expressions (A.13) and (A.15) for $ \underline{\rho} = 0 $.
Expression (A.16) is a limit form of expressions (A.15) and (A.17) for $ \overline{\rho} = 0 $.
The five settings: $ 0 < \underline{\rho} < \overline{\rho} $, $ 0 = \underline{\rho} < \overline{\rho} $, $ \underline{\rho} < 0 < \overline{\rho} $, $ \underline{\rho} < \overline{\rho} = 0 $, and $ \underline{\rho} < \overline{\rho} < 0 $ are shown in Figures A.2.6 - A.2.10, respectively. The panels show that the demand function $ Z_1 (p_1) $ is piecewise linear and monotonically decreasing in $ p_1 $.

\vskip 8 pt

Then we consider $ Z_1 (p_2) $, how the demand for asset 1 changes with $ p_2 $, and we fix the value of $ p_1 $ (and presume the sign of $ R_1 $).

Case 1: The price of asset 1 is given above the mean payoff, $ p_1 > \mu_1 $. According to different bounds of the correlation coefficient, we depict the demand function $ Z_1 (p_2) $.

For $ p_1 > \mu_1 $ and $ 0 < \underline{\rho} < \overline{\rho} $,
{\footnotesize \begin{eqnarray}
Z_{N 1}^* = \left\{ \begin{matrix}
\dfrac{\sigma_2^2 (\mu_1 - p_1) - \underline{\rho} \sigma_1 \sigma_2 (\mu_2 - p_2)}{\alpha \sigma_1^2 \sigma_2^2 (1 - \underline{\rho}^2)}, & \text{if} \qquad p_2 < \mu_2 - \underline{\rho} \dfrac{\sigma_2}{\sigma_1} (\mu_1 - p_1) \\
\dfrac{\mu_1 - p_1}{\alpha \sigma_1^2}, & \text{if} \qquad \mu_2 - \underline{\rho} \dfrac{\sigma_2}{\sigma_1} (\mu_1 - p_1) \leqslant p_2 \leqslant \mu_2 - \overline{\rho} \dfrac{\sigma_2}{\sigma_1} (\mu_1 - p_1) \\
\dfrac{\sigma_2^2 (\mu_1 - p_1) - \overline{\rho} \sigma_1 \sigma_2 (\mu_2 - p_2)}{\alpha \sigma_1^2 \sigma_2^2 (1 - \overline{\rho}^2)}, & \text{if} \qquad \mu_2 - \overline{\rho} \dfrac{\sigma_2}{\sigma_1} (\mu_1 - p_1) < p_2 < \mu_2 - \dfrac1{\overline{\rho}} \dfrac{\sigma_2}{\sigma_1} (\mu_1 - p_1) \\
0, & \text{if} \qquad \mu_2 - \dfrac1{\overline{\rho}} \dfrac{\sigma_2}{\sigma_1} (\mu_1 - p_1) \leqslant p_2 \leqslant \mu_2 - \dfrac1{\underline{\rho}} \dfrac{\sigma_2}{\sigma_1} (\mu_1 - p_1) \\
\dfrac{\sigma_2^2 (\mu_1 - p_1) - \underline{\rho} \sigma_1 \sigma_2 (\mu_2 - p_2)}{\alpha \sigma_1^2 \sigma_2^2 (1 - \underline{\rho}^2)}, & \text{if} \qquad \mu_2 - \dfrac1{\underline{\rho}} \dfrac{\sigma_2}{\sigma_1} (\mu_1 - p_1) < p_2.
\end{matrix} \right.
\end{eqnarray}}

For $ p_1 > \mu_1 $ and $ 0 = \underline{\rho} < \overline{\rho} $,
{\footnotesize \begin{eqnarray}
Z_{N 1}^* = \left\{ \begin{matrix}
\dfrac{\mu_1 - p_1}{\alpha \sigma_1^2}, & \text{if} \qquad p_2 \leqslant \mu_2 - \overline{\rho} \dfrac{\sigma_2}{\sigma_1} (\mu_1 - p_1) \\
\dfrac{\sigma_2^2 (\mu_1 - p_1) - \overline{\rho} \sigma_1 \sigma_2 (\mu_2 - p_2)}{\alpha \sigma_1^2 \sigma_2^2 (1 - \overline{\rho}^2)}, & \text{if} \qquad \mu_2 - \overline{\rho} \dfrac{\sigma_2}{\sigma_1} (\mu_1 - p_1) < p_2 < \mu_2 - \dfrac1{\overline{\rho}} \dfrac{\sigma_2}{\sigma_1} (\mu_1 - p_1) \\
0, & \text{if} \qquad \mu_2 - \dfrac1{\overline{\rho}} \dfrac{\sigma_2}{\sigma_1} (\mu_1 - p_1) \leqslant p_2.
\end{matrix} \right.
\end{eqnarray}}

For $ p_1 > \mu_1 $ and $ \underline{\rho} < 0 < \overline{\rho} $,
{\footnotesize \begin{eqnarray}
Z_{N 1}^* = \left\{ \begin{matrix}
0, & \text{if} \qquad p_2 \leqslant \mu_2 - \dfrac1{\underline{\rho}} \dfrac{\sigma_2}{\sigma_1} (\mu_1 - p_1) \\
\dfrac{\sigma_2^2 (\mu_1 - p_1) - \underline{\rho} \sigma_1 \sigma_2 (\mu_2 - p_2)}{\alpha \sigma_1^2 \sigma_2^2 (1 - \underline{\rho}^2)}, & \text{if} \qquad \mu_2 - \dfrac1{\underline{\rho}} \dfrac{\sigma_2}{\sigma_1} (\mu_1 - p_1) < p_2 < \mu_2 - \underline{\rho} \dfrac{\sigma_2}{\sigma_1} (\mu_1 - p_1) \\
\dfrac{\mu_1 - p_1}{\alpha \sigma_1^2}, & \text{if} \qquad \mu_2 - \underline{\rho} \dfrac{\sigma_2}{\sigma_1} (\mu_1 - p_1) \leqslant p_2 \leqslant \mu_2 - \overline{\rho} \dfrac{\sigma_2}{\sigma_1} (\mu_1 - p_1) \\
\dfrac{\sigma_2^2 (\mu_1 - p_1) - \overline{\rho} \sigma_1 \sigma_2 (\mu_2 - p_2)}{\alpha \sigma_1^2 \sigma_2^2 (1 - \overline{\rho}^2)}, & \text{if} \qquad \mu_2 - \overline{\rho} \dfrac{\sigma_2}{\sigma_1} (\mu_1 - p_1) < p_2 < \mu_2 - \dfrac1{\overline{\rho}} \dfrac{\sigma_2}{\sigma_1} (\mu_1 - p_1) \\
0, & \text{if} \qquad \mu_2 - \dfrac1{\overline{\rho}} \dfrac{\sigma_2}{\sigma_1} (\mu_1 - p_1) \leqslant p_2.
\end{matrix} \right.
\end{eqnarray}}

For $ p_1 > \mu_1 $ and $ \underline{\rho} < \overline{\rho} = 0 $,
{\footnotesize \begin{eqnarray}
Z_{N 1}^* = \left\{ \begin{matrix}
0, & \text{if} \qquad p_2 \leqslant \mu_2 - \dfrac1{\underline{\rho}} \dfrac{\sigma_2}{\sigma_1} (\mu_1 - p_1) \\
\dfrac{\sigma_2^2 (\mu_1 - p_1) - \underline{\rho} \sigma_1 \sigma_2 (\mu_2 - p_2)}{\alpha \sigma_1^2 \sigma_2^2 (1 - \underline{\rho}^2)}, & \text{if} \qquad \mu_2 - \dfrac1{\underline{\rho}} \dfrac{\sigma_2}{\sigma_1} (\mu_1 - p_1) < p_2 < \mu_2 - \underline{\rho} \dfrac{\sigma_2}{\sigma_1} (\mu_1 - p_1) \\
\dfrac{\mu_1 - p_1}{\alpha \sigma_1^2}, & \text{if} \qquad \mu_2 - \underline{\rho} \dfrac{\sigma_2}{\sigma_1} (\mu_1 - p_1) \leqslant p_2.
\end{matrix} \right.
\end{eqnarray}}

For $ p_1 > \mu_1 $ and $ \underline{\rho} < \overline{\rho} < 0 $,
{\footnotesize \begin{eqnarray}
Z_{N 1}^* = \left\{ \begin{matrix}
\dfrac{\sigma_2^2 (\mu_1 - p_1) - \overline{\rho} \sigma_1 \sigma_2 (\mu_2 - p_2)}{\alpha \sigma_1^2 \sigma_2^2 (1 - \overline{\rho}^2)}, & \text{if} \qquad p_2 < \mu_2 - \dfrac1{\overline{\rho}} \dfrac{\sigma_2}{\sigma_1} (\mu_1 - p_1) \\
0, & \text{if} \qquad \mu_2 - \dfrac1{\overline{\rho}} \dfrac{\sigma_2}{\sigma_1} (\mu_1 - p_1) \leqslant p_2 \leqslant \mu_2 - \dfrac1{\underline{\rho}} \dfrac{\sigma_2}{\sigma_1} (\mu_1 - p_1) \\
\dfrac{\sigma_2^2 (\mu_1 - p_1) - \underline{\rho} \sigma_1 \sigma_2 (\mu_2 - p_2)}{\alpha \sigma_1^2 \sigma_2^2 (1 - \underline{\rho}^2)}, & \text{if} \qquad \mu_2 - \dfrac1{\underline{\rho}} \dfrac{\sigma_2}{\sigma_1} (\mu_1 - p_1) < p_2 < \mu_2 - \underline{\rho} \dfrac{\sigma_2}{\sigma_1} (\mu_1 - p_1) \\
\dfrac{\mu_1 - p_1}{\alpha \sigma_1^2}, & \text{if} \qquad \mu_2 - \underline{\rho} \dfrac{\sigma_2}{\sigma_1} (\mu_1 - p_1) \leqslant p_2 \leqslant \mu_2 - \overline{\rho} \dfrac{\sigma_2}{\sigma_1} (\mu_1 - p_1) \\
\dfrac{\sigma_2^2 (\mu_1 - p_1) - \overline{\rho} \sigma_1 \sigma_2 (\mu_2 - p_2)}{\alpha \sigma_1^2 \sigma_2^2 (1 - \overline{\rho}^2)}, & \text{if} \qquad \mu_2 - \overline{\rho} \dfrac{\sigma_2}{\sigma_1} (\mu_1 - p_1) < p_2.
\end{matrix} \right.
\end{eqnarray}}

Expression (A.19) is a limit form of expressions (A.18) and (A.20) for $ \underline{\rho} = 0 $.
Expression (A.21) is a limit form of Expressions (A.20) and (A.22) for $ \overline{\rho} = 0 $.
Figures A2.11 - A2.15 show the five settings: $ 0 < \underline{\rho} < \overline{\rho} $, $ 0 = \underline{\rho} < \overline{\rho} $, $ \underline{\rho} < 0 < \overline{\rho} $, $ \underline{\rho} < \overline{\rho} = 0 $, and $ \underline{\rho} < \overline{\rho} < 0 $. The panels show that the demand function $ Z_1 (p_2) $ is piecewise linear, but its monotonicity depends upon the the extreme values of correlation coefficient. The na\"ive investor's demand function $ Z_1 (p_2) $ is monotonically increasing in $ p_2 $ for $ 0 \leqslant \underline{\rho} < \overline{\rho} $, and monotonically decreasing in $ p_2 $ for $ \underline{\rho} < \overline{\rho} \leqslant 0 $. However, it is not monotonic for $ \underline{\rho} < 0 < \overline{\rho} $. In fact, it is convex between $ \mu_2 - \dfrac1{\underline{\rho}} \dfrac{\sigma_2}{\sigma_1} (\mu_1 - p_1) $ and $ \mu_2 - \dfrac1{\overline{\rho}} \dfrac{\sigma_2}{\sigma_1} (\mu_1 - p_1) $ and equal to zero outside (the slope of the sophisticated investor's demand function depends on the value of $ \hat{\rho} $. If $ \hat{\rho} > 0 $, $ \hat{\rho} = 0 $, $ \hat{\rho} < 0 $, then the sophisticated investor's demand function is an increasing, constant, decreasing line with positive, zero, negative slope $ \hat{\rho} $, respectively).

Case 2: The price of asset 1 is given below the mean payoff, $ p_1 < \mu_1 $. According to different bounds of the correlation coefficient, we depict the demand function $ Z_1 (p_2) $.

For $ p_1 < \mu_1 $ and $ 0 < \underline{\rho} < \overline{\rho} $,
{\footnotesize \begin{eqnarray}
Z_{N 1}^* = \left\{ \begin{matrix}
\dfrac{\sigma_2^2 (\mu_1 - p_1) - \underline{\rho} \sigma_1 \sigma_2 (\mu_2 - p_2)}{\alpha \sigma_1^2 \sigma_2^2 (1 - \underline{\rho}^2)}, & \text{if} \qquad p_2 < \mu_2 - \dfrac1{\underline{\rho}} \dfrac{\sigma_2}{\sigma_1} (\mu_1 - p_1) \\
0, & \text{if} \qquad \mu_2 - \dfrac1{\underline{\rho}} \dfrac{\sigma_2}{\sigma_1} (\mu_1 - p_1) \leqslant p_2 \leqslant \mu_2 - \dfrac1{\overline{\rho}} \dfrac{\sigma_2}{\sigma_1} (\mu_1 - p_1) \\
\dfrac{\sigma_2^2 (\mu_1 - p_1) - \overline{\rho} \sigma_1 \sigma_2 (\mu_2 - p_2)}{\alpha \sigma_1^2 \sigma_2^2 (1 - \overline{\rho}^2)}, & \text{if} \qquad \mu_2 - \dfrac1{\overline{\rho}} \dfrac{\sigma_2}{\sigma_1} (\mu_1 - p_1) < p_2 < \mu_2 - \overline{\rho} \dfrac{\sigma_2}{\sigma_1} (\mu_1 - p_1) \\
\dfrac{\mu_1 - p_1}{\alpha \sigma_1^2}, & \text{if} \qquad \mu_2 - \overline{\rho} \dfrac{\sigma_2}{\sigma_1} (\mu_1 - p_1) \leqslant p_2 \leqslant \mu_2 - \underline{\rho} \dfrac{\sigma_2}{\sigma_1} (\mu_1 - p_1) \\
\dfrac{\sigma_2^2 (\mu_1 - p_1) - \underline{\rho} \sigma_1 \sigma_2 (\mu_2 - p_2)}{\alpha \sigma_1^2 \sigma_2^2 (1 - \underline{\rho}^2)}, & \text{if} \qquad \mu_2 - \underline{\rho} \dfrac{\sigma_2}{\sigma_1} (\mu_1 - p_1) < p_2.
\end{matrix} \right.
\end{eqnarray}}

For $ p_1 < \mu_1 $ and $ 0 = \underline{\rho} < \overline{\rho} $,
{\footnotesize \begin{eqnarray}
Z_{N 1}^* = \left\{ \begin{matrix}
0, & \text{if} \qquad p_2 \leqslant \mu_2 - \dfrac1{\overline{\rho}} \dfrac{\sigma_2}{\sigma_1} (\mu_1 - p_1) \\
\dfrac{\sigma_2^2 (\mu_1 - p_1) - \overline{\rho} \sigma_1 \sigma_2 (\mu_2 - p_2)}{\alpha \sigma_1^2 \sigma_2^2 (1 - \overline{\rho}^2)}, & \text{if} \qquad \mu_2 - \dfrac1{\overline{\rho}} \dfrac{\sigma_2}{\sigma_1} (\mu_1 - p_1) < p_2 < \mu_2 - \overline{\rho} \dfrac{\sigma_2}{\sigma_1} (\mu_1 - p_1) \\
\dfrac{\mu_1 - p_1}{\alpha \sigma_1^2}, & \text{if} \qquad \mu_2 - \overline{\rho} \dfrac{\sigma_2}{\sigma_1} (\mu_1 - p_1) \leqslant p_2.
\end{matrix} \right.
\end{eqnarray}}

For $ p_1 < \mu_1 $ and $ \underline{\rho} < 0 < \overline{\rho} $,
{\footnotesize \begin{eqnarray}
Z_{N 1}^* = \left\{ \begin{matrix}
0, & \text{if} \qquad p_2 \leqslant \mu_2 - \dfrac1{\overline{\rho}} \dfrac{\sigma_2}{\sigma_1} (\mu_1 - p_1) \\
\dfrac{\sigma_2^2 (\mu_1 - p_1) - \overline{\rho} \sigma_1 \sigma_2 (\mu_2 - p_2)}{\alpha \sigma_1^2 \sigma_2^2 (1 - \overline{\rho}^2)}, & \text{if} \qquad \mu_2 - \dfrac1{\overline{\rho}} \dfrac{\sigma_2}{\sigma_1} (\mu_1 - p_1) < p_2 < \mu_2 - \overline{\rho} \dfrac{\sigma_2}{\sigma_1} (\mu_1 - p_1) \\
\dfrac{\mu_1 - p_1}{\alpha \sigma_1^2}, & \text{if} \qquad \mu_2 - \overline{\rho} \dfrac{\sigma_2}{\sigma_1} (\mu_1 - p_1) \leqslant p_2 \leqslant \mu_2 - \underline{\rho} \dfrac{\sigma_2}{\sigma_1} (\mu_1 - p_1) \\
\dfrac{\sigma_2^2 (\mu_1 - p_1) - \underline{\rho} \sigma_1 \sigma_2 (\mu_2 - p_2)}{\alpha \sigma_1^2 \sigma_2^2 (1 - \underline{\rho}^2)}, & \text{if} \qquad \mu_2 - \underline{\rho} \dfrac{\sigma_2}{\sigma_1} (\mu_1 - p_1) < p_2 < \mu_2 - \dfrac1{\underline{\rho}} \dfrac{\sigma_2}{\sigma_1} (\mu_1 - p_1) \\
0, & \text{if} \qquad \mu_2 - \dfrac1{\underline{\rho}} \dfrac{\sigma_2}{\sigma_1} (\mu_1 - p_1) \leqslant p_2.
\end{matrix} \right.
\end{eqnarray}}

For $ p_1 < \mu_1 $ and $ \underline{\rho} < \overline{\rho} = 0 $,
{\footnotesize \begin{eqnarray}
Z_{N 1}^* = \left\{ \begin{matrix}
\dfrac{\mu_1 - p_1}{\alpha \sigma_1^2}, & \text{if} \qquad p_2 \leqslant \mu_2 - \underline{\rho} \dfrac{\sigma_2}{\sigma_1} (\mu_1 - p_1) \\
\dfrac{\sigma_2^2 (\mu_1 - p_1) - \underline{\rho} \sigma_1 \sigma_2 (\mu_2 - p_2)}{\alpha \sigma_1^2 \sigma_2^2 (1 - \underline{\rho}^2)}, & \text{if} \qquad \mu_2 - \underline{\rho} \dfrac{\sigma_2}{\sigma_1} (\mu_1 - p_1) < p_2 < \mu_2 - \dfrac1{\underline{\rho}} \dfrac{\sigma_2}{\sigma_1} (\mu_1 - p_1) \\
0, & \text{if} \qquad \mu_2 - \dfrac1{\underline{\rho}} \dfrac{\sigma_2}{\sigma_1} (\mu_1 - p_1) \leqslant p_2.
\end{matrix} \right.
\end{eqnarray}}

For $ p_1 < \mu_1 $ and $ \underline{\rho} < \overline{\rho} < 0 $,
{\footnotesize \begin{eqnarray}
Z_{N 1}^* = \left\{ \begin{matrix}
\dfrac{\sigma_2^2 (\mu_1 - p_1) - \overline{\rho} \sigma_1 \sigma_2 (\mu_2 - p_2)}{\alpha \sigma_1^2 \sigma_2^2 (1 - \overline{\rho}^2)}, & \text{if} \qquad p_2 < \mu_2 - \overline{\rho} \dfrac{\sigma_2}{\sigma_1} (\mu_1 - p_1) \\
\dfrac{\mu_1 - p_1}{\alpha \sigma_1^2}, & \text{if} \qquad \mu_2 - \overline{\rho} \dfrac{\sigma_2}{\sigma_1} (\mu_1 - p_1) \leqslant p_2 \leqslant \mu_2 - \underline{\rho} \dfrac{\sigma_2}{\sigma_1} (\mu_1 - p_1) \\
\dfrac{\sigma_2^2 (\mu_1 - p_1) - \underline{\rho} \sigma_1 \sigma_2 (\mu_2 - p_2)}{\alpha \sigma_1^2 \sigma_2^2 (1 - \underline{\rho}^2)}, & \text{if} \qquad \mu_2 - \underline{\rho} \dfrac{\sigma_2}{\sigma_1} (\mu_1 - p_1) < p_2 < \mu_2 - \dfrac1{\underline{\rho}} \dfrac{\sigma_2}{\sigma_1} (\mu_1 - p_1) \\
0, & \text{if} \qquad \mu_2 - \dfrac1{\underline{\rho}} \dfrac{\sigma_2}{\sigma_1} (\mu_1 - p_1) \leqslant p_2 \leqslant \mu_2 - \dfrac1{\overline{\rho}} \dfrac{\sigma_2}{\sigma_1} (\mu_1 - p_1) \\
\dfrac{\sigma_2^2 (\mu_1 - p_1) - \overline{\rho} \sigma_1 \sigma_2 (\mu_2 - p_2)}{\alpha \sigma_1^2 \sigma_2^2 (1 - \overline{\rho}^2)}, & \text{if} \qquad \mu_2 - \dfrac1{\overline{\rho}} \dfrac{\sigma_2}{\sigma_1} (\mu_1 - p_1) < p_2.
\end{matrix} \right.
\end{eqnarray}}

Expression (A.24) is a limit form of expressions (A.23) and (A.25) for $ \underline{\rho} = 0 $.
Expression (A.26) is a limit form of expressions (A.25) and (A.27) for $ \overline{\rho} = 0 $.
Figures A2.16 - A2.20 show the five settings: $ 0 < \underline{\rho} < \overline{\rho} $, $ 0 = \underline{\rho} < \overline{\rho} $, $ \underline{\rho} < 0 < \overline{\rho} $, $ \underline{\rho} < \overline{\rho} = 0 $, and $ \underline{\rho} < \overline{\rho} < 0 $. The panels show that the demand function $ Z_1 (p_2) $ is piecewise linear, but its monotonicity depends upon the the extreme values of correlation coefficient. The na\"ive investor's demand function $ Z_1 (p_2) $ is monotonically increasing in $ p_2 $ for $ 0 \leqslant \underline{\rho} < \overline{\rho} $, and monotonically decreasing in $ p_2 $ for $ \underline{\rho} < \overline{\rho} \leqslant 0 $. However, it is not monotonic for $ \underline{\rho} < 0 < \overline{\rho} $. In fact, it is concave $ \mu_2 - \dfrac1{\overline{\rho}} \dfrac{\sigma_2}{\sigma_1} (\mu_1 - p_1) $ and $ \mu_2 - \dfrac1{\underline{\rho}} \dfrac{\sigma_2}{\sigma_1} (\mu_1 - p_1) $ and equal to zero outside (the slope of the sophisticated investor's demand function depends on the value of $ \hat{\rho} $. If $ \hat{\rho} > 0 $, $ \hat{\rho} = 0 $, $ \hat{\rho} < 0 $, then the sophisticated investor's demand function is an increasing, constant, decreasing line with positive, zero, negative slope $ \hat{\rho} $, respectively).

\newpage

\centerline{\bf Figure A2.1 \quad Na\"ive tnvestor's demand function for given $ p_2 > \mu_2 $ and $ 0 < \underline{\rho} < \overline{\rho} $}

\begin{center}
\psset{xunit=2,yunit=0.5}
\begin{pspicture}(-2,-2.5)(6,1)
\put(-1,0){\vector(1,0){11}}
\put(0,-2){\vector(0,1){3}}
\rput(-0.08,-0.25){\scriptsize $O$}
\rput(5,-0.25){\scriptsize $ p_1 $}
\rput(-0.15,1.9){\scriptsize $ Z_{N 1}^* $}
\psline[linewidth=1.6pt,linecolor=red](5,-4.125)(4,-3)
\psline[linewidth=1.6pt,linecolor=magenta](4,-3)(3,-2)
\psline[linewidth=1.6pt,linecolor=yellow](3,-2)(1.5,0)
\psline[linewidth=1.6pt,linecolor=green](1.5,0)(1.33333333,0)
\psline[linewidth=1.6pt,linecolor=blue](1.33333333,0)(0,1.5)
\psline(4,-3)(4,0)
\psline(3,-2)(3,0)
%\psline(1.5,0)(1.5,0)
%\psline(1.33333333,0)(1.33333333,0)
\rput(0.8,-1.50){\tiny $ \mu_1 - \underline{\rho} \dfrac{\sigma_1}{\sigma_2} (\mu_2 - p_2) $}
\rput(1.75,1.50){\tiny $ \mu_1 - \overline{\rho} \dfrac{\sigma_1}{\sigma_2} (\mu_2 - p_2) $}
\rput(2.75,0.65){\tiny $ \mu_1 - \dfrac1{\overline{\rho}} \dfrac{\sigma_1}{\sigma_2} (\mu_2 - p_2) $}
\rput(4.1,0.65){\tiny $ \mu_1 - \dfrac1{\underline{\rho}} \dfrac{\sigma_1}{\sigma_2} (\mu_2 - p_2) $}
\put(2.5,-0.5){\vector(1,3){0.15}}
\put(3,0.5){\vector(0,-1){0.45}}
\rput(5.1,-4){\scriptsize $N$}
\psline[linewidth=1.6pt,linecolor=purple](4.72222222,-4)(0.17708333,1.5)
\rput(1.5,-2.50){\tiny $ \mu_1 - \hat{\rho} \dfrac{\sigma_1}{\sigma_2} (\mu_2 - p_2) $}
\psline(3.4,-2.4)(3.4,0)
\rput(3.4,1.50){\tiny $ \mu_1 - \dfrac1{\hat{\rho}} \dfrac{\sigma_1}{\sigma_2} (\mu_2 - p_2) $}
\put(2.8,-1.0){\vector(0,1){1.0}}
\put(6.8,0.5){\vector(0,-1){0.45}}
\rput(4.5,-4){\scriptsize $S$}
\end{pspicture}
\end{center}

\vskip 16 pt

\centerline{\bf Figure A2.2 \quad Na\"ive tnvestor's demand function for given $ p_2 > \mu_2 $ and $ 0 = \underline{\rho} < \overline{\rho} $}

\begin{center}
\psset{xunit=2,yunit=0.5}
\begin{pspicture}(-2,-2.5)(6,1)
\put(-1,0){\vector(1,0){11}}
\put(0,-2){\vector(0,1){3}}
\rput(-0.08,-0.25){\scriptsize $O$}
\rput(5,-0.25){\scriptsize $ p_1 $}
\rput(-0.15,1.9){\scriptsize $ Z_{N 1}^* $}
%\psline[linewidth=1.6pt,linecolor=red](5,-4)(4,-3)
\psline[linewidth=1.6pt,linecolor=magenta](5,-4)(3,-2)
\psline[linewidth=1.6pt,linecolor=yellow](3,-2)(1.5,0)
\psline[linewidth=1.6pt,linecolor=green](1.5,0)(1.33333333,0)
\psline[linewidth=1.6pt,linecolor=blue](1.33333333,0)(0,1.33333333)
%\psline[linewidth=1.6pt,linecolor=blue](1.33333333,0)(0,1.5)
%\psline(4,-3)(4,0)
\psline(3,-2)(3,0)
%\psline(1.5,0)(1.5,0)
%\psline(1.33333333,0)(1.33333333,0)
\rput(1.30,-0.25){\tiny $ \mu_1 $}
\rput(1.75,1.50){\tiny $ \mu_1 - \overline{\rho} \dfrac{\sigma_1}{\sigma_2} (\mu_2 - p_2) $}
\rput(2.65,0.65){\tiny $ \mu_1 - \dfrac1{\overline{\rho}} \dfrac{\sigma_1}{\sigma_2} (\mu_2 - p_2) $}
%\rput(4.1,0.65){\tiny $ \mu_1 - \dfrac1{\underline{\rho}} \dfrac{\sigma_1}{\sigma_2} (\mu_2 - p_2) $}
%\put(2.5,-0.5){\vector(1,3){0.15}}
\put(3,0.5){\vector(0,-1){0.45}}
\rput(5.1,-4){\scriptsize $N$}
\psline[linewidth=1.6pt,linecolor=purple](4.72222222,-4)(0.17708333,1.5)
\rput(1.5,-2.50){\tiny $ \mu_1 - \hat{\rho} \dfrac{\sigma_1}{\sigma_2} (\mu_2 - p_2) $}
\psline(3.4,-2.4)(3.4,0)
\rput(3.75,0.65){\tiny $ \mu_1 - \dfrac1{\hat{\rho}} \dfrac{\sigma_1}{\sigma_2} (\mu_2 - p_2) $}
\put(2.8,-1.0){\vector(0,1){1.0}}
%\put(6.8,0.5){\vector(0,-1){0.45}}
\rput(4.5,-4){\scriptsize $S$}
\end{pspicture}
\end{center}

\vskip 16 pt

\centerline{\bf Figure A2.3 \quad Na\"ive tnvestor's demand function for given $ p_2 > \mu_2 $ and $ \underline{\rho} < 0 < \overline{\rho} $}

\begin{center}
\psset{xunit=1.5,yunit=0.5}
\begin{pspicture}(-2,-3)(3,4)
\put(-4.5,0){\vector(1,0){11}}
\put(0,-1.5){\vector(0,1){3.5}}
\rput(-0.1,-0.25){\scriptsize $O$}
\rput(4.25,-0.25){\scriptsize $ p_1 $}
\rput(-0.15,4.0){\scriptsize $ Z_{N 1}^* $}
\psline[linewidth=1.6pt,linecolor=red](4,-3)(3,-2)
\psline[linewidth=1.6pt,linecolor=magenta](3,-2)(1.5,0)
\psline[linewidth=1.6pt,linecolor=yellow](1.5,0)(0.66666667,0)
\psline[linewidth=1.6pt,linecolor=green](0.66666667,0)(-2,3)
\psline[linewidth=1.6pt,linecolor=blue](-2,3)(-3,4)
\psline(3,-2)(3,0)
%\psline(1.5,0)(1.5,0)
%\psline(1.33333333,0)(1.33333333,0)
\psline(-2,3)(-2,0)
\rput(-2,-0.60){\tiny $ \mu_1 - \dfrac1{\underline{\rho}} \dfrac{\sigma_1}{\sigma_2} (\mu_2 - p_2) $}
\rput(0.66666667,-0.60){\tiny $ \mu_1 - \underline{\rho} \dfrac{\sigma_1}{\sigma_2} (\mu_2 - p_2) $}
\rput(2,0.60){\tiny $ \mu_1 - \overline{\rho} \dfrac{\sigma_1}{\sigma_2} (\mu_2 - p_2) $}
\rput(3.5,0.60){\tiny $ \mu_1 - \dfrac1{\overline{\rho}} \dfrac{\sigma_1}{\sigma_2} (\mu_2 - p_2) $}
\rput(-2.75,3.45){\scriptsize $N$}
\psline[linewidth=1.6pt,linecolor=purple](4.0625,-3)(-2.8888889,4)
\rput(1.16666667,2.35){\tiny $ \mu_1 - \hat{\rho} \dfrac{\sigma_1}{\sigma_2} (\mu_2 - p_2) $}
\put(1.66666667,0.8){\vector(0,-1){0.75}}
\rput(-2.5,3.9){\scriptsize $S$}
\end{pspicture}
\end{center}

%\vskip 16 pt

\centerline{\bf Figure A2.4 \quad Na\"ive tnvestor's demand function for given $ p_2 > \mu_2 $ and $ \underline{\rho} < \overline{\rho} = 0 $}

\begin{center}
\psset{xunit=2,yunit=0.5}
\begin{pspicture}(-2,-1)(1,4)
\put(-6,0){\vector(1,0){10.5}}
\put(0,-1.0){\vector(0,1){3.0}}
\rput(0.08,-0.25){\scriptsize $O$}
\rput(2.25,-0.25){\scriptsize $ p_1 $}
\rput(-0.15,4){\scriptsize $ Z_{N 1}^* $}
\psline[linewidth=1.6pt,linecolor=red](1.66666667,-1)(0.66666667,0)
\psline[linewidth=1.6pt,linecolor=magenta](0.66666667,0)(0.5,0)
\psline[linewidth=1.6pt,linecolor=yellow](0.5,0)(-1,2)
\psline[linewidth=1.6pt,linecolor=green](-1,2)(-3,4)
%\psline[linewidth=1.6pt,linecolor=blue](-2,3)(-3,4)
%\psline(0.66666667,0)(0.66666667,0)
%\psline(0.5,0)(0.5,0)
\psline(-1,2)(-1,0)
%\psline(-2,3)(-2,0)
%\rput(-2.25,-0.65){\tiny $ \mu_1 - \dfrac1{\overline{\rho}} \dfrac{\sigma_1}{\sigma_2} (\mu_2 - p_2) $}
\rput(-0.675,-0.65){\tiny $ \mu_1 - \dfrac1{\underline{\rho}} \dfrac{\sigma_1}{\sigma_2} (\mu_2 - p_2) $}
\rput(0.5,-1.50){\tiny $ \mu_1 - \underline{\rho} \dfrac{\sigma_1}{\sigma_2} (\mu_2 - p_2) $}
\rput(0.7,0.25){\tiny $ \mu_1 $}
\put(1,-0.5){\vector(0,1){0.4}}
%\put(1.875,0.5){\vector(-1,-1){0.5}}
\rput(-2.5,3.3){\scriptsize $N$}
\psline[linewidth=1.6pt,linecolor=purple](1.40972222,-1)(-2.72222222,4)
\rput(0.58333333,2.5){\tiny $ \mu_1 - \hat{\rho} \dfrac{\sigma_1}{\sigma_2} (\mu_2 - p_2) $}
\put(1.16666667,1.0){\vector(0,-1){0.95}}
\psline(-1.4,2.4)(-1.4,0)
\rput(-1.75,-0.65){\tiny $ \mu_1 - \dfrac1{\hat{\rho}} \dfrac{\sigma_1}{\sigma_2} (\mu_2 - p_2) $}
%\put(-2.8,-0.5){\vector(0,1){0.4}}
\rput(-2.5,4){\scriptsize $S$}
\end{pspicture}
\end{center}

\vskip 16 pt

\centerline{\bf Figure A2.5 \quad Na\"ive tnvestor's demand function for given $ p_2 > \mu_2 $ and $ \underline{\rho} < \overline{\rho} < 0 $}

\begin{center}
\psset{xunit=2,yunit=0.5}
\begin{pspicture}(-2,-1)(1,4)
\put(-6,0){\vector(1,0){10.5}}
\put(0,-1.0){\vector(0,1){3.0}}
\rput(0.08,-0.25){\scriptsize $O$}
\rput(2.25,-0.25){\scriptsize $ p_1 $}
\rput(-0.15,4){\scriptsize $ Z_{N 1}^* $}
\psline[linewidth=1.6pt,linecolor=red](1.66666667,-1.125)(0.66666667,0)
\psline[linewidth=1.6pt,linecolor=magenta](0.66666667,0)(0.5,0)
\psline[linewidth=1.6pt,linecolor=yellow](0.5,0)(-1,2)
\psline[linewidth=1.6pt,linecolor=green](-1,2)(-2,3)
\psline[linewidth=1.6pt,linecolor=blue](-2,3)(-3,4.125)
%\psline(0.66666667,0)(0.66666667,0)
%\psline(0.5,0)(0.5,0)
\psline(-1,2)(-1,0)
\psline(-2,3)(-2,0)
\rput(-2.25,-0.65){\tiny $ \mu_1 - \dfrac1{\overline{\rho}} \dfrac{\sigma_1}{\sigma_2} (\mu_2 - p_2) $}
\rput(-0.75,-0.65){\tiny $ \mu_1 - \dfrac1{\underline{\rho}} \dfrac{\sigma_1}{\sigma_2} (\mu_2 - p_2) $}
\rput(0.5,-1.50){\tiny $ \mu_1 - \underline{\rho} \dfrac{\sigma_1}{\sigma_2} (\mu_2 - p_2) $}
\rput(1.2,1.50){\tiny $ \mu_1 - \overline{\rho} \dfrac{\sigma_1}{\sigma_2} (\mu_2 - p_2) $}
\put(1,-0.5){\vector(0,1){0.4}}
\put(1.875,0.5){\vector(-1,-1){0.5}}
\rput(-2.5,3.3){\scriptsize $N$}
\psline[linewidth=1.6pt,linecolor=purple](1.40972222,-1)(-2.72222222,4)
\rput(0.58333333,2.5){\tiny $ \mu_1 - \hat{\rho} \dfrac{\sigma_1}{\sigma_2} (\mu_2 - p_2) $}
\put(1.16666667,1.0){\vector(0,-1){0.95}}
\psline(-1.4,2.4)(-1.4,0)
\rput(-1.4,-1.50){\tiny $ \mu_1 - \dfrac1{\hat{\rho}} \dfrac{\sigma_1}{\sigma_2} (\mu_2 - p_2) $}
\put(-2.8,-0.5){\vector(0,1){0.4}}
\rput(-2.5,4){\scriptsize $S$}
\end{pspicture}
\end{center}

\newpage

\centerline{\bf Figure A2.6 \quad Na\"ive tnvestor's demand function for given $ p_2 < \mu_2 $ and $ 0 < \underline{\rho} < \overline{\rho} $}

\begin{center}
\psset{xunit=2,yunit=0.5}
\begin{pspicture}(-2,-1)(1,4)
\put(-7,0){\vector(1,0){12}}
\put(0,-1){\vector(0,1){3}}
\rput(0.08,-0.25){\scriptsize $O$}
\rput(2.5,-0.25){\scriptsize $ p_1 $}
\rput(-0.15,4){\scriptsize $ Z_{N 1}^* $}
\psline[linewidth=1.6pt,linecolor=red](1.66666667,-1.125)(0.66666667,0)
\psline[linewidth=1.6pt,linecolor=magenta](0.66666667,0)(0.5,0)
\psline[linewidth=1.6pt,linecolor=yellow](0.5,0)(-1,2)
\psline[linewidth=1.6pt,linecolor=green](-1,2)(-2,3)
\psline[linewidth=1.6pt,linecolor=blue](-2,3)(-3,4.125)
\psline(-2,3)(-2,0)
\psline(-1,2)(-1,0)
%\psline(0.5,0)(0.5,0)
%\psline(0.66666667,0)(0.66666667,0)
\rput(-2.25,-0.65){\tiny $ \mu_1 - \dfrac1{\underline{\rho}} \dfrac{\sigma_1}{\sigma_2} (\mu_2 - p_2) $}
\rput(-0.75,-0.65){\tiny $ \mu_1 - \dfrac1{\overline{\rho}} \dfrac{\sigma_1}{\sigma_2} (\mu_2 - p_2) $}
\rput(0.5,-1.5){\tiny $ \mu_1 - \overline{\rho} \dfrac{\sigma_1}{\sigma_2} (\mu_2 - p_2) $}
\rput(1.25,1.5){\tiny $ \mu_1 - \underline{\rho} \dfrac{\sigma_1}{\sigma_2} (\mu_2 - p_2) $}
\put(1,-0.5){\vector(0,1){0.4}}
\put(1.875,0.50){\vector(-1,-1){0.50}}
\rput(-3.1,4.2){\scriptsize $N$}
\psline[linewidth=1.6pt,linecolor=purple](1.40972222,-1)(-2.72222222,4)
\rput(0.58333333,2.5){\tiny $ \mu_1 - \hat{\rho} \dfrac{\sigma_1}{\sigma_2} (\mu_2 - p_2) $}
\put(1.15,1.0){\vector(0,-1){0.95}}
\psline(-1.4,2.4)(-1.4,0)
\rput(-1.4,-1.5){\tiny $ \mu_1 - \dfrac1{\hat{\rho}} \dfrac{\sigma_1}{\sigma_2} (\mu_2 - p_2) $}
\put(-2.8,-0.5){\vector(0,1){0.4}}
\rput(-2.75,4.2){\scriptsize $S$}
\end{pspicture}
\end{center}

\vskip 8 pt

\centerline{\bf Figure A2.7 \quad Na\"ive tnvestor's demand function for given $ p_2 < \mu_2 $ and $ 0 = \underline{\rho} < \overline{\rho} $}

\begin{center}
\psset{xunit=2,yunit=0.5}
\begin{pspicture}(-2,-1)(1,4)
\put(-7,0){\vector(1,0){12}}
\put(0,-1){\vector(0,1){3}}
\rput(0.08,-0.25){\scriptsize $O$}
\rput(2.5,-0.25){\scriptsize $ p_1 $}
\rput(-0.15,4){\scriptsize $ Z_{N 1}^* $}
\psline[linewidth=1.6pt,linecolor=red](1.66666667,-1)(0.66666667,0)
\psline[linewidth=1.6pt,linecolor=magenta](0.66666667,0)(0.5,0)
\psline[linewidth=1.6pt,linecolor=yellow](0.5,0)(-1,2)
\psline[linewidth=1.6pt,linecolor=green](-1,2)(-3,4)
%\psline[linewidth=1.6pt,linecolor=blue](-2,3)(-3,4.125)
%\psline(-2,3)(-2,0)
\psline(-1,2)(-1,0)
%\psline(0.5,0)(0.5,0)
%\psline(0.66666667,0)(0.66666667,0)
%\rput(-2.25,-0.65){\tiny $ \mu_1 - \dfrac1{\underline{\rho}} \dfrac{\sigma_1}{\sigma_2} (\mu_2 - p_2) $}
\rput(-0.675,-0.65){\tiny $ \mu_1 - \dfrac1{\overline{\rho}} \dfrac{\sigma_1}{\sigma_2} (\mu_2 - p_2) $}
\rput(0.5,-1.5){\tiny $ \mu_1 - \overline{\rho} \dfrac{\sigma_1}{\sigma_2} (\mu_2 - p_2) $}
\rput(0.725,0.25){\scriptsize $ \mu_1 $}
\put(1,-0.5){\vector(0,1){0.4}}
%\put(1.875,0.50){\vector(-1,-1){0.50}}
\rput(-3.1,4.2){\scriptsize $N$}
\psline[linewidth=1.6pt,linecolor=purple](1.40972222,-1)(-2.72222222,4)
\rput(0.58333333,2.5){\tiny $ \mu_1 - \hat{\rho} \dfrac{\sigma_1}{\sigma_2} (\mu_2 - p_2) $}
\put(1.15,1.0){\vector(0,-1){0.95}}
\psline(-1.4,2.4)(-1.4,0)
\rput(-1.75,-0.65){\tiny $ \mu_1 - \dfrac1{\hat{\rho}} \dfrac{\sigma_1}{\sigma_2} (\mu_2 - p_2) $}
%\put(-2.8,-0.5){\vector(0,1){0.4}}
\rput(-2.75,4.2){\scriptsize $S$}
\end{pspicture}
\end{center}

\vskip 8 pt

\centerline{\bf Figure A2.8 \quad Na\"ive tnvestor's demand function for given $ p_2 < \mu_2 $ and $ \underline{\rho} < 0 < \overline{\rho} $}

\begin{center}
\psset{xunit=1.5,yunit=0.5}
\begin{pspicture}(-3,-3)(3,4)
\put(-6,0){\vector(1,0){12}}
\put(0,-1.5){\vector(0,1){3.5}}
\rput(-0.1,-0.25){\scriptsize $O$}
\rput(4,-0.25){\scriptsize $ p_1 $}
\rput(-0.15,4.5){\scriptsize $ Z_{N 1}^* $}
\psline[linewidth=1.6pt,linecolor=red](4,-3)(3,-2)
\psline[linewidth=1.6pt,linecolor=magenta](3,-2)(1.5,0)
\psline[linewidth=1.6pt,linecolor=yellow](1.5,0)(0.66666667,0)
\psline[linewidth=1.6pt,linecolor=green](0.66666667,0)(-2,3)
\psline[linewidth=1.6pt,linecolor=blue](-2,3)(-3,4)
\psline(-2,3)(-2,0)
%\psline(1.33333333,0)(1.33333333,0)
%\psline(1.5,0)(1.5,0)
\psline(3,-2)(3,0)
\rput(-2,-0.60){\tiny $ \mu_1 - \dfrac1{\overline{\rho}} \dfrac{\sigma_1}{\sigma_2} (\mu_2 - p_2) $}
\rput(0.66666667,-0.60){\tiny $ \mu_1 - \overline{\rho} \dfrac{\sigma_1}{\sigma_2} (\mu_2 - p_2) $}
\rput(1.75,0.60){\tiny $ \mu_1 - \underline{\rho} \dfrac{\sigma_1}{\sigma_2} (\mu_2 - p_2) $}
\rput(3.25,0.60){\tiny $ \mu_1 - \dfrac1{\underline{\rho}} \dfrac{\sigma_1}{\sigma_2} (\mu_2 - p_2) $}
\rput(-3.1,4){\scriptsize $N$}
\psline[linewidth=1.6pt,linecolor=purple](4.0625,-3)(-2.8888889,4)
\rput(1.16666667,2){\tiny $ \mu_1 - \hat{\rho} \dfrac{\sigma_1}{\sigma_2} (\mu_2 - p_2) $}
\put(1.6,0.7){\vector(0,-1){0.65}}
\rput(-2.75,4.1){\scriptsize $S$}
\end{pspicture}
\end{center}

%\vskip 16 pt

\centerline{\bf Figure A2.9 \quad Na\"ive tnvestor's demand function for given $ p_2 < \mu_2 $ and $ \underline{\rho} < \overline{\rho} = 0 $}

\begin{center}
\psset{xunit=2,yunit=0.5}
\begin{pspicture}(0,-4)(5,2)
\put(-1,0){\vector(1,0){12}}
\put(0,-2){\vector(0,1){3}}
\rput(-0.1,-0.2){\scriptsize $O$}
\rput(5.5,-0.25){\scriptsize $ p_1 $}
\rput(-0.15,1.9){\scriptsize $ Z_{N 1}^* $}
%\psline[linewidth=1.6pt,linecolor=red](5,-4.125)(4,-3)
\psline[linewidth=1.6pt,linecolor=magenta](5,-4)(3,-2)
\psline[linewidth=1.6pt,linecolor=yellow](3,-2)(1.5,0)
\psline[linewidth=1.6pt,linecolor=green](1.5,0)(1.33333333,0)
\psline[linewidth=1.6pt,linecolor=blue](1.33333333,0)(0,1.33333333)
%\psline(1.33333333,0)(1.33333333,0)
%\psline(1.5,0)(1.5,0)
\psline(3,-2)(3,0)
%\psline(4,-3)(4,0)
\rput(1.33333333,-0.25){\tiny $ \mu_1 $}
\rput(1.55,1.5){\tiny $ \mu_1 - \underline{\rho} \dfrac{\sigma_1}{\sigma_2} (\mu_2 - p_2) $}
\rput(2.75,0.65){\tiny $ \mu_1 - \dfrac1{\underline{\rho}} \dfrac{\sigma_1}{\sigma_2} (\mu_2 - p_2) $}
%\rput(4.25,0.65){\tiny $ \mu_1 - \dfrac1{\overline{\rho}} \dfrac{\sigma_1}{\sigma_2} (\mu_2 - p_2) $}
%\put(2.15,-0.5){\vector(1,1){0.5}}
\put(3.05,0.5){\vector(0,-1){0.45}}
\rput(5.1,-4){\scriptsize $N$}
\psline[linewidth=1.6pt,linecolor=purple](4.72222222,-4)(0.17708333,1.5)
\rput(1.41666667,-2.5){\tiny $ \mu_1 - \hat{\rho} \dfrac{\sigma_1}{\sigma_2} (\mu_2 - p_2) $}
\put(2.83333333,-1){\vector(0,1){0.95}}
\psline(3.4,-2.4)(3.4,0)
\rput(3.4,1.5){\tiny $ \mu_1 - \dfrac1{\hat{\rho}} \dfrac{\sigma_1}{\sigma_2} (\mu_2 - p_2) $}
\put(6.8,0.5){\vector(0,-1){0.45}}
\rput(4.5,-4){\scriptsize $S$}
\end{pspicture}
\end{center}

\vskip 8 pt

\centerline{\bf Figure A2.10 \quad Na\"ive tnvestor's demand function for given $ p_2 < \mu_2 $ and $ \underline{\rho} < \overline{\rho} < 0 $}

\begin{center}
\psset{xunit=2,yunit=0.5}
\begin{pspicture}(0,-4)(5,2)
\put(-1,0){\vector(1,0){12}}
\put(0,-2){\vector(0,1){3}}
\rput(-0.1,-0.2){\scriptsize $O$}
\rput(5.5,-0.25){\scriptsize $ p_1 $}
\rput(-0.15,1.9){\scriptsize $ Z_{N 1}^* $}
\psline[linewidth=1.6pt,linecolor=red](5,-4.125)(4,-3)
\psline[linewidth=1.6pt,linecolor=magenta](4,-3)(3,-2)
\psline[linewidth=1.6pt,linecolor=yellow](3,-2)(1.5,0)
\psline[linewidth=1.6pt,linecolor=green](1.5,0)(1.33333333,0)
\psline[linewidth=1.6pt,linecolor=blue](1.33333333,0)(0,1.5)
%\psline(1.33333333,0)(1.33333333,0)
%\psline(1.5,0)(1.5,0)
\psline(3,-2)(3,0)
\psline(4,-3)(4,0)
\rput(0.88333333,-1.5){\tiny $ \mu_1 - \overline{\rho} \dfrac{\sigma_1}{\sigma_2} (\mu_2 - p_2) $}
\rput(1.55,1.5){\tiny $ \mu_1 - \underline{\rho} \dfrac{\sigma_1}{\sigma_2} (\mu_2 - p_2) $}
\rput(2.75,0.65){\tiny $ \mu_1 - \dfrac1{\underline{\rho}} \dfrac{\sigma_1}{\sigma_2} (\mu_2 - p_2) $}
\rput(4.25,0.65){\tiny $ \mu_1 - \dfrac1{\overline{\rho}} \dfrac{\sigma_1}{\sigma_2} (\mu_2 - p_2) $}
\put(2.15,-0.5){\vector(1,1){0.5}}
\put(3.05,0.5){\vector(0,-1){0.45}}
\rput(5.1,-4){\scriptsize $N$}
\psline[linewidth=1.6pt,linecolor=purple](4.72222222,-4)(0.17708333,1.5)
\rput(1.41666667,-2.5){\tiny $ \mu_1 - \hat{\rho} \dfrac{\sigma_1}{\sigma_2} (\mu_2 - p_2) $}
\put(2.83333333,-1){\vector(0,1){0.95}}
\psline(3.4,-2.4)(3.4,0)
\rput(3.4,1.5){\tiny $ \mu_1 - \dfrac1{\hat{\rho}} \dfrac{\sigma_1}{\sigma_2} (\mu_2 - p_2) $}
\put(6.8,0.5){\vector(0,-1){0.45}}
\rput(4.5,-4){\scriptsize $S$}
\end{pspicture}
\end{center}

\newpage

\centerline{\bf Figure A2.11 \quad Na\"ive tnvestor's demand function for given $ p_1 > \mu_1 $ and $ 0 < \underline{\rho} < \overline{\rho} $}

\begin{center}
\psset{xunit=2,yunit=0.5}
\begin{pspicture}(-1,-2)(6,2.5)
\put(-1,0){\vector(1,0){12}}
\put(0,-1){\vector(0,1){2.5}}
\rput(-0.08,-0.25){\scriptsize $O$}
\rput(5.5,-0.25){\scriptsize $ p_2 $}
\rput(-0.15,2.9){\scriptsize $ Z_{N 1}^* $}
%\psline[linewidth=1.6pt,linecolor=red](5,4.125)(4,0)
\psline[linewidth=1.6pt,linecolor=red](4.6,2.475)(4,0)
\psline[linewidth=1.6pt,linecolor=magenta](4,0)(3,0)
\psline[linewidth=1.6pt,linecolor=yellow](3,0)(1.5,-1)
\psline[linewidth=1.6pt,linecolor=green](1.5,-1)(1.33333333,-1)
\psline[linewidth=1.6pt,linecolor=blue](1.33333333,-1)(0,-1.5)
\psline(1.33333333,-1)(1.33333333,0)
\psline(1.5,-1)(1.5,0)
%\psline(3,0)(3,0)
%\psline(4,0)(4,0)
\rput(0.88333333,0.65){\tiny $ \mu_2 - \underline{\rho} \dfrac{\sigma_2}{\sigma_1} (\mu_1 - p_1) $}
\rput(1.95,0.65){\tiny $ \mu_2 - \overline{\rho} \dfrac{\sigma_2}{\sigma_1} (\mu_1 - p_1) $}
\rput(3,0.65){\tiny $ \mu_2 - \dfrac1{\overline{\rho}} \dfrac{\sigma_2}{\sigma_1} (\mu_1 - p_1) $}
\rput(4.2,-0.65){\tiny $ \mu_2 - \dfrac1{\underline{\rho}} \dfrac{\sigma_2}{\sigma_1} (\mu_1 - p_1) $}
\rput(4.675,2.475){\scriptsize $N$}
\psline[linewidth=1.6pt,linecolor=purple](5.38333333,1)(0.425,-1.5)
\psline(1.41666667,-1)(1.41666667,0)
\rput(1.41666667,2.0){\tiny $ \mu_2 - \hat{\rho} \dfrac{\sigma_2}{\sigma_1} (\mu_1 - p_1) $}
\put(2.83333333,0.75){\vector(0,-1){0.7}}
\rput(3.4,-1.5){\tiny $ \mu_2 - \dfrac1{\hat{\rho}} \dfrac{\sigma_2}{\sigma_1} (\mu_1 - p_1) $}
\put(6.8,-0.5){\vector(0,1){0.5}}
\rput(5.33333333,0.8){\scriptsize $S$}
\end{pspicture}
\end{center}

\centerline{\bf Figure A2.12 \quad Na\"ive tnvestor's demand function for given $ p_1 > \mu_1 $ and $ 0 = \underline{\rho} < \overline{\rho} $}

\begin{center}
\psset{xunit=2,yunit=0.5}
\begin{pspicture}(-1,-2)(6,2.5)
\put(-1,0){\vector(1,0){12}}
\put(0,-1){\vector(0,1){2.5}}
\rput(-0.08,-0.25){\scriptsize $O$}
\rput(5.5,-0.25){\scriptsize $ p_2 $}
\rput(-0.15,2.9){\scriptsize $ Z_{N 1}^* $}
%\psline[linewidth=1.6pt,linecolor=red](5,4.125)(4,0)
%\psline[linewidth=1.6pt,linecolor=red](4.6,2.475)(4,0)
\psline[linewidth=1.6pt,linecolor=magenta](5,0)(3,0)
\psline[linewidth=1.6pt,linecolor=yellow](3,0)(1.5,-1)
\psline[linewidth=1.6pt,linecolor=green](1.5,-1)(0,-1)
%\psline[linewidth=1.6pt,linecolor=blue](1.33333333,-1)(0,-1.5)
%\psline(1.33333333,-1)(1.33333333,0)
\psline(1.5,-1)(1.5,0)
%\psline(3,0)(3,0)
%\psline(4,0)(4,0)
%\rput(0.88333333,0.65){\tiny $ \mu_2 - \underline{\rho} \dfrac{\sigma_2}{\sigma_1} (\mu_1 - p_1) $}
\rput(1.95,0.65){\tiny $ \mu_2 - \overline{\rho} \dfrac{\sigma_2}{\sigma_1} (\mu_1 - p_1) $}
\rput(3,0.65){\tiny $ \mu_2 - \dfrac1{\overline{\rho}} \dfrac{\sigma_2}{\sigma_1} (\mu_1 - p_1) $}
%\rput(4.2,-0.65){\tiny $ \mu_2 - \dfrac1{\underline{\rho}} \dfrac{\sigma_2}{\sigma_1} (\mu_1 - p_1) $}
\rput(4.875,-0.25){\scriptsize $N$}
\psline[linewidth=1.6pt,linecolor=purple](5.38333333,1)(0.425,-1.5)
\psline(1.41666667,-1)(1.41666667,0)
\rput(1.41666667,2.0){\tiny $ \mu_2 - \hat{\rho} \dfrac{\sigma_2}{\sigma_1} (\mu_1 - p_1) $}
\put(2.83333333,0.75){\vector(0,-1){0.7}}
\rput(3.4,-1.5){\tiny $ \mu_2 - \dfrac1{\hat{\rho}} \dfrac{\sigma_2}{\sigma_1} (\mu_1 - p_1) $}
\put(6.8,-0.5){\vector(0,1){0.5}}
\rput(5.33333333,1.25){\scriptsize $S$}
\end{pspicture}
\end{center}

\centerline{\bf Figure A2.13 \quad Na\"ive tnvestor's demand function for given $ p_1 > \mu_1 $ and $ \underline{\rho} < 0 < \overline{\rho} $}

\begin{center}
\psset{xunit=1.5,yunit=1.0}
\begin{pspicture}(-4,-2)(5,0.75)
\put(-5,0){\vector(1,0){11.75}}
\put(0,-1.75){\vector(0,1){2.75}}
\rput(-0.1,-0.125){\scriptsize $O$}
\rput(4.5,-0.125){\scriptsize $ p_2 $}
\rput(-0.15,0.875){\scriptsize $ Z_{N 1}^* $}
\psline[linewidth=1.6pt,linecolor=red](4,0)(3,0)
\psline[linewidth=1.6pt,linecolor=magenta](3,0)(1.5,-1)
\psline[linewidth=1.6pt,linecolor=yellow](1.5,-1)(0.66666667,-1)
\psline[linewidth=1.6pt,linecolor=green](0.66666667,-1)(-2,0)
\psline[linewidth=1.6pt,linecolor=blue](-2,0)(-3,0)
%\psline(-2,0)(-2,0)
\psline(0.66666667,-1)(0.66666667,0)
\psline(1.5,-1)(1.5,0)
%\psline(3,0)(3,0)
\rput(-2,0.35){\tiny $ \mu_2 - \dfrac1{\underline{\rho}} \dfrac{\sigma_2}{\sigma_1} (\mu_1 - p_1) $}
\rput(0.26666667,0.35){\tiny $ \mu_2 - \underline{\rho} \dfrac{\sigma_2}{\sigma_1} (\mu_1 - p_1) $}
\rput(2.05,0.875){\tiny $ \mu_2 - \overline{\rho} \dfrac{\sigma_2}{\sigma_1} (\mu_1 - p_1) $}
\rput(3,0.35){\tiny $ \mu_2 - \dfrac1{\overline{\rho}} \dfrac{\sigma_2}{\sigma_1} (\mu_1 - p_1) $}
\put(2.25,0.625){\vector(0,-1){0.5}}
\rput(-0.75,-0.6){\scriptsize $N$}
\psline[linewidth=1.6pt,linecolor=purple](-3,-1.71428571)(4,-0.51428571)
\psline(1.16666667,-1)(1.16666667,0)
\rput(0.66666667,0.875){\tiny $ \mu_2 - \hat{\rho} \dfrac{\sigma_2}{\sigma_1} (\mu_1 - p_1) $}
\put(1.75,0.625){\vector(0,-1){0.5}}
\rput(-0.75,-1.2){\scriptsize $S$}
\rput(3,-1.25){\scriptsize $ \hat{\rho} > 0 $}
\end{pspicture}
\end{center}

\begin{center}
\psset{xunit=1.5,yunit=1.0}
\begin{pspicture}(-3.5,-2)(6,0.75)
\put(-4,0){\vector(1,0){12}}
\put(0,-1.75){\vector(0,1){2.75}}
\rput(-0.1,-0.125){\scriptsize $O$}
\rput(5.25,-0.125){\scriptsize $ p_2 $}
\rput(-0.15,0.875){\scriptsize $ Z_{N 1}^* $}
\psline[linewidth=1.6pt,linecolor=red](5,0)(4,0)
\psline[linewidth=1.6pt,linecolor=magenta](4,0)(1.33333333,-1)
\psline[linewidth=1.6pt,linecolor=yellow](1.33333333,-1)(0.5,-1)
\psline[linewidth=1.6pt,linecolor=green](0.5,-1)(-1,0)
\psline[linewidth=1.6pt,linecolor=blue](-1,0)(-2,0)
%\psline(-1,0)(-1,0)
\psline(0.5,-1)(0.5,0)
\psline(1.33333333,-1)(1.33333333,0)
%\psline(4,0)(4,0)
\rput(-1.5,0.35){\tiny $ \mu_2 - \dfrac1{\underline{\rho}} \dfrac{\sigma_2}{\sigma_1} (\mu_1 - p_1) $}
\rput(0,0.35){\tiny $ \mu_2 - \underline{\rho} \dfrac{\sigma_2}{\sigma_1} (\mu_1 - p_1) $}
\rput(1.83333333,0.35){\tiny $ \mu_2 - \overline{\rho} \dfrac{\sigma_2}{\sigma_1} (\mu_1 - p_1) $}
\rput(4,0.35){\tiny $ \mu_2 - \dfrac1{\overline{\rho}} \dfrac{\sigma_2}{\sigma_1} (\mu_1 - p_1) $}
\rput(2.75,-0.6){\scriptsize $N$}
\psline[linewidth=1.6pt,linecolor=purple](5,-1.71428571)(-2,-0.51428571)
\psline(0.83333333,-1)(0.83333333,0)
\rput(0.83333333,0.875){\tiny $ \mu_2 - \hat{\rho} \dfrac{\sigma_2}{\sigma_1} (\mu_1 - p_1) $}
\put(1.23333333,0.625){\vector(0,-1){0.5}}
\rput(2.75,-1.2){\scriptsize $S$}
\rput(-1.5,-1.25){\scriptsize $ \hat{\rho} < 0 $}
\end{pspicture}
\end{center}

\centerline{\bf Figure A2.14 \quad Na\"ive tnvestor's demand function for given $ p_1 > \mu_1 $ and $ \underline{\rho} < \overline{\rho} = 0 $}

\begin{center}
\psset{xunit=2,yunit=1}
\begin{pspicture}(-3,-1.5)(2,1)
\put(-7,0){\vector(1,0){12}}
\put(0,-1.75){\vector(0,1){3}}
\rput(-0.1,-0.1){\scriptsize $O$}
\rput(2.45,-0.175){\scriptsize $ p_2 $}
\rput(-0.15,0.95){\scriptsize $ Z_{N 1}^* $}
%\psline[linewidth=1.6pt,linecolor=red](2,-1.5)(0.66666667,-1)
\psline[linewidth=1.6pt,linecolor=magenta](2,-1)(0.5,-1)
\psline[linewidth=1.6pt,linecolor=yellow](0.5,-1)(-1,0)
\psline[linewidth=1.6pt,linecolor=green](-1,0)(-3,0)
%\psline[linewidth=1.6pt,linecolor=blue](-2,0)(-3,0.375)
%\psline(-2,0)(-2,0)
%\psline(-1,0)(-1,0)
\psline(0.5,-1)(0.5,0)
%\psline(0.66666667,-1)(0.66666667,0)
%\rput(-2.25,-0.35){\tiny $ \mu_2 - \dfrac1{\overline{\rho}} \dfrac{\sigma_2}{\sigma_1} (\mu_1 - p_1) $}
\rput(-1.125,0.35){\tiny $ \mu_2 - \dfrac1{\underline{\rho}} \dfrac{\sigma_2}{\sigma_1} (\mu_1 - p_1) $}
\rput(0,0.35){\tiny $ \mu_2 - \underline{\rho} \dfrac{\sigma_2}{\sigma_1} (\mu_1 - p_1) $}
%\rput(1.15,0.35){\tiny $ \mu_2 - \overline{\rho} \dfrac{\sigma_2}{\sigma_1} (\mu_1 - p_1) $}
%\put(0.8,0.4){\vector(1,-2){0.2}}
%\put(1.53333333,0.4){\vector(-1,-2){0.2}}
\rput(-2.5,-0.175){\scriptsize $N$}
\psline[linewidth=1.6pt,linecolor=purple](2,-1.71428571)(-3,0.80672269)
\psline(0.58333333,-1)(0.58333333,0)
\rput(0.58333333,1.0){\tiny $ \mu_2 - \hat{\rho} \dfrac{\sigma_2}{\sigma_1} (\mu_1 - p_1) $}
\put(1.16666667,0.75){\vector(0,-1){0.75}}
\rput(-1.4,-1.0){\tiny $ \mu_2 - \dfrac1{\hat{\rho}} \dfrac{\sigma_2}{\sigma_1} (\mu_1 - p_1) $}
\put(-2.8,-0.75){\vector(0,1){0.75}}
\rput(-2.5,0.80672269){\scriptsize $S$}
\end{pspicture}
\end{center}

\centerline{\bf Figure A2.15 \quad Na\"ive tnvestor's demand function for given $ p_1 > \mu_1 $ and $ \underline{\rho} < \overline{\rho} < 0 $}

\begin{center}
\psset{xunit=2,yunit=1}
\begin{pspicture}(-3,-1.5)(2,1)
\put(-7,0){\vector(1,0){12}}
\put(0,-1.75){\vector(0,1){3}}
\rput(-0.1,-0.1){\scriptsize $O$}
\rput(2.45,-0.175){\scriptsize $ p_2 $}
\rput(-0.15,0.95){\scriptsize $ Z_{N 1}^* $}
\psline[linewidth=1.6pt,linecolor=red](2,-1.5)(0.66666667,-1)
\psline[linewidth=1.6pt,linecolor=magenta](0.66666667,-1)(0.5,-1)
\psline[linewidth=1.6pt,linecolor=yellow](0.5,-1)(-1,0)
\psline[linewidth=1.6pt,linecolor=green](-1,0)(-2,0)
\psline[linewidth=1.6pt,linecolor=blue](-2,0)(-3,0.375)
%\psline(-2,0)(-2,0)
%\psline(-1,0)(-1,0)
\psline(0.5,-1)(0.5,0)
\psline(0.66666667,-1)(0.66666667,0)
\rput(-2.25,-0.35){\tiny $ \mu_2 - \dfrac1{\overline{\rho}} \dfrac{\sigma_2}{\sigma_1} (\mu_1 - p_1) $}
\rput(-1.125,0.35){\tiny $ \mu_2 - \dfrac1{\underline{\rho}} \dfrac{\sigma_2}{\sigma_1} (\mu_1 - p_1) $}
\rput(0,0.35){\tiny $ \mu_2 - \underline{\rho} \dfrac{\sigma_2}{\sigma_1} (\mu_1 - p_1) $}
\rput(1.15,0.35){\tiny $ \mu_2 - \overline{\rho} \dfrac{\sigma_2}{\sigma_1} (\mu_1 - p_1) $}
%\put(0.8,0.4){\vector(1,-2){0.2}}
%\put(1.53333333,0.4){\vector(-1,-2){0.2}}
\rput(-3.1,0.375){\scriptsize $N$}
\psline[linewidth=1.6pt,linecolor=purple](2,-1.71428571)(-3,0.80672269)
\psline(0.58333333,-1)(0.58333333,0)
\rput(0.58333333,1.0){\tiny $ \mu_2 - \hat{\rho} \dfrac{\sigma_2}{\sigma_1} (\mu_1 - p_1) $}
\put(1.16666667,0.75){\vector(0,-1){0.75}}
\rput(-1.4,-1.0){\tiny $ \mu_2 - \dfrac1{\hat{\rho}} \dfrac{\sigma_2}{\sigma_1} (\mu_1 - p_1) $}
\put(-2.8,-0.75){\vector(0,1){0.75}}
\rput(-3.1,0.80672269){\scriptsize $S$}
\end{pspicture}
\end{center}

\newpage

\centerline{\bf Figure A2.16 \quad Na\"ive tnvestor's demand function for given $ p_1 < \mu_1 $ and $ 0 < \underline{\rho} < \overline{\rho} $}

\begin{center}
\psset{xunit=2,yunit=0.5}
\begin{pspicture}(-3,-1.5)(2,2.5)
\put(-7,0){\vector(1,0){12}}
\put(0,-1){\vector(0,1){2.5}}
\rput(0.1,0.125){\scriptsize $O$}
\rput(2.5,-0.125){\scriptsize $ p_2 $}
\rput(0.15,2.675){\scriptsize $ Z_{N 1}^* $}
\psline[linewidth=1.6pt,linecolor=red](2,1.5)(0.66666667,1)
\psline[linewidth=1.6pt,linecolor=magenta](0.66666667,1)(0.5,1)
\psline[linewidth=1.6pt,linecolor=yellow](0.5,1)(-1,0)
\psline[linewidth=1.6pt,linecolor=green](-1,0)(-2,0)
\psline[linewidth=1.6pt,linecolor=blue](-2,0)(-3,-0.375)
%\psline(-2,0)(-2,0)
%\psline(-1,0)(-1,0)
\psline(0.5,1)(0.5,0)
\psline(0.66666667,1)(0.66666667,0)
\rput(-2,0.65){\tiny $ \mu_2 - \dfrac1{\underline{\rho}} \dfrac{\sigma_2}{\sigma_1} (\mu_1 - p_1) $}
\rput(-1,-0.65){\tiny $ \mu_2 - \dfrac1{\overline{\rho}} \dfrac{\sigma_2}{\sigma_1} (\mu_1 - p_1) $}
\rput(0.05,-0.65){\tiny $ \mu_2 - \overline{\rho} \dfrac{\sigma_2}{\sigma_1} (\mu_1 - p_1) $}
\rput(1.16666667,-0.65){\tiny $ \mu_2 - \underline{\rho} \dfrac{\sigma_2}{\sigma_1} (\mu_1 - p_1) $}
\rput(-3.1,-0.375){\scriptsize $N$}
\psline[linewidth=1.6pt,linecolor=purple](2,1.71428571)(-3,-0.80672269)
\psline(0.58333333,1)(0.58333333,0)
\rput(0.58333333,-1.5){\tiny $ \mu_2 - \hat{\rho} \dfrac{\sigma_2}{\sigma_1} (\mu_1 - p_1) $}
\put(1.16666667,-0.625){\vector(0,1){0.6}}
\rput(-1.4,2.0){\tiny $ \mu_2 - \dfrac1{\hat{\rho}} \dfrac{\sigma_2}{\sigma_1} (\mu_1 - p_1) $}
\put(-2.8,0.6){\vector(0,-1){0.6}}
\rput(-3.1,-0.80672269){\scriptsize $S$}
\end{pspicture}
\end{center}

\centerline{\bf Figure A2.17 \quad Na\"ive tnvestor's demand function for given $ p_1 < \mu_1 $ and $ 0 = \underline{\rho} < \overline{\rho} $}

\begin{center}
\psset{xunit=2,yunit=0.5}
\begin{pspicture}(-3,-1.5)(2,2.5)
\put(-7,0){\vector(1,0){12}}
\put(0,-1){\vector(0,1){2.5}}
\rput(0.1,0.125){\scriptsize $O$}
\rput(2.5,-0.125){\scriptsize $ p_2 $}
\rput(0.15,2.675){\scriptsize $ Z_{N 1}^* $}
%\psline[linewidth=1.6pt,linecolor=red](2,1.5)(0.66666667,1)
\psline[linewidth=1.6pt,linecolor=magenta](2,1)(0.5,1)
\psline[linewidth=1.6pt,linecolor=yellow](0.5,1)(-1,0)
\psline[linewidth=1.6pt,linecolor=green](-1,0)(-3,0)
%\psline[linewidth=1.6pt,linecolor=blue](-2,0)(-3,-0.375)
%\psline(-2,0)(-2,0)
%\psline(-1,0)(-1,0)
\psline(0.5,1)(0.5,0)
%\psline(0.66666667,1)(0.66666667,0)
%\rput(-2,0.65){\tiny $ \mu_2 - \dfrac1{\underline{\rho}} \dfrac{\sigma_2}{\sigma_1} (\mu_1 - p_1) $}
\rput(-1,-0.65){\tiny $ \mu_2 - \dfrac1{\overline{\rho}} \dfrac{\sigma_2}{\sigma_1} (\mu_1 - p_1) $}
\rput(0.05,-0.65){\tiny $ \mu_2 - \overline{\rho} \dfrac{\sigma_2}{\sigma_1} (\mu_1 - p_1) $}
%\rput(1.16666667,-0.65){\tiny $ \mu_2 - \underline{\rho} \dfrac{\sigma_2}{\sigma_1} (\mu_1 - p_1) $}
\rput(-2.5,0.25){\scriptsize $N$}
\psline[linewidth=1.6pt,linecolor=purple](2,1.71428571)(-3,-0.80672269)
\psline(0.58333333,1)(0.58333333,0)
\rput(0.58333333,-1.5){\tiny $ \mu_2 - \hat{\rho} \dfrac{\sigma_2}{\sigma_1} (\mu_1 - p_1) $}
\put(1.16666667,-0.625){\vector(0,1){0.6}}
\rput(-1.75,0.65){\tiny $ \mu_2 - \dfrac1{\hat{\rho}} \dfrac{\sigma_2}{\sigma_1} (\mu_1 - p_1) $}
%\put(-2.8,0.6){\vector(0,-1){0.6}}
\rput(-2.5,-0.80672269){\scriptsize $S$}
\end{pspicture}
\end{center}

\centerline{\bf Figure A2.18 \quad Na\"ive tnvestor's demand function for given $ p_1 < \mu_1 $ and $ \underline{\rho} < 0 < \overline{\rho} $}

\begin{center}
\psset{xunit=1.5,yunit=1}
\begin{pspicture}(-2,-0.75)(5,2)
\put(-4,0){\vector(1,0){12}}
\put(0,-0.75){\vector(0,1){2.5}}
\rput(0.1,0.125){\scriptsize $O$}
\rput(5.25,-0.125){\scriptsize $ p_2 $}
\rput(-0.15,1.75){\scriptsize $ Z_{N 1}^* $}
\psline[linewidth=1.6pt,linecolor=red](5,0)(4,0)
\psline[linewidth=1.6pt,linecolor=magenta](4,0)(1.33333333,1)
\psline[linewidth=1.6pt,linecolor=yellow](1.33333333,1)(0.5,1)
\psline[linewidth=1.6pt,linecolor=green](0.5,1)(-1,0)
\psline[linewidth=1.6pt,linecolor=blue](-1,0)(-2,0)
%\psline(-1,0)(-1,0)
\psline(0.5,1)(0.5,0)
\psline(1.33333333,1)(1.33333333,0)
%\psline(4,0)(4,0)
\rput(-1.5,-0.35){\tiny $ \mu_2 - \dfrac1{\overline{\rho}} \dfrac{\sigma_2}{\sigma_1} (\mu_1 - p_1) $}
\rput(0,-0.35){\tiny $ \mu_2 - \overline{\rho} \dfrac{\sigma_2}{\sigma_1} (\mu_1 - p_1) $}
\rput(1.83333333,-0.35){\tiny $ \mu_2 - \underline{\rho} \dfrac{\sigma_2}{\sigma_1} (\mu_1 - p_1) $}
\rput(4,-0.35){\tiny $ \mu_2 - \dfrac1{\underline{\rho}} \dfrac{\sigma_2}{\sigma_1} (\mu_1 - p_1) $}
\rput(2.75,0.6){\scriptsize $N$}
\psline[linewidth=1.6pt,linecolor=purple](5,1.34265734)(-2,0.75524476)
\psline(0.916666667,1)(0.916666667,0)
\rput(0.91666667,-0.625){\tiny $ \mu_2 - \hat{\rho} \dfrac{\sigma_2}{\sigma_1} (\mu_1 - p_1) $}
\put(1.36,-0.5){\vector(0,1){0.4}}
\rput(2.75,1.0){\scriptsize $S$}
\rput(-1,1.25){\scriptsize $ \hat{\rho} > 0 $}
\end{pspicture}
\end{center}

\begin{center}
\psset{xunit=1.5,yunit=1}
\begin{pspicture}(-4,-0.75)(5,2)
\put(-5.5,0){\vector(1,0){12}}
\put(0,-0.75){\vector(0,1){2.5}}
\rput(0.1,0.125){\scriptsize $O$}
\rput(4.25,0.125){\scriptsize $ p_2 $}
\rput(0.18,1.75){\scriptsize $ Z_{N 1}^* $}
\psline[linewidth=1.6pt,linecolor=red](4,0)(3,0)
\psline[linewidth=1.6pt,linecolor=magenta](3,0)(1.5,1)
\psline[linewidth=1.6pt,linecolor=yellow](1.5,1)(0.66666667,1)
\psline[linewidth=1.6pt,linecolor=green](0.66666667,1)(-2,0)
\psline[linewidth=1.6pt,linecolor=blue](-2,0)(-3,0)
%\psline(-2,0)(-2,0)
\psline(0.66666667,1)(0.66666667,0)
\psline(1.5,1)(1.5,0)
%\psline(3,0)(3,0)
\rput(-2,-0.35){\tiny $ \mu_2 - \dfrac1{\overline{\rho}} \dfrac{\sigma_2}{\sigma_1} (\mu_1 - p_1) $}
\rput(0.26666667,-0.35){\tiny $ \mu_2 - \overline{\rho} \dfrac{\sigma_2}{\sigma_1} (\mu_1 - p_1) $}
\rput(2,-0.35){\tiny $ \mu_2 - \underline{\rho} \dfrac{\sigma_2}{\sigma_1} (\mu_1 - p_1) $}
\rput(3.5,-0.35){\tiny $ \mu_2 - \dfrac1{\underline{\rho}} \dfrac{\sigma_2}{\sigma_1} (\mu_1 - p_1) $}
\rput(-0.75,0.6){\tiny $N$}
\psline[linewidth=1.6pt,linecolor=purple](-3,1.34265734)(4,0.75524476)
\psline(1.08333333,1)(1.08333333,0)
\rput(1.08333333,-0.625){\tiny $ \mu_2 - \hat{\rho} \dfrac{\sigma_2}{\sigma_1} (\mu_1 - p_1) $}
\put(1.625,-0.5){\vector(0,1){0.4}}
\rput(-0.75,1.0){\scriptsize $S$}
\rput(4,1.25){\scriptsize $ \hat{\rho} < 0 $}
\end{pspicture}
\end{center}

%\vskip 16 pt

\centerline{\bf Figure A2.19 \quad Na\"ive tnvestor's demand function for given $ p_1 < \mu_1 $ and $ \underline{\rho} < \overline{\rho} = 0 $}

\begin{center}
\psset{xunit=2,yunit=1}
\begin{pspicture}(0,-1)(5,2)
\put(-1,0){\vector(1,0){12}}
\put(0,-1){\vector(0,1){3}}
\rput(-0.1,-0.1){\scriptsize $O$}
\rput(5.5,-0.125){\scriptsize $ p_2 $}
\rput(-0.15,1.85){\scriptsize $ Z_{N 1}^* $}
%\psline[linewidth=1.6pt,linecolor=red](5,-0.375)(4,0)
\psline[linewidth=1.6pt,linecolor=magenta](5,0)(3,0)
\psline[linewidth=1.6pt,linecolor=yellow](3,0)(1.5,1)
\psline[linewidth=1.6pt,linecolor=green](1.5,1)(0,1)
%\psline[linewidth=1.6pt,linecolor=blue](1.33333333,1)(0,1.5)
%\psline(1.33333333,1)(1.33333333,0)
\psline(1.5,1)(1.5,0)
%\psline(3,0)(3,0)
%\psline(4,0)(4,0)
%\rput(0.88333333,-0.35){\tiny $ \mu_2 - \overline{\rho} \dfrac{\sigma_2}{\sigma_1} (\mu_1 - p_1) $}
\rput(1.95,-0.35){\tiny $ \mu_2 - \underline{\rho} \dfrac{\sigma_2}{\sigma_1} (\mu_1 - p_1) $}
\rput(3,-0.35){\tiny $ \mu_2 - \dfrac1{\underline{\rho}} \dfrac{\sigma_2}{\sigma_1} (\mu_1 - p_1) $}
%\rput(4,0.35){\tiny $ \mu_2 - \dfrac1{\overline{\rho}} \dfrac{\sigma_2}{\sigma_1} (\mu_1 - p_1) $}
\rput(4.5,0.20){\scriptsize $N$}
\psline[linewidth=1.6pt,linecolor=purple](0,1.71428571)(5,-0.80672269)
\psline(1.41666667,1)(1.41666667,0)
\rput(1.41666667,-1.0){\tiny $ \mu_2 - \hat{\rho} \dfrac{\sigma_2}{\sigma_1} (\mu_1 - p_1) $}
\put(2.83333333,-0.75){\vector(0,1){0.75}}
\rput(3.4,1.0){\tiny $ \mu_2 - \dfrac1{\hat{\rho}} \dfrac{\sigma_2}{\sigma_1} (\mu_1 - p_1) $}
\put(6.8,0.75){\vector(0,-1){0.75}}
\rput(4.5,-0.80672269){\scriptsize $S$}
\end{pspicture}
\end{center}

\centerline{\bf Figure A2.20 \quad Na\"ive tnvestor's demand function for given $ p_1 < \mu_1 $ and $ \underline{\rho} < \overline{\rho} < 0 $}

\begin{center}
\psset{xunit=2,yunit=1}
\begin{pspicture}(0,-1)(5,2)
\put(-1,0){\vector(1,0){12}}
\put(0,-1){\vector(0,1){3}}
\rput(-0.1,-0.1){\scriptsize $O$}
\rput(5.5,-0.125){\scriptsize $ p_2 $}
\rput(-0.15,1.85){\scriptsize $ Z_{N 1}^* $}
\psline[linewidth=1.6pt,linecolor=red](5,-0.375)(4,0)
\psline[linewidth=1.6pt,linecolor=magenta](4,0)(3,0)
\psline[linewidth=1.6pt,linecolor=yellow](3,0)(1.5,1)
\psline[linewidth=1.6pt,linecolor=green](1.5,1)(1.33333333,1)
\psline[linewidth=1.6pt,linecolor=blue](1.33333333,1)(0,1.5)
\psline(1.33333333,1)(1.33333333,0)
\psline(1.5,1)(1.5,0)
%\psline(3,0)(3,0)
%\psline(4,0)(4,0)
\rput(0.88333333,-0.35){\tiny $ \mu_2 - \overline{\rho} \dfrac{\sigma_2}{\sigma_1} (\mu_1 - p_1) $}
\rput(1.95,-0.35){\tiny $ \mu_2 - \underline{\rho} \dfrac{\sigma_2}{\sigma_1} (\mu_1 - p_1) $}
\rput(3,-0.35){\tiny $ \mu_2 - \dfrac1{\underline{\rho}} \dfrac{\sigma_2}{\sigma_1} (\mu_1 - p_1) $}
\rput(4,0.35){\tiny $ \mu_2 - \dfrac1{\overline{\rho}} \dfrac{\sigma_2}{\sigma_1} (\mu_1 - p_1) $}
\rput(5.1,-0.375){\scriptsize $N$}
\psline[linewidth=1.6pt,linecolor=purple](0,1.71428571)(5,-0.80672269)
\psline(1.41666667,1)(1.41666667,0)
\rput(1.41666667,-1.0){\tiny $ \mu_2 - \hat{\rho} \dfrac{\sigma_2}{\sigma_1} (\mu_1 - p_1) $}
\put(2.83333333,-0.75){\vector(0,1){0.75}}
\rput(3.4,1.0){\tiny $ \mu_2 - \dfrac1{\hat{\rho}} \dfrac{\sigma_2}{\sigma_1} (\mu_1 - p_1) $}
\put(6.8,0.75){\vector(0,-1){0.75}}
\rput(5.1,-0.80672269){\scriptsize $S$}
\end{pspicture}
\end{center}

Easley and O'Hara (2009) claim that the sophisticated investor always holds a larger amount (in absolute value) of the risky asset than does the na\"ive investor because for any given parameters, these investors evaluate the tradeoff between mean and variance equivalently.\footnote{\baselineskip1.3em Both sophisticated and na\"ive investors avoid risk and require compensation in expected payoff in order to hold risk. However, the na\"ive investor also avoids ambiguity in the distribution of payoffs, and so as long as the set of possible means and variances is non-degenerate, she further reduces the size of this position in the risky asset.} But this claim does not hold in our expression (10) and Figures A2.1 - A2.10 and A2.11 - A2.20.\footnote{\baselineskip1.3em This is an interesting and counter intuitive feature that draws our attention when we observe the demand curves of the two types of investors.} Figures A2.1 - A2.20 show that the sophisticated investor's demand function intersects the na\"ive investor's demand function on certain intervals of prices. Specifically, Figures A2.1 - A2.10 report that the sophisticated investor's demand function intersects the na\"ive investor's demand function on intervals:
$$ \left[ \mu_1 - \overline{\rho} \dfrac{\sigma_1}{\sigma_2} (\mu_2 - p_2), \mu_1 - \underline{\rho} \dfrac{\sigma_1}{\sigma_2} (\mu_2 - p_2) \right] \ \text{and} \ \left[ \mu_1 - \dfrac1{\underline{\rho}} \dfrac{\sigma_1}{\sigma_2} (\mu_2 - p_2), \mu_1 - \dfrac1{\overline{\rho}} \dfrac{\sigma_1}{\sigma_2} (\mu_2 - p_2) \right] $$
for $ p_2 < \mu_2 $ and on intervals:
$$ \left[ \mu_1 - \underline{\rho} \dfrac{\sigma_1}{\sigma_2} (\mu_2 - p_2), \mu_1 - \overline{\rho} \dfrac{\sigma_1}{\sigma_2} (\mu_2 - p_2) \right] \ \text{and} \ \left[ \mu_1 - \dfrac1{\overline{\rho}} \dfrac{\sigma_1}{\sigma_2} (\mu_2 - p_2), \mu_1 - \dfrac1{\underline{\rho}} \dfrac{\sigma_1}{\sigma_2} (\mu_2 - p_2) \right] $$
for $ p_2 > \mu_2 $. Figures A2.11 - A2.20 illustrate that the sophisticated investor's demand function intersects the na\"ive investor's demand function on intervals:
$$ \left[ \mu_2 - \overline{\rho} \dfrac{\sigma_2}{\sigma_1} (\mu_1 - p_1), \mu_2 - \underline{\rho} \dfrac{\sigma_2}{\sigma_1} (\mu_1 - p_1) \right] \ \text{and} \ \left[ \mu_2 - \dfrac1{\underline{\rho}} \dfrac{\sigma_2}{\sigma_1} (\mu_1 - p_1), \mu_2 - \dfrac1{\overline{\rho}} \dfrac{\sigma_2}{\sigma_1} (\mu_1 - p_1) \right] $$
for $ p_1 < \mu_1 $ and on intervals:
$$ \left[ \mu_2 - \underline{\rho} \dfrac{\sigma_2}{\sigma_1} (\mu_1 - p_1), \mu_2 - \overline{\rho} \dfrac{\sigma_2}{\sigma_1} (\mu_1 - p_1) \right] \ \text{and} \ \left[ \mu_2 - \dfrac1{\overline{\rho}} \dfrac{\sigma_2}{\sigma_1} (\mu_1 - p_1), \mu_2 - \dfrac1{\underline{\rho}} \dfrac{\sigma_2}{\sigma_1} (\mu_1 - p_1) \right] $$
for $ p_1 > \mu_1 $. Therefore, it is possible that the sophisticated investor does not always hold a larger amount (in absolute value) of the risky asset than the na\"ive investor on the above intervals. This suggests that ambiguity-averse investors might not be as conservative as depicted in the literature. Sometimes the extreme values of ambiguity set on correlation coefficients help na\"ive investors to choose right investment strategies.

%\newpage

\subsection*{A.3 \quad Proof of Proposition 1}

\quad \
The sophisticated investor's demand function for risky assets can be rewritten as:
{\small \begin{eqnarray}
Z_S^* = \left( \begin{matrix} Z_{S 1}^* \\ Z_{S 2}^* \end{matrix} \right) = \left\{ \begin{matrix}
\dfrac1{\alpha (1 - {\hat \rho}^2)} \left( \begin{matrix} \dfrac{R_1 - {\hat \rho} R_2}{\sigma_1} \\ \dfrac{R_2 - {\hat \rho} R_1}{\sigma_2} \end{matrix} \right), & \text{if} \quad (S1.1) \left\{ \begin{matrix} R_1 < {\hat \rho} R_2 \\ R_2 > {\hat \rho} R_1 \end{matrix} \right. \quad \text{or} \quad (S1.2) \left\{ \begin{matrix} R_1 > {\hat \rho} R_2 \\ R_2 < {\hat \rho} R_1 \end{matrix} \right. \\
\dfrac1{\alpha} \left( \begin{matrix} 0 \\ \dfrac{R_2}{\sigma_2} \end{matrix} \right), & \text{if} \quad (S-.2) \left\{ \begin{matrix} R_1 = {\hat \rho} R_2 \\ R_2 < 0 \end{matrix} \right. \quad \text{or} \quad (S+.2) \left\{ \begin{matrix} R_1 = {\hat \rho} R_2 \\ R_2 > 0 \end{matrix} \right. \\
\dfrac1{\alpha} \left( \begin{matrix} \dfrac{R_1}{\sigma_1} \\ 0 \end{matrix} \right), & \text{if} \quad (S-.1) \left\{ \begin{matrix} R_1 < 0 \\ R_2 = {\hat \rho} R_1 \end{matrix} \right. \quad \text{or} \quad (S+.1) \left\{ \begin{matrix} R_1 > 0 \\ R_2 = {\hat \rho} R_1 \end{matrix} \right. \\
\dfrac1{\alpha (1 - {\hat \rho}^2)} \left( \begin{matrix} \dfrac{R_1 - {\hat \rho} R_2}{\sigma_1} \\ \dfrac{R_2 - {\hat \rho} R_1}{\sigma_2} \end{matrix} \right), & \text{if} \quad (S2.1) \left\{ \begin{matrix} R_1 < {\hat \rho} R_2 \\ R_2 < {\hat \rho} R_1 \end{matrix} \right. \quad \text{or} \quad (S2.2) \left\{ \begin{matrix} R_1 > {\hat \rho} R_2 \\ R_2 > {\hat \rho} R_1. \end{matrix} \right.
\end{matrix} \right.
\end{eqnarray}}
\noindent from equation (6), and the plane $ R_1 - O - R_2 $ is divided into eight zones in (A.28). In addition,
\begin{eqnarray*}
& Z_{S1}^* < 0 \quad \text{and} \quad Z_{S2}^* > 0 \quad \text{in} \quad (S1.1) & \\
& Z_{S1}^* > 0 \quad \text{and} \quad Z_{S2}^* < 0 \quad \text{in} \quad (S1.2) & \\
& Z_{S1}^* < 0 \quad \text{and} \quad Z_{S2}^* < 0 \quad \text{in} \quad (S2.1) & \\
& Z_{S1}^* > 0 \quad \text{and} \quad Z_{S2}^* > 0 \quad \text{in} \quad (S2.2) & 
\end{eqnarray*}
and
\begin{eqnarray*}
& Z_{S1}^* < 0 \quad \text{and} \quad Z_{S2}^* = 0 \quad \text{in} \quad (S-.1) & \\
& Z_{S1}^* > 0 \quad \text{and} \quad Z_{S2}^* = 0 \quad \text{in} \quad (S+.1) & \\
& Z_{S1}^* = 0 \quad \text{and} \quad Z_{S2}^* < 0 \quad \text{in} \quad (S-.2) & \\
& Z_{S1}^* = 0 \quad \text{and} \quad Z_{S2}^* > 0 \quad \text{in} \quad (S+.2). &
\end{eqnarray*}

Since
\begin{eqnarray*}
(N1.1) \subseteq (S1.1) & \qquad & (S-.1) \subseteq (N-.1) \\
(N1.2) \subseteq (S1.2) & \qquad & (S+.1) \subseteq (N+.1) \\
(N2.1) \subseteq (S2.1) & \qquad & (S-.2) \subseteq (N-.2) \\
(N2.2) \subseteq (S2.2) & \qquad & (S+.2) \subseteq (N+.2), 
\end{eqnarray*}
then
\begin{eqnarray*}
Z_{N1}^* < 0 \quad \text{and} \quad Z_{N2}^* > 0 \quad & & \quad Z_{S1}^* < 0 \quad \text{and} \quad Z_{S2}^* > 0 \quad \text{in} \quad (N1.1) \\
Z_{N1}^* > 0 \quad \text{and} \quad Z_{N2}^* < 0 \quad & & \quad Z_{S1}^* > 0 \quad \text{and} \quad Z_{S2}^* < 0 \quad \text{in} \quad (N1.2) \\
Z_{N1}^* < 0 \quad \text{and} \quad Z_{N2}^* < 0 \quad & & \quad Z_{S1}^* < 0 \quad \text{and} \quad Z_{S2}^* < 0 \quad \text{in} \quad (N2.1) \\
Z_{N1}^* > 0 \quad \text{and} \quad Z_{N2}^* > 0 \quad & & \quad Z_{S1}^* > 0 \quad \text{and} \quad Z_{S2}^* > 0 \quad \text{in} \quad (N2.2) 
\end{eqnarray*}
and hence $ Z_{S i}^* Z_{N i}^* > 0 $ for $ i = 1, 2 $ on $ (N1.1) \cup (N1.2) \cup (N2.1) \cup (N2.2) $.

As we know, 
$ Z_{S1}^* < 0 $ on $ [(S1.1) \cup (S-.1) \cup (S2.1)] $ and $ (N-.1) \subset [(S1.1) \cup (S-.1) \cup (S2.1)] $, then $ Z_{S1}^* < 0 $ on (N-.1); 
$ Z_{S1}^* > 0 $ on $ [(S1.2) \cup (S+.1) \cup (S2.2)] $ and $ (N+.1) \subset [(S1.2) \cup (S+.1) \cup (S2.2)] $, then $ Z_{S1}^* > 0 $ on (N+.1);
$ Z_{S2}^* < 0 $ on $ [(S2.1) \cup (S-.2) \cup (S1.2)] $ and $ (N-.2) \subset [(S2.1) \cup (S-.2) \cup (S1.2)] $, then $ Z_{S1}^* < 0 $ on (N-.2);
$ Z_{S2}^* > 0 $ on $ [(S1.1) \cup (S+.2) \cup (S2.2)] $ and $ (N+.2) \subset [(S1.1) \cup (S+.2) \cup (S2.2)] $, then $ Z_{S1}^* > 0 $ on (N+.2).
%{\footnotesize \begin{eqnarray*}
%Z_{S1}^* < 0 \ \text{on} \ [(S1.1) \cup (S-.1) \cup (S2.1)] \ \text{and} \ (N-.1) \subset [(S1.1) \cup (S-.1) \cup (S2.1)], & \text{then} & Z_{S1}^* < 0 \ \text{on} \ (N-.1) \\
%Z_{S1}^* > 0 \ \text{on} \ [(S1.2) \cup (S+.1) \cup (S2.2)] \ \text{and} \ (N+.1) \subset [(S1.2) \cup (S+.1) \cup (S2.2)], & \text{then} & Z_{S1}^* > 0 \ \text{on} \ (N+.1) \\
%Z_{S2}^* < 0 \ \text{on} \ [(S2.1) \cup (S-.2) \cup (S1.2)] \ \text{and} \ (N-.2) \subset [(S2.1) \cup (S-.2) \cup (S1.2)], & \text{then} & Z_{S1}^* < 0 \ \text{on} \ (N-.2) \\
%Z_{S2}^* > 0 \ \text{on} \ [(S1.1) \cup (S+.2) \cup (S2.2)] \ \text{and} \ (N+.2) \subset [(S1.1) \cup (S+.2) \cup (S2.2)], & \text{then} & Z_{S1}^* > 0 \ \text{on} \ (N+.2), 
%\end{eqnarray*}}
Therefore
\begin{eqnarray*}
Z_{N1}^* < 0 \quad \text{and} \quad Z_{N2}^* = 0 \quad & & \quad Z_{S1}^* < 0 \quad \text{in} \quad (N-.1) \\
Z_{N1}^* > 0 \quad \text{and} \quad Z_{N2}^* = 0 \quad & & \quad Z_{S1}^* > 0 \quad \text{in} \quad (N+.1) \\
Z_{N1}^* = 0 \quad \text{and} \quad Z_{N2}^* < 0 \quad & & \quad Z_{S2}^* < 0 \quad \text{in} \quad (N-.2) \\
Z_{N1}^* = 0 \quad \text{and} \quad Z_{N2}^* > 0 \quad & & \quad Z_{S2}^* > 0 \quad \text{in} \quad (N+.2)
\end{eqnarray*}
and hence $ Z_{S 1}^* Z_{N 1}^* > 0 $ and $ Z_{S 2}^* Z_{N 2}^* = 0 $ on $ (N-.1) \cup (N+.1) $ and $ Z_{S 1}^* Z_{N 1}^* = 0 $ and $ Z_{S 2}^* Z_{N 2}^* > 0 $ on $ (N-.2) \cup (N+.2) $.
Therefore, $ Z_{S i}^* Z_{N i}^* \geqslant 0 $ for $ i = 1, 2 $.

%\newpage

\subsection*{A.4 \quad Comparison of demand functions for sophisticated and na\"ive Investors}

\quad 
Equations (6) and (10) provide the demand functions for sophisticated and na\"ive investors. Now we check the different between them.

{\bf Scenario 1}: $ (N1.1) \left\{ \begin{matrix} R_1 < \underline{\rho} R_2 \\ R_2 > \underline{\rho} R_1 \end{matrix} \right. $ or $ (N1.2) \left\{ \begin{matrix} R_1 > \underline{\rho} R_2 \\ R_2 < \underline{\rho} R_1 \end{matrix} \right. $
\begin{eqnarray*}
\alpha Z_S^* = \dfrac1{1 - {\hat \rho}^2} \left( \begin{matrix} \dfrac{R_1 - {\hat \rho} R_2}{\sigma_1} \\ \dfrac{R_2 - {\hat \rho} R_1}{\sigma_2} \end{matrix} \right) \qquad \text{and} \qquad \alpha Z_N^* = \left( \begin{matrix} \dfrac{R_1 - \underline{\rho} R_2}{\sigma_1 (1 - \underline{\rho}^2)} \\ \dfrac{R_2 - \underline{\rho} R_1}{\sigma_2 (1 - \underline{\rho}^2)} \end{matrix} \right)
\end{eqnarray*}
\begin{eqnarray*}
\alpha (Z_N^* - Z_S^*) = \left( \begin{matrix} \dfrac{R_1 - \underline{\rho} R_2}{\sigma_1 (1 - \underline{\rho}^2)} - \dfrac{R_1 - {\hat \rho} R_2}{\sigma_1 (1 - {\hat \rho}^2)} \\ \dfrac{R_2 - \underline{\rho} R_1}{\sigma_2 (1 - \underline{\rho}^2)} - \dfrac{R_2 - {\hat \rho} R_1}{\sigma_2 (1 - {\hat \rho}^2)} \end{matrix} \right) = \left( \begin{matrix} - \dfrac{({\hat \rho} - \underline{\rho}) [(\underline{\rho} + {\hat \rho}) R_1 - (1 + \underline{\rho} {\hat \rho}) R_2]}{\sigma_1 (1 - \underline{\rho}^2) (1 - {\hat \rho}^2)} \\ - \dfrac{({\hat \rho} - \underline{\rho}) [(\underline{\rho} + {\hat \rho}) R_2 - (1 + \underline{\rho} {\hat \rho}) R_1]}{\sigma_2 (1 - \underline{\rho}^2) (1 - {\hat \rho}^2)} \end{matrix} \right)
\end{eqnarray*}
\begin{eqnarray*}
\dfrac{\alpha (1 - \underline{\rho}^2) (1 - {\hat \rho}^2)}{({\hat \rho} - \underline{\rho}) (1 + \underline{\rho} {\hat \rho})} (Z_N^* - Z_S^*) = \left( \begin{matrix} \dfrac{1}{\sigma_1} \left[ R_2 - \dfrac{\underline{\rho} + {\hat \rho}}{1 + \underline{\rho} {\hat \rho}} R_1 \right] \\ \dfrac{1}{\sigma_2} \left[ R_1 - \dfrac{\underline{\rho} + {\hat \rho}}{1 + \underline{\rho} {\hat \rho}} R_2 \right] \end{matrix} \right)
\end{eqnarray*}

{\it Setting 1}: $ (N1.1) \left\{ \begin{matrix} R_1 < \underline{\rho} R_2 \\ R_2 > \underline{\rho} R_1 \end{matrix} \right. $ is equivalent to $ \left\{ \begin{matrix} Z_{N 1}^* < 0 \\ Z_{N 2}^* > 0 \end{matrix} \right. $

In Case 1: $ 0 < \underline{\rho} < {\hat \rho} < \dfrac{\underline{\rho} + {\hat \rho}}{1 + \underline{\rho} {\hat \rho}} $, $ 0 > Z_{N 1}^* > Z_{S 1}^* $ and $ 0 < Z_{N 2}^* < Z_{S 2}^* $.
\begin{itemize}
\item If $ R_1 < 0 $, then $ R_2 > \underline{\rho} R_1 > \dfrac{\underline{\rho} + {\hat \rho}}{1 + \underline{\rho} {\hat \rho}} R_1 $, and hence $ Z_{N 1}^* - Z_{S 1}^* > 0 $.
\item If $ R_1 > 0 $, then $ R_2 > \dfrac{1}{\underline{\rho}} R_1 > \dfrac{\underline{\rho} + {\hat \rho}}{1 + \underline{\rho} {\hat \rho}} R_1 $, and hence $ Z_{N 1}^* - Z_{S 1}^* > 0 $.
\item If $ R_2 < 0 $, then $ R_1 < \dfrac{1}{\underline{\rho}} R_2 < \dfrac{\underline{\rho} + {\hat \rho}}{1 + \underline{\rho} {\hat \rho}} R_2 $, and hence $ Z_{N 2}^* - Z_{S 2}^* < 0 $.
\item If $ R_2 > 0 $, then $ R_1 < \underline{\rho} R_2 < \dfrac{\underline{\rho} + {\hat \rho}}{1 + \underline{\rho} {\hat \rho}} R_2 $, and hence $ Z_{N 2}^* - Z_{S 2}^* < 0 $.
\end{itemize}

In Case 2: $ 0 = \underline{\rho} < {\hat \rho} = \dfrac{\underline{\rho} + {\hat \rho}}{1 + \underline{\rho} {\hat \rho}} $, $ 0 > Z_{N 1}^* > Z_{S 1}^* $ and $ 0 < Z_{N 2}^* < Z_{S 2}^* $.

$ \left\{ \begin{matrix} R_1 < \underline{\rho} R_2 \\ R_2 > \underline{\rho} R_1 \end{matrix} \right. $ is equivalent to $ \left\{ \begin{matrix} R_1 < 0 \\ R_2 > 0 \end{matrix} \right. $ is equivalent to $ \left\{ \begin{matrix} Z_{N 1}^* - Z_{S 1}^* > 0 \\ Z_{N 2}^* - Z_{S 2}^* < 0 \end{matrix} \right. $

In Case 3: $ \underline{\rho} < 0 < {\hat \rho} $, then $ \underline{\rho} < \dfrac{\underline{\rho} + {\hat \rho}}{1 + \underline{\rho} {\hat \rho}} < {\hat \rho} $, hence $ \left\{ \begin{matrix} R_1 < 0 \\ \underline{\rho} R_1 < R_2 < \dfrac{1}{\underline{\rho}} R_1 \end{matrix} \right. $ and $ \left\{ \begin{matrix} R_2 > 0 \\ \dfrac{1}{\underline{\rho}} R_2 < R_1 < \underline{\rho} R_2 \end{matrix} \right. $.
Therefore $ 0 > Z_{N 1}^* > Z_{S 1}^* $ and $ 0 < Z_{N 2}^* < Z_{S 2}^* $.
\begin{itemize}
\item If $ R_1 < 0 $, then $ R_2 > \underline{\rho} R_1 > \dfrac{\underline{\rho} + {\hat \rho}}{1 + \underline{\rho} {\hat \rho}} R_1 $, and hence $ Z_{N 1}^* - Z_{S 1}^* > 0 $.
\item If $ R_2 > 0 $, then $ R_1 < \underline{\rho} R_2 < \dfrac{\underline{\rho} + {\hat \rho}}{1 + \underline{\rho} {\hat \rho}} R_2 $, and hence $ Z_{N 2}^* - Z_{S 2}^* < 0 $.
\end{itemize}

In Case 4: $ \dfrac{\underline{\rho} + {\hat \rho}}{1 + \underline{\rho} {\hat \rho}} = \underline{\rho} < {\hat \rho} = 0 $, $ 0 > Z_{N 1}^* > Z_{S 1}^* $ and $ 0 < Z_{N 2}^* < Z_{S 2}^* $.

$ \left\{ \begin{matrix} R_1 < \underline{\rho} R_2 \\ R_2 > \underline{\rho} R_1 \end{matrix} \right. $ is equivalent to $ \left\{ \begin{matrix} Z_{N 1}^* - Z_{S 1}^* > 0 \\ Z_{N 2}^* - Z_{S 2}^* < 0 \end{matrix} \right. $

In Case 5: $ \dfrac{\underline{\rho} + {\hat \rho}}{1 + \underline{\rho} {\hat \rho}} < \underline{\rho} < {\hat \rho} < 0 $, hence $ \left\{ \begin{matrix} R_1 < 0 \\ \underline{\rho} R_1 < R_2 < \dfrac{1}{\underline{\rho}} R_1 \end{matrix} \right. $ and $ \left\{ \begin{matrix} R_2 > 0 \\ \dfrac{1}{\underline{\rho}} R_2 < R_1 < \underline{\rho} R_2 \end{matrix} \right. $.
Therefore $ 0 > Z_{N 1}^* > Z_{S 1}^* $ for $ \left\{ \begin{matrix} R_1 < 0 \\ \dfrac{\underline{\rho} + {\hat \rho}}{1 + \underline{\rho} {\hat \rho}} R_1 < R_2 < \dfrac{1}{\underline{\rho}} R_1 \end{matrix} \right. $ and \hhred{$ Z_{N 1}^* < Z_{S 1}^* $ for $ \left\{ \begin{matrix} R_1 < 0 \\ \underline{\rho} R_1 < R_2 < \dfrac{\underline{\rho} + {\hat \rho}}{1 + \underline{\rho} {\hat \rho}} R_1 \end{matrix} \right. $}; 
$ 0 < Z_{N 2}^* < Z_{S 2}^* $ for $ \left\{ \begin{matrix} R_2 > 0 \\ \dfrac{1}{\underline{\rho}} R_2 < R_1 < \dfrac{\underline{\rho} + {\hat \rho}}{1 + \underline{\rho} {\hat \rho}} R_2 \end{matrix} \right. $ and \hhred{$ Z_{N 2}^* > Z_{S 2}^* $ for $ \left\{ \begin{matrix} R_2 > 0 \\ \dfrac{\underline{\rho} + {\hat \rho}}{1 + \underline{\rho} {\hat \rho}} R_2 < R_1 < \underline{\rho} R_2 \end{matrix} \right. $}.

{\it Setting 2}: $ (N1.2) \left\{ \begin{matrix} R_1 > \underline{\rho} R_2 \\ R_2 < \underline{\rho} R_1 \end{matrix} \right. $ is equivalent to $ \left\{ \begin{matrix} Z_{N 1}^* > 0 \\ Z_{N 2}^* < 0 \end{matrix} \right. $

In Case 1: $ 0 < \underline{\rho} < {\hat \rho} < \dfrac{\underline{\rho} + {\hat \rho}}{1 + \underline{\rho} {\hat \rho}} $, $ 0 < Z_{N 1}^* < Z_{S 1}^* $ and $ 0 > Z_{N 2}^* > Z_{S 2}^* $.
\begin{itemize}
\item If $ R_1 < 0 $, then $ R_2 < \dfrac{1}{\underline{\rho}} R_1 < \dfrac{\underline{\rho} + {\hat \rho}}{1 + \underline{\rho} {\hat \rho}} R_1 $, and hence $ Z_{N 1}^* - Z_{S 1}^* < 0 $.
\item If $ R_1 > 0 $, then $ R_2 < \underline{\rho} R_1 < \dfrac{\underline{\rho} + {\hat \rho}}{1 + \underline{\rho} {\hat \rho}} R_1 $, and hence $ Z_{N 1}^* - Z_{S 1}^* < 0 $.
\item If $ R_2 < 0 $, then $ R_1 > \underline{\rho} R_2 > \dfrac{\underline{\rho} + {\hat \rho}}{1 + \underline{\rho} {\hat \rho}} R_2 $, and hence $ Z_{N 2}^* - Z_{S 2}^* > 0 $.
\item If $ R_2 > 0 $, then $ R_1 > \dfrac{1}{\underline{\rho}} R_2 > \dfrac{\underline{\rho} + {\hat \rho}}{1 + \underline{\rho} {\hat \rho}} R_2 $, and hence $ Z_{N 2}^* - Z_{S 2}^* > 0 $.
\end{itemize}

In Case 2: $ 0 = \underline{\rho} < {\hat \rho} = \dfrac{\underline{\rho} + {\hat \rho}}{1 + \underline{\rho} {\hat \rho}} $, $ 0 < Z_{N 1}^* < Z_{S 1}^* $ and $ 0 > Z_{N 2}^* > Z_{S 2}^* $.

$ \left\{ \begin{matrix} R_1 > \underline{\rho} R_2 \\ R_2 < \underline{\rho} R_1 \end{matrix} \right. $ is equivalent to $ \left\{ \begin{matrix} R_1 > 0 \\ R_2 < 0 \end{matrix} \right. $, then $ \left\{ \begin{matrix} Z_{N 1}^* - Z_{S 1}^* < 0 \\ Z_{N 2}^* - Z_{S 2}^* > 0 \end{matrix} \right. $

In Case 3: $ \underline{\rho} < 0 < {\hat \rho} $, then $ \underline{\rho} < \dfrac{\underline{\rho} + {\hat \rho}}{1 + \underline{\rho} {\hat \rho}} < {\hat \rho} $, hence $ \left\{ \begin{matrix} R_1 > 0 \\ \dfrac{1}{\underline{\rho}} R_1 < R_2 < \underline{\rho} R_1 \end{matrix} \right. $ and $ \left\{ \begin{matrix} R_2 < 0 \\ \underline{\rho} R_2 < R_1 < \dfrac{1}{\underline{\rho}} R_2 \end{matrix} \right. $.
Therefore $ 0 < Z_{N 1}^* < Z_{S 1}^* $ and $ 0 > Z_{N 2}^* > Z_{S 2}^* $.
\begin{itemize}
\item If $ R_1 > 0 $, then $ R_2 < \underline{\rho} R_1 < \dfrac{\underline{\rho} + {\hat \rho}}{1 + \underline{\rho} {\hat \rho}} R_1 $, and hence $ Z_{N 1}^* - Z_{S 1}^* < 0 $.
\item If $ R_2 < 0 $, then $ R_1 > \underline{\rho} R_2 > \dfrac{\underline{\rho} + {\hat \rho}}{1 + \underline{\rho} {\hat \rho}} R_2 $, and hence $ Z_{N 2}^* - Z_{S 2}^* > 0 $.
\end{itemize}

In Case 4: $ \dfrac{\underline{\rho} + {\hat \rho}}{1 + \underline{\rho} {\hat \rho}} = \underline{\rho} < {\hat \rho} = 0 $, $ 0 < Z_{N 1}^* < Z_{S 1}^* $ and $ 0 > Z_{N 2}^* > Z_{S 2}^* $.

$ \left\{ \begin{matrix} R_1 > \underline{\rho} R_2 \\ R_2 < \underline{\rho} R_1 \end{matrix} \right. $ is equivalent to $ \left\{ \begin{matrix} Z_{N 1}^* - Z_{S 1}^* < 0 \\ Z_{N 2}^* - Z_{S 2}^* > 0 \end{matrix} \right. $

In Case 5: $ \dfrac{\underline{\rho} + {\hat \rho}}{1 + \underline{\rho} {\hat \rho}} < \underline{\rho} < {\hat \rho} < 0 $, hence $ \left\{ \begin{matrix} R_1 > 0 \\ \dfrac{1}{\underline{\rho}} R_1 < R_2 < \underline{\rho} R_1 \end{matrix} \right. $ and $ \left\{ \begin{matrix} R_2 < 0 \\ \underline{\rho} R_2 < R_1 < \dfrac{1}{\underline{\rho}} R_2 \end{matrix} \right. $.
Therefore $ 0 < Z_{N 1}^* < Z_{S 1}^* $ for $ \left\{ \begin{matrix} R_1 > 0 \\ \dfrac{1}{\underline{\rho}} R_1 < R_2 < \dfrac{\underline{\rho} + {\hat \rho}}{1 + \underline{\rho} {\hat \rho}} R_1 \end{matrix} \right. $ and \hhred{$ Z_{N 1}^* > Z_{S 1}^* $ for $ \left\{ \begin{matrix} R_1 > 0 \\ \dfrac{\underline{\rho} + {\hat \rho}}{1 + \underline{\rho} {\hat \rho}} R_1 < R_2 < \underline{\rho} R_1 \end{matrix} \right. $}; 
$ 0 > Z_{N 2}^* > Z_{S 2}^* $ for $ \left\{ \begin{matrix} R_2 < 0 \\ \dfrac{\underline{\rho} + {\hat \rho}}{1 + \underline{\rho} {\hat \rho}} R_2 < R_1 < \underline{\rho} R_2 \end{matrix} \right. $ and \hhred{$ Z_{N 2}^* < Z_{S 2}^* $ for $ \left\{ \begin{matrix} R_2 < 0 \\ \dfrac{1}{\underline{\rho}} R_2 < R_1 < \dfrac{\underline{\rho} + {\hat \rho}}{1 + \underline{\rho} {\hat \rho}} R_2 \end{matrix} \right. $}.

{\bf Scenario 2}: $ (N2.1) \left\{ \begin{matrix} R_1 < \overline{\rho} R_2 \\ R_2 < \overline{\rho} R_1 \end{matrix} \right. $ or $ (N2.2) \left\{ \begin{matrix} R_1 > \overline{\rho} R_2 \\ R_2 > \overline{\rho} R_1 \end{matrix} \right. $
\begin{eqnarray*}
\alpha Z_S^* = \dfrac1{1 - {\hat \rho}^2} \left( \begin{matrix} \dfrac{R_1 - {\hat \rho} R_2}{\sigma_1} \\ \dfrac{R_2 - {\hat \rho} R_1}{\sigma_2} \end{matrix} \right) \qquad \text{and} \qquad \alpha Z_N^* = \left( \begin{matrix} \dfrac{R_1 - \overline{\rho} R_2}{\sigma_1 (1 - \overline{\rho}^2)} \\ \dfrac{R_2 - \overline{\rho} R_1}{\sigma_2 (1 - \overline{\rho}^2)} \end{matrix} \right)
\end{eqnarray*}
\begin{eqnarray*}
\alpha (Z_N^* - Z_S^*) = \left( \begin{matrix} \dfrac{R_1 - \overline{\rho} R_2}{\sigma_1 (1 - \overline{\rho}^2)} - \dfrac{R_1 - {\hat \rho} R_2}{\sigma_1 (1 - {\hat \rho}^2)} \\ \dfrac{R_2 - \overline{\rho} R_1}{\sigma_2 (1 - \overline{\rho}^2)} - \dfrac{R_2 - {\hat \rho} R_1}{\sigma_2 (1 - {\hat \rho}^2)} \end{matrix} \right) = \left( \begin{matrix} \dfrac{(\overline{\rho} - {\hat \rho}) [(\overline{\rho} + {\hat \rho}) R_1 - (1 + \overline{\rho} {\hat \rho}) R_2]}{\sigma_1 (1 - \overline{\rho}^2) (1 - {\hat \rho}^2)} \\ \dfrac{(\overline{\rho} - {\hat \rho}) [(\overline{\rho} + {\hat \rho}) R_2 - (1 + \overline{\rho} {\hat \rho}) R_1]}{\sigma_2 (1 - \overline{\rho}^2) (1 - {\hat \rho}^2)} \end{matrix} \right)
\end{eqnarray*}
\begin{eqnarray*}
\dfrac{\alpha (1 - \overline{\rho}^2) (1 - {\hat \rho}^2)}{(\overline{\rho} - {\hat \rho}) (1 + \overline{\rho} {\hat \rho})} (Z_N^* - Z_S^*) = \left( \begin{matrix} - \dfrac{1}{\sigma_1} \left[ R_2 - \dfrac{\overline{\rho} + {\hat \rho}}{1 + \overline{\rho} {\hat \rho}} R_1 \right] \\ - \dfrac{1}{\sigma_2} \left[ R_1 - \dfrac{\overline{\rho} + {\hat \rho}}{1 + \overline{\rho} {\hat \rho}} R_2 \right] \end{matrix} \right)
\end{eqnarray*}

{\it Setting 3}: $ (N2.1) \left\{ \begin{matrix} R_1 < \overline{\rho} R_2 \\ R_2 < \overline{\rho} R_1 \end{matrix} \right. $ is equivalent to $ \left\{ \begin{matrix} Z_{N 1}^* < 0 \\ Z_{N 2}^* < 0 \end{matrix} \right. $

In Case 1: $ 0 < {\hat \rho} < \overline{\rho} < \dfrac{\overline{\rho} + {\hat \rho}}{1 + \overline{\rho} {\hat \rho}} $, 
hence $ \left\{ \begin{matrix} R_1 < 0 \\ \dfrac{1}{\overline{\rho}} R_1 < R_2 < \overline{\rho} R_1 \end{matrix} \right. $ and $ \left\{ \begin{matrix} R_2 < 0 \\ \dfrac{1}{\overline{\rho}} R_2 < R_1 < \overline{\rho} R_2 \end{matrix} \right. $.
Therefore $ 0 > Z_{N 1}^* > Z_{S 1}^* $ for $ \left\{ \begin{matrix} R_1 < 0 \\ \dfrac{1}{\overline{\rho}} R_1 < R_2 < \dfrac{\overline{\rho} + {\hat \rho}}{1 + \overline{\rho} {\hat \rho}} R_1 \end{matrix} \right. $ and \hhred{$ Z_{N 1}^* < Z_{S 1}^* $ for $ \left\{ \begin{matrix} R_1 < 0 \\ \dfrac{\overline{\rho} + {\hat \rho}}{1 + \overline{\rho} {\hat \rho}} R_1 < R_2 < \overline{\rho} R_1 \end{matrix} \right. $}; 
$ 0 > Z_{N 2}^* > Z_{S 2}^* $ for $ \left\{ \begin{matrix} R_2 < 0 \\ \dfrac{1}{\overline{\rho}} R_2 < R_1 < \dfrac{\overline{\rho} + {\hat \rho}}{1 + \overline{\rho} {\hat \rho}} R_2 \end{matrix} \right. $ and \hhred{$ Z_{N 2}^* < Z_{S 2}^* $ for $ \left\{ \begin{matrix} R_2 < 0 \\ \dfrac{\overline{\rho} + {\hat \rho}}{1 + \overline{\rho} {\hat \rho}} R_2 < R_1 < \overline{\rho} R_2 \end{matrix} \right. $}.

In Case 2: $ 0 = {\hat \rho} < \overline{\rho} = \dfrac{\overline{\rho} + {\hat \rho}}{1 + \overline{\rho} {\hat \rho}} $, $ 0 > Z_{N 1}^* > Z_{S 1}^* $ and $ 0 > Z_{N 2}^* > Z_{S 2}^* $.

$ \left\{ \begin{matrix} R_1 < \overline{\rho} R_2 \\ R_2 < \overline{\rho} R_1 \end{matrix} \right. $ is equivalent to $ \left\{ \begin{matrix} Z_{N 1}^* - Z_{S 1}^* > 0 \\ Z_{N 2}^* - Z_{S 2}^* > 0 \end{matrix} \right. $

In Case 3: $ {\hat \rho} < 0 < \overline{\rho} $, then $ {\hat \rho} < \dfrac{\overline{\rho} + {\hat \rho}}{1 + \overline{\rho} {\hat \rho}} < \overline{\rho} $, hence $ \left\{ \begin{matrix} R_1 < 0 \\ \overline{\rho} R_1 < R_2 < \dfrac{1}{\overline{\rho}} R_1 \end{matrix} \right. $ and $ \left\{ \begin{matrix} R_2 < 0 \\ \dfrac{1}{\overline{\rho}} R_2 < R_1 < \overline{\rho} R_2 \end{matrix} \right. $.
Therefore $ 0 > Z_{N 1}^* > Z_{S 1}^* $ and $ 0 > Z_{N 2}^* > Z_{S 2}^* $.
\begin{itemize}
\item If $ R_1 < 0 $, then $ R_2 < \overline{\rho} R_1 < \dfrac{\overline{\rho} + {\hat \rho}}{1 + \overline{\rho} {\hat \rho}} R_1 $, and hence $ Z_{N 1}^* - Z_{S 1}^* > 0 $.
\item If $ R_2 < 0 $, then $ R_1 < \overline{\rho} R_2 < \dfrac{\overline{\rho} + {\hat \rho}}{1 + \overline{\rho} {\hat \rho}} R_2 $, and hence $ Z_{N 2}^* - Z_{S 2}^* > 0 $.
\end{itemize}

In Case 4: $ \dfrac{\overline{\rho} + {\hat \rho}}{1 + \overline{\rho} {\hat \rho}} = {\hat \rho} < \overline{\rho} = 0 $, $ 0 > Z_{N 1}^* > Z_{S 1}^* $ and $ 0 > Z_{N 2}^* > Z_{S 2}^* $.

$ \left\{ \begin{matrix} R_1 < \overline{\rho} R_2 \\ R_2 < \overline{\rho} R_1 \end{matrix} \right. $ is equivalent to $ \left\{ \begin{matrix} R_1 < 0 \\ R_2 < 0 \end{matrix} \right. $, then $ \left\{ \begin{matrix} Z_{N 1}^* - Z_{S 1}^* > 0 \\ Z_{N 2}^* - Z_{S 2}^* > 0 \end{matrix} \right. $

In Case 5: $ \dfrac{\overline{\rho} + {\hat \rho}}{1 + \overline{\rho} {\hat \rho}} < {\hat \rho} < \overline{\rho} < 0 $, $ 0 > Z_{N 1}^* > Z_{S 1}^* $ and $ 0 > Z_{N 2}^* > Z_{S 2}^* $.

$ \left\{ \begin{matrix} R_1 < \overline{\rho} R_2 \\ R_2 < \overline{\rho} R_1 \end{matrix} \right. $ is equivalent to $ \left\{ \begin{matrix} R_1 < 0 \\ R_2 < \overline{\rho} R_1 \end{matrix} \right. $ and $ \left\{ \begin{matrix} R_1 > 0 \\ R_2 < \dfrac{1}{\overline{\rho}} R_1 \end{matrix} \right. $. 
\begin{itemize}
\item $ \left\{ \begin{matrix} R_1 < 0 \\ R_2 < \overline{\rho} R_1 \end{matrix} \right. $ implies $ R_2 < \overline{\rho} R_1 < \dfrac{\overline{\rho} + {\hat \rho}}{1 + \overline{\rho} {\hat \rho}} R_1 $, and hence $ Z_{N 1}^* - Z_{S 1}^* > 0 $.
\item $ \left\{ \begin{matrix} R_1 > 0 \\ R_2 < \dfrac{1}{\overline{\rho}} R_1 \end{matrix} \right. $ implies $ R_2 < \dfrac{1}{\overline{\rho}} R_1 < \dfrac{\overline{\rho} + {\hat \rho}}{1 + \overline{\rho} {\hat \rho}} R_1 $, and hence $ Z_{N 1}^* - Z_{S 1}^* > 0 $.
\end{itemize}

$ \left\{ \begin{matrix} R_1 < \overline{\rho} R_2 \\ R_2 < \overline{\rho} R_1 \end{matrix} \right. $ is equivalent to $ \left\{ \begin{matrix} R_2 < 0 \\ R_1 < \overline{\rho} R_2 \end{matrix} \right. $ and $ \left\{ \begin{matrix} R_2 > 0 \\ R_1 < \dfrac{1}{\overline{\rho}} R_2 \end{matrix} \right. $. 
\begin{itemize}
\item $ \left\{ \begin{matrix} R_2 < 0 \\ R_1 < \overline{\rho} R_2 \end{matrix} \right. $ implies $ R_1 < \overline{\rho} R_2 < \dfrac{\overline{\rho} + {\hat \rho}}{1 + \overline{\rho} {\hat \rho}} R_2 $, and hence $ Z_{N 2}^* - Z_{S 2}^* > 0 $.
\item $ \left\{ \begin{matrix} R_2 > 0 \\ R_1 < \dfrac{1}{\overline{\rho}} R_2 \end{matrix} \right. $ implies $ R_1 < \dfrac{1}{\overline{\rho}} R_2 < \dfrac{\overline{\rho} + {\hat \rho}}{1 + \overline{\rho} {\hat \rho}} R_2 $, and hence $ Z_{N 2}^* - Z_{S 2}^* > 0 $.
\end{itemize}

\begin{eqnarray*}
\dfrac{\alpha (1 - \overline{\rho}^2) (1 - {\hat \rho}^2)}{(\overline{\rho} - {\hat \rho}) (1 + \overline{\rho} {\hat \rho})} (Z_N^* - Z_S^*) = \left( \begin{matrix} - \dfrac{1}{\sigma_1} \left[ R_2 - \dfrac{\overline{\rho} + {\hat \rho}}{1 + \overline{\rho} {\hat \rho}} R_1 \right] \\ - \dfrac{1}{\sigma_2} \left[ R_1 - \dfrac{\overline{\rho} + {\hat \rho}}{1 + \overline{\rho} {\hat \rho}} R_2 \right] \end{matrix} \right)
\end{eqnarray*}

{\it Setting 4}: $ (N2.2) \left\{ \begin{matrix} R_1 > \overline{\rho} R_2 \\ R_2 > \overline{\rho} R_1 \end{matrix} \right. $ is equivalent to $ \left\{ \begin{matrix} Z_{N 1}^* > 0 \\ Z_{N 2}^* > 0 \end{matrix} \right. $

In Case 1: $ 0 < {\hat \rho} < \overline{\rho} < \dfrac{\overline{\rho} + {\hat \rho}}{1 + \overline{\rho} {\hat \rho}} $, 
hence $ \left\{ \begin{matrix} R_1 > 0 \\ \overline{\rho} R_1 < R_2 < \dfrac{1}{\overline{\rho}} R_1 \end{matrix} \right. $ and $ \left\{ \begin{matrix} R_2 > 0 \\ \overline{\rho} R_2 < R_1 < \dfrac{1}{\overline{\rho}} R_2 \end{matrix} \right. $.
Therefore \hhred{$ Z_{N 1}^* > Z_{S 1}^* $ for $ \left\{ \begin{matrix} R_1 > 0 \\ \overline{\rho} R_1 < R_2 < \dfrac{\overline{\rho} + {\hat \rho}}{1 + \overline{\rho} {\hat \rho}} R_1 \end{matrix} \right. $} and $ 0 < Z_{N 1}^* < Z_{S 1}^* $ for $ \left\{ \begin{matrix} R_1 > 0 \\ \dfrac{\overline{\rho} + {\hat \rho}}{1 + \overline{\rho} {\hat \rho}} R_1 < R_2 < \dfrac{1}{\overline{\rho}} R_1 \end{matrix} \right. $; 
\hhred{$ Z_{N 2}^* > Z_{S 2}^* $ for $ \left\{ \begin{matrix} R_2 > 0 \\ \overline{\rho} R_2 < R_1 < \dfrac{\overline{\rho} + {\hat \rho}}{1 + \overline{\rho} {\hat \rho}} R_2 \end{matrix} \right. $} and $ 0 < Z_{N 2}^* < Z_{S 2}^* $ for $ \left\{ \begin{matrix} R_2 > 0 \\ \dfrac{\overline{\rho} + {\hat \rho}}{1 + \overline{\rho} {\hat \rho}} R_2 < R_1 < \dfrac{1}{\overline{\rho}} R_2 \end{matrix} \right. $.

In Case 2: $ 0 = {\hat \rho} < \overline{\rho} = \dfrac{\overline{\rho} + {\hat \rho}}{1 + \overline{\rho} {\hat \rho}} $, $ 0 < Z_{N 1}^* < Z_{S 1}^* $ and $ 0 < Z_{N 2}^* < Z_{S 2}^* $.

$ \left\{ \begin{matrix} R_1 > \overline{\rho} R_2 \\ R_2 > \overline{\rho} R_1 \end{matrix} \right. $ is equivalent to $ \left\{ \begin{matrix} Z_{N 1}^* - Z_{S 1}^* < 0 \\ Z_{N 2}^* - Z_{S 2}^* < 0 \end{matrix} \right. $

In Case 3: $ {\hat \rho} < 0 < \overline{\rho} $, then $ {\hat \rho} < \dfrac{\overline{\rho} + {\hat \rho}}{1 + \overline{\rho} {\hat \rho}} < \overline{\rho} $, hence $ \left\{ \begin{matrix} R_1 > 0 \\ \overline{\rho} R_1 < R_2 < \dfrac{1}{\overline{\rho}} R_1 \end{matrix} \right. $ and $ \left\{ \begin{matrix} R_2 > 0 \\ \overline{\rho} R_2 < R_1 < \dfrac{1}{\overline{\rho}} R_2 \end{matrix} \right. $.
Therefore $ 0 < Z_{N 1}^* < Z_{S 1}^* $ and $ 0 < Z_{N 2}^* < Z_{S 2}^* $.
\begin{itemize}
\item If $ R_1 > 0 $, then $ R_2 > \overline{\rho} R_1 > \dfrac{\overline{\rho} + {\hat \rho}}{1 + \overline{\rho} {\hat \rho}} R_1 $, and hence $ Z_{N 1}^* - Z_{S 1}^* < 0 $.
\item If $ R_2 > 0 $, then $ R_1 > \overline{\rho} R_2 > \dfrac{\overline{\rho} + {\hat \rho}}{1 + \overline{\rho} {\hat \rho}} R_2 $, and hence $ Z_{N 2}^* - Z_{S 2}^* < 0 $.
\end{itemize}

In Case 4: $ \dfrac{\overline{\rho} + {\hat \rho}}{1 + \overline{\rho} {\hat \rho}} = {\hat \rho} < \overline{\rho} = 0 $, $ 0 < Z_{N 1}^* < Z_{S 1}^* $ and $ 0 < Z_{N 2}^* < Z_{S 2}^* $.

$ \left\{ \begin{matrix} R_1 > \overline{\rho} R_2 \\ R_2 > \overline{\rho} R_1 \end{matrix} \right. $ is equivalent to $ \left\{ \begin{matrix} R_1 > 0 \\ R_2 > 0 \end{matrix} \right. $, then $ \left\{ \begin{matrix} Z_{N 1}^* - Z_{S 1}^* < 0 \\ Z_{N 2}^* - Z_{S 2}^* < 0 \end{matrix} \right. $

In Case 5: $ \dfrac{\overline{\rho} + {\hat \rho}}{1 + \overline{\rho} {\hat \rho}} < {\hat \rho} < \overline{\rho} < 0 $, $ 0 < Z_{N 1}^* < Z_{S 1}^* $ and $ 0 < Z_{N 2}^* < Z_{S 2}^* $.

$ \left\{ \begin{matrix} R_1 > \overline{\rho} R_2 \\ R_2 > \overline{\rho} R_1 \end{matrix} \right. $ is equivalent to $ \left\{ \begin{matrix} R_1 < 0 \\ R_2 > \dfrac{1}{\overline{\rho}} R_1 \end{matrix} \right. $ and $ \left\{ \begin{matrix} R_1 > 0 \\ R_2 > \overline{\rho} R_1 \end{matrix} \right. $. 
\begin{itemize}
\item $ \left\{ \begin{matrix} R_1 < 0 \\ R_2 > \dfrac{1}{\overline{\rho}} R_1 \end{matrix} \right. $ implies $ R_2 > \dfrac{1}{\overline{\rho}} R_1 > \dfrac{\overline{\rho} + {\hat \rho}}{1 + \overline{\rho} {\hat \rho}} R_1 $, and hence $ Z_{N 1}^* - Z_{S 1}^* < 0 $.
\item $ \left\{ \begin{matrix} R_1 > 0 \\ R_2 > \overline{\rho} R_1 \end{matrix} \right. $ implies $ R_2 > \overline{\rho} R_1 > \dfrac{\overline{\rho} + {\hat \rho}}{1 + \overline{\rho} {\hat \rho}} R_1 $, and hence $ Z_{N 1}^* - Z_{S 1}^* < 0 $.
\end{itemize}

$ \left\{ \begin{matrix} R_1 > \overline{\rho} R_2 \\ R_2 > \overline{\rho} R_1 \end{matrix} \right. $ is equivalent to $ \left\{ \begin{matrix} R_2 < 0 \\ R_1 > \dfrac{1}{\overline{\rho}} R_2 \end{matrix} \right. $ and $ \left\{ \begin{matrix} R_2 > 0 \\ R_1 > \overline{\rho} R_2 \end{matrix} \right. $. 
\begin{itemize}
\item $ \left\{ \begin{matrix} R_2 < 0 \\ R_1 > \dfrac{1}{\overline{\rho}} R_2 \end{matrix} \right. $ implies $ R_1 > \dfrac{1}{\overline{\rho}} R_2 > \dfrac{\overline{\rho} + {\hat \rho}}{1 + \overline{\rho} {\hat \rho}} R_2 $, and hence $ Z_{N 2}^* - Z_{S 2}^* < 0 $.
\item $ \left\{ \begin{matrix} R_2 > 0 \\ R_1 > \overline{\rho} R_2 \end{matrix} \right. $ implies $ R_1 > \overline{\rho} R_2 > \dfrac{\overline{\rho} + {\hat \rho}}{1 + \overline{\rho} {\hat \rho}} R_2 $, and hence $ Z_{N 2}^* - Z_{S 2}^* < 0 $.
\end{itemize}

{\bf Scenario 3}: $ (N-.1) \left\{ \begin{matrix} R_1 < 0 \\ \overline{\rho} R_1 \leqslant R_2 \leqslant \underline{\rho} R_1 \end{matrix} \right. $ or $ (N+.1) \left\{ \begin{matrix} R_1 > 0 \\ \underline{\rho} R_1 \leqslant R_2 \leqslant \overline{\rho} R_1 \end{matrix} \right. $
\begin{eqnarray*}
\alpha Z_S^* = \dfrac1{1 - {\hat \rho}^2} \left( \begin{matrix} \dfrac{R_1 - {\hat \rho} R_2}{\sigma_1} \\ \dfrac{R_2 - {\hat \rho} R_1}{\sigma_2} \end{matrix} \right) \qquad \text{and} \qquad \alpha Z_N^* = \left( \begin{matrix} \dfrac{R_1}{\sigma_1} \\ 0 \end{matrix} \right)
\end{eqnarray*}
\begin{eqnarray*}
\alpha (Z_N^* - Z_S^*) = \left( \begin{matrix} \dfrac{R_1}{\sigma_1} - \dfrac{R_1 - {\hat \rho} R_2}{\sigma_1 (1 - {\hat \rho}^2)} \\ - \dfrac{R_2 - {\hat \rho} R_1}{\sigma_2 (1 - {\hat \rho}^2)} \end{matrix} \right) = \left( \begin{matrix} \dfrac{{\hat \rho} (R_2 - {\hat \rho} R_1)}{\sigma_1 (1 - {\hat \rho}^2)} \\ - \dfrac{R_2 - {\hat \rho} R_1}{\sigma_2 (1 - {\hat \rho}^2)} \end{matrix} \right) = \dfrac{R_2 - {\hat \rho} R_1}{1 - {\hat \rho}^2} \left( \begin{matrix} \dfrac{{\hat \rho}}{\sigma_1} \\ - \dfrac{1}{\sigma_2} \end{matrix} \right)
\end{eqnarray*}

{\it Setting 5}: $ (N-.1) \left\{ \begin{matrix} R_1 < 0 \\ \overline{\rho} R_1 \leqslant R_2 \leqslant \underline{\rho} R_1 \end{matrix} \right. $ is equivalent to $ \left\{ \begin{matrix} Z_{N 1}^* < 0 \\ Z_{N 2}^* = 0 \end{matrix} \right. $

In Case 1: $ 0 < {\hat \rho} $, \hhred{$ \left\{ \begin{matrix} R_1 < 0 \\ \overline{\rho} R_1 \leqslant R_2 < {\hat \rho} R_1 \end{matrix} \right. $ is equivalent to $ \left\{ \begin{matrix} Z_{N 1}^* - Z_{S 1}^* < 0 \\ Z_{N 2}^* - Z_{S 2}^* > 0 \end{matrix} \right. $ or $ \left\{ \begin{matrix} Z_{N 1}^* < Z_{S 1}^* \\ 0 = Z_{N 2}^* > Z_{S 2}^* \end{matrix} \right. $.}
$ \left\{ \begin{matrix} R_1 < 0 \\ {\hat \rho} R_1 < R_2 \leqslant \underline{\rho} R_1 \end{matrix} \right. $ is equivalent to $ \left\{ \begin{matrix} Z_{N 1}^* - Z_{S 1}^* > 0 \\ Z_{N 2}^* - Z_{S 2}^* < 0 \end{matrix} \right. $ or $ \left\{ \begin{matrix} 0 > Z_{N 1}^* > Z_{S 1}^* \\ 0 = Z_{N 2}^* < Z_{S 2}^* \end{matrix} \right. $.

In Case 2: $ 0 = {\hat \rho} = 0 $, $ \left\{ \begin{matrix} R_1 < 0 \\ \overline{\rho} R_1 \leqslant R_2 < 0 \end{matrix} \right. $ is equivalent to $ \left\{ \begin{matrix} Z_{N 1}^* - Z_{S 1}^* = 0 \\ Z_{N 2}^* - Z_{S 2}^* > 0 \end{matrix} \right. $ or $ \left\{ \begin{matrix} 0 > Z_{N 1}^* = Z_{S 1}^* \\ 0 = Z_{N 2}^* > Z_{S 2}^* \end{matrix} \right. $.
$ \left\{ \begin{matrix} R_1 < 0 \\ 0 < R_2 \leqslant \underline{\rho} R_1 \end{matrix} \right. $ is equivalent to $ \left\{ \begin{matrix} Z_{N 1}^* - Z_{S 1}^* = 0 \\ Z_{N 2}^* - Z_{S 2}^* < 0 \end{matrix} \right. $ or $ \left\{ \begin{matrix} 0 > Z_{N 1}^* = Z_{S 1}^* \\ 0 = Z_{N 2}^* < Z_{S 2}^* \end{matrix} \right. $.

In Case 3: $ {\hat \rho} < 0 $, $ \left\{ \begin{matrix} R_1 < 0 \\ \overline{\rho} R_1 \leqslant R_2 < {\hat \rho} R_1 \end{matrix} \right. $ is equivalent to $ \left\{ \begin{matrix} Z_{N 1}^* - Z_{S 1}^* > 0 \\ Z_{N 2}^* - Z_{S 2}^* > 0 \end{matrix} \right. $ or $ \left\{ \begin{matrix} 0 > Z_{N 1}^* > Z_{S 1}^* \\ 0 = Z_{N 2}^* > Z_{S 2}^* \end{matrix} \right. $.
\hhred{$ \left\{ \begin{matrix} R_1 < 0 \\ {\hat \rho} R_1 < R_2 \leqslant \underline{\rho} R_1 \end{matrix} \right. $ is equivalent to $ \left\{ \begin{matrix} Z_{N 1}^* - Z_{S 1}^* < 0 \\ Z_{N 2}^* - Z_{S 2}^* < 0 \end{matrix} \right. $ or $ \left\{ \begin{matrix} Z_{N 1}^* < Z_{S 1}^* \\ 0 = Z_{N 2}^* < Z_{S 2}^* \end{matrix} \right. $.}

{\it Setting 6}: $ (N+.1) \left\{ \begin{matrix} R_1 > 0 \\ \underline{\rho} R_1 \leqslant R_2 \leqslant \overline{\rho} R_1 \end{matrix} \right. $ is equivalent to $ \left\{ \begin{matrix} Z_{N 1}^* > 0 \\ Z_{N 2}^* = 0 \end{matrix} \right. $

In Case 1: $ 0 < {\hat \rho} $, \hhred{$ \left\{ \begin{matrix} R_1 > 0 \\ \underline{\rho} R_1 \leqslant R_2 < {\hat \rho} R_1 \end{matrix} \right. $ is equivalent to $ \left\{ \begin{matrix} Z_{N 1}^* - Z_{S 1}^* < 0 \\ Z_{N 2}^* - Z_{S 2}^* > 0 \end{matrix} \right. $ or $ \left\{ \begin{matrix} Z_{N 1}^* < Z_{S 1}^* \\ 0 = Z_{N 2}^* > Z_{S 2}^* \end{matrix} \right. $.}
$ \left\{ \begin{matrix} R_1 > 0 \\ {\hat \rho} R_1 < R_2 \leqslant \overline{\rho} R_1 \end{matrix} \right. $ is equivalent to $ \left\{ \begin{matrix} Z_{N 1}^* - Z_{S 1}^* > 0 \\ Z_{N 2}^* - Z_{S 2}^* < 0 \end{matrix} \right. $ or $ \left\{ \begin{matrix} 0 > Z_{N 1}^* > Z_{S 1}^* \\ 0 = Z_{N 2}^* < Z_{S 2}^* \end{matrix} \right. $.

In Case 2: $ 0 = {\hat \rho} = 0 $, $ \left\{ \begin{matrix} R_1 < 0 \\ \underline{\rho} R_1 \leqslant R_2 < 0 \end{matrix} \right. $ is equivalent to $ \left\{ \begin{matrix} Z_{N 1}^* - Z_{S 1}^* = 0 \\ Z_{N 2}^* - Z_{S 2}^* > 0 \end{matrix} \right. $ or $ \left\{ \begin{matrix} 0 > Z_{N 1}^* = Z_{S 1}^* \\ 0 = Z_{N 2}^* > Z_{S 2}^* \end{matrix} \right. $.
$ \left\{ \begin{matrix} R_1 < 0 \\ 0 < R_2 \leqslant \overline{\rho} R_1 \end{matrix} \right. $ is equivalent to $ \left\{ \begin{matrix} Z_{N 1}^* - Z_{S 1}^* = 0 \\ Z_{N 2}^* - Z_{S 2}^* < 0 \end{matrix} \right. $ or $ \left\{ \begin{matrix} 0 > Z_{N 1}^* = Z_{S 1}^* \\ 0 = Z_{N 2}^* < Z_{S 2}^* \end{matrix} \right. $.

In Case 3: $ {\hat \rho} < 0 $, $ \left\{ \begin{matrix} R_1 > 0 \\ \underline{\rho} R_1 \leqslant R_2 < {\hat \rho} R_1 \end{matrix} \right. $ is equivalent to $ \left\{ \begin{matrix} Z_{N 1}^* - Z_{S 1}^* > 0 \\ Z_{N 2}^* - Z_{S 2}^* > 0 \end{matrix} \right. $ or $ \left\{ \begin{matrix} 0 > Z_{N 1}^* > Z_{S 1}^* \\ 0 = Z_{N 2}^* > Z_{S 2}^* \end{matrix} \right. $.
\hhred{$ \left\{ \begin{matrix} R_1 > 0 \\ {\hat \rho} R_1 < R_2 \leqslant \overline{\rho} R_1 \end{matrix} \right. $ is equivalent to $ \left\{ \begin{matrix} Z_{N 1}^* - Z_{S 1}^* < 0 \\ Z_{N 2}^* - Z_{S 2}^* < 0 \end{matrix} \right. $ or $ \left\{ \begin{matrix} Z_{N 1}^* < Z_{S 1}^* \\ 0 = Z_{N 2}^* < Z_{S 2}^* \end{matrix} \right. $.}

{\bf Scenario 4}: $ (N-.2) \left\{ \begin{matrix} \overline{\rho} R_2 \leqslant R_1 \leqslant \underline{\rho} R_2 \\ R_2 < 0 \end{matrix} \right. $ or $ (N+.2) \left\{ \begin{matrix} \underline{\rho} R_2 \leqslant R_1 \leqslant \overline{\rho} R_2 \\ R_2 > 0 \end{matrix} \right. $
\begin{eqnarray*}
\alpha Z_S^* = \dfrac1{1 - {\hat \rho}^2} \left( \begin{matrix} \dfrac{R_1 - {\hat \rho} R_2}{\sigma_1} \\ \dfrac{R_2 - {\hat \rho} R_1}{\sigma_2} \end{matrix} \right) \qquad \text{and} \qquad \alpha Z_N^* = \left( \begin{matrix} 0 \\ \dfrac{R_2}{\sigma_2} \end{matrix} \right)
\end{eqnarray*}
\begin{eqnarray*}
\alpha (Z_N^* - Z_S^*) = \left( \begin{matrix} - \dfrac{R_1 - {\hat \rho} R_2}{\sigma_1 (1 - {\hat \rho}^2)} \\ \dfrac{R_2}{\sigma_2} - \dfrac{R_2 - {\hat \rho} R_1}{\sigma_2 (1 - {\hat \rho}^2)} \end{matrix} \right) = \left( \begin{matrix} - \dfrac{R_1 - {\hat \rho} R_2}{\sigma_1 (1 - {\hat \rho}^2)} \\ \dfrac{{\hat \rho} (R_1 - {\hat \rho} R_2)}{\sigma_2 (1 - {\hat \rho}^2)} \end{matrix} \right) = \dfrac{R_1 - {\hat \rho} R_2}{1 - {\hat \rho}^2} \left( \begin{matrix} - \dfrac{1}{\sigma_1} \\ \dfrac{{\hat \rho}}{\sigma_2} \end{matrix} \right)
\end{eqnarray*}

{\it Setting 7}: $ (N-.2) \left\{ \begin{matrix} \overline{\rho} R_2 \leqslant R_1 \leqslant \underline{\rho} R_2 \\ R_2 < 0 \end{matrix} \right. $ is equivalent to $ \left\{ \begin{matrix} Z_{N 1}^* = 0 \\ Z_{N 2}^* < 0 \end{matrix} \right. $

In Case 1: $ 0 < {\hat \rho} $, \hhred{$ \left\{ \begin{matrix} \overline{\rho} R_2 \leqslant R_1 < {\hat \rho} R_2 \\ R_2 < 0 \end{matrix} \right. $ is equivalent to $ \left\{ \begin{matrix} Z_{N 1}^* - Z_{S 1}^* > 0 \\ Z_{N 2}^* - Z_{S 2}^* < 0 \end{matrix} \right. $ or $ \left\{ \begin{matrix} 0 = Z_{N 1}^* > Z_{S 1}^* \\ Z_{N 2}^* < Z_{S 2}^* \end{matrix} \right. $.} $ \left\{ \begin{matrix} {\hat \rho} R_2 < R_1 \leqslant \underline{\rho} R_2 \\ R_2 < 0 \end{matrix} \right. $ is equivalent to $ \left\{ \begin{matrix} Z_{N 1}^* - Z_{S 1}^* < 0 \\ Z_{N 2}^* - Z_{S 2}^* > 0 \end{matrix} \right. $ or $ \left\{ \begin{matrix} 0 = Z_{N 1}^* < Z_{S 1}^* \\ 0 > Z_{N 2}^* > Z_{S 2}^* \end{matrix} \right. $.

In Case 2: $ 0 = {\hat \rho} = 0 $, $ \left\{ \begin{matrix} \overline{\rho} R_2 \leqslant R_1 < 0 \\ R_2 < 0 \end{matrix} \right. $ is equivalent to $ \left\{ \begin{matrix} Z_{N 1}^* - Z_{S 1}^* > 0 \\ Z_{N 2}^* - Z_{S 2}^* = 0 \end{matrix} \right. $ or $ \left\{ \begin{matrix} 0 = Z_{N 1}^* > Z_{S 1}^* \\ 0 > Z_{N 2}^* = Z_{S 2}^* \end{matrix} \right. $. $ \left\{ \begin{matrix} 0 < R_1 \leqslant \underline{\rho} R_2 \\ R_2 < 0 \end{matrix} \right. $ is equivalent to $ \left\{ \begin{matrix} Z_{N 1}^* - Z_{S 1}^* < 0 \\ Z_{N 2}^* - Z_{S 2}^* = 0 \end{matrix} \right. $ or $ \left\{ \begin{matrix} 0 = Z_{N 1}^* < Z_{S 1}^* \\ 0 > Z_{N 2}^* = Z_{S 2}^* \end{matrix} \right. $.

In Case 3: $ {\hat \rho} < 0 $, $ \left\{ \begin{matrix} \overline{\rho} R_2 \leqslant R_1 < {\hat \rho} R_2 \\ R_2 < 0 \end{matrix} \right. $ is equivalent to $ \left\{ \begin{matrix} Z_{N 1}^* - Z_{S 1}^* > 0 \\ Z_{N 2}^* - Z_{S 2}^* > 0 \end{matrix} \right. $ or $ \left\{ \begin{matrix} 0 = Z_{N 1}^* > Z_{S 1}^* \\ 0 > Z_{N 2}^* > Z_{S 2}^* \end{matrix} \right. $. \hhred{$ \left\{ \begin{matrix} {\hat \rho} R_2 < R_1 \leqslant \underline{\rho} R_2 \\ R_2 < 0 \end{matrix} \right. $ is equivalent to $ \left\{ \begin{matrix} Z_{N 1}^* - Z_{S 1}^* < 0 \\ Z_{N 2}^* - Z_{S 2}^* < 0 \end{matrix} \right. $ or $ \left\{ \begin{matrix} 0 = Z_{N 1}^* < Z_{S 1}^* \\ Z_{N 2}^* < Z_{S 2}^* \end{matrix} \right. $.}

{\it Setting 8}: $ (N+.2) \left\{ \begin{matrix} \underline{\rho} R_2 \leqslant R_1 \leqslant \overline{\rho} R_2 \\ R_2 > 0 \end{matrix} \right. $ is equivalent to $ \left\{ \begin{matrix} Z_{N 1}^* = 0 \\ Z_{N 2}^* > 0 \end{matrix} \right. $

In Case 1: $ 0 < {\hat \rho} $, \hhred{$ \left\{ \begin{matrix} \underline{\rho} R_2 \leqslant R_1 < {\hat \rho} R_2 \\ R_2 > 0 \end{matrix} \right. $ is equivalent to $ \left\{ \begin{matrix} Z_{N 1}^* - Z_{S 1}^* < 0 \\ Z_{N 2}^* - Z_{S 2}^* > 0 \end{matrix} \right. $ or $ \left\{ \begin{matrix} 0 = Z_{N 1}^* < Z_{S 1}^* \\ Z_{N 2}^* > Z_{S 2}^* \end{matrix} \right. $.} $ \left\{ \begin{matrix} {\hat \rho} R_2 < R_1 \leqslant \overline{\rho} R_2 \\ R_2 > 0 \end{matrix} \right. $ is equivalent to $ \left\{ \begin{matrix} Z_{N 1}^* - Z_{S 1}^* > 0 \\ Z_{N 2}^* - Z_{S 2}^* < 0 \end{matrix} \right. $ or $ \left\{ \begin{matrix} 0 = Z_{N 1}^* > Z_{S 1}^* \\ 0 < Z_{N 2}^* < Z_{S 2}^* \end{matrix} \right. $.

In Case 2: $ 0 = {\hat \rho} = 0 $, $ \left\{ \begin{matrix} \underline{\rho} R_2 \leqslant R_1 < 0 \\ R_2 > 0 \end{matrix} \right. $ is equivalent to $ \left\{ \begin{matrix} Z_{N 1}^* - Z_{S 1}^* < 0 \\ Z_{N 2}^* - Z_{S 2}^* = 0 \end{matrix} \right. $ or $ \left\{ \begin{matrix} 0 = Z_{N 1}^* < Z_{S 1}^* \\ 0 < Z_{N 2}^* = Z_{S 2}^* \end{matrix} \right. $. $ \left\{ \begin{matrix} 0 < R_1 \leqslant \overline{\rho} R_2 \\ R_2 > 0 \end{matrix} \right. $ is equivalent to $ \left\{ \begin{matrix} Z_{N 1}^* - Z_{S 1}^* > 0 \\ Z_{N 2}^* - Z_{S 2}^* = 0 \end{matrix} \right. $ or $ \left\{ \begin{matrix} 0 = Z_{N 1}^* > Z_{S 1}^* \\ 0 < Z_{N 2}^* = Z_{S 2}^* \end{matrix} \right. $.

In Case 3: $ {\hat \rho} < 0 $, $ \left\{ \begin{matrix} \underline{\rho} R_2 \leqslant R_1 < {\hat \rho} R_2 \\ R_2 > 0 \end{matrix} \right. $ is equivalent to $ \left\{ \begin{matrix} Z_{N 1}^* - Z_{S 1}^* < 0 \\ Z_{N 2}^* - Z_{S 2}^* < 0 \end{matrix} \right. $ or $ \left\{ \begin{matrix} 0 = Z_{N 1}^* < Z_{S 1}^* \\ 0 < Z_{N 2}^* < Z_{S 2}^* \end{matrix} \right. $. \hhred{$ \left\{ \begin{matrix} {\hat \rho} R_2 < R_1 \leqslant \overline{\rho} R_2 \\ R_2 > 0 \end{matrix} \right. $ is equivalent to $ \left\{ \begin{matrix} Z_{N 1}^* - Z_{S 1}^* > 0 \\ Z_{N 2}^* - Z_{S 2}^* > 0 \end{matrix} \right. $ or $ \left\{ \begin{matrix} 0 = Z_{N 1}^* > Z_{S 1}^* \\ Z_{N 2}^* > Z_{S 2}^* \end{matrix} \right. $.}

\subsection*{A.5 \quad Equilibrium conditions}

\quad \ 
For simplicity, we first adopt the demands from equations (6) and (10) in equilibrium to solve the Sharpe ratio $ R_i $ for $ i = 1, 2 $. According to equation (10), we follow the four possible scenarios to calculate equilibrium Sharpe ratios.

\subsubsection*{A.5.1 \quad No equilibrium for scenario 1: $ Z_{N 1}^* Z_{N 2}^* < 0 $}

\quad \ 
Scenario 1: $ Z_{N 1}^* Z_{N 2}^* < 0 $ is split into two settings: $ \left\{ \begin{matrix} Z_{N 1}^* < 0 \\ Z_{N 2}^* > 0 \end{matrix} \right. $ and $ \left\{ \begin{matrix} Z_{N 1}^* > 0 \\ Z_{N 2}^* < 0 \end{matrix} \right. $, which correspond to the environments in equation (10): $ \left\{ \begin{matrix} R_1 < \underline{\rho} R_2 \\ R_2 > \underline{\rho} R_1 \end{matrix} \right. $ and $ \left\{ \begin{matrix} R_1 > \underline{\rho} R_2 \\ R_2 < \underline{\rho} R_1 \end{matrix} \right. $; then we obtain the equilibrium condition:
\begin{eqnarray}
(1 - \theta) \left( \begin{matrix} \dfrac{R_1 - \hat \rho R_2}{\alpha \sigma_1 (1 - {\hat \rho}^2)} \\ \dfrac{R_2 - \hat \rho R_1}{\alpha \sigma_2 (1 - {\hat \rho}^2)} \end{matrix} \right) + \theta \left( \begin{matrix} \dfrac{R_1 - \underline{\rho} R_2}{\alpha \sigma_1 (1 - \underline{\rho}^2)} \\ \dfrac{R_2 - \underline{\rho} R_1}{\alpha \sigma_2 (1 - \underline{\rho}^2)} \end{matrix} \right) = \left( \begin{matrix} Z_1^0 \\ Z_2^0 \end{matrix} \right)
\end{eqnarray}
or
\begin{eqnarray*}
& \quad \left[ \dfrac{1 - \theta}{1 - {\hat \rho}^2} + \dfrac{\theta}{1 - \underline{\rho}^2} \right] R_1 - \left[ \dfrac{1 - \theta}{1 - {\hat \rho}^2} {\hat \rho} + \dfrac{\theta}{1 - \underline{\rho}^2} \underline{\rho} \right] R_2 = \alpha \sigma_1 Z_1^0 & \\
& - \left[ \dfrac{1 - \theta}{1 - {\hat \rho}^2} {\hat \rho} + \dfrac{\theta}{1 - \underline{\rho}^2} \underline{\rho} \right] R_1 + \left[ \dfrac{1 - \theta}{1 - {\hat \rho}^2} + \dfrac{\theta}{1 - \underline{\rho}^2} \right] R_2 = \alpha \sigma_2 Z_2^0. & 
\end{eqnarray*}
The equilibrium Sharpe ratios are given by:
\begin{eqnarray}
& R_1 = \alpha \dfrac{\left[ \dfrac{1 - \theta}{1 - {\hat \rho}^2} + \dfrac{\theta}{1 - \underline{\rho}^2} \right] \sigma_1 Z_1^0 + \left[ \dfrac{1 - \theta}{1 - {\hat \rho}^2} {\hat \rho} + \dfrac{\theta}{1 - \underline{\rho}^2} \underline{\rho} \right] \sigma_2 Z_2^0}{\left[ \dfrac{1 - \theta}{1 - {\hat \rho}^2} + \dfrac{\theta}{1 - \underline{\rho}^2} \right]^2 - \left[ \dfrac{1 - \theta}{1 - {\hat \rho}^2} {\hat \rho} + \dfrac{\theta}{1 - \underline{\rho}^2} \underline{\rho} \right]^2} & \\
& R_2 = \alpha \dfrac{\left[ \dfrac{1 - \theta}{1 - {\hat \rho}^2} {\hat \rho} + \dfrac{\theta}{1 - \underline{\rho}^2} \underline{\rho} \right] \sigma_1 Z_1^0 + \left[ \dfrac{1 - \theta}{1 - {\hat \rho}^2} + \dfrac{\theta}{1 - \underline{\rho}^2} \right] \sigma_2 Z_2^0}{\left[ \dfrac{1 - \theta}{1 - {\hat \rho}^2} + \dfrac{\theta}{1 - \underline{\rho}^2} \right]^2 - \left[ \dfrac{1 - \theta}{1 - {\hat \rho}^2} {\hat \rho} + \dfrac{\theta}{1 - \underline{\rho}^2} \underline{\rho} \right]^2}. &
\end{eqnarray}

$ R_1 < \underline{\rho} R_2 $ is equivalent to 
{\footnotesize \begin{eqnarray*}
\left[ \dfrac{1 - \theta}{1 - {\hat \rho}^2} + \dfrac{\theta}{1 - \underline{\rho}^2} \right] \sigma_1 Z_1^0 + \left[ \dfrac{1 - \theta}{1 - {\hat \rho}^2} {\hat \rho} + \dfrac{\theta}{1 - \underline{\rho}^2} \underline{\rho} \right] \sigma_2 Z_2^0 < \underline{\rho} \left\{ \left[ \dfrac{1 - \theta}{1 - {\hat \rho}^2} {\hat \rho} + \dfrac{\theta}{1 - \underline{\rho}^2} \underline{\rho} \right] \sigma_1 Z_1^0 + \left[ \dfrac{1 - \theta}{1 - {\hat \rho}^2} + \dfrac{\theta}{1 - \underline{\rho}^2} \right] \sigma_2 Z_2^0 \right\},
\end{eqnarray*}}
which implies a contradiction:
\begin{eqnarray*}
0 < \left[ \dfrac{1 - \theta}{1 - {\hat \rho}^2} (1 - \underline{\rho} {\hat \rho}) + \theta \right] \sigma_1 Z_1^0 < \dfrac{1 - \theta}{1 - {\hat \rho}^2} (\underline{\rho} - {\hat \rho}) \sigma_2 Z_2^0 \leqslant 0.
\end{eqnarray*}
$ R_2 < \underline{\rho} R_1 $ is equivalent to 
{\footnotesize \begin{eqnarray*}
\left[ \dfrac{1 - \theta}{1 - {\hat \rho}^2} {\hat \rho} + \dfrac{\theta}{1 - \underline{\rho}^2} \underline{\rho} \right] \sigma_1 Z_1^0 + \left[ \dfrac{1 - \theta}{1 - {\hat \rho}^2} + \dfrac{\theta}{1 - \underline{\rho}^2} \right] \sigma_2 Z_2^0 < \underline{\rho} \left\{ \left[ \dfrac{1 - \theta}{1 - {\hat \rho}^2} + \dfrac{\theta}{1 - \underline{\rho}^2} \right] \sigma_1 Z_1^0 + \left[ \dfrac{1 - \theta}{1 - {\hat \rho}^2} {\hat \rho} + \dfrac{\theta}{1 - \underline{\rho}^2} \underline{\rho} \right] \sigma_2 Z_2^0 \right\},
\end{eqnarray*}}
which implies a contradiction:
\begin{eqnarray*}
0 \leqslant \dfrac{1 - \theta}{1 - {\hat \rho}^2} ({\hat \rho} - \underline{\rho}) \sigma_1 Z_1^0 < - \left[ \dfrac{1 - \theta}{1 - {\hat \rho}^2} (1 - \underline{\rho} {\hat \rho}) + \theta \right] \sigma_2 Z_2^0 < 0.
\end{eqnarray*}

The two fallacies negate the existence of equilibrium.

\vskip 8 pt

{\bf Proposition A5.1}. If na\"ive investors short sell one risky asset and long buy another one, $ Z_{N 1}^* Z_{N 2}^* < 0 $, then there is not general equilibrium.

\subsubsection*{A.5.2 \quad Equilibrium conditions for scenario 2: $ Z_{N 1}^* = 0 $}

\quad \ 
Scenario $ Z_{N 1}^* = 0 $ is split into two settings: $ \left\{ \begin{matrix} Z_{N 1}^* = 0 \\ Z_{N 2}^* < 0 \end{matrix} \right. $ and $ \left\{ \begin{matrix} Z_{N 1}^* = 0 \\ Z_{N 2}^* > 0 \end{matrix} \right. $, which correspond to the environments in equation (10): $ \left\{ \begin{matrix} \overline{\rho} R_2 \leqslant R_1 \leqslant \underline{\rho} R_2 \\ R_2 < 0 \end{matrix} \right. $ and $ \left\{ \begin{matrix} \underline{\rho} R_2 \leqslant R_1 \leqslant \overline{\rho} R_2 \\ R_2 > 0 \end{matrix} \right. $; then we obtain the equilibrium condition:
\begin{eqnarray}
(1 - \theta) \left( \begin{matrix} \dfrac{R_1 - {\hat \rho} R_2}{\alpha \sigma_1 (1 - \hat \rho^2)} \\ \dfrac{R_2 - {\hat \rho} R_1}{\alpha \sigma_2 (1 - {\hat \rho}^2)} \end{matrix} \right) + \theta \left( \begin{matrix} 0 \\ \dfrac{R_2}{\alpha \sigma_2} \end{matrix} \right) = \left( \begin{matrix} Z_1^0 \\ Z_2^0 \end{matrix} \right)
\end{eqnarray}
or
%\begin{eqnarray*}
%& (1 - \theta) (R_1 - {\hat \rho} R_2) = \alpha \sigma_1 (1 - {\hat \rho}^2) Z_1^0 & \\
%& (1 - \theta) (R_2 - {\hat \rho} R_1) + \theta (1 - {\hat \rho}^2) R_2 = \alpha \sigma_2 (1 - {\hat \rho}^2) Z_2^0 &
%\end{eqnarray*}
%that is,
\begin{eqnarray*}
& (1 - \theta) R_1 - (1 - \theta) {\hat \rho} R_2 = \alpha (1 - {\hat \rho}^2) \sigma_1 Z_1^0 & \\
& - (1 - \theta) {\hat \rho} R_1 + (1 - \theta {\hat \rho}^2) R_2 = \alpha (1 - {\hat \rho}^2) \sigma_2 Z_2^0. &
\end{eqnarray*}
The equilibrium Sharpe ratios are given by:
\begin{eqnarray}
%& R_1 = \dfrac{\alpha (1 - {\hat \rho}^2) \sigma_1 Z_1^0}{1 - \theta} + {\hat \rho} R_2 = \alpha \dfrac{(1 - \theta {\hat \rho}^2) \sigma_1 Z_1^0 + (1 - \theta) {\hat \rho} \sigma_2 Z_2^0}{1 - \theta} & \\
& R_1 = \alpha \dfrac{(1 - \theta {\hat \rho}^2) \sigma_1 Z_1^0 + (1 - \theta) {\hat \rho} \sigma_2 Z_2^0}{1 - \theta} & \\
& R_2 = \alpha ({\hat \rho} \sigma_1 Z_1^0 + \sigma_2 Z_2^0). &
\end{eqnarray}

%$ R_2 < 0 $ is equivalent to $ {\hat \rho} \sigma_1 Z_1^0 + \sigma_2 Z_2^0 < 0 $ or $ {\hat \rho} < - \dfrac{\sigma_2 Z_2^0}{\sigma_1 Z_1^0} $.
$ R_2 < 0 $ is equivalent to $ {\hat \rho} < - \dfrac{\sigma_2 Z_2^0}{\sigma_1 Z_1^0} $.
$ \overline{\rho} R_2 \leqslant R_1 \leqslant \underline{\rho} R_2 $ is equivalent to 
\begin{eqnarray*}
\overline{\rho} ({\hat \rho} \sigma_1 Z_1^0 + \sigma_2 Z_2^0) \leqslant \dfrac{(1 - \theta {\hat \rho}^2) \sigma_1 Z_1^0 + (1 - \theta) {\hat \rho} \sigma_2 Z_2^0}{1 - \theta} \leqslant \underline{\rho} ({\hat \rho} \sigma_1 Z_1^0 + \sigma_2 Z_2^0)
\end{eqnarray*}
or
\begin{eqnarray*}
\dfrac{(1 - \theta) \overline{\rho} {\hat \rho} - (1 - \theta {\hat \rho}^2)}{(1 - \theta) (\overline{\rho} - {\hat \rho})} \leqslant - \dfrac{\sigma_2 Z_2^0}{\sigma_1 Z_1^0} \qquad \text{and} \qquad \dfrac{(1 - \theta {\hat \rho}^2) - (1 - \theta) \underline{\rho} {\hat \rho}}{(1 - \theta) ({\hat \rho} - \underline{\rho})} \leqslant - \dfrac{\sigma_2 Z_2^0}{\sigma_1 Z_1^0}.
\end{eqnarray*}
%\begin{eqnarray*}
%[(1 - \theta) \overline{\rho} {\hat \rho} - (1 - \theta {\hat \rho}^2)] \sigma_1 Z_1^0 \leqslant (1 - \theta) ({\hat \rho} - \overline{\rho}) \sigma_2 Z_2^0 \qquad \dfrac{(1 - \theta) \overline{\rho} {\hat \rho} - (1 - \theta {\hat \rho}^2)}{(1 - \theta) (\overline{\rho} - {\hat \rho})} \leqslant - \dfrac{\sigma_2 Z_2^0}{\sigma_1 Z_1^0}
%\end{eqnarray*} 
%\begin{eqnarray*}
%[(1 - \theta {\hat \rho}^2) - (1 - \theta) \underline{\rho} {\hat \rho}] \sigma_1 Z_1^0 \leqslant (1 - \theta) (\underline{\rho} - {\hat \rho}) \sigma_2 Z_2^0 \qquad \dfrac{(1 - \theta {\hat \rho}^2) - (1 - \theta) \underline{\rho} {\hat \rho}}{(1 - \theta) ({\hat \rho} - \underline{\rho})} \leqslant - \dfrac{\sigma_2 Z_2^0}{\sigma_1 Z_1^0}
%\end{eqnarray*}
Since $ \dfrac{(1 - \theta) \overline{\rho} {\hat \rho} - (1 - \theta {\hat \rho}^2)}{(1 - \theta) (\overline{\rho} - {\hat \rho})} < {\hat \rho} < \dfrac{(1 - \theta {\hat \rho}^2) - (1 - \theta) \underline{\rho} {\hat \rho}}{(1 - \theta) ({\hat \rho} - \underline{\rho})} $, then $ \left\{ \begin{matrix} R_2 < 0 \\ \overline{\rho} R_2 \leqslant R_1 \leqslant \underline{\rho} R_2 \end{matrix} \right. $ is equivalent to $ \dfrac{(1 - \theta {\hat \rho}^2) - (1 - \theta) \underline{\rho} {\hat \rho}}{(1 - \theta) ({\hat \rho} - \underline{\rho})} \leqslant - \dfrac{\sigma_2 Z_2^0}{\sigma_1 Z_1^0} $, which implies a contradiction: $ 1 < \dfrac{1 - \theta {\hat \rho}^2}{1 - \theta} < \underline{\rho} {\hat \rho} $.

%$ R_2 > 0 $ is equivalent to $ {\hat \rho} \sigma_1 Z_1^0 + \sigma_2 Z_2^0 > 0 $ or $ - {\hat \rho} < \dfrac{\sigma_2 Z_2^0}{\sigma_1 Z_1^0} $.
$ R_2 > 0 $ is equivalent to $ - {\hat \rho} < \dfrac{\sigma_2 Z_2^0}{\sigma_1 Z_1^0} $.
$ \underline{\rho} R_2 \leqslant R_1 \leqslant \overline{\rho} R_2 $ is equivalent to 
\begin{eqnarray*}
\underline{\rho} ({\hat \rho} \sigma_1 Z_1^0 + \sigma_2 Z_2^0) \leqslant \dfrac{(1 - \theta {\hat \rho}^2) \sigma_1 Z_1^0 + (1 - \theta) {\hat \rho} \sigma_2 Z_2^0}{1 - \theta} \leqslant \overline{\rho} ({\hat \rho} \sigma_1 Z_1^0 + \sigma_2 Z_2^0)
\end{eqnarray*}
or
\begin{eqnarray*}
\dfrac{(1 - \theta) \underline{\rho} {\hat \rho} - (1 - \theta {\hat \rho}^2)}{(1 - \theta) ({\hat \rho} - \underline{\rho})} \leqslant \dfrac{\sigma_2 Z_2^0}{\sigma_1 Z_1^0} \qquad \text{and} \qquad \dfrac{(1 - \theta {\hat \rho}^2) - (1 - \theta) \overline{\rho} {\hat \rho}}{(1 - \theta) (\overline{\rho} - {\hat \rho})} \leqslant \dfrac{\sigma_2 Z_2^0}{\sigma_1 Z_1^0}.
\end{eqnarray*}
%\begin{eqnarray*}
%[(1 - \theta) \underline{\rho} {\hat \rho} - (1 - \theta {\hat \rho}^2)] \sigma_1 Z_1^0 \leqslant (1 - \theta) ({\hat \rho} - \underline{\rho}) \sigma_2 Z_2^0 \qquad \dfrac{(1 - \theta) \underline{\rho} {\hat \rho} - (1 - \theta {\hat \rho}^2)}{(1 - \theta) ({\hat \rho} - \underline{\rho})} \leqslant \dfrac{\sigma_2 Z_2^0}{\sigma_1 Z_1^0}
%\end{eqnarray*} 
%\begin{eqnarray*}
%[(1 - \theta {\hat \rho}^2) - (1 - \theta) \overline{\rho} {\hat \rho}] \sigma_1 Z_1^0 \leqslant (1 - \theta) (\overline{\rho} - {\hat \rho}) \sigma_2 Z_2^0 \qquad \dfrac{(1 - \theta {\hat \rho}^2) - (1 - \theta) \overline{\rho} {\hat \rho}}{(1 - \theta) (\overline{\rho} - {\hat \rho})} \leqslant \dfrac{\sigma_2 Z_2^0}{\sigma_1 Z_1^0}
%\end{eqnarray*}
Since $ \dfrac{(1 - \theta) \underline{\rho} {\hat \rho} - (1 - \theta {\hat \rho}^2)}{(1 - \theta) ({\hat \rho} - \underline{\rho})} < - {\hat \rho} < \dfrac{(1 - \theta {\hat \rho}^2) - (1 - \theta) \overline{\rho} {\hat \rho}}{(1 - \theta) (\overline{\rho} - {\hat \rho})} $, then $ \left\{ \begin{matrix} R_2 > 0 \\ \underline{\rho} R_2 \leqslant R_1 \leqslant \overline{\rho} R_2 \end{matrix} \right. $ is equivalent to $ \dfrac{(1 - \theta {\hat \rho}^2) - (1 - \theta) \overline{\rho} {\hat \rho}}{(1 - \theta) (\overline{\rho} - {\hat \rho})} \leqslant \dfrac{\sigma_2 Z_2^0}{\sigma_1 Z_1^0} $.
%In this setting, equilibrium Sharpe ratios (A.5.5) and (A.5.6) can be written as
%\begin{eqnarray*}
%& \dfrac{\mu_1 - p_1}{\sigma_1} = R_1 = \alpha \dfrac{(1 - \theta {\hat \rho}^2) \sigma_1 Z_1^0 + (1 - \theta) {\hat \rho} \sigma_2 Z_2^0}{1 - \theta} & \\
%& \dfrac{\mu_2 - p_2}{\sigma_2} = R_2 = \alpha ({\hat \rho} \sigma_1 Z_1^0 + \sigma_2 Z_2^0) &
%\end{eqnarray*}
%then we obtain equilibrium prices for the two risky assets
%\begin{eqnarray}
%& p_1 = \mu_1 - \alpha \sigma_1 \dfrac{(1 - \theta {\hat \rho}^2) \sigma_1 Z_1^0 + (1 - \theta) {\hat \rho} \sigma_2 Z_2^0}{1 - \theta} & \\
%& p_2 = \mu_2 - \alpha \sigma_2 ({\hat \rho} \sigma_1 Z_1^0 + \sigma_2 Z_2^0) &
%\end{eqnarray}
In this setting, equilibrium Sharpe ratios (A.33) and (A.34) imply equilibrium prices for the two risky assets:
\begin{eqnarray}
& p_1 = \mu_1 - \alpha \sigma_1 \dfrac{(1 - \theta {\hat \rho}^2) \sigma_1 Z_1^0 + (1 - \theta) {\hat \rho} \sigma_2 Z_2^0}{1 - \theta} & \\
& p_2 = \mu_2 - \alpha \sigma_2 ({\hat \rho} \sigma_1 Z_1^0 + \sigma_2 Z_2^0) &
\end{eqnarray}
in which na\"ive investors will buy risky asset 2 but not trade risky asset 1. In equilibrium, a sophisticated investor holds positions of risky assets:
\begin{eqnarray}
& Z_{S 1}^* = \dfrac{R_1 - \hat \rho R_2}{\alpha \sigma_1 (1 - \hat \rho^2)} = \dfrac{Z_1^0}{1 - \theta} & \\
& Z_{S 2}^* = \dfrac{R_2 - \hat \rho R_1}{\alpha \sigma_2 (1 - \hat \rho^2)} = Z_2^0 - \dfrac{\theta}{1 - \theta} {\hat \rho} \dfrac{\sigma_1}{\sigma_2} Z_1^0 &
\end{eqnarray}
and a na\"ive investor holds positions of risky assets:
\begin{eqnarray}
& Z_{N 1}^* = 0 & \\
& Z_{N 2}^* = \dfrac{R_2}{\alpha \sigma_2} = \dfrac{{\hat \rho} \sigma_1 Z_1^0 + \sigma_2 Z_2^0}{\sigma_2} = {\hat \rho} \dfrac{\sigma_1}{\sigma_2} Z_1^0 + Z_2^0 > 0. &
\end{eqnarray}

We summarize the above analysis and obtain the existence of equilibrium for $ \dfrac{\sigma_1 Z_1^0}{\sigma_2 Z_2^0} \leqslant \dfrac{(1 - \theta) (\overline{\rho} - {\hat \rho})}{(1 - \theta {\hat \rho}^2) - (1 - \theta) \overline{\rho} {\hat \rho}} $.

\vskip 8 pt

{\bf Proposition A5.2}. If na\"ive investors do not trade risky asset 1, $ Z_{N 1}^* = 0 $, then there is not general equilibrium when na\"ive investors short sell risky asset 2, $ Z_{N 2}^* < 0 $, and there exists general equilibrium when na\"ive investors long buy risky asset 2, $ Z_{N 2}^* > 0 $. The equilibrium prices for the two risky assets are given by (A.35) - (A.36) when $ \dfrac{\sigma_1 Z_1^0}{\sigma_2 Z_2^0} \leqslant \dfrac{(1 - \theta) (\overline{\rho} - {\hat \rho})}{(1 - \theta {\hat \rho}^2) - (1 - \theta) \overline{\rho} {\hat \rho}} $.

\vskip 8 pt

Since $ \dfrac{(1 - \theta) (\overline{\rho} - {\hat \rho})}{(1 - \theta {\hat \rho}^2) - (1 - \theta) \overline{\rho} {\hat \rho}} < \dfrac{1 - \theta}{\theta {\hat \rho}} $ for $ \hat \rho > 0 $, then $ Z_{S 1}^* > 0 $ and $ Z_{S 2}^* > 0 $. In equilibrium, a sophisticated investor holds positive positions of risky assets.

\subsubsection*{A.5.3 \quad Equilibrium conditions for scenario 3: $ Z_{N 2}^* = 0 $}

\quad \ 
Scenario $ Z_{N 2}^* = 0 $ is split into two settings: $ \left\{ \begin{matrix} Z_{N 1}^* < 0 \\ Z_{N 2}^* = 0 \end{matrix} \right. $ and $ \left\{ \begin{matrix} Z_{N 1}^* > 0 \\ Z_{N 2}^* = 0 \end{matrix} \right. $, which correspond to the environments in equation (10): $ \left\{ \begin{matrix} R_1 < 0 \\ \overline{\rho} R_1 \leqslant R_2 \leqslant \underline{\rho} R_1 \end{matrix} \right. $ and $ \left\{ \begin{matrix} R_1 > 0 \\ \underline{\rho} R_1 \leqslant R_2 \leqslant \overline{\rho} R_1 \end{matrix} \right. $; then we obtain the equilibrium condition:
\begin{eqnarray}
(1 - \theta) \left( \begin{matrix} \dfrac{R_1 - \hat \rho R_2}{\alpha \sigma_1 (1 - \hat \rho^2)} \\ \dfrac{R_2 - \hat \rho R_1}{\alpha \sigma_2 (1 - \hat \rho^2)} \end{matrix} \right) + \theta \left( \begin{matrix} \dfrac{R_1}{\alpha \sigma_1} \\ 0 \end{matrix} \right) = \left( \begin{matrix} Z_1^0 \\ Z_2^0 \end{matrix} \right)
\end{eqnarray}
or
%\begin{eqnarray*}
%& (1 - \theta) (R_1 - {\hat \rho} R_2) + \theta (1 - {\hat \rho}^2) R_1 = \alpha \sigma_1 (1 - {\hat \rho}^2) Z_1^0 & \\
%& (1 - \theta) (R_2 - {\hat \rho} R_1) = \alpha \sigma_2 (1 - {\hat \rho}^2) Z_2^0 &
%\end{eqnarray*}
%that is,
\begin{eqnarray*}
& (1 - \theta {\hat \rho}^2) R_1 - (1 - \theta) {\hat \rho} R_2 = \alpha (1 - {\hat \rho}^2) \sigma_1 Z_1^0 & \\
& - (1 - \theta) {\hat \rho} R_1 + (1 - \theta) R_2 = \alpha (1 - {\hat \rho}^2) \sigma_2 Z_2^0. &
\end{eqnarray*}
The equilibrium Sharpe ratios are given by:
\begin{eqnarray}
& R_1 = \alpha (\sigma_1 Z_1^0 + {\hat \rho} \sigma_2 Z_2^0) & \\
& R_2 = \alpha \dfrac{(1 - \theta) {\hat \rho} \sigma_1 Z_1^0 + (1 - \theta {\hat \rho}^2) \sigma_2 Z_2^0}{1 - \theta}. &
%& R_2 = \dfrac{\alpha (1 - {\hat \rho}^2) \sigma_2 Z_2^0}{1 - \theta} + {\hat \rho} R_1 = \alpha \dfrac{(1 - \theta) {\hat \rho} \sigma_1 Z_1^0 + (1 - \theta {\hat \rho}^2) \sigma_2 Z_2^0}{1 - \theta}. &
\end{eqnarray}

%$ R_1 < 0 $ is equivalent to $ \sigma_1 Z_1^0 + {\hat \rho} \sigma_2 Z_2^0 < 0 $ or $ {\hat \rho} < - \dfrac{\sigma_1 Z_1^0}{\sigma_2 Z_2^0} $.
$ R_1 < 0 $ is equivalent to $ {\hat \rho} < - \dfrac{\sigma_1 Z_1^0}{\sigma_2 Z_2^0} $.
$ \overline{\rho} R_1 \leqslant R_2 \leqslant \underline{\rho} R_1 $ is equivalent to 
\begin{eqnarray*}
\overline{\rho} (\sigma_1 Z_1^0 + {\hat \rho} \sigma_2 Z_2^0) \leqslant \dfrac{(1 - \theta) {\hat \rho} \sigma_1 Z_1^0 + (1 - \theta {\hat \rho}^2) \sigma_2 Z_2^0}{1 - \theta} \leqslant \underline{\rho} (\sigma_1 Z_1^0 + {\hat \rho} \sigma_2 Z_2^0)
\end{eqnarray*}
%\begin{eqnarray*}
%[(1 - \theta) \overline{\rho} {\hat \rho} - (1 - \theta {\hat \rho}^2)] \sigma_2 Z_2^0 \leqslant (1 - \theta) ({\hat \rho} - \overline{\rho}) \sigma_1 Z_1^0 \qquad \dfrac{(1 - \theta) \overline{\rho} {\hat \rho} - (1 - \theta {\hat \rho}^2)}{(1 - \theta) (\overline{\rho} - {\hat \rho})} \leqslant - \dfrac{\sigma_1 Z_1^0}{\sigma_2 Z_2^0}
%\end{eqnarray*} 
%\begin{eqnarray*}
%[(1 - \theta {\hat \rho}^2) - (1 - \theta) \underline{\rho} {\hat \rho}] \sigma_2 Z_2^0 \leqslant (1 - \theta) (\underline{\rho} - {\hat \rho}) \sigma_1 Z_1^0 \qquad \dfrac{(1 - \theta {\hat \rho}^2) - (1 - \theta) \underline{\rho} {\hat \rho}}{(1 - \theta) ({\hat \rho} - \underline{\rho})} \leqslant - \dfrac{\sigma_1 Z_1^0}{\sigma_2 Z_2^0}
%\end{eqnarray*}
or
\begin{eqnarray*}
\dfrac{(1 - \theta) \overline{\rho} {\hat \rho} - (1 - \theta {\hat \rho}^2)}{(1 - \theta) (\overline{\rho} - {\hat \rho})} \leqslant - \dfrac{\sigma_1 Z_1^0}{\sigma_2 Z_2^0} \qquad \text{and} \qquad \dfrac{(1 - \theta {\hat \rho}^2) - (1 - \theta) \underline{\rho} {\hat \rho}}{(1 - \theta) ({\hat \rho} - \underline{\rho})} \leqslant - \dfrac{\sigma_1 Z_1^0}{\sigma_2 Z_2^0}.
\end{eqnarray*}
Since $ \dfrac{(1 - \theta) \overline{\rho} {\hat \rho} - (1 - \theta {\hat \rho}^2)}{(1 - \theta) (\overline{\rho} - {\hat \rho})} < {\hat \rho} < \dfrac{(1 - \theta {\hat \rho}^2) - (1 - \theta) \underline{\rho} {\hat \rho}}{(1 - \theta) ({\hat \rho} - \underline{\rho})} $, then $ \left\{ \begin{matrix} R_1 < 0 \\ \overline{\rho} R_1 \leqslant R_2 \leqslant \underline{\rho} R_1 \end{matrix} \right. $ is equivalent to $ \dfrac{(1 - \theta {\hat \rho}^2) - (1 - \theta) \underline{\rho} {\hat \rho}}{(1 - \theta) ({\hat \rho} - \underline{\rho})} \leqslant - \dfrac{\sigma_1 Z_1^0}{\sigma_2 Z_2^0} $, which implies a contradiction: $ 1 < \dfrac{1 - \theta {\hat \rho}^2}{1 - \theta} < \underline{\rho} {\hat \rho} $.

%$ R_1 > 0 $ is equivalent to $ \sigma_1 Z_1^0 + {\hat \rho} \sigma_2 Z_2^0 > 0 $ or $ - {\hat \rho} < \dfrac{\sigma_1 Z_1^0}{\sigma_2 Z_2^0} $.
$ R_1 > 0 $ is equivalent to $ - {\hat \rho} < \dfrac{\sigma_1 Z_1^0}{\sigma_2 Z_2^0} $.
$ \underline{\rho} R_1 \leqslant R_2 \leqslant \overline{\rho} R_1 $ is equivalent to 
\begin{eqnarray*}
\underline{\rho} (\sigma_1 Z_1^0 + {\hat \rho} \sigma_2 Z_2^0) \leqslant \dfrac{(1 - \theta) {\hat \rho} \sigma_1 Z_1^0 + (1 - \theta {\hat \rho}^2) \sigma_2 Z_2^0}{1 - \theta} \leqslant \overline{\rho} (\sigma_1 Z_1^0 + {\hat \rho} \sigma_2 Z_2^0)
\end{eqnarray*} 
%\begin{eqnarray*}
%[(1 - \theta) \underline{\rho} {\hat \rho} - (1 - \theta {\hat \rho}^2)] \sigma_2 Z_2^0 \leqslant (1 - \theta) ({\hat \rho} - \underline{\rho}) \sigma_1 Z_1^0 \qquad \dfrac{(1 - \theta) \underline{\rho} {\hat \rho} - (1 - \theta {\hat \rho}^2)}{(1 - \theta) ({\hat \rho} - \underline{\rho})} \leqslant \dfrac{\sigma_1 Z_1^0}{\sigma_2 Z_2^0}
%\end{eqnarray*} 
%\begin{eqnarray*}
%[(1 - \theta {\hat \rho}^2) - (1 - \theta) \overline{\rho} {\hat \rho}] \sigma_2 Z_2^0 \leqslant (1 - \theta) (\overline{\rho} - {\hat \rho}) \sigma_1 Z_1^0 \qquad \dfrac{(1 - \theta {\hat \rho}^2) - (1 - \theta) \overline{\rho} {\hat \rho}}{(1 - \theta) (\overline{\rho} - {\hat \rho})} \leqslant \dfrac{\sigma_1 Z_1^0}{\sigma_2 Z_2^0}
%\end{eqnarray*}
or
\begin{eqnarray*}
\dfrac{(1 - \theta) \underline{\rho} {\hat \rho} - (1 - \theta {\hat \rho}^2)}{(1 - \theta) ({\hat \rho} - \underline{\rho})} \leqslant \dfrac{\sigma_1 Z_1^0}{\sigma_2 Z_2^0} \qquad \text{and} \qquad \dfrac{(1 - \theta {\hat \rho}^2) - (1 - \theta) \overline{\rho} {\hat \rho}}{(1 - \theta) (\overline{\rho} - {\hat \rho})} \leqslant \dfrac{\sigma_1 Z_1^0}{\sigma_2 Z_2^0}.
\end{eqnarray*}
Since $ \dfrac{(1 - \theta) \underline{\rho} {\hat \rho} - (1 - \theta {\hat \rho}^2)}{(1 - \theta) ({\hat \rho} - \underline{\rho})} < - {\hat \rho} < \dfrac{(1 - \theta {\hat \rho}^2) - (1 - \theta) \overline{\rho} {\hat \rho}}{(1 - \theta) (\overline{\rho} - {\hat \rho})} $, then $ \left\{ \begin{matrix} R_1 > 0 \\ \underline{\rho} R_1 \leqslant R_2 \leqslant \overline{\rho} R_1 \end{matrix} \right. $ is equivalent to $ \dfrac{(1 - \theta {\hat \rho}^2) - (1 - \theta) \overline{\rho} {\hat \rho}}{(1 - \theta) (\overline{\rho} - {\hat \rho})} \leqslant \dfrac{\sigma_1 Z_1^0}{\sigma_2 Z_2^0} $.
%In this setting, equilibrium Sharpe ratios (A.5.14) and (A.5.15) can be written as
%\begin{eqnarray*}
%& \dfrac{\mu_1 - p_1}{\sigma_1} = R_1 = \alpha (\sigma_1 Z_1^0 + {\hat \rho} \sigma_2 Z_2^0) & \\
%& \dfrac{\mu_2 - p_2}{\sigma_2} = R_2 = \alpha \dfrac{(1 - \theta) {\hat \rho} \sigma_1 Z_1^0 + (1 - \theta {\hat \rho}^2) \sigma_2 Z_2^0}{1 - \theta} &
%\end{eqnarray*}
%and then we obtain equilibrium prices for the two risky assets
In this setting, equilibrium Sharpe ratios (A.42) and (A.43) imply equilibrium prices for the two risky assets
\begin{eqnarray}
& p_1 = \mu_1 - \alpha \sigma_1 (\sigma_1 Z_1^0 + {\hat \rho} \sigma_2 Z_2^0) & \\
& p_2 = \mu_2 - \alpha \sigma_2 \dfrac{(1 - \theta) {\hat \rho} \sigma_1 Z_1^0 + (1 - \theta {\hat \rho}^2) \sigma_2 Z_2^0}{1 - \theta} &
\end{eqnarray}
in which na\"ive investors buy risky asset 1 but do not trade risky asset 2. In equilibrium, a sophisticated investor holds positions of risky assets:
\begin{eqnarray}
& Z_{S 1}^* = \dfrac{R_1 - \hat \rho R_2}{\alpha \sigma_1 (1 - \hat \rho^2)} = Z_1^0 - \dfrac{\theta}{1 - \theta} {\hat \rho} \dfrac{\sigma_2}{\sigma_1} Z_2^0 & \\
& Z_{S 2}^* = \dfrac{R_2 - \hat \rho R_1}{\alpha \sigma_2 (1 - \hat \rho^2)} = \dfrac{Z_2^0}{1 - \theta} &
\end{eqnarray}
and a na\"ive investor holds positions of risky assets:
\begin{eqnarray}
& Z_{N 1}^* = \dfrac{R_1}{\alpha \sigma_1} = \dfrac{\sigma_1 Z_1^0 + {\hat \rho} \sigma_2 Z_2^0}{\sigma_1} = Z_1^0 + {\hat \rho} \dfrac{\sigma_2}{\sigma_1} Z_2^0 > 0 & \\
& Z_{N 2}^* = 0. &
\end{eqnarray}

We summarize the above analysis and obtain the existence of equilibrium for $ \dfrac{(1 - \theta {\hat \rho}^2) - (1 - \theta) \overline{\rho} {\hat \rho}}{(1 - \theta) (\overline{\rho} - {\hat \rho})} \leqslant \dfrac{\sigma_1 Z_1^0}{\sigma_2 Z_2^0} $.

\vskip 8 pt

{\bf Proposition A5.3}. If na\"ive investors do not trade risky asset 2, $ Z_{N 2}^* = 0 $, then there is not general equilibrium when na\"ive investors short sell risky asset 1, $ Z_{N 1}^* < 0 $, and there exists general equilibrium when na\"ive investors long buy risky asset 1, $ Z_{N 1}^* > 0 $. The equilibrium prices for the two risky assets are given by (A.44) - (A.45) when $ \dfrac{(1 - \theta {\hat \rho}^2) - (1 - \theta) \overline{\rho} {\hat \rho}}{(1 - \theta) (\overline{\rho} - {\hat \rho})} \leqslant \dfrac{\sigma_1 Z_1^0}{\sigma_2 Z_2^0} $.

\vskip 8 pt

Since $ \dfrac{\theta {\hat \rho}}{1 - \theta} < \dfrac{(1 - \theta {\hat \rho}^2) - (1 - \theta) \overline{\rho} {\hat \rho}}{(1 - \theta) (\overline{\rho} - {\hat \rho})} $, then $ Z_{S 1}^* > 0 $ and $ Z_{S 2}^* > 0 $. In equilibrium, a sophisticated investor holds positive positions of risky assets.

\subsubsection*{A.5.4 \quad Equilibrium conditions for scenario 4: $ Z_{N 1}^* Z_{N 2}^* > 0 $}

\quad \ 
Scenario $ Z_{N 1}^* Z_{N 2}^* > 0 $ is split into two settings: $ \left\{ \begin{matrix} Z_{N 1}^* < 0 \\ Z_{N 2}^* < 0 \end{matrix} \right. $ and $ \left\{ \begin{matrix} Z_{N 1}^* > 0 \\ Z_{N 2}^* > 0 \end{matrix} \right. $, which correspond to the environments in equation (10): $ \left\{ \begin{matrix} R_1 < \overline{\rho} R_2 \\ R_2 < \overline{\rho} R_1 \end{matrix} \right. $ and $ \left\{ \begin{matrix} R_1 > \overline{\rho} R_2 \\ R_2 > \overline{\rho} R_1 \end{matrix} \right. $; then we obtain the equilibrium condition:
\begin{eqnarray}
(1 - \theta) \left( \begin{matrix} \dfrac{R_1 - \hat \rho R_2}{\alpha \sigma_1 (1 - \hat \rho^2)} \\ \dfrac{R_2 - \hat \rho R_1}{\alpha \sigma_2 (1 - \hat \rho^2)} \end{matrix} \right) + \theta \left( \begin{matrix} \dfrac{R_1 - \overline{\rho} R_2}{\alpha \sigma_1 (1 - \overline{\rho}^2)} \\ \dfrac{R_2 - \overline{\rho} R_1}{\alpha \sigma_2 (1 - \overline{\rho}^2)} \end{matrix} \right) = \left( \begin{matrix} Z_1^0 \\ Z_2^0 \end{matrix} \right)
\end{eqnarray}
or
\begin{eqnarray*}
& \quad \left[ \dfrac{1 - \theta}{1 - {\hat \rho}^2} + \dfrac{\theta}{1 - \overline{\rho}^2} \right] R_1 - \left[ \dfrac{1 - \theta}{1 - {\hat \rho}^2} {\hat \rho} + \dfrac{\theta}{1 - \overline{\rho}^2} \overline{\rho} \right] R_2 = \alpha \sigma_1 Z_1^0 & \\
& - \left[ \dfrac{1 - \theta}{1 - {\hat \rho}^2} {\hat \rho} + \dfrac{\theta}{1 - \overline{\rho}^2} \overline{\rho} \right] R_1 + \left[ \dfrac{1 - \theta}{1 - {\hat \rho}^2} + \dfrac{\theta}{1 - \overline{\rho}^2} \right] R_2 = \alpha \sigma_2 Z_2^0. & 
\end{eqnarray*}
The equilibrium Sharpe ratios are given by:
\begin{eqnarray}
& R_1 = \alpha \dfrac{\left[ \dfrac{1 - \theta}{1 - {\hat \rho}^2} + \dfrac{\theta}{1 - \overline{\rho}^2} \right] \sigma_1 Z_1^0 + \left[ \dfrac{1 - \theta}{1 - {\hat \rho}^2} {\hat \rho} + \dfrac{\theta}{1 - \overline{\rho}^2} \overline{\rho} \right] \sigma_2 Z_2^0}{\left[ \dfrac{1 - \theta}{1 - {\hat \rho}^2} + \dfrac{\theta}{1 - \overline{\rho}^2} \right]^2 - \left[ \dfrac{1 - \theta}{1 - {\hat \rho}^2} {\hat \rho} + \dfrac{\theta}{1 - \overline{\rho}^2} \overline{\rho} \right]^2} & \\
& R_2 = \alpha \dfrac{\left[ \dfrac{1 - \theta}{1 - {\hat \rho}^2} {\hat \rho} + \dfrac{\theta}{1 - \overline{\rho}^2} \overline{\rho} \right] \sigma_1 Z_1^0 + \left[ \dfrac{1 - \theta}{1 - {\hat \rho}^2} + \dfrac{\theta}{1 - \overline{\rho}^2} \right] \sigma_2 Z_2^0}{\left[ \dfrac{1 - \theta}{1 - {\hat \rho}^2} + \dfrac{\theta}{1 - \overline{\rho}^2} \right]^2 - \left[ \dfrac{1 - \theta}{1 - {\hat \rho}^2} {\hat \rho} + \dfrac{\theta}{1 - \overline{\rho}^2} \overline{\rho} \right]^2}. &
\end{eqnarray}

$ R_1 < \overline{\rho} R_2 $ is equivalent to 
{\footnotesize \begin{eqnarray*}
\left[ \dfrac{1 - \theta}{1 - {\hat \rho}^2} + \dfrac{\theta}{1 - \overline{\rho}^2} \right] \sigma_1 Z_1^0 + \left[ \dfrac{1 - \theta}{1 - {\hat \rho}^2} {\hat \rho} + \dfrac{\theta}{1 - \overline{\rho}^2} \overline{\rho} \right] \sigma_2 Z_2^0 < \overline{\rho} \left\{ \left[ \dfrac{1 - \theta}{1 - {\hat \rho}^2} {\hat \rho} + \dfrac{\theta}{1 - \overline{\rho}^2} \overline{\rho} \right] \sigma_1 Z_1^0 + \left[ \dfrac{1 - \theta}{1 - {\hat \rho}^2} + \dfrac{\theta}{1 - \overline{\rho}^2} \right] \sigma_2 Z_2^0 \right\}
\end{eqnarray*}}
\begin{eqnarray*}
\left[ \dfrac{1 - \theta}{1 - {\hat \rho}^2} (1 - \overline{\rho} {\hat \rho}) + \theta \right] \sigma_1 Z_1^0 < \dfrac{1 - \theta}{1 - {\hat \rho}^2} (\overline{\rho} - {\hat \rho}) \sigma_2 Z_2^0.
\end{eqnarray*}
$ R_2 < \overline{\rho} R_1 $ is equivalent to 
{\footnotesize \begin{eqnarray*}
\left[ \dfrac{1 - \theta}{1 - {\hat \rho}^2} {\hat \rho} + \dfrac{\theta}{1 - \overline{\rho}^2} \overline{\rho} \right] \sigma_1 Z_1^0 + \left[ \dfrac{1 - \theta}{1 - {\hat \rho}^2} + \dfrac{\theta}{1 - \overline{\rho}^2} \right] \sigma_2 Z_2^0 < \overline{\rho} \left\{ \left[ \dfrac{1 - \theta}{1 - {\hat \rho}^2} + \dfrac{\theta}{1 - \overline{\rho}^2} \right] \sigma_1 Z_1^0 + \left[ \dfrac{1 - \theta}{1 - {\hat \rho}^2} {\hat \rho} + \dfrac{\theta}{1 - \overline{\rho}^2} \overline{\rho} \right] \sigma_2 Z_2^0 \right\}
\end{eqnarray*}}
\begin{eqnarray*}
\left[ \dfrac{1 - \theta}{1 - {\hat \rho}^2} (1 - \overline{\rho} {\hat \rho}) + \theta \right] \sigma_2 Z_2^0 < \dfrac{1 - \theta}{1 - {\hat \rho}^2} (\overline{\rho} - {\hat \rho}) \sigma_1 Z_1^0.
\end{eqnarray*}
$ \left\{ \begin{matrix} R_1 < \overline{\rho} R_2 \\ R_2 < \overline{\rho} R_1 \end{matrix} \right. $ is equivalent to a contradiction:
\begin{eqnarray*}
\dfrac{\dfrac{1 - \theta}{1 - {\hat \rho}^2} (1 - \overline{\rho} {\hat \rho}) + \theta}{\dfrac{1 - \theta}{1 - {\hat \rho}^2} (\overline{\rho} - {\hat \rho})} < \dfrac{\sigma_2 Z_2^0}{\sigma_1 Z_1^0} < \dfrac{\dfrac{1 - \theta}{1 - {\hat \rho}^2} (\overline{\rho} - {\hat \rho})}{\dfrac{1 - \theta}{1 - {\hat \rho}^2} (1 - \overline{\rho} {\hat \rho}) + \theta}.
\end{eqnarray*}

$ R_1 > \overline{\rho} R_2 $ is equivalent to 
{\footnotesize \begin{eqnarray*}
\left[ \dfrac{1 - \theta}{1 - {\hat \rho}^2} + \dfrac{\theta}{1 - \overline{\rho}^2} \right] \sigma_1 Z_1^0 + \left[ \dfrac{1 - \theta}{1 - {\hat \rho}^2} {\hat \rho} + \dfrac{\theta}{1 - \overline{\rho}^2} \overline{\rho} \right] \sigma_2 Z_2^0 > \overline{\rho} \left\{ \left[ \dfrac{1 - \theta}{1 - {\hat \rho}^2} {\hat \rho} + \dfrac{\theta}{1 - \overline{\rho}^2} \overline{\rho} \right] \sigma_1 Z_1^0 + \left[ \dfrac{1 - \theta}{1 - {\hat \rho}^2} + \dfrac{\theta}{1 - \overline{\rho}^2} \right] \sigma_2 Z_2^0 \right\}
\end{eqnarray*}}
\begin{eqnarray*}
\left[ \dfrac{1 - \theta}{1 - {\hat \rho}^2} (1 - \overline{\rho} {\hat \rho}) + \theta \right] \sigma_1 Z_1^0 > \dfrac{1 - \theta}{1 - {\hat \rho}^2} (\overline{\rho} - {\hat \rho}) \sigma_2 Z_2^0
\end{eqnarray*}
$ R_2 > \overline{\rho} R_1 $ is equivalent to 
{\footnotesize \begin{eqnarray*}
\left[ \dfrac{1 - \theta}{1 - {\hat \rho}^2} {\hat \rho} + \dfrac{\theta}{1 - \overline{\rho}^2} \overline{\rho} \right] \sigma_1 Z_1^0 + \left[ \dfrac{1 - \theta}{1 - {\hat \rho}^2} + \dfrac{\theta}{1 - \overline{\rho}^2} \right] \sigma_2 Z_2^0 > \overline{\rho} \left\{ \left[ \dfrac{1 - \theta}{1 - {\hat \rho}^2} + \dfrac{\theta}{1 - \overline{\rho}^2} \right] \sigma_1 Z_1^0 + \left[ \dfrac{1 - \theta}{1 - {\hat \rho}^2} {\hat \rho} + \dfrac{\theta}{1 - \overline{\rho}^2} \overline{\rho} \right] \sigma_2 Z_2^0 \right\}
\end{eqnarray*}}
\begin{eqnarray*}
\left[ \dfrac{1 - \theta}{1 - {\hat \rho}^2} (1 - \overline{\rho} {\hat \rho}) + \theta \right] \sigma_2 Z_2^0 > \dfrac{1 - \theta}{1 - {\hat \rho}^2} (\overline{\rho} - {\hat \rho}) \sigma_1 Z_1^0.
\end{eqnarray*}
$ \left\{ \begin{matrix} R_1 > \overline{\rho} R_2 \\ R_2 > \overline{\rho} R_1 \end{matrix} \right. $ is equivalent to 
\begin{eqnarray*}
\dfrac{\dfrac{1 - \theta}{1 - {\hat \rho}^2} (\overline{\rho} - {\hat \rho})}{\dfrac{1 - \theta}{1 - {\hat \rho}^2} (1 - \overline{\rho} {\hat \rho}) + \theta} < \dfrac{\sigma_2 Z_2^0}{\sigma_1 Z_1^0} < \dfrac{\dfrac{1 - \theta}{1 - {\hat \rho}^2} (1 - \overline{\rho} {\hat \rho}) + \theta}{\dfrac{1 - \theta}{1 - {\hat \rho}^2} (\overline{\rho} - {\hat \rho})}.
\end{eqnarray*}
In this setting, equilibrium Sharpe ratios (A.51) and (A.52) can be written as
\begin{eqnarray*}
& \dfrac{\mu_1 - p_1}{\sigma_1} = R_1 = \alpha \dfrac{\left[ \dfrac{1 - \theta}{1 - {\hat \rho}^2} + \dfrac{\theta}{1 - \overline{\rho}^2} \right] \sigma_1 Z_1^0 + \left[ \dfrac{1 - \theta}{1 - {\hat \rho}^2} {\hat \rho} + \dfrac{\theta}{1 - \overline{\rho}^2} \overline{\rho} \right] \sigma_2 Z_2^0}{\left[ \dfrac{1 - \theta}{1 - {\hat \rho}^2} + \dfrac{\theta}{1 - \overline{\rho}^2} \right]^2 - \left[ \dfrac{1 - \theta}{1 - {\hat \rho}^2} {\hat \rho} + \dfrac{\theta}{1 - \overline{\rho}^2} \overline{\rho} \right]^2} & \\
& \dfrac{\mu_2 - p_2}{\sigma_2} = R_2 = \alpha \dfrac{\left[ \dfrac{1 - \theta}{1 - {\hat \rho}^2} {\hat \rho} + \dfrac{\theta}{1 - \overline{\rho}^2} \overline{\rho} \right] \sigma_1 Z_1^0 + \left[ \dfrac{1 - \theta}{1 - {\hat \rho}^2} + \dfrac{\theta}{1 - \overline{\rho}^2} \right] \sigma_2 Z_2^0}{\left[ \dfrac{1 - \theta}{1 - {\hat \rho}^2} + \dfrac{\theta}{1 - \overline{\rho}^2} \right]^2 - \left[ \dfrac{1 - \theta}{1 - {\hat \rho}^2} {\hat \rho} + \dfrac{\theta}{1 - \overline{\rho}^2} \overline{\rho} \right]^2} &
\end{eqnarray*}
and then we obtain equilibrium prices for the two risky assets:
\begin{eqnarray}
& p_1 = \mu_1 - \alpha \sigma_1 \dfrac{\left[ \dfrac{1 - \theta}{1 - {\hat \rho}^2} + \dfrac{\theta}{1 - \overline{\rho}^2} \right] \sigma_1 Z_1^0 + \left[ \dfrac{1 - \theta}{1 - {\hat \rho}^2} {\hat \rho} + \dfrac{\theta}{1 - \overline{\rho}^2} \overline{\rho} \right] \sigma_2 Z_2^0}{\left[ \dfrac{1 - \theta}{1 - {\hat \rho}^2} + \dfrac{\theta}{1 - \overline{\rho}^2} \right]^2 - \left[ \dfrac{1 - \theta}{1 - {\hat \rho}^2} {\hat \rho} + \dfrac{\theta}{1 - \overline{\rho}^2} \overline{\rho} \right]^2} & \\
& p_2 = \mu_2 - \alpha \sigma_2 \dfrac{\left[ \dfrac{1 - \theta}{1 - {\hat \rho}^2} {\hat \rho} + \dfrac{\theta}{1 - \overline{\rho}^2} \overline{\rho} \right] \sigma_1 Z_1^0 + \left[ \dfrac{1 - \theta}{1 - {\hat \rho}^2} + \dfrac{\theta}{1 - \overline{\rho}^2} \right] \sigma_2 Z_2^0}{\left[ \dfrac{1 - \theta}{1 - {\hat \rho}^2} + \dfrac{\theta}{1 - \overline{\rho}^2} \right]^2 - \left[ \dfrac{1 - \theta}{1 - {\hat \rho}^2} {\hat \rho} + \dfrac{\theta}{1 - \overline{\rho}^2} \overline{\rho} \right]^2} &
\end{eqnarray}
which is a unique equilibrium in which na\"ive investors hold positive positions of risky assets. In equilibrium, a sophisticated investor holds positions of risky assets:
\begin{eqnarray}
& & Z_{S 1}^* = \dfrac1{\sigma_1 (1 - {\hat \rho}^2)} \dfrac{\left[ (1 - \theta) + \dfrac{\theta}{1 - \overline{\rho}^2} (1 - {\hat \rho} \overline{\rho}) \right] \sigma_1 Z_1^0 + \dfrac{\theta}{1 - \overline{\rho}^2} (\overline{\rho} - {\hat \rho}) \sigma_2 Z_2^0}{\left[ \dfrac{1 - \theta}{1 - {\hat \rho}^2} + \dfrac{\theta}{1 - \overline{\rho}^2} \right]^2 - \left[ \dfrac{1 - \theta}{1 - {\hat \rho}^2} {\hat \rho} + \dfrac{\theta}{1 - \overline{\rho}^2} \overline{\rho} \right]^2} \\
& & Z_{S 2}^* = \dfrac1{\sigma_2 (1 - {\hat \rho}^2)} \dfrac{\dfrac{\theta}{1 - \overline{\rho}^2} (\overline{\rho} - {\hat \rho}) \sigma_1 Z_1^0 + \left[ (1 - \theta) + \dfrac{\theta}{1 - \overline{\rho}^2} (1 - {\hat \rho} \overline{\rho}) \right] \sigma_2 Z_2^0}{\left[ \dfrac{1 - \theta}{1 - {\hat \rho}^2} + \dfrac{\theta}{1 - \overline{\rho}^2} \right]^2 - \left[ \dfrac{1 - \theta}{1 - {\hat \rho}^2} {\hat \rho} + \dfrac{\theta}{1 - \overline{\rho}^2} \overline{\rho} \right]^2}
\end{eqnarray}
and a na\"ive investor holds positions of risky assets:
\begin{eqnarray}
& & Z_{N 1}^* = \dfrac1{\sigma_1 (1 - \overline{\rho}^2)} \dfrac{\left[ \dfrac{1 - \theta}{1 - {\hat \rho}^2} (1 - {\hat \rho} \overline{\rho}) + \theta \right] \sigma_1 Z_1^0 + \dfrac{1 - \theta}{1 - {\hat \rho}^2} ({\hat \rho} - \overline{\rho}) \sigma_2 Z_2^0}{\left[ \dfrac{1 - \theta}{1 - {\hat \rho}^2} + \dfrac{\theta}{1 - \overline{\rho}^2} \right]^2 - \left[ \dfrac{1 - \theta}{1 - {\hat \rho}^2} {\hat \rho} + \dfrac{\theta}{1 - \overline{\rho}^2} \overline{\rho} \right]^2} > 0 \\
& & Z_{N 2}^* = \dfrac1{\sigma_2 (1 - \overline{\rho}^2)} \dfrac{\dfrac{1 - \theta}{1 - {\hat \rho}^2} ({\hat \rho} - \overline{\rho}) \sigma_1 Z_1^0 + \left[ \dfrac{1 - \theta}{1 - {\hat \rho}^2} (1 - {\hat \rho} \overline{\rho}) + \theta \right] \sigma_2 Z_2^0}{\left[ \dfrac{1 - \theta}{1 - {\hat \rho}^2} + \dfrac{\theta}{1 - \overline{\rho}^2} \right]^2 - \left[ \dfrac{1 - \theta}{1 - {\hat \rho}^2} {\hat \rho} + \dfrac{\theta}{1 - \overline{\rho}^2} \overline{\rho} \right]^2} > 0.
\end{eqnarray}

We summarize the above analysis and obtain the existence of equilibrium for $ \dfrac{(1 - \theta) (\overline{\rho} - {\hat \rho})}{(1 - \theta {\hat \rho}^2) - (1 - \theta) \overline{\rho} {\hat \rho}} < \dfrac{\sigma_1 Z_1^0}{\sigma_2 Z_2^0} < \dfrac{(1 - \theta {\hat \rho}^2) - (1 - \theta) \overline{\rho} {\hat \rho}}{(1 - \theta) (\overline{\rho} - {\hat \rho})} $.

\vskip 8 pt

{\bf Proposition A5.4}. If na\"ive investors short sell or long buy the two risky assets simultaneously, $ Z_{N 1}^* Z_{N 2}^* > 0 $, and then there is not general equilibrium when na\"ive investors short sell them, $ Z_{N i}^* < 0 $ for $ i = 1, 2 $ and there exists general equilibrium when na\"ive investors long buy them, $ Z_{N i}^* > 0 $ for $ i = 1, 2 $. The equilibrium prices for the two risky assets are given by (A.53) - (A.54) when $ \dfrac{(1 - \theta) (\overline{\rho} - {\hat \rho})}{(1 - \theta {\hat \rho}^2) - (1 - \theta) \overline{\rho} {\hat \rho}} < \dfrac{\sigma_1 Z_1^0}{\sigma_2 Z_2^0} < \dfrac{(1 - \theta {\hat \rho}^2) - (1 - \theta) \overline{\rho} {\hat \rho}}{(1 - \theta) (\overline{\rho} - {\hat \rho})} $.

\vskip 8 pt

In equilibrium, sophisticated investors always hold positive positions of risky assets: $ Z_{S 1}^* > 0 $ and $ Z_{S 2}^* > 0 $. 

\subsection*{A.6 \quad Comparison of equilibrium positions for sophisticated and na\"ive investors}

\quad
In equilibrium, na\"ive investors might hold larger positions (long or short) than the sophisticated investors. Specifically, we have the following results.

\underline{Scenario 1}: For a non-participating equilibrium in asset 1, the quality ratio is small, $ E_{1 2} \leqslant h (\theta, \overline{\rho}, {\hat \rho}) $, and hence $ Z_{N 1}^* = 0 < \dfrac{Z_1^0}{1 - \theta} = Z_{S 1}^* $ and $ 0 < Z_{S 2}^* = Z_2^0 - \dfrac{\theta}{1 - \theta} {\hat \rho} \dfrac{\sigma_1}{\sigma_2} Z_1^0 < {\hat \rho} \dfrac{\sigma_1}{\sigma_2} Z_1^0 + Z_2^0 = Z_{N 2}^* $.

\underline{Scenario 2}: For a participating equilibrium in both assets, the quality ratio is moderate, $ h (\theta, \overline{\rho}, {\hat \rho}) < E_{1 2} < H (\theta, \overline{\rho}, {\hat \rho}) $, 
\begin{eqnarray*}
& Z_{N 1}^* - Z_{S 1}^* = \dfrac{1}{\sigma_1} D (\theta, \overline{\rho}, {\hat \rho}) \left\{ \left[ (1 - \theta) \overline{\rho} + \theta {\hat \rho} \right] \sigma_1 Z_1^0 - \sigma_2 Z_2^0 \right\} & \\
& Z_{N 2}^* - Z_{S 2}^* = \dfrac{1}{\sigma_2} D (\theta, \overline{\rho}, {\hat \rho}) \left\{ \left[ (1 - \theta) \overline{\rho} + \theta {\hat \rho} \right] \sigma_2 Z_2^0 - \sigma_1 Z_1^0 \right\}. &
\end{eqnarray*}
If $ {\hat \rho} < 0 $, then $ (1 - \theta) \overline{\rho} + \theta {\hat \rho} < h (\theta, \overline{\rho}, {\hat \rho}) < E_{2 1} < H (\theta, \overline{\rho}, {\hat \rho}) $, and hence $ Z_{N 1}^* < Z_{S 1}^* $ and $ Z_{N 2}^* < Z_{S 2}^* $.
If $ 0 < {\hat \rho} $, then $ h (\theta, \overline{\rho}, {\hat \rho}) < (1 - \theta) \overline{\rho} + \theta {\hat \rho} < H (\theta, \overline{\rho}, {\hat \rho}) $. Therefore, we have:
\begin{itemize}
\item If $ h (\theta, \overline{\rho}, {\hat \rho}) < E_{2 1} < (1 - \theta) \overline{\rho} + \theta {\hat \rho} $, then $ Z_{N 1}^* > Z_{S 1}^* $;
\item If $ (1 - \theta) \overline{\rho} + \theta {\hat \rho} < E_{2 1} < H (\theta, \overline{\rho}, {\hat \rho}) $, then $ Z_{N 1}^* < Z_{S 1}^* $;
\item If $ h (\theta, \overline{\rho}, {\hat \rho}) < E_{1 2} < (1 - \theta) \overline{\rho} + \theta {\hat \rho} $, then $ Z_{N 2}^* > Z_{S 2}^* $;
\item If $ (1 - \theta) \overline{\rho} + \theta {\hat \rho} < E_{1 2} < H (\theta, \overline{\rho}, {\hat \rho}) $, then $ Z_{N 2}^* < Z_{S 2}^* $.
\end{itemize}

\underline{Scenario 3}: For a non-participating equilibrium in asset 2, the quality ratio is big, $ H (\theta, \overline{\rho}, {\hat \rho}) \leqslant E_{1 2} $, and hence $ Z_{N 2}^* = 0 < \dfrac{Z_2^0}{1 - \theta} = Z_{S 2}^* $ and $ 0 < Z_{S 1}^* = Z_1^0 - \dfrac{\theta}{1 - \theta} {\hat \rho} \dfrac{\sigma_2}{\sigma_1} Z_2^0 < Z_1^0 + {\hat \rho} \dfrac{\sigma_2}{\sigma_1} Z_2^0 = Z_{N 1}^* $.

\subsection*{A.7 \quad Participating equilibrium prices change with fraction of na\"ive investors}

\quad
In Theorem 2, the equilibrium Sharpe ratio for the risky assets is:
{\footnotesize \begin{eqnarray*}
R_1 (\theta, \overline{\rho}, {\hat \rho}) = & \dfrac{\mu_1 - p_1}{\sigma_1} & = \alpha \left[ Q (\theta, \overline{\rho}, {\hat \rho}) \sigma_1 Z_1^0 + q (\theta, \overline{\rho}, {\hat \rho}) \sigma_2 Z_2^0 \right] \\
R_2 (\theta, \overline{\rho}, {\hat \rho}) = & \dfrac{\mu_2 - p_2}{\sigma_2} & = \alpha \left[ q (\theta, \overline{\rho}, {\hat \rho}) \sigma_1 Z_1^0 + Q (\theta, \overline{\rho}, {\hat \rho}) \sigma_2 Z_2^0 \right]
\end{eqnarray*}}
then
{\footnotesize \begin{eqnarray*}
\dfrac{1}{\alpha Q^2 (\theta, \overline{\rho}, {\hat \rho})} \dfrac{\partial R_1 (\theta, \overline{\rho}, {\hat \rho})}{\partial \theta} & = & \left\{ \left[ \dfrac{- 1}{1 - {\hat \rho}^2} {\hat \rho} + \dfrac1{1 - \overline{\rho}^2} \overline{\rho} \right] \sigma_2 Z_2^0 - \left[ \dfrac{- 1}{1 - {\hat \rho}^2} + \dfrac1{1 - \overline{\rho}^2} \right] \sigma_1 Z_1^0 \right\} \left\{ \Gamma^2 (\theta, \overline{\rho}, {\hat \rho}) + 1 \right\} \\
& & + 2 \left\{ \left[ \dfrac{- 1}{1 - {\hat \rho}^2} {\hat \rho} + \dfrac1{1 - \overline{\rho}^2} \overline{\rho} \right] \sigma_1 Z_1^0 - \left[ \dfrac{- 1}{1 - {\hat \rho}^2} + \dfrac1{1 - \overline{\rho}^2} \right] \sigma_2 Z_2^0 \right\} \Gamma (\theta, \overline{\rho}, {\hat \rho}) \\
\dfrac{1}{\alpha Q^2 (\theta, \overline{\rho}, {\hat \rho})} \dfrac{\partial R_2 (\theta, \overline{\rho}, {\hat \rho})}{\partial \theta} & = & \left\{ \left[ \dfrac{- 1}{1 - {\hat \rho}^2} {\hat \rho} + \dfrac1{1 - \overline{\rho}^2} \overline{\rho} \right] \sigma_1 Z_1^0 - \left[ \dfrac{- 1}{1 - {\hat \rho}^2} + \dfrac1{1 - \overline{\rho}^2} \right] \sigma_2 Z_2^0 \right\} \left\{ \Gamma^2 (\theta, \overline{\rho}, {\hat \rho}) + 1 \right\} \\
& & + 2 \left\{ \left[ \dfrac{- 1}{1 - {\hat \rho}^2} {\hat \rho} + \dfrac1{1 - \overline{\rho}^2} \overline{\rho} \right] \sigma_2 Z_2^0 - \left[ \dfrac{- 1}{1 - {\hat \rho}^2} + \dfrac1{1 - \overline{\rho}^2} \right] \sigma_1 Z_1^0 \right\} \Gamma (\theta, \overline{\rho}, {\hat \rho})
\end{eqnarray*}}
where {\footnotesize $ \Gamma (\theta, \overline{\rho}, {\hat \rho}) \equiv \dfrac{\dfrac{1 - \theta}{1 - {\hat \rho}^2} {\hat \rho} + \dfrac{\theta}{1 - \overline{\rho}^2} \overline{\rho}}{\dfrac{1 - \theta}{1 - {\hat \rho}^2} + \dfrac{\theta}{1 - \overline{\rho}^2}} $}.

We first explore the monotonicity of {\footnotesize $ R_1 $} in {\footnotesize $ \theta $} in the following three scenarios.

Scenario 1: {\footnotesize $ \left[ \dfrac{- 1}{1 - {\hat \rho}^2} {\hat \rho} + \dfrac1{1 - \overline{\rho}^2} \overline{\rho} \right] = \left[ \dfrac{- 1}{1 - {\hat \rho}^2} + \dfrac1{1 - \overline{\rho}^2} \right] E_{2 1} $}, then
{\footnotesize \begin{eqnarray*}
\left[ \dfrac{- 1}{1 - {\hat \rho}^2} {\hat \rho} + \dfrac1{1 - \overline{\rho}^2} \overline{\rho} \right] \sigma_2 Z_2^0 - \left[ \dfrac{- 1}{1 - {\hat \rho}^2} + \dfrac1{1 - \overline{\rho}^2} \right] \sigma_1 Z_1^0 = \dfrac{\overline{\rho} - {\hat \rho}}{\overline{\rho} + {\hat \rho}} \sigma_1 Z_1^0
\end{eqnarray*}}
and hence {\footnotesize $ \dfrac{1}{\alpha} \dfrac{\partial R_1 (\theta, \overline{\rho}, {\hat \rho})}{\partial \theta} = Q^2 (\theta, \overline{\rho}, {\hat \rho}) \dfrac{\overline{\rho} - {\hat \rho}}{\overline{\rho} + {\hat \rho}} \sigma_1 Z_1^0 \left\{ 1 + \Gamma^2 (\theta, \overline{\rho}, {\hat \rho}) \right\} $}. Thus {\footnotesize $ \text{sign} \dfrac{\partial R_1 (\theta, \overline{\rho}, {\hat \rho})}{\partial \theta} = \text{sign} \left\{ \overline{\rho} + {\hat \rho} \right\} $}.

Scenario 2: {\footnotesize $ \left[ \dfrac{- 1}{1 - {\hat \rho}^2} {\hat \rho} + \dfrac1{1 - \overline{\rho}^2} \overline{\rho} \right] \neq \left[ \dfrac{- 1}{1 - {\hat \rho}^2} + \dfrac1{1 - \overline{\rho}^2} \right] E_{2 1} $}, then
{\footnotesize \begin{eqnarray*}
& \dfrac1{1 + \Gamma^2 (\theta, \overline{\rho}, {\hat \rho})} \dfrac1{\sigma_1 Z_1^0 \left\{ \left[ \dfrac{- 1}{1 - {\hat \rho}^2} {\hat \rho} + \dfrac1{1 - \overline{\rho}^2} \overline{\rho} \right] - \left[ \dfrac{- 1}{1 - {\hat \rho}^2} + \dfrac1{1 - \overline{\rho}^2} \right] E_{2 1} \right\}} \dfrac{1}{\alpha Q^2 (\theta, \overline{\rho}, {\hat \rho})} \dfrac{\partial R_1 (\theta, \overline{\rho}, {\hat \rho})}{\partial \theta} & \\
& = \dfrac{\left[ \dfrac{- 1}{1 - {\hat \rho}^2} {\hat \rho} + \dfrac1{1 - \overline{\rho}^2} \overline{\rho} \right] E_{2 1} - \left[ \dfrac{- 1}{1 - {\hat \rho}^2} + \dfrac1{1 - \overline{\rho}^2} \right]}{\left[ \dfrac{- 1}{1 - {\hat \rho}^2} {\hat \rho} + \dfrac1{1 - \overline{\rho}^2} \overline{\rho} \right] - \left[ \dfrac{- 1}{1 - {\hat \rho}^2} + \dfrac1{1 - \overline{\rho}^2} \right] E_{2 1}} + \dfrac{2 \Gamma (\theta, \overline{\rho}, {\hat \rho})}{1 + \Gamma^2 (\theta, \overline{\rho}, {\hat \rho})}. &
\end{eqnarray*}}
and {\footnotesize $ \dfrac{\partial}{\partial \theta} \left( \dfrac{2 \Gamma (\theta, \overline{\rho}, {\hat \rho})}{1 + \Gamma^2 (\theta, \overline{\rho}, {\hat \rho})} \right) = \dfrac{2 \{ 1 - \Gamma^2 (\theta, \overline{\rho}, {\hat \rho}) \}}{\{ 1 + \Gamma^2 (\theta, \overline{\rho}, {\hat \rho}) \}^2} \dfrac{\partial \Gamma (\theta, \overline{\rho}, {\hat \rho})}{\partial \theta} > 0 $} from {\footnotesize $ \dfrac{\partial \Gamma (\theta, \overline{\rho}, {\hat \rho})}{\partial \theta} = \dfrac{\dfrac{\overline{\rho} - \hat{\rho}}{(1 - {\hat \rho}^2) (1 - \overline{\rho}^2)}}{\left[ \dfrac{1 - \theta}{1 - {\hat \rho}^2} + \dfrac{\theta}{1 - \overline{\rho}^2} \right]^2} > 0 $}, that is, {\footnotesize $ \dfrac{2 \Gamma (\theta, \overline{\rho}, {\hat \rho})}{1 + \Gamma^2 (\theta, \overline{\rho}, {\hat \rho})} $} is increasing in $ \theta $.

Case 2.1: {\footnotesize $ \left[ \dfrac{- 1}{1 - {\hat \rho}^2} {\hat \rho} + \dfrac1{1 - \overline{\rho}^2} \overline{\rho} \right] < \left[ \dfrac{- 1}{1 - {\hat \rho}^2} + \dfrac1{1 - \overline{\rho}^2} \right] E_{2 1} $}, that is, {\footnotesize $ {\hat \rho} + \overline{\rho} > 0 \ \text{and} \ E_{2 1} > \dfrac{1 + {\hat \rho} \overline{\rho}}{{\hat \rho} + \overline{\rho}} $}, then
{\footnotesize \begin{eqnarray*}
\left[ \dfrac{- 1}{1 - {\hat \rho}^2} {\hat \rho} + \dfrac1{1 - \overline{\rho}^2} \overline{\rho} \right] E_{2 1} - \left[ \dfrac{- 1}{1 - {\hat \rho}^2} + \dfrac1{1 - \overline{\rho}^2} \right] = \dfrac{\overline{\rho} - {\hat \rho}}{(1 - {\hat \rho}^2) (1 - \overline{\rho}^2)} \left\{ \left[ 1 + {\hat \rho} \overline{\rho} \right] E_{2 1} - \left[ {\hat \rho} + \overline{\rho} \right] \right\} > 0
\end{eqnarray*}}
and
{\footnotesize \begin{eqnarray*}
\text{sign} \dfrac{\partial R_1 (\theta, \overline{\rho}, {\hat \rho})}{\partial \theta} = - \text{sign} \left\{ \dfrac{\left[ \dfrac{- 1}{1 - {\hat \rho}^2} {\hat \rho} + \dfrac1{1 - \overline{\rho}^2} \overline{\rho} \right] E_{2 1} - \left[ \dfrac{- 1}{1 - {\hat \rho}^2} + \dfrac1{1 - \overline{\rho}^2} \right]}{\left[ \dfrac{- 1}{1 - {\hat \rho}^2} {\hat \rho} + \dfrac1{1 - \overline{\rho}^2} \overline{\rho} \right] - \left[ \dfrac{- 1}{1 - {\hat \rho}^2} + \dfrac1{1 - \overline{\rho}^2} \right] E_{2 1}} + \dfrac{2 \Gamma (\theta, \overline{\rho}, {\hat \rho})}{1 + \Gamma^2 (\theta, \overline{\rho}, {\hat \rho})} \right\}.
\end{eqnarray*}}

On the other hand, the condition {\footnotesize $ E_{2 1} > \dfrac{1 + {\hat \rho} \overline{\rho}}{{\hat \rho} + \overline{\rho}} $} is equivalent to:
{\footnotesize \begin{eqnarray*}
\dfrac{\left[ \dfrac{- 1}{1 - {\hat \rho}^2} {\hat \rho} + \dfrac1{1 - \overline{\rho}^2} \overline{\rho} \right] E_{2 1} - \left[ \dfrac{- 1}{1 - {\hat \rho}^2} + \dfrac1{1 - \overline{\rho}^2} \right]}{\left[ \dfrac{- 1}{1 - {\hat \rho}^2} {\hat \rho} + \dfrac1{1 - \overline{\rho}^2} \overline{\rho} \right] - \left[ \dfrac{- 1}{1 - {\hat \rho}^2} + \dfrac1{1 - \overline{\rho}^2} \right] E_{2 1}} < - \dfrac{1 + {\hat \rho} \overline{\rho}}{{\hat \rho} + \overline{\rho}},
\end{eqnarray*}}
and hence:
{\footnotesize \begin{eqnarray*}
\dfrac{\left[ \dfrac{- 1}{1 - {\hat \rho}^2} {\hat \rho} + \dfrac1{1 - \overline{\rho}^2} \overline{\rho} \right] E_{2 1} - \left[ \dfrac{- 1}{1 - {\hat \rho}^2} + \dfrac1{1 - \overline{\rho}^2} \right]}{\left[ \dfrac{- 1}{1 - {\hat \rho}^2} {\hat \rho} + \dfrac1{1 - \overline{\rho}^2} \overline{\rho} \right] - \left[ \dfrac{- 1}{1 - {\hat \rho}^2} + \dfrac1{1 - \overline{\rho}^2} \right] E_{2 1}} + \dfrac{2 \Gamma (\theta, \overline{\rho}, {\hat \rho})}{1 + \Gamma^2 (\theta, \overline{\rho}, {\hat \rho})} < - \dfrac{1 + {\hat \rho} \overline{\rho}}{{\hat \rho} + \overline{\rho}}  + \overline{\rho} = - \dfrac{1 - \overline{\rho}^2}{{\hat \rho} + \overline{\rho}} < 0.
\end{eqnarray*}}
Therefore, {\footnotesize $ \dfrac{\partial R_1 (\theta, \overline{\rho}, {\hat \rho})}{\partial \theta} > 0 $}.

%Case 2.2: {\footnotesize $ \left[ \dfrac{- 1}{1 - {\hat \rho}^2} {\hat \rho} + \dfrac1{1 - \overline{\rho}^2} \overline{\rho} \right] > \left[ \dfrac{- 1}{1 - {\hat \rho}^2} + \dfrac1{1 - \overline{\rho}^2} \right] E_{2 1} $}, that is, {\footnotesize $ {\hat \rho} + \overline{\rho} < 0 $} or {\footnotesize $ {\hat \rho} + \overline{\rho} > 0 \ \text{and} \ E_{2 1} < \dfrac{1 + {\hat \rho} \overline{\rho}}{{\hat \rho} + \overline{\rho}} $}, then
Case 2.2: {\footnotesize $ \left[ \dfrac{- 1}{1 - {\hat \rho}^2} {\hat \rho} + \dfrac1{1 - \overline{\rho}^2} \overline{\rho} \right] > \left[ \dfrac{- 1}{1 - {\hat \rho}^2} + \dfrac1{1 - \overline{\rho}^2} \right] E_{2 1} $}, that is, {\footnotesize $ {\hat \rho} + \overline{\rho} > 0 \ \text{and} \ E_{2 1} < \dfrac{1 + {\hat \rho} \overline{\rho}}{{\hat \rho} + \overline{\rho}} $}, then
{\footnotesize \begin{eqnarray*}
\text{sign} \dfrac{\partial R_1 (\theta, \overline{\rho}, {\hat \rho})}{\partial \theta} = \text{sign} \left\{ \dfrac{\left[ \dfrac{- 1}{1 - {\hat \rho}^2} {\hat \rho} + \dfrac1{1 - \overline{\rho}^2} \overline{\rho} \right] E_{2 1} - \left[ \dfrac{- 1}{1 - {\hat \rho}^2} + \dfrac1{1 - \overline{\rho}^2} \right]}{\left[ \dfrac{- 1}{1 - {\hat \rho}^2} {\hat \rho} + \dfrac1{1 - \overline{\rho}^2} \overline{\rho} \right] - \left[ \dfrac{- 1}{1 - {\hat \rho}^2} + \dfrac1{1 - \overline{\rho}^2} \right] E_{2 1}} + \dfrac{2 \Gamma (\theta, \overline{\rho}, {\hat \rho})}{1 + \Gamma^2 (\theta, \overline{\rho}, {\hat \rho})} \right\}.
\end{eqnarray*}}

Case 2.2.1: {\footnotesize $ \dfrac{\left[ \dfrac{- 1}{1 - {\hat \rho}^2} {\hat \rho} + \dfrac1{1 - \overline{\rho}^2} \overline{\rho} \right] E_{2 1} - \left[ \dfrac{- 1}{1 - {\hat \rho}^2} + \dfrac1{1 - \overline{\rho}^2} \right]}{\left[ \dfrac{- 1}{1 - {\hat \rho}^2} {\hat \rho} + \dfrac1{1 - \overline{\rho}^2} \overline{\rho} \right] - \left[ \dfrac{- 1}{1 - {\hat \rho}^2} + \dfrac1{1 - \overline{\rho}^2} \right] E_{2 1}} < - \overline{\rho} $}, then {\footnotesize $ \dfrac{\partial R_1 (\theta, \overline{\rho}, {\hat \rho})}{\partial \theta} < 0 $} for {\footnotesize $ E_{2 1} < \hat{\rho} $}.

Case 2.2.2: {\footnotesize $ - \overline{\rho} < \dfrac{\left[ \dfrac{- 1}{1 - {\hat \rho}^2} {\hat \rho} + \dfrac1{1 - \overline{\rho}^2} \overline{\rho} \right] E_{2 1} - \left[ \dfrac{- 1}{1 - {\hat \rho}^2} + \dfrac1{1 - \overline{\rho}^2} \right]}{\left[ \dfrac{- 1}{1 - {\hat \rho}^2} {\hat \rho} + \dfrac1{1 - \overline{\rho}^2} \overline{\rho} \right] - \left[ \dfrac{- 1}{1 - {\hat \rho}^2} + \dfrac1{1 - \overline{\rho}^2} \right] E_{2 1}} < - \hat{\rho} $}, then there exists a unique {\footnotesize $ \theta^* \in [0, 1] $} such that {\footnotesize $ \dfrac{\left[ \dfrac{- 1}{1 - {\hat \rho}^2} {\hat \rho} + \dfrac1{1 - \overline{\rho}^2} \overline{\rho} \right] E_{2 1} - \left[ \dfrac{- 1}{1 - {\hat \rho}^2} + \dfrac1{1 - \overline{\rho}^2} \right]}{\left[ \dfrac{- 1}{1 - {\hat \rho}^2} {\hat \rho} + \dfrac1{1 - \overline{\rho}^2} \overline{\rho} \right] - \left[ \dfrac{- 1}{1 - {\hat \rho}^2} + \dfrac1{1 - \overline{\rho}^2} \right] E_{2 1}} + \dfrac{2 \Gamma (\theta^*, \overline{\rho}, {\hat \rho})}{1 + \Gamma^2 (\theta^*, \overline{\rho}, {\hat \rho})} = 0 $}. Therefore\footnote{\baselineskip1.3em There exists a unique {\footnotesize $ \rho^* \in [\hat{\rho}, \overline{\rho}] $} such that {\footnotesize $ \dfrac{\partial R_1 (\theta, \overline{\rho}, {\hat \rho})}{\partial \theta} < 0 $} for {\footnotesize $ E_{2 1} < \rho^* $} and {\footnotesize $ \dfrac{\partial R_1 (\theta, \overline{\rho}, {\hat \rho})}{\partial \theta} > 0 $} for {\footnotesize $ E_{2 1} > \rho^* $}.} {\footnotesize $ \dfrac{\partial R_1 (\theta, \overline{\rho}, {\hat \rho})}{\partial \theta} < 0 $} for {\footnotesize $ \theta < \theta^* $} and {\footnotesize $ \dfrac{\partial R_1 (\theta, \overline{\rho}, {\hat \rho})}{\partial \theta} > 0 $} for {\footnotesize $ \theta > \theta^* $}.

Case 2.2.3: {\footnotesize $ - \hat{\rho} < \dfrac{\left[ \dfrac{- 1}{1 - {\hat \rho}^2} {\hat \rho} + \dfrac1{1 - \overline{\rho}^2} \overline{\rho} \right] E_{2 1} - \left[ \dfrac{- 1}{1 - {\hat \rho}^2} + \dfrac1{1 - \overline{\rho}^2} \right]}{\left[ \dfrac{- 1}{1 - {\hat \rho}^2} {\hat \rho} + \dfrac1{1 - \overline{\rho}^2} \overline{\rho} \right] - \left[ \dfrac{- 1}{1 - {\hat \rho}^2} + \dfrac1{1 - \overline{\rho}^2} \right] E_{2 1}} $}, then {\footnotesize $ \dfrac{\partial R_1 (\theta, \overline{\rho}, {\hat \rho})}{\partial \theta} > 0 $} for {\footnotesize $ E_{2 1} > \overline{\rho} $}.

\subsection*{A.8 \quad Participating equilibrium prices change with maximum correlation coefficient for na\"ive investors}

\quad
In Theorem 2, the equilibrium Sharpe ratio for the risky assets is:
{\footnotesize \begin{eqnarray*}
R_1 (\theta, \overline{\rho}, {\hat \rho}) = & \dfrac{\mu_1 - p_1}{\sigma_1} & = \alpha \left[ Q (\theta, \overline{\rho}, {\hat \rho}) \sigma_1 Z_1^0 + q (\theta, \overline{\rho}, {\hat \rho}) \sigma_2 Z_2^0 \right] \\
R_2 (\theta, \overline{\rho}, {\hat \rho}) = & \dfrac{\mu_2 - p_2}{\sigma_2} & = \alpha \left[ q (\theta, \overline{\rho}, {\hat \rho}) \sigma_1 Z_1^0 + Q (\theta, \overline{\rho}, {\hat \rho}) \sigma_2 Z_2^0 \right].
\end{eqnarray*}}
then
{\footnotesize \begin{eqnarray*}
& & \dfrac{1}{\alpha \theta Q^2 (\theta, \overline{\rho}, {\hat \rho})} \dfrac{(1 - \overline{\rho}^2)^2}{1 + \overline{\rho}^2} \dfrac{\partial R_1 (\theta, \overline{\rho}, {\hat \rho})}{\partial \overline{\rho}} \\
& = & \sigma_1 Z_1^0 \left[ E_{2 1} - \dfrac{2 \overline{\rho}}{1 + \overline{\rho}^2} \right] \left\{ 1 + \Gamma^2 (\theta, \overline{\rho}, {\hat \rho}) \right\} + 2 \sigma_2 Z_2^0 \left[ E_{1 2} - \dfrac{2 \overline{\rho}}{1 + \overline{\rho}^2} \right] \Gamma (\theta, \overline{\rho}, {\hat \rho}) \\
%& = & \sigma_2 Z_2^0 \left\{ 1 + \Gamma^2 (\theta, \overline{\rho}, {\hat \rho}) \right\} \left\{ \left[ 1 - \dfrac{2 \overline{\rho}}{1 + \overline{\rho}^2} E_{1 2} \right] + \left[ E_{1 2} - \dfrac{2 \overline{\rho}}{1 + \overline{\rho}^2} \right] \dfrac{2 \Gamma (\theta, \overline{\rho}, {\hat \rho})}{1 + \Gamma^2 (\theta, \overline{\rho}, {\hat \rho})} \right\} \\
& = & \sigma_1 Z_1^0 \left\{ 1 + \Gamma^2 (\theta, \overline{\rho}, {\hat \rho}) \right\} \left[ 1 - \dfrac{2 \overline{\rho}}{1 + \overline{\rho}^2} \dfrac{2 \Gamma (\theta, \overline{\rho}, {\hat \rho})}{1 + \Gamma^2 (\theta, \overline{\rho}, {\hat \rho})} 
\right] \left\{ E_{2 1} - \dfrac{\dfrac{2 \overline{\rho}}{1 + \overline{\rho}^2} - \dfrac{2 \Gamma (\theta, \overline{\rho}, {\hat \rho})}{1 + \Gamma^2 (\theta, \overline{\rho}, {\hat \rho})}}{1 - \dfrac{2 \overline{\rho}}{1 + \overline{\rho}^2} \dfrac{2 \Gamma (\theta, \overline{\rho}, {\hat \rho})}{1 + \Gamma^2 (\theta, \overline{\rho}, {\hat \rho})}} \right\} \\
& & \dfrac{1}{\alpha \theta Q^2 (\theta, \overline{\rho}, {\hat \rho})} \dfrac{(1 - \overline{\rho}^2)^2}{1 + \overline{\rho}^2} \dfrac{\partial R_2 (\theta, \overline{\rho}, {\hat \rho})}{\partial \overline{\rho}} \\
& = & \sigma_2 Z_2^0 \left[ E_{1 2} - \dfrac{2 \overline{\rho}}{1 + \overline{\rho}^2} \right] \left\{ 1 + \Gamma^2 (\theta, \overline{\rho}, {\hat \rho}) \right\} + 2 \sigma_1 Z_1^0 \left[ E_{2 1} - \dfrac{2 \overline{\rho}}{1 + \overline{\rho}^2} \right] \Gamma (\theta, \overline{\rho}, {\hat \rho}) \\
%& = & \sigma_1 Z_1^0 \left\{ 1 + \Gamma^2 (\theta, \overline{\rho}, {\hat \rho}) \right\} \left\{ \left[ 1 - \dfrac{2 \overline{\rho}}{1 + \overline{\rho}^2} E_{2 1} \right] + \left[ E_{2 1} - \dfrac{2 \overline{\rho}}{1 + \overline{\rho}^2} \right] \dfrac{2 \Gamma (\theta, \overline{\rho}, {\hat \rho})}{1 + \Gamma^2 (\theta, \overline{\rho}, {\hat \rho})} \right\} \\
& = & \sigma_2 Z_2^0 \left\{ 1 + \Gamma^2 (\theta, \overline{\rho}, {\hat \rho}) \right\} \left[ 1 - \dfrac{2 \overline{\rho}}{1 + \overline{\rho}^2} \dfrac{2 \Gamma (\theta, \overline{\rho}, {\hat \rho})}{1 + \Gamma^2 (\theta, \overline{\rho}, {\hat \rho})} 
\right] \left\{ E_{1 2} - \dfrac{\dfrac{2 \overline{\rho}}{1 + \overline{\rho}^2} - \dfrac{2 \Gamma (\theta, \overline{\rho}, {\hat \rho})}{1 + \Gamma^2 (\theta, \overline{\rho}, {\hat \rho})}}{1 - \dfrac{2 \overline{\rho}}{1 + \overline{\rho}^2} \dfrac{2 \Gamma (\theta, \overline{\rho}, {\hat \rho})}{1 + \Gamma^2 (\theta, \overline{\rho}, {\hat \rho})}} \right\}
\end{eqnarray*}}
where {\footnotesize $ \Gamma (\theta, \overline{\rho}, {\hat \rho}) \equiv \dfrac{\dfrac{1 - \theta}{1 - {\hat \rho}^2} {\hat \rho} + \dfrac{\theta}{1 - \overline{\rho}^2} \overline{\rho}}{\dfrac{1 - \theta}{1 - {\hat \rho}^2} + \dfrac{\theta}{1 - \overline{\rho}^2}} $}.

We first explore the monotonicity of {\footnotesize $ R_1 $} in {\footnotesize $ \overline{\rho} $}.
{\footnotesize \begin{eqnarray*}
\dfrac{\partial}{\partial \overline{\rho}} \left( \dfrac{2 \overline{\rho}}{1 + \overline{\rho}^2} \right) = \dfrac{2 (1 - \overline{\rho}^2)}{(1 + \overline{\rho}^2)^2} > 0 \qquad \text{and} \qquad \dfrac{\partial}{\partial \overline{\rho}} \left( \dfrac{2 \Gamma (\theta, \overline{\rho}, {\hat \rho})}{1 + \Gamma^2 (\theta, \overline{\rho}, {\hat \rho})} \right) = \dfrac{2 \{ 1 - \Gamma^2 (\theta, \overline{\rho}, {\hat \rho}) \}}{\{ 1 + \Gamma^2 (\theta, \overline{\rho}, {\hat \rho}) \}^2} \dfrac{\partial \Gamma (\theta, \overline{\rho}, {\hat \rho})}{\partial \overline{\rho}} > 0
\end{eqnarray*}}
from {\footnotesize $ \dfrac{\partial \Gamma (\theta, \overline{\rho}, {\hat \rho})}{\partial \overline{\rho}} = \dfrac{\theta}{(1 - \overline{\rho}^2)^2} \dfrac{1 + \dfrac{1 - \theta}{1 - {\hat \rho}^2} (\overline{\rho} - \hat{\rho})^2}{\left[ \dfrac{1 - \theta}{1 - {\hat \rho}^2} + \dfrac{\theta}{1 - \overline{\rho}^2} \right]^2} > 0 $}. Then
{\footnotesize \begin{eqnarray*}
& & \left[ 1 - \dfrac{2 \overline{\rho}}{1 + \overline{\rho}^2} \dfrac{2 \Gamma (\theta, \overline{\rho}, {\hat \rho})}{1 + \Gamma^2 (\theta, \overline{\rho}, {\hat \rho})} \right]^2 \dfrac{\partial}{\partial \overline{\rho}} \left( \dfrac{\dfrac{2 \overline{\rho}}{1 + \overline{\rho}^2} - \dfrac{2 \Gamma (\theta, \overline{\rho}, {\hat \rho})}{1 + \Gamma^2 (\theta, \overline{\rho}, {\hat \rho})}}{1 - \dfrac{2 \overline{\rho}}{1 + \overline{\rho}^2} \dfrac{2 \Gamma (\theta, \overline{\rho}, {\hat \rho})}{1 + \Gamma^2 (\theta, \overline{\rho}, {\hat \rho})}} \right) \\
%& = & \left\{ \dfrac{2 (1 - \overline{\rho}^2)}{(1 + \overline{\rho}^2)^2} - \dfrac{2 \{ 1 - \Gamma^2 (\theta, \overline{\rho}, {\hat \rho}) \}}{\{ 1 + \Gamma^2 (\theta, \overline{\rho}, {\hat \rho}) \}^2} \dfrac{\partial \Gamma (\theta, \overline{\rho}, {\hat \rho})}{\partial \overline{\rho}} \right\} \left[ 1 - \dfrac{2 \overline{\rho}}{1 + \overline{\rho}^2} \dfrac{2 \Gamma (\theta, \overline{\rho}, {\hat \rho})}{1 + \Gamma^2 (\theta, \overline{\rho}, {\hat \rho})} \right] \\
%& & + \left[ \dfrac{2 \overline{\rho}}{1 + \overline{\rho}^2} - \dfrac{2 \Gamma (\theta, \overline{\rho}, {\hat \rho})}{1 + \Gamma^2 (\theta, \overline{\rho}, {\hat \rho})} \right] \left\{ \dfrac{2 (1 - \overline{\rho}^2)}{(1 + \overline{\rho}^2)^2} \dfrac{2 \Gamma (\theta, \overline{\rho}, {\hat \rho})}{1 + \Gamma^2 (\theta, \overline{\rho}, {\hat \rho})} + \dfrac{2 \overline{\rho}}{1 + \overline{\rho}^2} \dfrac{2 \{ 1 - \Gamma^2 (\theta, \overline{\rho}, {\hat \rho}) \}}{\{ 1 + \Gamma^2 (\theta, \overline{\rho}, {\hat \rho}) \}^2} \dfrac{\partial \Gamma (\theta, \overline{\rho}, {\hat \rho})}{\partial \overline{\rho}} \right\} \\
%& = & \dfrac{2 (1 - \overline{\rho}^2)}{(1 + \overline{\rho}^2)^2} \left[ 1 - \left( \dfrac{2 \Gamma (\theta, \overline{\rho}, {\hat \rho})}{1 + \Gamma^2 (\theta, \overline{\rho}, {\hat \rho})} \right)^2 \right] - \dfrac{2 \{ 1 - \Gamma^2 (\theta, \overline{\rho}, {\hat \rho}) \}}{\{ 1 + \Gamma^2 (\theta, \overline{\rho}, {\hat \rho}) \}^2} \left[ 1 - \left( \dfrac{2 \overline{\rho}}{1 + \overline{\rho}^2} \right)^2 \right] \dfrac{\partial \Gamma (\theta, \overline{\rho}, {\hat \rho})}{\partial \overline{\rho}} \\
& = & \dfrac{2 (1 - \overline{\rho}^2)}{(1 + \overline{\rho}^2)^2} \dfrac{\{ 1 - \Gamma^2 (\theta, \overline{\rho}, {\hat \rho}) \}^2}{\{ 1 + \Gamma^2 (\theta, \overline{\rho}, {\hat \rho}) \}^2} - \dfrac{2 \{ 1 - \Gamma^2 (\theta, \overline{\rho}, {\hat \rho}) \}}{\{ 1 + \Gamma^2 (\theta, \overline{\rho}, {\hat \rho}) \}^2} \dfrac{(1 - \overline{\rho}^2)^2}{(1 + \overline{\rho}^2)^2} \dfrac{\partial \Gamma (\theta, \overline{\rho}, {\hat \rho})}{\partial \overline{\rho}} \\
& = & 2 \dfrac{1 - \overline{\rho}^2}{(1 + \overline{\rho}^2)^2} \dfrac{1 - \Gamma^2 (\theta, \overline{\rho}, {\hat \rho})}{\{ 1 + \Gamma^2 (\theta, \overline{\rho}, {\hat \rho}) \}^2} \left\{ \{ 1 - \Gamma^2 (\theta, \overline{\rho}, {\hat \rho}) \} - (1 - \overline{\rho}^2) \dfrac{\theta}{(1 - \overline{\rho}^2)^2} \dfrac{1 + \dfrac{1 - \theta}{1 - {\hat \rho}^2} (\overline{\rho} - \hat{\rho})^2}{\left[ \dfrac{1 - \theta}{1 - {\hat \rho}^2} + \dfrac{\theta}{1 - \overline{\rho}^2} \right]^2} \right\} \\
%& = & 2 \dfrac{1 - \overline{\rho}^2}{(1 + \overline{\rho}^2)^2} \dfrac{1 - \Gamma^2 (\theta, \overline{\rho}, {\hat \rho})}{\{ 1 + \Gamma^2 (\theta, \overline{\rho}, {\hat \rho}) \}^2} \left\{ \dfrac{\dfrac{1 - \theta}{1 - {\hat \rho}} + \dfrac{\theta}{1 - \overline{\rho}}}{\dfrac{1 - \theta}{1 - {\hat \rho}^2} + \dfrac{\theta}{1 - \overline{\rho}^2}} \dfrac{\dfrac{1 - \theta}{1 + {\hat \rho}} + \dfrac{\theta}{1 + \overline{\rho}}}{\dfrac{1 - \theta}{1 - {\hat \rho}^2} + \dfrac{\theta}{1 - \overline{\rho}^2}} - \dfrac{\theta}{1 - \overline{\rho}^2} \dfrac{1 + \dfrac{1 - \theta}{1 - {\hat \rho}^2} (\overline{\rho} - \hat{\rho})^2}{\left[ \dfrac{1 - \theta}{1 - {\hat \rho}^2} + \dfrac{\theta}{1 - \overline{\rho}^2} \right]^2} \right\} \\
%& = & \dfrac{2 \dfrac{1 - \overline{\rho}^2}{(1 + \overline{\rho}^2)^2} \dfrac{1 - \Gamma^2 (\theta, \overline{\rho}, {\hat \rho})}{\{ 1 + \Gamma^2 (\theta, \overline{\rho}, {\hat \rho}) \}^2}}{\left[ \dfrac{1 - \theta}{1 - {\hat \rho}^2} + \dfrac{\theta}{1 - \overline{\rho}^2} \right]^2} \left\{ \left[ \dfrac{1 - \theta}{1 - {\hat \rho}} + \dfrac{\theta}{1 - \overline{\rho}} \right] \left[ \dfrac{1 - \theta}{1 + {\hat \rho}} + \dfrac{\theta}{1 + \overline{\rho}} \right] - \dfrac{\theta}{1 - \overline{\rho}^2} \left[ 1 + \dfrac{1 - \theta}{1 - {\hat \rho}^2} (\overline{\rho} - \hat{\rho})^2 \right] \right\} \\
& = & \dfrac{2 \dfrac{1 - \overline{\rho}^2}{(1 + \overline{\rho}^2)^2} \dfrac{1 - \Gamma^2 (\theta, \overline{\rho}, {\hat \rho})}{\{ 1 + \Gamma^2 (\theta, \overline{\rho}, {\hat \rho}) \}^2}}{\left[ \dfrac{1 - \theta}{1 - {\hat \rho}^2} + \dfrac{\theta}{1 - \overline{\rho}^2} \right]^2} \dfrac{1 - \theta}{1 - {\hat \rho}^2} > 0,
\end{eqnarray*}}
that is, {\footnotesize $ \dfrac{\partial}{\partial \overline{\rho}} \left( \dfrac{\dfrac{2 \overline{\rho}}{1 + \overline{\rho}^2} - \dfrac{2 \Gamma (\theta, \overline{\rho}, {\hat \rho})}{1 + \Gamma^2 (\theta, \overline{\rho}, {\hat \rho})}}{1 - \dfrac{2 \overline{\rho}}{1 + \overline{\rho}^2} \dfrac{2 \Gamma (\theta, \overline{\rho}, {\hat \rho})}{1 + \Gamma^2 (\theta, \overline{\rho}, {\hat \rho})}} \right) > 0 $} with
{\footnotesize $ \max\limits_{\overline{\rho} \in [\hat{\rho}, 1]} \dfrac{\dfrac{2 \overline{\rho}}{1 + \overline{\rho}^2} - \dfrac{2 \Gamma (\theta, \overline{\rho}, {\hat \rho})}{1 + \Gamma^2 (\theta, \overline{\rho}, {\hat \rho})}}{1 - \dfrac{2 \overline{\rho}}{1 + \overline{\rho}^2} \dfrac{2 \Gamma (\theta, \overline{\rho}, {\hat \rho})}{1 + \Gamma^2 (\theta, \overline{\rho}, {\hat \rho})}} = \dfrac{2 (1 - \theta) (1 + \theta \hat{\rho})}{(1 - \theta)^2 + (1 + \theta\hat{\rho})^2} $}.

Thus {\footnotesize $ \dfrac{\partial R_1 (\theta, \overline{\rho}, {\hat \rho})}{\partial \overline{\rho}} > 0 $} when {\footnotesize $ E_{1 2} \geqslant \dfrac{2 (1 - \theta) (1 + \theta \hat{\rho})}{(1 - \theta)^2 + (1 + \theta\hat{\rho})^2} $}, and there exists {\footnotesize $ \overline{\rho}_1^* \in [{\hat \rho}, 1] $} such that {\footnotesize $ \dfrac{\partial R_1 (\theta, \overline{\rho}, {\hat \rho})}{\partial \overline{\rho}} < 0 $} for {\footnotesize $ \overline{\rho} < \overline{\rho}_1^* $} and {\footnotesize $ \dfrac{\partial R_1 (\theta, \overline{\rho}, {\hat \rho})}{\partial \overline{\rho}} > 0 $} for {\footnotesize $ \overline{\rho} > \overline{\rho}_1^* $} when {\footnotesize $ E_{1 2} < \dfrac{2 (1 - \theta) (1 + \theta \hat{\rho})}{(1 - \theta)^2 + (1 + \theta\hat{\rho})^2} $}. 

\vskip 8 pt

Similarly, we can explore the monotonicity of $ R_2 $ in $ \overline{\rho} $.
{\footnotesize $ \dfrac{\partial R_2 (\theta, \overline{\rho}, {\hat \rho})}{\partial \overline{\rho}} > 0 $} when {\footnotesize $ E_{2 1} \geqslant \dfrac{2 (1 - \theta) (1 + \theta \hat{\rho})}{(1 - \theta)^2 + (1 + \theta\hat{\rho})^2} $}, and there exists {\footnotesize $ \overline{\rho}_2^* \in [{\hat \rho}, 1] $} such that {\footnotesize $ \dfrac{\partial R_2 (\theta, \overline{\rho}, {\hat \rho})}{\partial \overline{\rho}} < 0 $} for {\footnotesize $ \overline{\rho} < \overline{\rho}_2^* $} and {\footnotesize $ \dfrac{\partial R_1 (\theta, \overline{\rho}, {\hat \rho})}{\partial \overline{\rho}} > 0 $} for {\footnotesize $ \overline{\rho} > \overline{\rho}_2^* $} when {\footnotesize $ E_{2 1} < \dfrac{2 (1 - \theta) (1 + \theta \hat{\rho})}{(1 - \theta)^2 + (1 + \theta\hat{\rho})^2} $}. 

%\end{document}

%\newpage
%\vskip 32 pt

%\noindent {\Large Acknowledgements} 

%This work is supported by the Social Sciences and Humanities Research Council of Canada (SSHRC) Insight Development Grant ($\#$ 430-2012-0698), the National Natural Science Foundation of China (NSFC Grant Number: 71273271 and 71573220) and the Major Basic Research Plan of Renmin University of China (Grant Number: 14XNL001).

\vskip 16 pt

\begin{thebibliography}{99}

\bibitem{A} A. Cevdet Aydemir (2008). {\it Risk Sharing and Counter-Cyclical Variation in Market Correlations}. Journal of Economic Dynamics and Control, 32 (10), 3084-3112.

%\bibitem{B} Rolf Banz (1981). {\it The Relationship between Return And Market Value of Common Stocks}. Journal of Financial Economics, 9 (1), 3-18.

%\bibitem{BT} Nicholas Barberis and Richard Thaler (2003). {\it A Survey of Behavioral Finance}. Handbook of the Economics of Finance, 1, 1053-1128.

\bibitem{BEW} Tim Bollerslev, Robert F. Engle and Jeffrey M. Wooldridge (1988). {\it A Capital Asset Pricing Model with Time-Varying Covariances}. Journal of Political Economy, 96 (1), 116-131.

\bibitem{CK} Ricardo J. Caballero and A. Krishnamurthy (2007). {\it Collective Risk Management in a Flight to Quality Episode}. Journal of Finance, 63 (5), 2195-2230.

%\bibitem{CC} John Y. Campbell and John Cochrane (1999). {\it By Force of Habit: A Consumption-Based Explanation of Aggregate Stock Market Behavior}. Journal of Political Economy, 107 (2), 205-251.

%\bibitem{CY} John Y. Campbell (2006). {\it Household Finance}. Journal of Finance, 61 (4), 1553-1604.

\bibitem{CWZ} H. Henry Cao, Tan Wang and Harold H. Zhang (2005). {\it Model Uncertainty, Limited Market Participation and Asset Prices}. Review of Financial Studies, 18 (4), 1219-1251.

\bibitem{CE} Zengjing Chen and Larry G. Epstein (2002). {\it Ambiguity, Risk, and Asset Payoffs in Continuous Time}. Econometrica 70 (6), 1403-1043.

\bibitem{CG} Scott Condie and Jayant V. Ganguli (2011). {\it Ambiguity and Rational Expectations Equilibria}. Review of Economic Studies, 78 (3), 821-845.

%\bibitem{DW} James Dow and Sergio Ribeiro da Costa Werlang. (1992). {\it Uncertainty Aversion, Risk Aversion and the Optimal Choice of Portfolio}. Econometrica, 60 (1), 197-204.

\bibitem{DS} Darrell Duffie and Kenneth J. Singleton (2003). {\bf Credit Risk: Pricing, Measurement, and Management}. Princeton University Press.

\bibitem{EO} David Easley and Maureen O'Hara (2009). {\it Ambiguity and Nonparticipation: The Role of Regulation}. Review of Financial Studies, 22 (5), 1817-1846.

\bibitem{EO} David Easley and Maureen O'Hara (2010). {\it Microstructure and Ambiguity}. Journal of Finance, 65 (5), 1817-1843.

\bibitem{EOY} David Easley, Maureen O'Hara and Liyan Yang (2014). {\it Opaque Trading, Disclosure and Asset Prices: Implications for Hedge Fund Regulation}. Review of Financial Studies, 27 (4), 1190-1237.

%\bibitem{E} Daniel Ellsberg (1961). {\it Risk, Ambiguity and Savage Axioms}. Quarterly Journal of Economics, 75 (4), 643-669.

%\bibitem{EMS} Paul Embrechts, Alexander McNeil and Daniel Straumann (2015). {\it Correlation and Dependency in Risk Management: Properties and Pitfalls}. Risk Management Value at Risk and Beyond, 1 (1), 176-223.

\bibitem{EJ} Larry G. Epstein and Shaolin Ji (2012). {\it Ambiguous Volatility and Asset Pricing in Continuous Time}. Review of Financial Studies, 26 (7), 1740-1786.

\bibitem{EJ} Larry G. Epstein and Shaolin Ji (2013). {\it Ambiguous Volatility, Possibility and Utility in Continuous Time}. Journal of Mathematical Economics, 50 (1), 269-282.

\bibitem{EJ} Larry G. Epstein and Martin Schneider (2008). {\it Ambiguity, Information Quality and Asset Pricing}. Journal of Finance, 63 (1), 197-228.

%\bibitem{FF} Eugene Fama and Kenneth French (1992). {\it The Cross-section of Expected Stock Returns}. Journal of Finance, 47(2), 427-465.

%\bibitem{FASB} Financial Accounting Standards Board (1982). {\it Statement of Financial Accounting Standards No. 57, Related Party Disclosures}. Norwalk, CT: FASB (FAS 57).

%\bibitem{GMM} Paolo Ghirardato, Fabio Maccheroni and Massimo Marinacci (2004). {\it Differentiating Ambiguity and Ambiguity Attitude}. Journal of Economic Theory, 118 (2), 133-173.

\bibitem{GS} Itzhak Gilboa and David Schmeidler (1989). {\it Maximum Expected Utility Theory with Non-unique Prior}. Journal of
Mathematical Economics, 18 (2), 141-53.

%\bibitem{GJS} Elena Gouskova, F. Thomas Juster and Frank P. Stafford (2004). {\it Exploring the Changing Nature of U.S. Stock Market Participation, 1994-1999}. Working Paper, University of Michigan.

%\bibitem{HKS} Harrison Hong, Jeffrey D. Kubik and Jeremy C. Stein (2004). {\it Social Interaction and Stock-Market Participation}. Journal of Finance, 49 (1), 137-63.

\bibitem{I} Phillip K. Illeditsch (2009). {\it Ambiguous Information, Risk Aversion, and Asset Pricing}. Working Paper, University of Pennsylvania.

%\bibitem{KT} Markku Kaustia and Sami Torstila (2011). {\it Stock Market Aversion? Political Preferences and Stock Market Participation}. Journal of Financial Economics, 100 (1), 98-112.

%\bibitem{KMM} Peter Klibanoff, Massimo Marinacci and Sujoy Mukerji (2005). {\it A Smooth Model of Decision Making under Ambiguity}. Econometrica, 76 (6), 1849-1892.

\bibitem{K} Frank H. Knight (1921). {\bf Risk, Ambiguity, and Profit}. Boston, MA: Houghton Mifflin.

\bibitem{KM} Mark J. Kohlbeck and Brian W. Mayhew (2004). {\it Related Party Transactions} (September 15, 2004). AAA 2005 FARS Meeting Paper. Available at SSRN: http://ssrn.com/abstract=591285 or http://dx.doi.org/10.2139/ssrn.591285

\bibitem{L} John Litner (1965). {\it The Valuation of Risk Assets and the Selection of Risky Investments in Stock Portfolios and Capital Budgets}. Review of Economics and Statistics 47 (1), 13-37.

%\bibitem{MMR} Fabio Maccheroni, Massimo Marinacci and Aldo Rustichini (2006), {\it Ambiguity Aversion, Robustness, and the Variational Representation of Preferences}. Econometrica 74 (6), 1447-1498.

\bibitem{M} Harry Markowitz (1952). {\it Portfolio Selection}. Journal of Finance, 7 (1), 77-91.

%\bibitem{MP} Rajnish Mehra and Edward Prescott (1985). {\it The Equity Premium: A Puzzle}. Journal of Monetary Economics, 15 (2), 145-161.

%\bibitem{M} Robert C. Merton (1969). {\it Lifetime Portfolio Selection under Uncertainty: The Continuous-Time Case}. Review of Economics and Statistics, 51 (3), 247-257.

%\bibitem{P} Monica Paiella (2007). {\it The Forgone Gains of Incomplete Portfolios}. Review of Financial Studies, 20 (5), 1623-1646.

%\bibitem{P} Valery Polkovnichenko (2005). {\it Household Portfolio Diversification: A Case for Rank-Dependent Preferences}. Review of Financial Studies, 18 (4), 1467-1502.

\bibitem{S} Paul A. Samuelson (1969). {\it Lifetime Portfolio Selection By Dynamic Stochastic Programming}. Review of Economics and Statistics, 51 (3), 239-246.

%\bibitem{S} Leonard J. Savage (1954). {\bf The Foundations of Statistics}. New York: Dover Publications, Wiley.

\bibitem{S} David Schmeidler (1989). {\it Subjective Probability and Expected Utility without Additivity}. Econometrica, 57 (3), 571-587.

\bibitem{S} William F. Sharpe (1964). {\it Capital Asset Prices: A Theory of Capital Market Equilibrium under Conditions of Risk}. Journal of Finance 19 (2), 425-442.

\bibitem{S} William F. Sharpe (1966). {\it Mutual Fund Performance}. Journal of Business, 39 (1), 119-138.

\end{thebibliography}

\end{document}